% !TEX root = ../main.tex

\begin{theory}
    Функція $v \in L_2[a;b]$ (нагадаємо: елементи
    $L_2[a;b]$ --- це класи еквівалентних функцій, тому
    $v$ --- це представник відповідного класу) називається
    \uline{узагальненою похідною} функції $u \in L_2[a;b]$,
    якщо для кожної функції $w \in C_0^1 [a;b] = \set{w \in C^1[a;b] \mid \supp{w} \subset (a; b)}$
    має місце рівність $$\int\limits_a^b v \cdot w dx = -\int\limits_a^b u \cdot w^\prime dx$$
    Множина функцій з $L_2[a;b]$, для яких існує узагальнена похідна, утворює
    підпростір в $L_2[a;b]$, який позначається $H^1[a;b] = W_2^1[a;b]$.
    Узагальнена похідна функції $v$ позначається через $v^\prime$.
\end{theory}

\begin{exercise}
    Довести, що для функції $u \in H^1[a;b]$ узагальнена похідна єдина,
    тобто $(u' = v_1, u' = v_2) \Rightarrow (v_1 = v_2 \text{ майже скрізь})$.
\end{exercise}

\begin{exercise}
    Перевірити, що множина $H^1[a;b]$ --- лінійний простір, щільний в $L_2[a;b]$ (звичайно, за нормою $L_2[a;b]$).
\end{exercise}

\begin{exercise}
    Для функцій $u_1, u_2 \in H^1[a;b]$ покладемо
    $\dotprod{u_1}{u_2}_H = \int\limits_a^b u_1 u_2 dx +\int\limits_a^b u_1^{\prime} u_2^{\prime} dx$.
    Доведіть:
    \begin{enumerate}
        \item $\dotprod{\cdot}{\cdot}_H$ --- скалярний добуток в $H^1[a;b]$;
        \item множина неперервно диференційовних функцій $C^1[a;b]$ щільна в $H^1[a;b]$ за нормою, породженою цим скалярним добутком;
        \item $H_1[a;b]$ --- гільбертів простір за цим скалярним добутком.
    \end{enumerate}
\end{exercise}

\begin{exercise}\label{N:3_1_4}
    Довести сепарабельність гільбертового простору $H^1[a;b]$.
\end{exercise}

\begin{exercise}
    Довести, що вкладення $C^1[a;b] \to C[a;b]$ є компактним лінійним оператором.
\end{exercise}

\begin{exercise}
    Довести, що вкладення $H_1[a;b] \to L_2[a;b]$ є
    \begin{enumerate}
        \item обмеженим лінійним оператором;
        \item компактним лінійний оператором.
    \end{enumerate}
\end{exercise}

\begin{exercise}
    \begin{enumerate}
        \item Довести вкладення $A: H^1[a;b] \to C[a;b]$, тобто що кожний
        клас функцій $[u]$, що є елементом простору $H^1[a;b]$, містить і при тому
        єдину неперервну функцію.
        \item Довести, що відображення $A$, що переводить клас $[u]$
        у відповідну неперервну функцію, є обмеженим лінійним оператором:
        $A \in L(H^1[a;b], C[a;b])$.
        \item[в)*] Довести, що $A$ --- компактний лінійний оператор. 
    \end{enumerate}
\end{exercise}

\begin{exercise}
    Нехай $\{ f_\alpha\}$ --- передкомпактна множина в $C[a;b]$.
    \begin{enumerate}
        \item Довести, що для фіксованої функції $g \in L_2[a;b]$
        множина $\{ f_\alpha \cdot g\}$ передкомпактна в $L_2[a;b]$.
        \item Нехай $g \in H^1[a;b]$. Чи можна стверджувати, що множина
        $\{ f_\alpha \cdot g\}$ передкомпактна в $H^1[a;b]$?
    \end{enumerate}
\end{exercise}

\begin{exercise}
    Чи буде множина $M = \left\{ \frac{1}{n} \sin{(nt)}\right\}_{n=1}^{\infty}$
    передкомпактною в:
    \begin{enumerate}
        \item $C[0;1]$;
        \item $C^1[0;1]$;
        \item $H^1[0;1]$?
    \end{enumerate}
\end{exercise}

\begin{exercise}
    Довести, що вказаний функціонал на $H^1[-\pi; \pi]$
    є лінійним обмеженим і знайти його норму.
    \begin{enumerate}
        \item $\varphi(x) = \int\limits_{-\pi}^{\pi} (x(t) \sin(t) + x'(t) \cos(t)) dt$;
        \item $\varphi(x) = \int\limits_{-\pi}^{\pi} (x(t) \cos(t) + x'(t) \sin(t)) dt$;
        \item[в)*] $\varphi(x) = x(0)$ ($\varphi([u]) = u(0)$).
    \end{enumerate}
\end{exercise}

\begin{exercise}
    $x \in H^1[a;b]$, $y \in C^1[a;b]$. Довести: $xy \in H^1[a;b]$.
\end{exercise}

\begin{exercise}
    Довести, що $\mathring{H}^1[a;b] = \set{x \in H^1[a;b] \mid x(a) = x(b) = 0}$
    є замкненим підпростором в $H^1[a;b]$. Знайти $\left(\mathring{H}^1[a;b]\right)^{\perp}$. 
\end{exercise}

\begin{exercise}
    Довести, що множина $M = \left\{ x \in H^1[a;b] \mid \int\limits_a^b x(t) = 0\right\}$
    є замкненим підпростором в $H^1[a;b]$. Знайти $M^{\perp}$.
\end{exercise}

\begin{exercise}
    Які з функцій $\sgn t$, $|t|$ не лежать в $H^1[-\pi; \pi]$?
\end{exercise}

\begin{exercise}\label{N:3_1_15}
    Довести, що вкладення $H^1[0; \pi]$ в $C[0; \pi]$ є строгим:
    $\exists \; x \in C[0; \pi]$ такий, що $x \notin H^1[0;\pi]$.
\end{exercise}

\begin{exercise}
    Оператор $A$ в $H = H^1[0;1]$ визначено формулою $(Ax)(t) = t x(t)$.
    Знайти $\norm{A}$, $\mathrm{Ker}{A}$. Чи вірно, що $\mathrm{Im}{A} = H$,
    $\overline{\mathrm{Im}{A}} = H$?
\end{exercise}

\begin{exercise}\label{N:3_1_17}
    Оператор $H^1[0;1] \to L_2[0;1]$ визначено формулою $Ax = x'$.
    \begin{enumerate}
        \item Знайти $\mathrm{Im}{A}$, $\mathrm{Ker}{A}$, $\norm{A}$.
        \item Довести, що оператор $A$ є замкненим.
        \item Довести, що оператор $A$ є обмеженим.
        \item Чи буде оператор $A$ компактним?
    \end{enumerate}
\end{exercise}