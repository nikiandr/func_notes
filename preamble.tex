%%% Работа с языком
%\usepackage{cmap}					    % поиск в PDF				    % русские буквы в формулах
\usepackage[12pt]{extsizes}
\usepackage[T2A]{fontenc}			% кодировка
\usepackage[utf8]{inputenc}             % кодировка исходного текста
\usepackage[english,ukrainian]{babel}	% локализация и переносы	    
\usepackage{indentfirst}
\usepackage{mdframed}
\usepackage{ulem}
\usepackage{soulutf8}

\usepackage[a4paper, top=25mm, bottom=25mm, left=30mm, right=30mm]{geometry}

\usepackage{amsmath,amsfonts,amssymb,amsthm,mathtools} % AMS
\usepackage{dsfont} %для цифр в шрифте типа mathbb
\usepackage{braket} % для команды \set

\usepackage{makecell}% Прикольная настройка ячеек в tabular 
%http://ctan.math.utah.edu/ctan/tex-archive/macros/latex/contrib/makecell/makecell-rus.pdf

\usepackage{diagbox}% Разделение клеток tabular по диагонали

\usepackage{multicol} % Несколько колонок
\usepackage{wrapfig} % Картинки посреди текста
\usepackage{chngcntr} % для нумерации формул
\usepackage{enumitem} % чтоб можно было делать римскую нумерацию
\usepackage{mdwlist}
\usepackage{adjustbox}
\usepackage{xcolor} % градации цветов
\usepackage[psdextra]{hyperref}
\hypersetup{unicode=true}
\hypersetup{
    colorlinks=true,
    linkcolor=blue,
}
\setlist[enumerate]{nosep}


\makeatletter
\def\theorem@space@setup{\theorem@preskip=0pt
\theorem@postskip=0pt}
\def\thm@space@setup{\thm@preskip=0pt
\thm@postskip=0pt}
\makeatother

\theoremstyle{plain} % Это стиль по умолчанию, его можно не переопределять.
\newtheorem{theorem}{Теорема}
\newtheorem*{thm}{Теорема}
\newenvironment{theorem*}{%
  \begin{thm}
}{%
  \end{thm}
}
\newtheorem{proposition}{Твердження}

\theoremstyle{remark}
\newtheorem{exercise}{}[section]
\newenvironment{exercise*}
  {\renewcommand\theexercise{\thesection.\arabic{exercise}\rlap{$^*$}}%
   \exercise\edef\@currentlabel{\thesection.\arabic{exercise}}}
  {\endexercise}

%%%% Картинки
\usepackage{graphicx}
\usepackage{tikz}
\usepackage{pgfplots}
\usepackage{mathrsfs}
\pgfplotsset{compat=1.16}

\usetikzlibrary{patterns}
\usetikzlibrary{shapes.geometric}
\usetikzlibrary{arrows}
\usetikzlibrary{arrows.meta}
\usetikzlibrary{shapes.geometric}
\graphicspath{{pictures/}}
\DeclareGraphicsExtensions{.pdf,.png,.jpg}

\newmdenv[
  topline=false,
  bottomline=false,
  rightline=false,
  skipabove=\topsep,
  skipbelow=\topsep,
  linecolor=black,
  outerlinecolor=black,
  innerlinecolor=black
]{theory}


\newcommand{\norm}[1]{\left\lVert#1\right\rVert} % command for norm
\newcommand{\dotprod}[2]{\left(#1, #2\right)} % command for scalar
\newcommand{\inv}[1]{#1^{-1}}
\newcommand{\real}[0]{\mathbb{R}}
\newcommand{\complex}[0]{\mathbb{C}}
\newcommand{\natur}[0]{\mathbb{N}}
\newcommand{\integers}[0]{\mathbb{Z}}
\newcommand{\intl}[2]{\int\limits_{#1}^{#2}}
\newcommand{\suml}[2]{\sum\limits_{#1}^{#2}}

\DeclareMathOperator{\sgn}{sgn} %

\makeatletter
\def\ukr#1{\expandafter\@ukr\csname c@#1\endcsname}
\def\@ukr#1{\ifcase#1\or а\or б\or в\or
г\or ґ\or д\or е\or є\or ж\or з\or и\or і\or ї\or й\or к\or л\or м\or н
\or о\or п\or р\or с\or т\or у\or ф\or х\or ц\or ч\or ш\or щ\fi}
\makeatother
\AddEnumerateCounter{\ukr}{\@ukr}{Українська}
\setlist[enumerate, 1]{label=\ukr*), align = left}