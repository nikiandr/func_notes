\documentclass{extreport}

%%% Работа с языком
%\usepackage{cmap}					    % поиск в PDF				    % русские буквы в формулах
\usepackage[12pt]{extsizes}
\usepackage[T2A]{fontenc}			% кодировка
\usepackage[utf8]{inputenc}             % кодировка исходного текста
\usepackage[english,ukrainian]{babel}	% локализация и переносы	    
\usepackage{indentfirst}
\usepackage{mdframed}
\usepackage{ulem}

\usepackage[a4paper, top=25mm, bottom=25mm, left=30mm, right=30mm]{geometry}

\usepackage{amsmath,amsfonts,amssymb,amsthm,mathtools} % AMS
\usepackage{dsfont} %для цифр в шрифте типа mathbb
\usepackage{braket} % для команды \set

\usepackage{makecell}% Прикольная настройка ячеек в tabular 
%http://ctan.math.utah.edu/ctan/tex-archive/macros/latex/contrib/makecell/makecell-rus.pdf

\usepackage{diagbox}
% Разделение клеток tabular по диагонали

\usepackage{multicol} % Несколько колонок
\usepackage{wrapfig} % Картинки посреди текста
\usepackage{chngcntr} % для нумерации формул
\usepackage{enumitem} % чтоб можно было делать римскую нумерацию
\usepackage{mdwlist}
\usepackage{adjustbox}
\usepackage{xcolor} % градации цветов
\usepackage[psdextra]{hyperref}
\hypersetup{unicode=true}
\hypersetup{
    colorlinks=true,
    linkcolor=blue,
}
\setlist[enumerate]{nosep}

%%%% Теоремы, определения и т.д. и т.п.
\theoremstyle{plain} % Это стиль по умолчанию, его можно не переопределять.
\newtheorem{theorem}{Теорема}
\newtheorem*{theorem*}{Теорема}
\newtheorem{proposition}{Твердження}
%\renewcommand{\qedsymbol}{$\blacktriangle$}
 
% \theoremstyle{definition} % "Определение"
% \newtheorem{definition}{Означення}[section]
% \newtheorem*{definition*}{Означення}
% \newtheorem*{example}{Приклад}

% \theoremstyle{remark}
% \newtheorem*{remark}{Зауваження}
\theoremstyle{remark}
\newtheorem{exercise}{}[section]
\newenvironment{exercise*}
  {\renewcommand\theexercise{\thesection.\arabic{exercise}\rlap{$^*$}}%
   \exercise\edef\@currentlabel{\thesection.\arabic{exercise}}}
  {\endexercise}

%\counterwithout{equation}{chapter} % нумерация формул
%\counterwithin*{equation}{section}

%%%% Картинки
\usepackage{graphicx}
\usepackage{tikz}
\usepackage{pgfplots}
\pgfplotsset{compat=1.16}

\usetikzlibrary{patterns}
\usetikzlibrary{shapes.geometric}
\usetikzlibrary{arrows.meta}
\usetikzlibrary{shapes.geometric}
\graphicspath{{pictures/}}
\DeclareGraphicsExtensions{.pdf,.png,.jpg}

\newmdenv[
  topline=false,
  bottomline=false,
  rightline=false,
  skipabove=\topsep,
  skipbelow=\topsep,
  linecolor=black,
  outerlinecolor=black,
  innerlinecolor=black
]{theory}


\newcommand{\norm}[1]{\left\lVert#1\right\rVert} % command for norm
\newcommand{\dotprod}[2]{\left(#1, #2\right)} % command for scalar
\DeclareMathOperator{\sgn}{sgn} %

\makeatletter
\def\ukr#1{\expandafter\@ukr\csname c@#1\endcsname}
\def\@ukr#1{\ifcase#1\or а\or б\or в\or
г\or ґ\or д\or е\or є\or ж\or з\or и\or і\or ї\or й\or к\or л\or м\or н
\or о\or п\or р\or с\or т\or у\or ф\or х\or ц\or ч\or ш\or щ\fi}
\makeatother
\AddEnumerateCounter{\ukr}{\@ukr}{Українська}
\setlist[enumerate, 1]{label=\ukr*), align = left}

\author{Богданський Ю.В.}
\title{Задачник з функціонального аналізу}
\date{\today}

\begin{document}
    \maketitle
    \tableofcontents
    \chapter{Обмежені лінійні оператори в нормованих просторах}
        \section{Основні положення. Властивості.}
            % !TEX root = ../main.tex

\begin{theory}
    Нехай $(X, \norm{\cdot}_X)$, $(Y, \norm{\cdot}_Y)$ --- нормовані простори над полем $K$ (
        зазвичай $K = 
    \mathbb{R}$ або $\mathbb{C}$, 
    але обов'язково однакове для обох просторів). 

    Лінійний оператор $A: X \rightarrow Y$ (визначений на всьому X) називається 
    \underline{обмеженим}, якщо існує число $C > 0$, для якого $\forall x \in X$: 
    $\norm{Ax}_Y \leq C \norm{x}_X$. Надалі найчастіше нижні індекси при позначенні норми 
    не будемо ставити; за змістом формул зрозуміло, до якого простору належить відповідна 
    норма. Також при позначенні нормованого простору $(X, \norm{\cdot})$ будемо писати лише 
    літеру $X$, якщо за змістом задачі норма в $X$ не викликає сумніву. 

    Якщо $Y$ є основним полем ($Y=\mathbb{R}$ або $Y = \mathbb{C}$), то лінійний оператор 
    $\varphi: X \rightarrow Y$ прийнято називати \underline{(лінійним) функціоналом}.

    Норма лінійного обмеженого оператора задається формулою:
    $\norm{A} = \inf\left\{C\;|\;\forall x \in X : \norm{Ax}\leq C\norm{x} \right\}$.
    Аналогічно для обмеженого функціонала:
    $\norm{\varphi} = \inf\left\{C \;|\; \forall x \in X : |\varphi(x)| \leq C\norm{x} \right\}$.
\end{theory}

\begin{exercise}
    Нехай $A$ --- обмежений лінійний оператор з $X$ в $Y$ ($A: X \rightarrow Y$).
    Довести: $\norm{A} = \min\left\{C \mid \forall x \in X : \norm{Ax} \leq C\norm{x}\right\}$.
\end{exercise}

\begin{exercise}
    Нехай $A: X \rightarrow Y$ --- обмежений лінійний оператор.
    Довести: $\norm{A} = \sup\limits_{x \neq 0}\frac{\norm{Ax}}{\norm{x}} = 
    \sup\limits_{\norm{x} \leq 1}\norm{Ax} = \sup\limits_{\norm{x} = 1}\norm{Ax}$.
\end{exercise}

\begin{theory}
    Лінійний оператор $A: X \rightarrow Y$ називається \underline{неперервним}, якщо відображення 
    A є неперервним в кожній точці $x \in X$.
\end{theory}

\begin{exercise}
    Нехай $A$ --- лінійний оператор з $X$ в $Y$. Довести: 
    \begin{enumerate}[label=\alph*)]
        \item якщо $A$ --- обмежений оператор, то $A$ --- неперервний;
        \item якщо існує точка $x_0 \in X$, в якій $A$ --- неперервний, то $A$ --- обмежений.
    \end{enumerate}
\end{exercise}

\begin{theory}
    \emph{Висновок:} Для перевірки неперервності лінійного оператора достатньо 
    довести його неперервність лише в одній точці простору аргументу.
\end{theory}

\begin{exercise}
    З'ясувати, чи є наведені нижче функціонали в просторі $C{\left[-1; 1\right]}$ лінійними; 
    неперервними. В разі позитивної відповіді знайти їх норми. 
    \begin{enumerate}[label=\ukr*)]
        \item $\varphi(x) = x(0)$;
        \item $\varphi(x) = \int\limits_0^1x(t)dt$;
        \item $\varphi(x) = \frac{1}{2}(x(1) - x(-1))$;
        \item $\varphi(x) = \int\limits_0^1tx(t)dt$;
        \item $\varphi(x) = \int\limits_{-1}^1tx(t)dt$;
        \item $\varphi(x) = \norm{x} = \max\limits_{-1 \leq t \leq 1} |x(t)|$;
        \item $\varphi(x) = \int\limits_0^1|x(t)|dt$;
        \item $\varphi(x) = \int\limits_{-1}^1x(t)\cos(\pi t)dt$;
        \item $\varphi(x) = \int\limits_0^1tx^2(t)dt$;
        \item $\varphi(x) = \int\limits_{-1}^0x(t)dt - \int\limits_0^1x(t)dt$;
        \item $\varphi(x) = \int\limits_{-1}^1x(t^2)dt$.
    \end{enumerate}
\end{exercise}

\begin{exercise}\label{N:1_1_5}
    З'ясувати, чи є наведені нижче функціонали на $X$ лінійними; 
    неперервними. В разі позитивної відповіді знайти їх норми.
    \begin{enumerate}[label=\ukr*)]
        \item $X = \ell_1$, $\varphi(\vec{x}) = \sum\limits_{n=1}^{\infty} x_n $;
        \item $X = \ell_1$, $\varphi(\vec{x}) = \sum\limits_{n=1}^{\infty} |x_n| $;
        \item $X = \ell_2$, $\varphi(\vec{x}) = x_1 + x_2 $;
        \item $X = \ell_2$, $\varphi(\vec{x}) = x_1 - x_2 + x_3$;
        \item $X = \ell_2$, $\varphi(\vec{x}) = \sum\limits_{n=1}^{\infty} \frac{x_n}{n}$;
        \item $X = \ell_p$ ($1 \leq p \leq +\infty$), $\varphi(\vec{x}) = 
        \sum\limits_{n=1}^{\infty} \frac{x_n}{n}$;
        \item $X = L_2[0;1]$, $\varphi(x) = \int\limits_0^1 tx(t)dt$;
        \item $X = L_2[0;1]$, $\varphi(x) = \int\limits_0^1 |x(t)|^{\frac{1}{2}}dt$;
        \item $X = L_2[0;\frac{\pi}{2}]$,
        $\varphi(x) = \int\limits_0^{\frac{\pi}{2}} x(t)\sin(t)dt$;
        \item $X = C^1[0;1]$ $(\norm{x}_1 = \max\limits_{[0;1]}|x(t)| + 
        \max\limits_{[0;1]}|x^{\prime}(t)|)$, $\varphi(x) = x^{\prime}(0) + x(1)$.
    \end{enumerate}
\end{exercise}

\begin{exercise}\label{N:1_1_6}
    Лінійний функціонал $\varphi$ визначено на просторі $C[0, 1]$ формулою 
    $\varphi(x) = \int\limits_0^1 x(t)dt - x(0)$. Довести: $\norm{\varphi} = 2$, 
    але не існує такої функції $x_0 \in C[0, 1]$, що $\norm{x_0} = 1$, $|\varphi(x_0)| = 2$
\end{exercise}

\begin{exercise}
    Нехай $X = C^1[0, 1]$ з нормою $\norm{x} = \max\limits_{[0, 1]}|x(t)|$, 
    $\varphi(x) = x^\prime(0)$. Доведіть необмеженість функціонала $\varphi$.
\end{exercise}

\begin{exercise}
    Нехай $f$, $g$ --- лінійні функціонали на лінійному просторі $L$; $\mathrm{Ker} f = \mathrm{Ker} g$. 
    Довести: існує $\alpha \in K$, для якого $f = \alpha g$.
\end{exercise}

\begin{exercise}\label{N:1_1_9}
    Нехай $f$, $f_1$, ..., $f_n$ --- лінійні функціонали на лінійному просторі $L$. 
    Довести, що $f$ є лінійною комбінацією $f_1$, ..., $f_n$ тоді й тільки тоді, 
    коли $\bigcap\limits_{k=1}^n \mathrm{Ker} f_k \subset \mathrm{Ker} f$.
\end{exercise}

\begin{exercise}
    Довести, що будь-який оператор в нормованих просторах із скінченновимірною областю 
    визначення $X$ є неперервним і при цьому існує такий ненульовий вектор $x_0 \in X$, 
    для якого $\norm{Ax_0} = \norm{A}\cdot\norm{x_0}$.
\end{exercise}

\begin{exercise}
    Довести, що лінійний неперервний оператор $A: X \rightarrow Y$ залишається неперервним, 
    якщо в $X$ та $Y$ замінити норми на еквівалентні.
\end{exercise}

\begin{exercise}\label{N:1_1_12}
    З'ясувати, чи є наведені нижче оператори лінійними; 
    неперервними. В разі позитивної відповіді знайти їх норми.
    \begin{enumerate}[label=\ukr*)]
        \item $A: C[0; 1] \rightarrow C[0; 1]$, $(Ax)(t) = \int\limits_0^t x(\tau) d\tau$;
        \item $A: C[0; 1] \rightarrow C[0; 1]$, $(Ax)(t) = \int\limits_0^t \tau x(\tau) d\tau$;
        \item $A: C[0; 1] \rightarrow C[0; 1]$, $(Ax)(t) = \int\limits_0^t \tau x^2(\tau) d\tau$;
        \item $A: C[0; 1] \rightarrow C[0; 1]$, $(Ax)(t) = x(t^2)$;
        \item $A: C^1[0; 1] \rightarrow C[0; 1]$, $(Ax)(t) = x^\prime (t)$; 
        \item $A: \ell_2 \rightarrow \ell_2$, $A\vec{x} = (\underbrace{0,0,...,0}_n,
        x_1,x_2,...)$;
        \item $A: \ell_2 \rightarrow \ell_2$, $A\vec{x} = (x_2,x_3,x_4,...)$;
        \item $A: \ell_2 \rightarrow \ell_2$, $A\vec{x} = (x_1,0,x_3,0,x_5,...)$;
        \item $A: \ell_2 \rightarrow \ell_1$, $A\vec{x} = \vec{x}$;
        \item $A: \ell_1 \rightarrow \ell_1$, $A\vec{x} = (x_1+x_2, x_1-x_2, x_3, x_4, ...)$;
        \item $A: L_2[0; 1] \rightarrow L_2[0; 1]$, $(Ax)(t) = t \int\limits_0^1 x(s)ds$;
        \item $A: L_2[0; 1] \rightarrow L_2[0; 1]$, $(Ax)(t) = \varphi(t)x(t)$ 
        ($\varphi \in C[0;1]$);
        \item $A: L_2[0; 1] \rightarrow L_2[0; 1]$, $(Ax)(t) = \int\limits_0^1 tsx(s)ds$.
    \end{enumerate}
\end{exercise}
            % !TEX root = ../main.tex
\begin{exercise}
    Нехай $\left\{c_n\right\}$ --- числова послідовність. Довести, що оператор
    $A : \ell_2 \ni (x_1, x_2, ...) = \vec{x} \mapsto \vec{y} = (c_1x_1, c_2x_2, ...) \in \ell_2$
    є обмеженим (та визначеним на всьому $\ell_2$) тоді й тільки тоді, коли $c = \underset{n\in\mathbb{N}}{\sup} |c_n| < +\infty$, 
    і при цьому $\norm{A} = c$.
\end{exercise}

\begin{exercise}
    Нехай $K \in C\left( [a;b] \times [a;b]\right)$ і оператор $A: C\left( [a;b]\right) \rightarrow C\left( [a;b]\right)$
    визначено формулою $(Ax)(t) = \int\limits_a^b K(t, s) x(s) ds$. Довести лінійність і обмеженість оператора $A$.
\end{exercise}

\begin{exercise}
    Довести, що ядро лінійного неперервного оператора замкнене. Чи завжди замкнена його область значень?
\end{exercise}

\begin{exercise}
    Нехай $A: X \rightarrow Y$ --- обмежений лінійний оператор, $Z$ --- передкомпакт в $X$.
    Довести $A(Z)$ --- передкомпакт в $Y$. Чи завжди образ компакта під дією обмеженого оператора буде компактом?
\end{exercise}

\begin{exercise}\label{N:1_1_17}
    Нехай $\varphi$ --- лінійний функціонал на нормованому просторі $X$. 
    Довести, що $\varphi$ --- обмежений тоді й тільки тоді, коли його ядро замкнене. 
\end{exercise}

\begin{exercise*}
    Нехай $A$ --- лінійний оператор з $X$ в $Y$, $\mathrm{dim} Y < \infty$.
    Доведіть, що $A$ --- обмежений тоді й тільки тоді, коли його ядро замкнене.
\end{exercise*}

\begin{exercise}\label{N:1_1_19}
    Нехай $A$ --- лінійний оператор з $X$ в $Y$. Чи завжди з умови замкненості ядра $A$ випливає його обмеженість?
\end{exercise}

\begin{exercise}\label{N:1_1_20}
    Нехай $\varphi$ --- лінійний функціонал на нормованому просторі. Довести $\varphi$ --- неперервний тоді й тільки тоді, коли
    для будь-якої послідовності $x_n \rightarrow 0$ числова послідовність $\left\{\varphi(x_n)\right\}$ --- обмежена.
\end{exercise}

\begin{exercise}\label{N:1_1_21}
    Нехай $\varphi$ --- ненульовий лінійний функціонал на нормованому просторі $X$. Довести, що для будь-якого $x \in X$ виконується рівність
    $\rho \left( x, \mathrm{Ker}\varphi\right) = \inf \left\{\norm{x - y} \; | \; y \in \mathrm{Ker} \varphi\right\} = \frac{|\varphi(x)|}{\norm{\varphi}}$.
\end{exercise}

\begin{exercise}\label{N:1_1_22}
    Нехай $\varphi$ --- ненульовий лінійний функціонал на нормованому просторі $X$, $a \in K$ ($\mathbb{R}$ або $\mathbb{C}$).
    Позначимо $L = \left\{x\in X \; | \; \varphi(x) = a\right\}$. Довести $\rho(0, L) = \frac{|a|}{\norm{\varphi}}$.
\end{exercise}

\begin{theory}
    Обмежені лінійні оператори з нормованого простору $X$ в нормований простір $Y$ утворюють лінійний простір за поточковими операціями:
    $(A+B)x = Ax + Bx$, $(\lambda\cdot A)x = \lambda \cdot Ax$. Цей простір є нормований із стандартною операторною нормою.
    Найчастіше він позначається так: $L\left( X, Y\right)$ або $\left\{X \rightarrow Y\right\}$. В разі, якщо $X = Y$, застосовується
    скорочене позначення $L(X)$. Якщо $Y = K$ (основне поле), то $L\left( X, K\right)$ позначається $X^*$ і називається 
    \uline{простором, спряженим} до $X$.
\end{theory}

\begin{exercise}
    Довести, що в разі, якщо простір $Y$ є повним, простір $L\left( X, Y\right)$ також є повним.
    Зокрема, для кожного нормованого простору $X$ спряжений простір $X^*$ є банаховим.
\end{exercise}

\begin{exercise}
    Нехай $X$, $Y$, $Z$ --- нормовані простори. Довести:
    \begin{enumerate}[label=\ukr*)]
        \item $\left( A \in L\left( X, Y\right); B \in L\left( Y, Z\right)\right) \Rightarrow \left( B \circ A \in L\left( X, Z\right); \norm{B\circ A} \leq \norm{B} \cdot \norm{A}\right)$;
        \item $\left( A \in L(X), n\in\mathbb{N}\right) \Rightarrow \left( A^n \in L(X) ; \norm{A^n} \leq \norm{A}^n\right)$;
        \item $\left( X \text{ --- банахів}; A_n \in L(X), n \in \mathbb{N} ; \text{ ряд } \sum\limits_{n=1}^{\infty} \norm{A_n} \text{ --- збіжний}\right) \Rightarrow$ 

        $\Rightarrow \left( \exists A \in L(X): \norm{A - \sum\limits_{k=1}^n A_k} \rightarrow 0, n\rightarrow \infty \right)$.
    \end{enumerate}
\end{exercise}

\begin{exercise}
    Нехай $X$ --- банахів простір, $A \in L(X)$. $e^A = \exp{A} := I + \sum\limits_{n=1}^{\infty} \frac{1}{n!}A^n$, $I$ --- тотожний оператор.
    Довести: існування, $\exp{A} \in L(X)$, $\norm{e^A} \leq e^{\norm{A}}$.
\end{exercise}

\begin{exercise}\label{N:1_1_26}
    Нехай $X$ --- банахів простір, $A\in L(X)$. Довести, що ряд $\sum\limits_{k=0}^{\infty} A^k$ ($A^0 := I$)
    збігається тоді й тільки тоді, коли існує натуральне число $n$, для якого виконується нерівність $\norm{A^n} < 1$.
\end{exercise}

\begin{exercise}\label{N:1_1_27}
    Знайти образ та ядро оператора $A \in L\left( X, Y\right)$, визначеного формулою $Ax = \varphi(x) y$,
    де $\varphi \in X^*$ --- фіксований обмежений лінійний функціонал на $X$, $y$ --- фіксований вектор з $Y$.
    Знайти норму $\norm{A}$.
\end{exercise}

\begin{theory}
    \underline{Рангом} оператора $A$ називається число $\mathrm{rank}A := \mathrm{dim} \mathrm{Im}A$.
    В разі, якщо $\mathrm{rank} A < \infty$, оператор $A$ називається
    \uline{оператором скінченного рангу} або \uline{скінченновимірним оператором}.
\end{theory}

\begin{exercise}\label{N:1_1_28}
    Нехай $A \in L\left( X, Y\right)$. Довести: $A$ --- оператор скінченного рангу тоді й тільки тоді, коли він допускає представлення
    $Ax = \sum\limits_{k=1}^n \varphi_k(x) y_k$, де $\varphi_k \in X^*$, $y_k \in Y$.
\end{exercise}

\begin{theory}
    Оператори $A, B: \ell_2 \rightarrow \ell_2$, $A : \vec{x} \mapsto (0, x_1, x_2, ...)$, $B : \vec{x} \mapsto (x_2, x_3, x_4, ...)$
    називаються відповідно операторами \underline{правого} та \underline{лівого зсуву}.
\end{theory}

\begin{exercise}
    Знайти норми $\norm{A}$ та $\norm{B}$ операторів зсуву.
\end{exercise}

\begin{theory}
    Нормовані простори $(X, \norm{\cdot}_X)$, $(Y, \norm{\cdot}_Y)$ називаються 
    \underline{ізоморфними}, якщо існує лінійний оператор $A: X \rightarrow Y$, для якого $\mathrm{Ker}A = {0}$,
    $\mathrm{Im}A = Y$ і для кожного $x \in X$ має місце $\norm{Ax}_Y = \norm{x}_X$. Такий оператор
    називається \underline{ізоморфізмом}. Позначення: $X \cong Y$.
\end{theory}

\begin{exercise}\label{N:1_1_30}
    Довести наступні ізоморфізми:
    \begin{enumerate}[label=\ukr*)]
        \item $\ell_1^* \cong \ell_\infty$;
        \item $c_0^* \cong \ell_1$;
        \item $\left( 1<p<\infty\right) \Rightarrow \left( \ell_p^* \cong \ell_q, \text{ де } \frac{1}{p} + \frac{1}{q} = 1\right)$.
    \end{enumerate}
    Тут $c_0 = \left\{ \vec{x} = (x_1, x_2, ...) \; | \; \underset{n\rightarrow\infty}{\lim} x_n = 0\right\}$ 
    з нормою $\norm{\vec{x}} = \sup\left\{|x_n| \; | \; n \in \mathbb{N}\right\}$.
\end{exercise}

\begin{exercise*}\label{N:1_1_31}
    Нехай $X$ --- нормований простір, $X^*$ --- сепарабельний простір. Довести сепарабельність простору $X$.
    Чи можна стверджувати, що спряжений простір до сепарабельного простору $X$ завжди є сепарабельним?
\end{exercise*}
            % !TEX root = ../main.tex

\begin{exercise}
    Лінійний функціонал $\varphi: C[a; b] \rightarrow \mathbb{R}$ називається \underline{невід'ємним},
    якщо для кожної невід'ємної на $[a; b]$ функції $x \in C[a, b]$ має місце нерівність 
    $\varphi(x) \geq 0$. Довести, що невід'ємний функціонал $\varphi$ є неперервним і при цьому
    $\norm{\varphi} = \varphi(\mathds{1})$, де $\mathds{1}$ --- тотожно одинична функція.
\end{exercise}

\begin{exercise}
    Довести, що лінійний оператор $A$, який діє в нормованому просторі $X$, є неперервним тоді
    й тільки тоді, коли множина $\set{x \in X \mid \norm{Ax} < 1}$ має внутрішні точки.
\end{exercise}

\begin{exercise}
    Нехай $X$ --- банахів простір; $A \in L(X)$ та існує $c > 0$ таке, що для кожного $x \in X$
    виконується нерівність $\norm{Ax} \geq c \norm{x}$. Довести, що $\mathrm{Im} A$ є замкненим підпростором.
\end{exercise}

\begin{exercise}
    Довести, що після заміни норм в нормованих просторах $X$ і $Y$ на еквівалентні,
    нова норма в $L\left( X, Y\right)$ буде еквівалентна старій.
\end{exercise}

\begin{exercise}
    Нехай $X$ --- комплексний нормований простір, $\varphi \in X^*$. Довести $\norm{\varphi} = \underset{\norm{x}=1}{\sup}\mathrm{Re}\varphi(x)$.
\end{exercise}

\begin{exercise}
    Нехай лінійний функціонал, визначений на нормованому просторі  $X$ --- необмежений.
    Довести, що в будь-якому околі нуля він приймає всі дійсні значення.
\end{exercise}

\begin{exercise}
    Нехай $X,Y$ --- нормовані простори, $Z$ --- замкнений підпростір в $X$. Покладемо 
    $M = \set{A \in L(X,Y) \mid \mathrm{Ker} A = Z}$. Чи буде $M$ замкненим підпростором в $L\left( X, Y\right)$?
\end{exercise}

\begin{exercise}
    Нехай $X,Y$ --- нормовані простори; $Z$ --- замкнений підпростір в $X$. Покладемо 
    $M = \set{A \in L(X,Y) \mid \mathrm{Ker} A \supset Z}$. Чи буде $M$ замкненим підпростором в $L\left( X, Y\right)$?
\end{exercise}

\begin{exercise}
    Нехай $X$ --- нормований простір; $A \in L(X)$. Покладемо $M = \{B \in L(X) \;|\; AB=0\}$;
    $N = \set{B \in L(X) \mid AB=BA}$. Довести, що $M$ та $N$ --- замкнені підпростори в $L(X)$.
\end{exercise}

\begin{theory}
    \begin{theorem*}[Ган, Банах]
        Для кожного ненульового вектора $x$ нормованого простору $X$ існує $\varphi \in X^*$,
        для якого $\norm{\varphi} = 1$ та $\varphi(x) = \norm{x}$. 
    \end{theorem*}
\end{theory}

\begin{exercise} 
    \begin{enumerate}[label=\ukr*)]
        \item Нехай $X$ --- нормований простір; $x,y \in X$. Довести $(x=y) \Leftrightarrow 
        \big(\forall\varphi \in X^*: \varphi(x) = \varphi(y)\big)$;
        \item Нехай $X$ --- сепарабельний нормований простір. Довести існування зліченної сім'ї
        функціоналів $\varphi_n \in X^*$, що розділяють точки $X$, тобто 
        $(x\neq y) \Leftrightarrow \big(\exists n \in \mathbb{N}: \varphi_n(x) \neq \varphi_n(y)\big)$.
    \end{enumerate}
\end{exercise}

\begin{theory}
    Кожен вектор $x$ нормованого простору $X$ можна розглядати як функцію $\tilde{x}$, яка 
    діє на $X^*$ за правилом $\tilde{x}(\varphi)=\varphi(x)$.
\end{theory}

\begin{exercise}
    Доведіть:~
    \begin{enumerate}[label=\ukr*)]
        \item Для будь-якого $x \in X$ функція $\tilde{x}$ є лінійним обмеженим функціонал на просторі $X^*$
        і його норма, як елемента $X^{**}$, співпадає з нормою $x \in X$;
        \item відображення $\alpha: X \ni x \mapsto \tilde{x} \in X^{**}$ є ізометричним вкладенням
        простору $X$ в $X^{**}$, тобто $\alpha$ --- лінійний оператор; $\mathrm{Ker}\alpha = \{0\}$; 
        $\norm{\alpha(x)} = \norm{\tilde{x}} = \norm{x}$.
    \end{enumerate}
\end{exercise}

\begin{theory}
    В разі, якщо $\mathrm{Im}\alpha = X^{**}$, тобто $\alpha$ --- ізоморфізм просторів $X$ та $X^{**}$,
    вихідний простір $X$ називається \underline{рефлексивним}.
\end{theory}

\begin{exercise}
    Доведіть, що простори  $\ell_p$, де $1<p<\infty$, є рефлексивними, а простори $\ell_1$ та $\ell_\infty$ --- ні.
\end{exercise}

\begin{exercise}
    Чи буде простір $L(X)$, де $X=L_2[0;1]$, сепарабельним?
\end{exercise}

\begin{exercise}
    Нехай $X$, $Y$ --- нормовані простори; $A \in L\left( X, Y\right)$; $B, C \in L\left( Y, X\right)$;
    $AB = I_Y$; $CA = I_X$ ($I_X$ --- тотожній оператор в $X$).
    Довести $B = C = A^{-1} \in L\left( Y, X\right)$ --- обернений оператор до $A$.
\end{exercise}

\begin{theory}
    \begin{theorem*}[Банах, Штейнгауз; принцип рівномірної обмеженості]
        Нехай $X, Y$ --- нормовані простори; $X$ --- повний; $A_n \in L\left( X, Y\right),\; n\in \mathbb{N}$;
        $\forall x \in X, \; \exists C(x) > 0$ таке, що $\forall n \in \mathbb{N}: \norm{A_n x} \leq C(x)$.
        Тоді $\exists \; C>0: \forall n \in \mathbb{N}: \norm{A_n} \leq C$.
    \end{theorem*}
\end{theory}

\begin{exercise}
    Нехай $\alpha_n$ --- числова послідовність в $\mathbb{R}$, для якої 
    $\sum\limits^\infty_{k=1} \alpha^2_k =\infty$. Довести, що існує така послідовність 
    $\beta_n$ в $\mathbb{R}$, для якої $\sum\limits^\infty_{k=1} \beta^2_k < \infty$, 
    але $\sum\limits^\infty_{k=1} \alpha_k\beta_k$ --- розбіжний ряд.
\end{exercise}

\begin{exercise}
    Нехай $X$ --- нормований простір, $L$ --- замкнений підпростір $X$; $\mathrm{codim}L \geq 1$.
    Тоді $\exists \; \varphi \in X^*$; $\varphi \neq 0$, для деякого $L \subset \mathrm{Ker}\varphi$.
\end{exercise}

\begin{exercise}
    Нехай $X, Y$ --- банахові простори над полем $K$ ($K$ = $\mathbb{R}$ або $\mathbb{C}$).
    $\varphi: X \times Y \rightarrow K$ --- білінійний функціонал, що має наступні властивості:
    \begin{enumerate}[label=\ukr*)]
        \item $\forall y \in Y$: $\varphi(\,\cdot\,, y)$ --- неперервний по $x$;
        \item $\forall x \in X$: $\varphi(x, \,\cdot\,)$ --- неперервний по $y$.
    \end{enumerate}
    Довести існування константи $k$ такої, що для кожного $(x,y) \in X \times Y$ має місце нерівність
    $|\varphi(x,y)| \leq k \norm{x}\cdot\norm{y}$.
\end{exercise}
        \section{Геометрія гільбертового простору}
            % !TEX root = ../main.tex

\begin{theory}
    Нехай $H$ --- нескінченновимірний гільбертів простір (дійсний або комплексний); 
    $\{e_1, e_2, ...\}$ --- зліченна ортонормована система векторів в $H$ (тобто $\dotprod{e_i}{e_j} 
    = \delta_{ij}$). Для $x \in H$ покладемо $c_n = \dotprod{x}{e_n}$. Ряд $\sum\limits_{n=1}^\infty 
    c_n e_n$ називається \underline{рядом Фур'є} вектора $x$ за ортонормованою системою 
    $\{e_n\}$, $c_n$ --- \underline{коефіцієнтами Фур'є}.
\end{theory}

\begin{exercise}
    Довести \underline{нерівність Бесселя}: $\sum\limits_{n=1}^\infty |\dotprod{x}{e_n}|^2 \leq 
    \norm{x}^2$ ($\norm{x} = \sqrt{\dotprod{x}{x}}$).
\end{exercise}

\begin{theory}
    
    Ортонормована система $\{e_n\}$ в $H$ називається \underline{повною}, якщо її лінійна 
    оболонка (л.о.) щільна в $H$:
     $\forall x \in H, \;\forall \varepsilon > 0 \; \exists \; \{\alpha_1, ..., 
    \alpha_m\}$: $\norm{\sum\limits_{k=1}^m \alpha_k e_k - x} < \varepsilon$.

    Ортонормована система $\{e_n\}$ називається \underline{замкненою}, якщо для кожного 
    $x \in H$ має місце рівність: $\norm{x}^2 = \sum\limits_{n=1}^\infty |\dotprod{x}{e_n}|^2$ 
    (\underline{рівність Парсеваля}).

    Повна ортонормована система $\{e_n\}$ називається \uline{ортонормованим 
    базисом} в $H$.
\end{theory}

\begin{exercise}
    Нехай $\{e_n\}$ --- ортонормована система векторів в гільбертовому просторі $H$. 
    Довести еквівалентність трьох умов:
    \begin{enumerate}[label=\ukr*)]
        \item система векторів $\{e_n\}$ повна в $H$;
        \item система векторів $\{e_n\}$ замкнена в $H$;
        \item $\forall x \in H$ ряд $\sum\limits_{n=1}^\infty \dotprod{x}{e_n}e_n$ збігається до $x$
        (тобто $\norm{x - \sum\limits_{n=1}^m \dotprod{x}{e_n}e_n}
        \underset{m\rightarrow\infty}{\rightarrow}0$)
    \end{enumerate}
\end{exercise}

\begin{exercise}
    Довести еквівалентність двох умов:
    \begin{enumerate}[label=\ukr*)]
        \item $H$ --- сепарабельний гільбертів простір;
        \item в просторі $H$ існує ортонормований базис.
    \end{enumerate}
\end{exercise}

\begin{theory}
    Ортонормована система векторів $\{e_n\}$ в $H$ називається \underline{тотальною}, 
    якщо умова <<$\dotprod{x}{e_n} = 0$ $\forall n \in N$>> виконується лише для вектора $x=0$.
\end{theory}

\begin{exercise}[лема Ріса-Фішера]
    Нехай $\{e_n\}$ --- ортонормована система векторів в $H$, $\{c_n\}$ --- числова послідовність,
    для якої ряд $\sum\limits_{n=1}^\infty |c_n|^2$ --- збіжний. Тоді існує $x \in H$, 
    для якого при всіх $n \in \mathbb{N}$ має місце рівність: $c_n = \dotprod{x}{e_n}$ і при цьому 
    $\norm{x}^2 = \sum\limits_{n=1}^\infty |c_n|^2$.
\end{exercise}

\begin{exercise}
    Нехай $\{e_n\}$ --- ортонормована система в $H$. Тоді дві умови еквівалентні:
    \begin{enumerate}[label=\ukr*)]
        \item $\{e_n\}$ --- ортонормований базис в $H$;
        \item $\{e_n\}$ --- тотальна система векторів в $H$.
    \end{enumerate}
\end{exercise}

\begin{theory}
    Гільбертові простори $H_1$ і $H_2$ називають \underline{ізоморфними}, якщо існує 
    лінійний оператор $A: H_1 \rightarrow H_2$, для якого виконуються рівності: 
    $\mathrm{Ker} A = \{0\}$; $\mathrm{Im} A = H_2$; $\dotprod{Ax}{Ay}_2 = \dotprod{x}{y}_1$ (тут $x$, $y$ --- довільні 
    вектори в $H_1$; $\dotprod{\cdot}{\cdot}_k$ --- скалярний добуток в $H_k$). 
    Такий оператор $A$ називається \underline{ізоморфізмом}. Позначення: $H_1 \cong H_2 $.
\end{theory}

\begin{exercise}
    Нехай $A: H_1 \rightarrow H_2$ --- ізоморфізм. Доведіть, що в означенні ізоморфізма:
    \begin{enumerate}[label=\ukr*)]
        \item умова <<$\mathrm{Ker} A = \{0\}$>> є зайвою;
        \item умова <<$\dotprod{Ax}{Ay} = \dotprod{x}{y}$ для всіх $x, y \in H_1$>> може бути 
        замінена на таку: <<$\norm{Ax} = \norm{x}$ для всіх $x \in H_1$>>.
    \end{enumerate}
\end{exercise}
            % !TEX root = ../main.tex
\begin{exercise}
    Довести, що на множині гільбертових просторів відношення <<бути ізоморфним>>
    є відношенням еквівалентності, тобто виконуються наступні властивості:
    \begin{enumerate}[label=\ukr*)]
        \item $H \cong H$ (рефлексивність);
        \item $(H_1 \cong H_2) \Leftrightarrow (H_2 \cong H_1)$ (симетричність);
        \item $(H_1 \cong H_2, H_2 \cong H_3) \Rightarrow (H_1 \cong H_3)$ (транзитивність).
    \end{enumerate}
\end{exercise}

\begin{exercise}\label{N:1_2_8}
    Нехай $H_1$, $H_2$ --- два нескінченновимірні гільбертові простори над однаковим полем, $H_1$ --- сепарабельний.
    Довести $H_2$ --- сепарабельний простір тоді й тільки тоді, коли $H_1 \cong H_2$.
\end{exercise}

\begin{exercise}
    Нехай $H$ --- гільбертів простір, $x_n, y_n \in H \; (n \in \mathbb{N})$, $\norm{x_n} = \norm{y_n} = 1 \; \forall n \in \mathbb{N}$.
    Довести наступні твердження:
    \begin{enumerate}[label=\ukr*)]
        \item $(\dotprod{x_n}{y_n} \rightarrow 1) \Rightarrow (\norm{x_n - y_n} \rightarrow 0)$;
        \item $(\norm{x_n + y_n} \rightarrow 2) \Rightarrow (\norm{x_n - y_n} \rightarrow 0)$.
    \end{enumerate}
\end{exercise}

\begin{theory}
    Нехай $x \in H$, $M$ --- підмножина $H$. Вектор $x$ називається \uline{ортогональним до $M$},
    якщо $\dotprod{x}{y} = 0$ для кожного $y \in M$. Позначення: $x \perp M$.
    Множина всіх векторів, ортогональних заданій множині $M$, називається \uline{ортогональним доповненням до $M$} 
    та позначається $M^\perp$.
\end{theory}

\begin{exercise}
    Нехай $H$ --- гільбертів простір, $x\in H$, $M \subset H$, $x \perp M$.
    Довести:
    \begin{enumerate}[label=\ukr*)]
        \item $x \perp \overline{M}$ ($\overline{M}$ --- замикання $M$);
        \item $M^\perp$ --- замкнений підпростір в $H$;
        \item $\overline{M}^\perp = M^\perp$;
        \item $(\text{л.о. } M)^\perp = M^\perp$.
    \end{enumerate}
\end{exercise}

\begin{exercise}
    Нехай $L$ --- замкнений підпростір гільбертового простору $H$, $x \in H$. Довести:
    $x \perp L$ тоді й тільки тоді, коли для кожного $y \in L$ виконується нерівність $\norm{x} \leq \norm{x-y}$.
\end{exercise}

\begin{exercise}
    Нехай $M \subset H$. Довести: умова <<$M^\perp = \{0\}$>> еквівалентна <<$\text{л.о. } M$ щільна в $H$>>.
\end{exercise}

\begin{exercise}
    Перевірити взаємну ортогональність наступних систем векторів:
    \begin{enumerate}[label=\ukr*)]
        \item $\set{1, \cos{nt}, \sin{nt} \mid n \in \mathbb{N}}$, $H = L_2 [-\pi; \pi]$;
        \item $\set{\frac{d^n}{dt^n} ((t^2-1)^n) \mid n \in \{0 \} \cup \mathbb{N}}$, $H = L_2 [-1; 1]$.
    \end{enumerate}
\end{exercise}

\begin{exercise}\label{N:1_2_14}
    В комплексному просторі $H = L_2 [-\pi; \pi]$ знайти $M^\perp$ для наступних множин:
    \begin{enumerate}[label=\ukr*)]
        \item $M = \set{e^{i n t} \mid n \in \mathbb{Z}}$;
        \item $M = \set{e^{i n t} \mid n \in \mathbb{N}}$;
        \item $M = \set{\sin{n t} \mid n \in \mathbb{N}}$.
    \end{enumerate}
\end{exercise}

\begin{exercise}
    В просторі $L_2 [0; 1]$ знайти ортогональне доповнення до наступних множин:
    \begin{enumerate}[label=\ukr*)]
        \item $M = C[0; 1]$;
        \item $M = P[0;1]$ --- множина всіх многочленів, що визначені на $[0; 1]$;
        \item $M = \set{ x \in C[0;1] \mid x(0) = 0}$.
    \end{enumerate}
\end{exercise}

\begin{theory}
    Нехай $H_1$, $H_2$ --- замкнені підпростори в гільбертовому просторі $H$.
    $H$ називається \uline{сумою $H_1$ та $H_2$} ($H = H_1 + H_2$), якщо виконується умова
    $(x \in H) \Rightarrow (\exists \; x_1 \in H_1, x_2 \in H_2 : x = x_1 + x_2)$;
    \uline{ортогональною сумою $H_1$ та $H_2$} ($H = H_1 \oplus H_2$), якщо додатково виконується умова
    $(x \in H_1, y \in H_2) \Rightarrow ( \dotprod{x}{y} = 0)$.
\end{theory}
            % !TEX root = ../main.tex

\begin{exercise}\label{N:1_2_16}
    Нехай $H_1$, $H_2$ --- замкнені підпростори в $H$.
    Довести еквівалентність наступних тверджень:
    \begin{enumerate}[label=\ukr*)]
        \item $H = H_1 \oplus H_2$;
        \item $H_1 = H_2^\perp $;
        \item $H_2 = H_1^\perp $.
    \end{enumerate}
\end{exercise}

\begin{exercise}
    Довести, що при фіксованому $n$ множина $M = \set{ \vec{x} \in \ell_2
    \mid \sum\limits^n_{k=1} x_k=0 }$ є замкненим підпростором в $\ell_2$.
    Знайти такий замкнений підпростір $N$, що виконується рівність $\ell_2 = M \oplus N$.
\end{exercise}

\begin{exercise}\label{N:1_2_18}
    Нехай $M$ та $N$ --- підмножини гільбертового простору $H$. Довести наступні твердження:
    \begin{enumerate}[label=\ukr*)]
        \item $(M \subset N) \Rightarrow (N^\perp \subset M^\perp)$;
        \item $M \subset M^{\perp\perp}$, де $M^{\perp\perp}={(M^\perp)}^\perp$;
        \item $M = M^{\perp\perp}$ тоді й тільки тоді, коли $M$ ---
              замкнений підпростір в $H$;
        \item $M^\perp = M^{\perp\perp\perp}$;
        \item $(M \cap N)^\perp = \overline{M^\perp + N^\perp}$.
    \end{enumerate}
\end{exercise}

\begin{exercise}
    Нехай $M_\alpha$ --- сім'я підмножин гільбертового простору $H$. Довести:
    \begin{enumerate}[label=\ukr*)]
        \item $\Big( \bigcup\limits_\alpha M_\alpha \Big)^\perp = 
              \bigcap\limits_\alpha M_\alpha^\perp$;
        \item $\Big( \bigcap\limits_\alpha M_\alpha \Big)^\perp = 
              \overline{\text{л.о.}\Big(\bigcup\limits_\alpha M_\alpha^\perp\Big)}$.
    \end{enumerate}
\end{exercise}

\begin{exercise}
    Нехай $M$ --- замкнена опукла множина в дійсному гільбертовому просторі $H$.
    Довести, що вектор $y \in M$ задовольняє умову $\rho(x, M) = \norm{x-y}$,
    де $x \in H$ тоді й тільки тоді, коли для будь-якого $z \in M$ виконується
    нерівність $\dotprod{x-y}{y-z} \geq 0$.
\end{exercise}

\begin{exercise}\label{N:1_2_21}
    В просторі $\ell_2$ знайти замкнену підмножину, в якій немає вектора з найменшою нормою.
\end{exercise}

\begin{exercise}
    В просторі $L_2[0;1]$ знайти відстань від елемента $x_0(t) = t^2$ до підпростору
    $L = \set{x \in L_2[0;1] \mid \int\limits_0^1x(t)dt = 0}$.
\end{exercise}

\begin{exercise}
    В просторі $\ell_2$ знайти відстань $\rho_n(\vec{x}_0, L_n)$ від вектора $\vec{x}_0 = (1, 0, 0, ...)$ до
    підпростору $L_n = \set{ \vec{x}\in \ell_2 \mid \sum\limits^n_{k=1} x_k = 0}$.
    Чому дорівнює $\lim\limits_{n \to \infty} \rho_n(\vec{x_0}, L_n)$?
\end{exercise}

\begin{exercise}\label{N:1_2_24}
    Нехай $M$ --- замкнена опукла множина в гільбертовому просторі $H$; $x \in H$.
    Довести, що існує, і при тому єдиний, вектор $y \in M$ для якого $\norm{x-y} =
    \rho(x, M)$.
\end{exercise}

\begin{exercise}
    Довести, що \underline{система Радемахера} $f_n(t) = \sgn\sin(2^n \pi t)$,
    $n=0,1,2,\dots$ ортонормована, але не є повною в $L_2[0;1]$.
\end{exercise}

\begin{exercise}\label{N:1_2_26}
    Нехай $\{x_n\}$ --- ортогональна система векторів гільбертового простору $H$.
    Довести еквівалентність наступних трьох умов:
    \begin{enumerate}[label=\ukr*)]
        \item ряд $\sum\limits^\infty_{n=1} x_n$ збігається;
        \item для кожного $y \in H$ ряд $\sum\limits^\infty_{n=1} \dotprod{x_n}{y}$ збігається;
        \item ряд $\sum\limits^\infty_{n=1} \norm{x_n}^2$ збігається.
    \end{enumerate}
\end{exercise}

\begin{exercise}\label{N:1_2_27}
    В лінійному просторі послідовностей $\vec{x} = (x_1, x_2, \dots)$
    $(x_n \in \mathbb{R})$ таких, що $\sum\limits^\infty_{n=1} x_n^2 < \infty$,
    покладемо $\dotprod{x}{y} \coloneqq \sum\limits^\infty_{k=1} \lambda_k x_k y_k$,
    де $\lambda_k \in \mathbb{R}$; $0 < \lambda_k < 1$. Перевірте, що остання формула
    коректно визначає скалярний добуток. Чи буде одержаний евклідів простір гільбертовим?
\end{exercise}

\begin{exercise}
    Нехай $\{e_n\}$ --- ортонормована система в гільбертовому просторі $H$,
    $\{\lambda_n\}$ --- числова послідовність. Доведіть, що ряд $\sum\limits^\infty_{n=1}
    \lambda_n e_n$ збігається в $H$ тоді й тільки тоді, коли $\sum\limits^\infty_{n=1}
    |\lambda_n|^2 < \infty$
\end{exercise}
            % !TEX root = ../main.tex

\begin{exercise}
    Довести, що множина $M = \left\{ \vec{x} \in \ell_2 | \sum\limits_{n = 1}^{\infty}x_n = 0\right\}$
    щільна в $\ell_2$.
\end{exercise}

\begin{exercise}\label{N:1_2_30}
    Нехай $M$ --- замкнена опукла множина в гільбертовому просторі $H$. Довести, що в $M$
    існує і притому єдиний вектор із найменшою нормою.
\end{exercise}

\begin{exercise}
    На просторі $C[0;1]$ розглянемо функціонал $\varphi$, який визначено формулою: 
    $\varphi(x) = \int\limits_{0}^{\frac{1}{2}}x(t)dt - \int\limits_{\frac{1}{2}}^{1}x(t)dt$. Тоді
    $M = \left\{x \in C[0; 1] \mid \varphi(x) = 1\right\}$ опукла замкнена множина, яка не містить найближчого до 
    нуля елемента. Довести. Порівняйте із задачею \ref{N:1_2_30}.
\end{exercise}

\begin{exercise}\label{N:1_2_32}
    Нехай послідовність неперервно диференційовних функцій утворює ортонормовану систему
    в $L_2[0; 2\pi]$. Доведіть, що похідні цих функцій не можуть бути обмеженими у сукупності.
\end{exercise}

\begin{exercise}
    Побудувати конкретний ізоморфізм просторів $L_2[0;1]$ та $\ell_2$.
\end{exercise}

\begin{theory}
    Нехай $L$ --- замкнений підпростір гільбертового простору $H$; $x \in H$. Вектор
    $y \in L$ називається \uline{(ортогональною) проекцією} вектора $x$ на $L$, якщо
    $x - y \bot L$. Вектор $x - y$ називається \uline{ортогональною складовою} при
    проектуванні $x$ на $L$. Позначення: $y = pr_L x$; $x - y = ort_L x$.
\end{theory}

\begin{exercise}
    Нехай $L$ --- замкнений підпростір $H$, $x \in H$, $y = pr_L x$, $z \in L, z \neq y$.
    Доведіть: $\norm{x - y} < \norm{x - z}$ (екстремальна властивість ортогональної проекції).
\end{exercise}

\begin{exercise}
    Нехай $L$ --- замкнений підпростір $H$; $x \in H$. Довести існування та єдиність ${pr}_L x$:
    \begin{enumerate}[label=\ukr*)]
        \item в разі, якщо $L$ --- сепарабельний підпростір;
        \item для загального випадку.
    \end{enumerate}
\end{exercise}

\begin{exercise}
    Нехай $L$ --- замкнений підпростір гільбертового простору $H$, $M$ --- замкнений підпростір $L$, $x \in H$.
    Доведіть:
    \begin{enumerate}[label=\ukr*)]
        \item $L$ --- гільбертів простір, що успадковує скалярний добуток простору $H$;
        \item $M$ є замкненим підпростором $H$;
        \item ${pr}_M({pr}_L x) = {pr}_M x$.
    \end{enumerate}
\end{exercise}
        \section{Обмежені лінійні функціонали та оператори в гільбертових просторах}
            % !TEX root = ../main.tex

\begin{exercise}
    Нехай $L$ --- замкнений підпростір в $H$, $L \neq \{0\}$. 
    Відображення $P: H \rightarrow H$ визначене формулою 
    $P: x \mapsto {pr}_L x$. Довести:
    \begin{enumerate}[label=\ukr*)]
        \item $P$ --- лінійний обмежений оператор в $H$;
        \item $\norm{P} = 1$, $\mathrm{Im}P = L$, $\mathrm{Ker} P = L^\bot$;
        \item $P^2 = P$, $L = \set{x \mid x=Px}$, $\dotprod{Px}{y}=\dotprod{x}{Py}$,
        $\dotprod{Px}{x} = \norm{Px}^2$ ($\forall x,y \in H$);
        \item оператор $Q: x \mapsto {pr}_{L^\bot}x$ пов'язаний з $P$ 
        співвідношенням: $Q = I-P$ (тут $I$ --- тотожній оператор в $H$);
        \item $PQ = QP = 0$, $\mathrm{KerP} = \mathrm{Im}Q$, $\mathrm{Im}P = \mathrm{Ker}Q$;
        \item якщо $M$ --- замкнений підпростір в $L$ і для $x \in H$:
        $P_1 x = {pr}_M x$, то $P_1 P = P P_1 = P_1$,
        $\dotprod{P_1x}{x} = \dotprod{Px}{x}$ ($\forall x \in H$).
    \end{enumerate}
\end{exercise}

\begin{theory}
    Лінійний оператор $P$ із задачі 1.3.1 називається \uline{ортопроектором}.
\end{theory}

\begin{exercise}
    Нехай $X$ --- нормований простір, $P \in L(X)$.
    $P$ називається \uline{проектором}, якщо $P^2=P$. Доведіть наступні 
    властивості (обмеженого) проектора:
    \begin{enumerate}[label=\ukr*)]
        \item $M = KerP$ --- замкнений підпростір;
        \item оператор $Q = I - P$ також є (обмеженим) проектором;
        \item $L=ImP=KerQ=\set{x \mid Px=x}$ --- замкнений підпростір в $X$;
        \item $X = L \dotplus M$ (тобто $\forall x \in X$ має, і притому 
        єдиний розклад: $x=x_1+x_2$, де $x_1 \in L$, $x_2 \in M$).
    \end{enumerate}
\end{exercise}

\begin{theory}
    Оператор $P$ називається \uline{проектором на $L$ паралельно $M$}.
\end{theory}

\begin{exercise}[теорема Фреше-Ріса]
    Нехай $H$ --- гільбертів простір над полем $K$ ($K=\mathbb{R}$ 
    або $\mathbb{C}$), $y \in H$, $\varphi_y : H \rightarrow K$ визначено 
    формулою $\phi_y(x) = \dotprod{x}{y}$. Довести:
    \begin{enumerate}[label=\ukr*)]
        \item $\varphi_y \in H^*$, $\norm{\phi_y} = \norm{y}$;
        \item $\forall \varphi \in H^*$ $\exists! \; y\in H$, для якого 
        $\varphi = \varphi_y$ (при цьому $y\in(Ker\varphi)^\bot$).
    \end{enumerate}
\end{exercise}

\begin{exercise}
    Застосувати теорему Фреше-Ріса для розв'язання задачі \ref{N:1_1_5}
    (в, г, ґ, е, ж).
\end{exercise}

\begin{exercise}
    Нехай $H$ --- гільбертів простір; $A$ --- лінійний оператор. 
    Довести, що $A \in L(H)$ тоді й тільки тоді, коли існує 
    $C > 0$ таке, що для кожного $x, y \in H$ має місце нерівність:
    $|\dotprod{Ax}{y}| \leq C\norm{x}\norm{y}$. При цьому $\norm{A} = 
    \underset{x,y \neq 0}{\sup} \frac{|\dotprod{Ax}{y}|}{\norm{x}\norm{y}}
    =\underset{\norm{x},\norm{y} \leq 1}{\sup} |\dotprod{Ax}{y}| = 
    \underset{\norm{x} = \norm{y} = 1}{\sup} |\dotprod{Ax}{y}|$.
\end{exercise}

\begin{theory}
    Нехай $A \in L(H)$. \uline{Спряженим оператором} до $A$ називається 
    такий оператор $B \in L(H)$, для якого при всіх $x, y \in H$ 
    виконується рівність $\dotprod{Ax}{y} = \dotprod{x}{By}$. 
    Позначення: $B = A^*$. 
\end{theory}

\begin{exercise}
    Довести коректність означення $A^*$ (тобто для кожного $A\in L(H)$ 
    $\exists! \; A^* \in L(H)$).
\end{exercise}

\begin{exercise}
    Нехай $A, B \in L(H)$, $\alpha \in K$. Доведіть наступні властивості:
    \begin{enumerate}[label=\ukr*)]
        \item $(A+B)^* = A^* + B^*$;
        \item $0^*=0$; $I^* = I$;
        \item $(\alpha A)^* = \overline{\alpha}A^*$
        \item $(A^*)^* = A^{**} = A$;
        \item якщо існує $A^{-1} \in L(H)$, то існує $(A^*)^{-1} \in 
        L(H)$ і при цьому $(A^*)^{-1} = (A^{-1})^*$;
        \item $\norm{A^*} = \norm{A}$.
    \end{enumerate}
\end{exercise}

\begin{theory}
    Оператор $A \in L(H)$ називається \uline{самоспряженим}, якщо 
    $A = A^*$.
\end{theory}

\begin{exercise}
    Нехай $A, B \in L(H)$; $A, B$ --- самоспряжені. Довести:
    \begin{enumerate}[label=\ukr*)]
        \item $A+B$ --- самоспряжений;
        \item $\alpha \in \mathbb{R}, \alpha A $ 
        --- самоспряжений;
        \item $0, I$ --- самоспряжені;
        \item $AB$ -- самоспряжений $\Rightarrow AB = BA$;
        \item якщо існує $A^{-1} \in L(H)$, то $A^{-1}$ --- 
        самоспряжений.
    \end{enumerate}
\end{exercise}

\begin{exercise}
    Знайти спряжений оператор до $A: H\rightarrow H$ в наступних 
    прикладах:
    \begin{enumerate}[label=\ukr*)]
        \item $H = \ell_2$; $A\vec{x} = (\alpha_1x_1, \alpha_2x_2, ...)$
        ($\alpha_k \in \mathbb{C}$, $\underset{k \in \mathbb{N}}{\sup}|\alpha_k| < \infty$);
        \item $H = \ell_2$; $A\vec{x} = (0, x_1, x_2, ...)$;
        \item $H = \ell_2$; $A\vec{x} = (x_5, x_6, x_7, ...)$;
        \item $H = \ell_2$; $A\vec{x} = (x_1, x_2, x_3,0,0,...)$;
        \item $H = \ell_2$; $A\vec{x} = (\underbrace{0,...,0}_m, 
        x_1, 0, 0, ...)$;
        \item $H = \ell_2$; $A\vec{x} = (2x_1 + 5x_2, x_2, x_3, ...)$;
        \item $H = \ell_2$; $A\vec{x} = (0, 0, x_1 + x_2, x_1 - x_2, 0,  
        0, ...)$;
        \item $H = L_2[0,1]$; $(Ax)(t) = \int\limits_0^t x(\tau) d\tau$;
        \item $H = L_2[0,1]$; $(Ax)(t) = x(t^\alpha)$, $\alpha \in (0,1)$;
        \item $H = L_2[0,1]$; $(Ax)(t) = \int\limits_0^1 e^{t+\tau}x(\tau) 
        d\tau$;
        \item $H = L_2[0,1]$; $(Ax)(t) = \int\limits_0^1 
        sin(t+2\tau) x(\tau) d\tau$;
    \end{enumerate}
\end{exercise}
            % !TEX root = ../main.tex

\begin{exercise}
    Нехай $A \in L(H)$ --- самоспряжений оператор.
    Довести: $\norm{A} = \underset{\norm{x} = 1}{\sup}(Ax, x)$.
\end{exercise}

\begin{exercise}
    Нехай $A \in L(H)$.
    \begin{enumerate}[label=\ukr*)]
        \item довести рівності: $(\mathrm{Im}A)^\perp = \mathrm{Ker}A^*$, $(\mathrm{Ker}A)^\perp = \overline{\mathrm{Im}A^*}$;
        \item навести приклад, коли $(\mathrm{Ker}A)^\perp \neq \mathrm{Im}A^*$.
    \end{enumerate}
\end{exercise}

\begin{exercise}
    Нехай $A \in L(H)$. Довести наступні твердження:
    \begin{enumerate}[label=\ukr*)]
        \item $\mathrm{Ker}(A A^*) = \mathrm{Ker}(A^*)$;
        \item $\mathrm{Ker}(A^* A) = \mathrm{Ker}(A)$;
        \item $\overline{\mathrm{Im} (A A^*)} = \overline{\mathrm{Im} (A)}$;
        \item $\norm{A^* A} = \norm{A}^2$.
    \end{enumerate}
\end{exercise}

\begin{exercise}
    Нехай $A \in L(H)$. Довести еквівалентність двох умов:
    \begin{enumerate}[label=\ukr*)]
        \item $A A^* = A^* A$;
        \item $\forall x \in H : \norm{Ax} = \norm{A^{*}x}$.
    \end{enumerate}
\end{exercise}

\begin{theory}
    Оператор $A \in L(H)$, для якого $A A^* = A^* A$, називається \uline{нормальним}.
\end{theory}

\begin{exercise}
    Нехай $A$ --- нормальний оператор в $H$. Довести:
    \begin{enumerate}[label=\ukr*)]
        \item $\mathrm{Ker} A = \mathrm{Ker} A^* = (\mathrm{Im} A)^\perp$;
        \item $\norm{A^2} = \norm{A}^2$;
        \item $(A^2 = 0) \Leftrightarrow (A = 0)$;
        \item $\left( \exists \; A^{-1} \in L(H)\right) \Rightarrow (A^{-1} \text{ --- нормальний оператор})$.
    \end{enumerate}
\end{exercise}

\begin{exercise}
    Нехай $H$ --- комплексний лінійний простір, $A \in L(H)$. Довести: 
    $A$ --- самоспряжений оператор тоді й тільки тоді, коли 
    \uline{квадратична форма} $(Ax, x)$ набуває лише дійсних значень.
\end{exercise}

\begin{theory}
    Самоспряжений оператор $A \in L(H)$ називається \uline{невід'ємним} ($A \geq 0$),
    якщо квадратична форма $(Ax, x)$ на $H$ набуває лише невід'ємних значень.

    Самоспряжені оператори $A, B \in L(H)$ задовольняють нерівність $A \geq B$, якщо $A - B \geq 0$.
\end{theory}

\begin{exercise}
    Довести, що ортопроектор --- невід'ємний оператор.
\end{exercise}

\begin{exercise}
    Нехай $P \in L(H)$, $P^2 = P$. Довести еквівалентність наступних умов:
    \begin{enumerate}[label=\ukr*)]
        \item $P$ --- ортопроектор;
        \item $P = P^*$;
        \item $P P^* = P^* P$;
        \item $\mathrm{Im} P = (\mathrm{Ker} P)^\perp$;
        \item $\forall x \in H: (Px, x) = \norm{Px}^2$.
    \end{enumerate}
\end{exercise}

\begin{exercise}
    Нехай $P_1$, $P_2$ --- ортопроектори в $H$, $H_k = \mathrm{Im} P_k$ ($k = 1, 2$).
    Довести:
    \begin{enumerate}[label=\ukr*)]
        \item $(P_1 + P_2 \text{ --- ортопроектор}) \Leftrightarrow (P_1 P_2 = 0) \Leftrightarrow (H_1 \perp H_2)$.
        При цьому $P_1 + P_2$ --- ортопроектор на $H_1 \oplus H_2$;
        \item $(P_1 P_2 \text{ --- ортопроектор}) \Leftrightarrow (P_1 P_2 = P_2 P_1)$.
        При цьому $P_1 + P_2$ --- ортопроектор на $H_1 \cap H_2$;
        \item $(P_1 - P_2) \Leftrightarrow (P_1 \geq P_2) \Leftrightarrow (H_1 \supset H_2)$.
        При цьому $P_1 - P_2$ --- ортопроектор на $H_1 \ominus (H_1 \cap H_2) = H_1 \cap H_2^\perp$;
        \item $(P_1 P_2 = P_2) \Leftarrow (P_2 P_1 = P_2) \Leftrightarrow (P_1 \geq P_2) \Leftrightarrow (H_1 \supset H_2)$.
    \end{enumerate}
\end{exercise}

\begin{exercise}\label{N:1_3_19}
    Нехай $P_1$, $P_2$ --- ортопроектори в гільбертовому просторі, $H_k = \mathrm{Im} P_k$ ($k = 1, 2$). Довести:
    \begin{enumerate}[label=\ukr*)]
        \item $(H_1 \subset H_2; \norm{P_1 - P_2} < 1) \Rightarrow (P_1 = P_2)$;
        \item навести приклад ортопроекторів $P_1 \neq P_2$, для яких $\norm{P_1 - P_2} < 1$.
    \end{enumerate}
\end{exercise}

\begin{exercise}
    В позначеннях задачі \ref{N:1_3_19}:
    \begin{enumerate}[label=\ukr*)]
        \item довести: $(\norm{P_2 - P_1} < 1) \Rightarrow (\dim H_1 = \dim H_2)$;
        \item навести приклад таких ортопроекторів $P_1$, $P_2$, для яких $\norm{P_2 - P_1} = 1$ та $\dim H_1 \neq \dim H_2$.
    \end{enumerate}
\end{exercise}

\begin{exercise}
    Нехай $A \in L(H), A \geq 0, x, y \in H$. Довести:
    \begin{enumerate}[label=\ukr*)]
        \item $|(Ax, y)|^2 \leq (Ax, x) \cdot (Ay, y)$;
        \item $\norm{A}^2 \leq \norm{A} \cdot (Ax, x)$.
    \end{enumerate}
\end{exercise}

\begin{exercise}
    Нехай $A \in L(H)$. Довести:
    \begin{enumerate}[label=\ukr*)]
        \item $(A \geq 0, \exists \; A^{-1} \in L(H)) \Rightarrow (\exists \; \lambda > 0 : A \geq \lambda I)$;
        \item $\exists \; (I + A^* A)^{-1} \in L(H)$.
    \end{enumerate}
\end{exercise}

\begin{exercise}
    Нехай $A \in L(H)$. Довести еквівалентність умов:
    \begin{enumerate}[label=\ukr*)]
        \item $\exists \; A^{-1} \in L(H)$;
        \item $\exists \; \alpha, \beta > 0 : A A^{*} \geq \alpha I, A^{*} A \geq \beta I$.
    \end{enumerate}
\end{exercise}
            % !TEX root = ../main.tex

\begin{exercise}
    Нехай $A \in L(H)$ --- самоспряжений оператор. Довести, що $A = 0$ тоді,
    й тільки тоді, коли для кожного $x \in H$ виконується рівність $\dotprod{Ax}{x}=0$.
    Навести приклад дійсного гільбертового простору $H$ і ненульового оператора
    $A \in L(H)$, для якого $\dotprod{Ax}{x}=0$ для всіх $x \in H$.
\end{exercise}

\begin{exercise}\label{N:1_3_25}
    Нехай $A \in L(H)$ і для кожного $B \in L(H)$ має місце рівність $AB = BA$.
    Довести, що існує число $\alpha$ таке, що $A = \alpha I$.
\end{exercise}

\begin{exercise}
    Нехай $H$ --- комплексний гільбертів простір. Для оператора $A \in L(H)$ позначимо
    \uline{дійсну} та \uline{уявну} частину формулами $\mathfrak{Re}A = \frac{1}{2} (A + A^*)$;
    $\mathfrak{Im}A = \frac{1}{2i} (A - A^*)$. Доведіть:
    \begin{enumerate}[label=\ukr*)]
        \item $\mathfrak{Re}A$ та $\mathfrak{Im}A$ --- самоспряжені оператори;
        \item якщо $A$ --- нормальний оператор, то $\norm{A} = 
        \sqrt{\norm{ (\mathfrak{Re}A)^2 + (\mathfrak{Im}A)^2 }}$.
    \end{enumerate}
\end{exercise}

\begin{exercise}
    Як повинні бути пов'язані між собою замкнені підпростори $H_1, H_2 \subset H$,
    щоб ортопроектори $P_1$ і $P_2$ на ці підпростори комутували?
\end{exercise}

\begin{exercise}
    Нехай $A, B \in L(H)$ --- самоспряжені оператори; $A \geq 0$; $B \geq 0$.
    Доведіть:
    \begin{enumerate}
        \item $A + B \geq 0$;
        \item[б)*] $(AB = BA) \Rightarrow (AB \geq 0)$.
    \end{enumerate}
\end{exercise}

\begin{exercise}
    Нехай $A \in L(H)$, $A$ --- самоспряжений оператор; $n \in \mathbb{N}$.
    Довести $\mathrm{Ker}A^n = \mathrm{Ker}A$.
\end{exercise}

\begin{theory}
    Лінійний оператор $U$ в $H$ називається \uline{унітарним}, якщо
    $U$ --- лінійний ізоморфізм $H$ на $H$, і при цьому $(x, y \in H) \Rightarrow 
    \left(\dotprod{Ux}{Uy} = \dotprod{x}{y}\right)$. За іншою термінологією, оператор
    $U$ у випадку дійсного простору $H$ називають \uline{ортогональним}.
\end{theory}

\begin{exercise}
    Нехай лінійний оператор $U:H \to H$ задовольняє умови:
    \begin{enumerate}[label=\ukr*)]
        \item $\mathrm{Im}U = H$;
        \item $\norm{Ux} = \norm{x}$ для кожного $x \in H$.
    \end{enumerate}
    Доведіть: $U$ --- унітарний оператор.
\end{exercise}

\begin{exercise}
    Довести, що оператор $U \in L(H)$ ($H$ --- комплексний) є унітарним в тому,
    й тільки в тому разі, якщо виконуються дві умови:
    \begin{enumerate}[label=\ukr*)]
        \item $U$ --- нормальний;
        \item $(\mathfrak{Re}U)^2 + (\mathfrak{Im}U)^2 = I$.
    \end{enumerate}
\end{exercise}

\begin{exercise}
    Нехай $y, z \in H$; оператор $A: H \to H$ визначено формулою: $Ax = \dotprod{x}{y} \cdot z$.
    Знайти $A^*$ та з'ясувати за яких умов на вектори $y$ та $z$ оператор $A$ буде:
    \begin{enumerate}[label=\ukr*)]
        \item нормальним;
        \item самоспряженим;
        \item додатним;
        \item унітарним.
    \end{enumerate}
\end{exercise}

\begin{exercise}\label{N:1_3_33}
    Довести: спряжений оператор до скінченновимірного також має скінченний ранг.
\end{exercise}

\begin{exercise}
    Нехай $A$, $B$ --- самоспряжені оператори в $H$; $A \geq 0$.
    Довести $BAB \geq 0$.
\end{exercise}
            % !TEX root = ../main.tex

\begin{exercise}
    Нехай $A, B$ --- самоспряжені оператори в $H$; $A \geq B$; $B \geq A$.
    Довести: $A = B$.
\end{exercise}

\begin{exercise}
    Нехай $A$ --- самоспряжений оператор в $H$; $\lambda \geq 0$; $0 \leq A \leq \lambda \mathbb{I}$.
    Довести: $\norm{A} \leq \lambda$.
\end{exercise}

\begin{exercise}
    Нехай $A$ --- самоспряжений оператор в $H$. Довести:
    $ (A \geq 0; \exists A^{-1} \in L(H)) \Rightarrow (A^{-1} \geq 0)$.
\end{exercise}

\begin{exercise}
    Нехай $A$ --- самоспряжений оператор в $H$; $A \geq 0$. Довести еквівалентність наступних умов:
    \begin{enumerate}[label=\ukr*)]
        \item $\overline{\rm \mathrm{Im}A} = H$;
        \item $\mathrm{Ker}A = \{0\}$;
        \item $(Ax, x) > 0$ для $\forall x \in H \setminus \{0\}$.
    \end{enumerate}
\end{exercise}

\begin{exercise}
    Нехай $A$ --- лінійний оператор в гільбертовому просторі $H$ і для $\forall x, y \in H$ виконується
    рівність: $(Ax, y) = (x, Ay)$. Довести: $A \in L(H)$ (а тому $A$ --- самоспряжений).
\end{exercise}

\begin{exercise}
    Нехай числова послідовність $\vec{a} = (a_1, a_2, \dots)$ така, що для кожного $\vec{x} \in \ell_2$
    ряд $\sum\limits_{n = 1}^\infty a_n x_n$ збігається. Тоді $\vec{a} \in \ell_2$ і формула 
    $\varphi(\vec{x}) = \sum\limits_{n = 1}^\infty a_n x_n$ задає неперервний лінійний функціонал на $\ell_2$.
    Довести.
\end{exercise}

\begin{exercise}
    Нехай $A$ --- самоспряжений оператор в $H$; $\exists m > 0: (Ax, x) \geq m \norm{x}^2$ для $\forall x \in H$.
    Довести: $\forall f \in H$ рівняння $Ax = f$ має і при тому єдиний розв'язок.
\end{exercise}

\begin{exercise}
    Використовуючи результат задачі \ref{N:1_3_41} довести розв'язність в дійсному просторі $L_2[0; 1]$ рівняння
    $x(t) = \int\limits_{0}^1 K(t, s)x(s)ds + f(t)$ для наступних функцій $K$:
    \begin{enumerate}[label=\ukr*)]
        \item $K(t, s) = -e^{ts}$;
        \item $K(t, s) = \sin(ts)$;
        \item $K(t, s) = -\sum\limits_{n = 1}^\infty \frac{\sin(nt)\sin(ns)}{n^2}$;
        \item $K(t, s) = -\sum\limits_{n = 1}^\infty t^n(1-t)s^n(1-s)$.
    \end{enumerate}
\end{exercise}

\begin{exercise}
    Для функціоналів на $L_2[0; 1]$ вказати такий вектор $h \in L_2[0; 1]$, що $\varphi(x) = (x, h)$ для $\forall x$:
    \begin{enumerate}[label=\ukr*)]
        \item $\varphi(x) = \int\limits_{0}^{\frac{1}{2}}x(s)ds$;
        \item $\varphi(x) = \int\limits_{0}^{\frac{1}{3}}x(s)ds - \int\limits_{\frac{1}{2}}^{1}x(s)ds$;
        \item $\varphi(x) = \int_Ax(s)ds$, де $A$ --- вимірна множина на $[0; 1]$.
    \end{enumerate}
\end{exercise}

\begin{exercise}
    Нехай $H$ --- нескінченновимірний гільбертів простір. Довести, що простір $L(H)$ не є сепарабельним.
\end{exercise}

\begin{exercise}
    Нехай $A \geq B \geq 0$. Довести: $\norm{A} \geq \norm{B}$.
\end{exercise}
        \section{Збіжність послідовностей векторів, функціоналів та операторів}
            % !TEX root = ../main.tex

\begin{theory}
    Нехай $X$ --- нормований простір; $X^*$ --- його спряжений. Послідовність $\{x_n\}$ 
    векторів простору $X$ називається \ul{сильно збіжною} до вектора $x \in X$, якщо 
    $\norm{x_n - x} \underset{n \rightarrow \infty}{\rightarrow} 0$. Це є звичайна збіжність 
    за нормою. Позначення : $x_n \rightarrow x$.

    Послідовність $x_n \in X$ називається \ul{слабо} (або \ul{слабко}) збіжною до $x \in X$, 
    якщо $\forall \varphi \in X^*$ : $\varphi(x_n) \underset{n \rightarrow \infty}{\rightarrow} 
    \varphi(x)$. Позначення : $x_n \underset{\text{сл.}}{\rightarrow} x$ 
    (або $x_n \rightharpoonup x$).

    Послідовність функціоналів $\varphi_n \in X$ \ul{сильно збігається} до $\varphi \in X^*$, 
    якщо $\norm{\varphi_n - \varphi} \underset{n \rightarrow \infty}{\rightarrow} 0$ 
    (збіжність за нормою в $X^*$). \ul{Слабка збіжність} $\varphi_n$ до $\varphi$ --- 
    це просто поточкова збіжність  $\forall x \in X$ : $\varphi_n(x) \rightarrow \varphi(x)$.
    Позначаємо її так : $\varphi_n \overset{*}{\rightarrow} \varphi$. 
    Її також називають *-слабкою, бо в $X^*$ є і інша слабка збіжність : $\forall \alpha 
    \in X^{**}$ : $\alpha(\varphi_n) \rightarrow \alpha(\varphi)$ (як в будь-якому 
    нормованому просторі).

    Для послідовності операторів $A_n \in L(X; Y)$ будемо розглядати \ul{збіжність за нормою} 
    (або \ul{рівномірну}) : $\norm{A_n - A} \rightarrow 0$, $n \rightarrow \infty$. 
    Позначення: $A_n \rightrightarrows A$.

    Також нам важлива \ul{сильна збіжність}, яка визначена умовою: 
    $A_n x \rightarrow Ax$ для $\forall x \in X$. Позначення: $A_n \overset{s}{\rightarrow} A$ 
    (або інакше: <<$A_n \rightarrow A$ сильно>>).

    Є ще <<слабка операторна збіжність>> $A_n \underset{\text{сл.}}{\rightarrow}$ 
    (або $A_n \rightharpoonup A$). Слабка збіжність за означенням --- це умова : 
    $\forall x \in X$, $\forall \varphi \in X^*$: $\varphi(A_nx) \rightarrow \varphi(Ax)$
\end{theory}

\begin{exercise}
    Доведіть, що у випадку гільбертового простору $H$ слабка збіжність $x_n 
    \underset{\text{сл.}}{\rightarrow} x$ ($x, x_n \in H$) рівносильна умові : 
    $\forall y \in H$ : $\dotprod{x_n}{y} \rightarrow \dotprod{x}{y}$.
\end{exercise}

\begin{exercise}
    Доведіть, що сильна збіжність послідовності $x_n$ нормованого простору $X$ гарантує 
    її слабку збіжність. Доведіть, що у випадку скінченновимірного $X$ має місце і зворотній 
    факт. Наведіть приклад нескінченновимірного простору $X$ і слабко збіжної послідовності 
    $x_n$, яка не має сильної границі.
\end{exercise}

\begin{exercise}
    Нехай $\{x_n\}$ --- слабо збіжна послідовність векторів гільбертового простору $H$. 
    Доведіть : 
    \begin{enumerate}[label=\ukr*)]
        \item послідовність $\{x_n\}$ --- обмежена;
        \item послідовність $\{x_n\}$ не може мати двох різних слабких 
        границь.
    \end{enumerate}
\end{exercise}

\begin{exercise}
    Нехай $\{x_n\}$ --- слабо збіжна послідовність векторів в нормованому 
    просторі $X$. Доведіть :
    \begin{enumerate}[label=\ukr*)]
        \item послідовність $\{x_n\}$ --- обмежена;
        \item послідовність $\{x_n\}$ не може мати двох різних 
        слабких границь.
    \end{enumerate}
\end{exercise}

\begin{exercise}
    Нехай $X$ - банахів простір; $\varphi_n \in X^*$; $\varphi_n$ --- 
    слабко збіжна. Доведіть:
    \begin{enumerate}[label=\ukr*)]
        \item послідовність $\{\varphi_n\}$ обмежена (за нормою в $X^*$);
        \item послідовність $\{\varphi_n\}$ не може мати двох різних слабких 
        границь.
    \end{enumerate}
\end{exercise}

\begin{exercise}
    Нехай $A_n$, $A \in L(X; Y)$. Доведіть: 
    \begin{enumerate}
        \item $(A_n \rightrightarrows A) \Rightarrow (A_n 
        \overset{s}{\rightarrow} A)$;
        \item взагалі кажучи : $(A_n 
        \overset{s}{\rightarrow} A) \nRightarrow (A_n \rightrightarrows A)$;

        Розгляньте приклад : $H = \ell_2$; $P_nx = (x_1, x_2, ..., x_n, 0, 0, ...)$;

        \item $(A_n \overset{s}{\rightarrow} A) \Rightarrow (\underset{n}{\sup}\norm{A_n} 
        < \infty$), ($X$ --- повний простір)

        \item $A_n$ не може мати двох різних сильних границь.
    \end{enumerate}
\end{exercise}

\begin{exercise}
    Наступні послідовності $x_n \in X$ дослідити на сильну і слабку збіжність : 
    \begin{enumerate}
        \item $X = \ell_2$ ; $\vec{x_n} = (1, \frac{1}{2}, \frac{1}{3}, ..., \frac{1}{n}, 0, 0
        , ...)$;
        \item $X = \ell_2$ ; $\vec{x_n} = ( \underbrace{0, ..., 0}_{n-1} ,
        1, \frac{1}{2}, \frac{1}{3}, ...)$;
        \item $X = \ell_2$ ; $\vec{x_n} = ( \underbrace{1, ..., 1}_{n-1} ,
        \frac{1}{n}, \frac{1}{n+1}, ...)$;
        \item $X = L_2[0, 1]$ ; $x_n(t) = t^n$;
        \item $X = L_2[0, 1]$ ; $x_n(t) =  \begin{cases}
            \sqrt{n} & t \in [0; \frac{1}{n}] \\
            0 & t \in (\frac{1}{n}; 1]
        \end{cases}$;
        \item $X = L_2[0, 1]$ ; $x_n(t) = e^{int}$;
        \item $X = L_2[0, 1]$ ; $x_n(t) = sin(2^nt)$;
        \item $X = \ell_p (1 < p < \infty)$; $x_n = ( \underbrace{0, ..., 0}_{n-1} ,
        1, \frac{1}{2}, \frac{1}{3}, ...)$;
        \item $X = c_0$; $x_n = ( \underbrace{0, ..., 0}_{n-1} ,
        1, \frac{1}{2}, \frac{1}{3}, ...)$;
    \end{enumerate}
\end{exercise}
            % !TEX root = ../main.tex

\begin{exercise}
    Нехай $X$ --- банахів простір, $\varphi, \varphi_n \in X^{*}$.
    Довести еквівалентність двох умов:
    \begin{enumerate}
        \item $\varphi_n \underset{\text{сл.}}{\rightarrow} \varphi$;
        \item $\underset{n}{\sup}\norm{\varphi_n} < \infty$, $\exists$ щільна
        в $X$ множина $Z$ така, що $(x \in Z) \Rightarrow (\varphi_n (x) \rightarrow \varphi(x))$.
    \end{enumerate}
\end{exercise}

\begin{exercise}
    Нехай $X$ --- нормований простір, $x, x_n \in X$.
    Довести еквівалентність двох умов:
    \begin{enumerate}
        \item $x_n \underset{\text{сл.}}{\rightarrow} x$;
        \item $\underset{n}{\sup}\norm{x_n} < \infty$, $\exists$ щільна
        в $X^{*}$ множина $Z$ така, що $(\varphi \in Z) \Rightarrow (\varphi (x_n) \rightarrow \varphi(x))$.
    \end{enumerate}
\end{exercise}

\begin{exercise}
    Нехай $X, Y$ --- нормовані простори, $\mathrm{dim}X < \infty$. Довести:
    \begin{enumerate}
        \item $\left(x_n, x \in X, \; x_n \underset{\text{сл.}}{\rightarrow} x\right) \Rightarrow \left(x_n \rightarrow x\right)$;
        \item $\left(\varphi_n, \; \varphi \in X^{*}, \varphi_n \underset{\text{сл.}}{\rightarrow} \varphi\right) \Rightarrow \left(\varphi_n \rightarrow \varphi\right)$;
        \item $\left(A_n, A \in L(X, Y), \; A_n \overset{s}{\rightarrow} A\right) \Rightarrow \left(A_n \rightrightarrows A\right)$.
    \end{enumerate}
\end{exercise}

\begin{exercise}
    Наступні послідовності операторів $A_n \in L(X)$ дослідити на сильну та рівномірну збіжність:
    \begin{enumerate}
        \item $X = \ell_2$; $A_n \vec{x} = (x_1, x_2, ..., x_n, 0, 0, ...)$;
        \item $X = \ell_2$; $A_n \vec{x} = (\underbrace{0, ..., 0}_{n}, x_1, x_2, ...)$;
        \item $X = \ell_2$; $A_n \vec{x} = (\underbrace{0, ..., 0}_{n}, x_1, 0, 0, ...)$;
        \item $X = \ell_2$; $A_n \vec{x} = (x_1, \frac{1}{2} x_2, \frac{1}{3} x_3, ...)$;
        \item $X = \ell_p \; (1 \leq p \leq \infty)$; $A_n \vec{x} = (\underbrace{0, ..., 0}_{n-1}, x_n, x_{n+1}, ...)$;
        \item $X = \ell_p \; (1 \leq p \leq \infty)$; $A_n \vec{x} = (\underbrace{0, ..., 0}_{n-1}, x_n, 0, 0, ...)$;
        \item $X = \ell_p \; (1 \leq p \leq \infty)$; $A_n \vec{x} = (x_n, x_{n-1}, ..., x_1, 0, 0, ...)$;
        \item $X = L_2 [0; 1]$; $(A_n x)(t) = x(t) \cos(nt)$;
        \item $X = C [0; 1]$; $(A_n x)(t) = t^n x(t)$;
        \item $X = C [0; 1]$; $(A_n x)(t) = e^{-nt} x(t)$;
        \item $X = C [0; 1]$; $(A_n x)(t) = t \cdot \int\limits_0^1 x(s) \sin^n (\frac{\pi s}{2}) ds$.
    \end{enumerate}
\end{exercise}

\begin{exercise}
    Нехай $X, Y$ --- банахові простори, $A, A_n \in L(X, Y)$.
    Довести еквівалентність двох умов:
    \begin{enumerate}
        \item $A_n \overset{s}{\rightarrow} A$;
        \item $\forall x \in X$ послідовність $\left\{ A_n x\right\}$ обмежена в $Y$ та $\exists$ щільна
        в $X$ множина $Z$, для якої $\forall z \in Z$ $A_n z \rightarrow A z$.
    \end{enumerate} 
\end{exercise}

\begin{exercise}\label{N:1_4_13}
    Нехай $X, Y$ --- гільбертові простори, $A : X \rightarrow Y$ --- лінійний оператор. 
    Довести еквівалентність трьох умов:
    \begin{enumerate}
        \item $( x_n \rightarrow x) \Rightarrow (A x_n \rightarrow A x)$;
        \item $(x_n \underset{\text{сл.}}{\rightarrow} x ) \Rightarrow (A x_n \underset{\text{сл.}}{\rightarrow} A x )$;
        \item $( x_n \rightarrow x) \Rightarrow (A x_n \underset{\text{сл.}}{\rightarrow} A x )$.
    \end{enumerate}
\end{exercise}

\begin{exercise}
    Нехай $H_1, H_2$ --- гільбертові простори, $A: H_1 \rightarrow H_2$ та $B: H_2 \rightarrow H_1$
    --- лінійні оператори, для яких $\forall x \in H_1, y \in H_2$ виконується $\dotprod{Ax}{y}_2 = \dotprod{x}{By}_1$.
    Доведіть обмеженість операторів $A$ і $B$.
    Зокрема, якщо $H_1 = H_2 = H$ та $A = B$, з рівності $\dotprod{Ax}{y} = \dotprod{x}{Ay}$,
    що виконується для всіх $x, y \in H$, випливає обмеженість (та самоспряженість) оператора $A$ (\ul{теорема Хелінгера-Тепліца}).
\end{exercise}

\begin{exercise*}
    \begin{enumerate}
        \item Доведіть твердження задачі \ref{N:1_4_13} для випадку довільних банахових просторів $X$ та $Y$.
        \item Нехай відображення $A: X \rightarrow Y$ нормованих просторів $X$ та $Y$ задовольняє умову
        $(\varphi \in Y^{*}) \Rightarrow (\varphi \circ A \in X^{*})$. Довести: $A \in L(X, Y)$.
    \end{enumerate}
\end{exercise*}

\begin{exercise}
    Нехай $H$ --- гільбертів простір. Довести:
    \begin{enumerate}
        \item $(x_n \underset{\text{сл.}}{\rightarrow} x, y_n \rightarrow y) \Rightarrow (\dotprod{x_n}{y_n} \rightarrow \dotprod{x}{y})$;
        \item $(x_n \rightarrow x) \Leftrightarrow (x_n \underset{\text{сл.}}{\rightarrow} x, \norm{x_n} \rightarrow \norm{x}) 
        \Leftrightarrow (x_n \underset{\text{сл.}}{\rightarrow} x,\; \underset{n \rightarrow \infty}{\overline{\lim}} \norm{x_n} \leq \norm{x})$.
    \end{enumerate}
\end{exercise}

\begin{exercise}
    Нехай $H$ --- гільбертів простір. Побудувати приклади послідовностей $\{x_n\}$ та $\{y_n\}$ в $H$, для яких виконується одна з наступних умов:
    \begin{enumerate}
        \item $x_n \underset{\text{сл.}}{\rightarrow} x$, $y_n \underset{\text{сл.}}{\rightarrow} y$, $\dotprod{x_n}{y_n} \not\rightarrow \dotprod{x}{y}$;
        \item $x_n \underset{\text{сл.}}{\rightarrow} x$, $y_n \underset{\text{сл.}}{\rightarrow} y$, $x_n \not\to x$, $y_n \not \to y$, $\dotprod{x_n}{y_n} \to \dotprod{x}{y}$.
    \end{enumerate}
\end{exercise}

\begin{exercise}
    Нехай $X$ --- банахів простір, $x, x_n \in X$, $\varphi, \varphi_n \in X^{*}$.
    Довести збіжність $\varphi_n(x_n) \to \varphi(x)$, якщо виконується одна з наступних умов:
    \begin{enumerate}
        \item $x_n \to x$, $\varphi_n \to \varphi$;
        \item $x_n \underset{\text{сл.}}{\rightarrow} x$, $\varphi_n \to \varphi$;
        \item $x_n \to x$, $\varphi_n \underset{\text{сл.}}\rightarrow \varphi$.
    \end{enumerate}
\end{exercise}
            % !TEX root = ../main.tex

\begin{exercise}
    Нехай $\{x_n\}$ --- ортогональна система векторів в гільбертовому просторі $H$.
    Довести еквівалентність наступних тверджень:
    \begin{enumerate}
        \item ряд $\sum\limits^\infty_{n=1} x_n$ збігається сильно;
        \item ряд $\sum\limits^\infty_{n=1} x_n$ збігається слабо (тобто слабо збігається
        послідовність його часткових сум);
        \item числовий ряд $\sum\limits^\infty_{n=1} \norm{x_n}^2$ --- збіжний.
    \end{enumerate}
\end{exercise}

\begin{theory}
    Послідовність $\{x_n\}$ в нормованому просторі $X$ називається \ul{слабо фундаментальною},
    якщо для кожного $\varphi \in X^*$ числова послідовність $\{\varphi(x_n)\}$ є фундаментальною.
    
    Послідовність операторів $A_n \in L(X,Y)$ називається \ul{сильно фундаментальною},
    якщо для кожного $x \in X$ послідовність векторів $\{A_n x\} \subset Y$ є фундаментальною
    за нормою.
\end{theory}

\begin{exercise}
    \begin{enumerate}
        \item Довести, що слабо збіжна послідовність векторів в банаховому просторі 
        є слабо фундаментальною;
        \item Довести, що послідовність векторів $\vec{x}_n = 
        (\underbrace{1,\dots,1}_{n},0,0,\dots)$ є слабо фундаментальною, але не слабо 
        збіжною в банаховому просторі $c_0$;
        \item Довести, що слабо фундаментальна послідовність в гільбертовому просторі 
        є слабо збіжною.
    \end{enumerate}
\end{exercise}

\begin{exercise}
    Довести, що послідовність $\vec{x}_n$ векторів простору $\ell_2$ слабо збігається тоді
    й тільки тоді, коли виконуються дві умови:
    \begin{enumerate}
        \item послідовність $\vec{x}_n = (x_n^1,x_n^2,\dots)$ рівномірно обмежена, тобто
        $\exists C > 0 \; \forall n$: $\norm{\vec{x}_n} \leq C$;
        \item $\forall k \in \mathbb{N}$ числова послідовність $\{x_n^k\}$ збігається при
        $n \to \infty$ (<<\ul{покоординатна збіжність}>>).
    \end{enumerate}
\end{exercise}

\begin{exercise}
    Нехай $X$, $Y$ --- банахові простори, послідовність операторів $A_n \in L(X,Y)$ сильно
    фундаментальна. Довести, що існує оператор $A \in L(X,Y)$, до якого послідовність $A_n$
    збігається сильно (<<\ul{повнота $L(X,Y)$ відносно сильної операторної збіжності}>>).
\end{exercise}

\begin{exercise}
    Нехай $X$, $Y$ --- банахові простори; $A, A_n \in L(X,Y)$; $x, x_n \in X$;
    $A_n \overset{S}{\to} A$; $x_n \to x$ (за нормою). Доведіть $A_n x_n \to Ax$ (за нормою).
\end{exercise}

\begin{exercise}
    Нехай $X$, $Y$, $Z$ --- нормовані простори; $A, A_n \in L(X,Y)$; $B, B_n \in L(Y,Z)$;
    $A_n \rightrightarrows A$; $B_n \rightrightarrows B$.
    Довести $B_n A_n \rightrightarrows BA$.
\end{exercise}

\begin{exercise}
    Нехай $X$, $Y$, $Z$ --- банахові простори; $A, A_n \in L(X,Y)$; $B, B_n \in L(Y,Z)$;
    $A_n \overset{S}{\to} A$; $B_n \overset{S}{\to} B$.
    Довести $B_n A_n \overset{S}{\to} BA$.
\end{exercise}

\begin{exercise}\label{N:1_4_26}
    Нехай $H$ --- сепарабельний гільбертів простір, $Z$ --- підмножина в $H$.
    Довести еквівалентність двох умов:
    \begin{enumerate}
        \item $Z$ --- обмежена множина в $H$;
        \item кожна послідовність точок $x_n \in Z$ містить слабо збіжну (в $H$) 
        підпослідовність.
    \end{enumerate}
\end{exercise}

\begin{theory}
    Умова б) називається умовою <<\uline{слабкої компактності}>> множини $Z$,
    а твердження задачі \ref{N:1_4_26} є спрощеним варіантом теореми Банаха-Алаоглу.
\end{theory}

\begin{exercise}
    $A_n$ --- самоспряжені обмежені оператори в гільбертовому просторі $H$,
    $A_n \overset{S}{\to} A$. Довести $A$ --- самоспряжений оператор.
    Якщо, на додачу, $A_n \geq 0$ при всіх $n$, то $A \geq 0$.
\end{exercise}

\begin{exercise}
    $A_n$ --- самоспряжені обмежені оператори в гільбертовому просторі $H$,
    $A_1 \leq A_2 \leq \dots$; $\exists C > 0 \; \forall n \in \mathbb{N}$: $\norm{A_n}\leq C$.
    Тоді існує самоспряжений оператор $A$ такий, що $A_n \overset{S}{\to} A$
    (аналог теореми Вейєрштрасса).
\end{exercise}

\begin{exercise}
    $A_n \in L(L_2[0;1])$. Оператори $A_n$ визначено формулою $(A_n x)(t) = a_n(t) x(t)$.
    Дослідити послідовність $\{A_n\}$ на сильну та рівномірну збіжність якщо:
    \begin{enumerate}
        \item $a_n(t) = t^n$;
        \item $a_n(t) = t^n (1-t)$.
    \end{enumerate}
\end{exercise}

\begin{exercise*}
    Довести, що в просторі $\ell_1$ сильна збіжність послідовності векторів співпадає зі слабкою.
\end{exercise*}
        \section{Цілком неперервні оператори}
            % !TEX root = ../main.tex

\begin{theory}
    Нехай $X$, $Y$ --- нормовані простори, $A: X \to Y$ --- лінійний оператор.
    Оператор $A$ називається \ul{цілком непереревним} (або \ul{компактним}),
    якщо для кожної обмеженої множини $Z \subset X$ її образ $A(Z)$ є передкомпактною
    множиною в $Y$. 
    Множина всіх компактних операторів з $X$ в $Y$ позначається через 
    \ul{$K(X,Y)$}, а якщо $X = Y$, то через \ul{$K(X)$}.
\end{theory}

\begin{exercise}
    Нехай $X$, $Y$ --- нормовані простори, $A: X \to Y$ --- лінійний оператор. Доведіть:
    \begin{enumerate}
        \item якщо образ кулі $B(0;1) = \set{x \in X \mid \norm{x} < 1}$ є передкомпактом в $Y$, то $A$ --- компактний оператор;
        \item якщо $A$ --- компактний оператор, то він обмежений, тобто $K(X,Y) \subset L(X,Y)$.
    \end{enumerate}
\end{exercise}

\begin{exercise}
    Нехай $X$, $Y$ --- нормовані простори, $A, B, A_n \in K(X,Y)$. Довести:
    \begin{enumerate}
        \item $A+B \in K(X, Y)$;
        \item $\lambda A \in K(X, Y)$ (тут $\lambda$ --- число);
        \item $\left( A_n \rightrightarrows C, C\in L(X, Y)\right) \Rightarrow \left( C \in K(X, Y)\right)$.
    \end{enumerate}
\end{exercise}

\begin{exercise}
    Нехай $X$, $Y$, $Z$ --- нормовані простори, $A \in L(X, Y)$, $B \in L(Y, Z)$.
    Довести, що якщо принаймні один з операторів $A$ або $B$ є компактним, то й оператор $BA$ --- компактний.
\end{exercise}

\begin{theory}
    Результати трьох задач вище приводять до висновку: для нормованого простору $X$ $K(X)$ 
    є замкненим двобічним ідеалом в операторній алгебрі $L(X)$.
\end{theory}

\begin{exercise}
    Нехай $X$, $Y$ --- нормовані простори, $A: X \to Y$ --- обмежений лінійний оператор
    скінченного рангу. Довести, що тоді $A \in K(X, Y)$.
\end{exercise}

\begin{exercise}
    Довести, що тотожній оператор в нормованому просторі $X$ є компактним
    тоді й тільки тоді, коли $\mathrm{dim} X < \infty$.
\end{exercise}

\begin{exercise}
    Нехай $A \in K(X)$, $\mathrm{dim} X = \infty$. Довести, що не існує оператора
    $B \in L(X)$, для якого виконується або рівність $AB = I$, або рівність $BA = I$ 
    ($I$ --- тотожний оператор в $X$).
\end{exercise}

\begin{exercise}
    Нехай $X$, $Y$ --- нормовані простори, $A \in K(X, Y)$. Довести, що простір $\left( \mathrm{Im} A, \norm{\cdot}_Y\right)$ --- сепарабельний.
\end{exercise}

\begin{exercise}
    Нехай $X$ та $Y$ --- банахові простори, $A \in K(X, Y)$. $Z \subset \mathrm{Im} A$, 
    $Z$ --- замкнений підпростір в $Y$. Довести: $\mathrm{dim} Z < \infty$.
\end{exercise}

\begin{exercise}
    Довести, що обмежений оператор проектування в банаховому просторі $X$
    є компактним тоді й тільки тоді, коли він скінченновимірний.
\end{exercise}

\begin{exercise}
    Чи вірно, що якщо в нескінченновимірному нормованому просторі $X$ для оператора $A \in L(X)$
    виконується рівність $A^2 = 0$, то $A$ --- компактний оператор?
\end{exercise}

\begin{exercise}
    Нехай $\{e_n\}$ --- ортонормована система в гільбертовому просторі $H$, $A \in K(H)$.
    Довести: $\norm{A e_n} \to 0$, $n \to \infty$.
\end{exercise}

\begin{exercise}
    Нехай $\{a_n\}$ --- числова послідовність. Довести, що оператор $A: \ell_2 \to \ell_2$,
    визначений формулою $A\vec{x} = (a_1 x_1, a_2 x_2, ...)$ є компактним тоді й тільки тоді,
    коли $a_n \to 0$.
\end{exercise}

\begin{exercise}
    Довести, що оператор вкладення $A: C^1 [a;b] \to C[a;b]$, $(Ax)(t) = x(t)$ є компактним.
\end{exercise}
            % !TEX root = ../main.tex

\begin{exercise}
    Довести, що оператор $A: C[a, b] \rightarrow C[a, b]$, визначений формулою : $(Ax)(t) = 
    f(t)x(t)$ ($f \in C[a, b]$; $\exists t \in [a, b] : f(t) \neq 0$) не є компактним.
\end{exercise}

\begin{exercise}
    З'ясувати, які з наведених нижче операторів $A : C[0, 1] \rightarrow C[0, 1]$ 
    є компактними:
    \begin{enumerate}
        \item $(Ax)(t) = x(0) + tx(\frac{1}{2}) + t^2 x(1)$;
        \item $(Ax)(t) = \int\limits_0^t sx(s) ds$
        \item $(Ax)(t) = x(t^2)$
        \item $(Ax)(t) = \int\limits_0^1 x(ts) ds$
    \end{enumerate}
\end{exercise}

\begin{exercise}
    З'ясувати, які з наведених нижче операторів $A : \ell_2 \rightarrow \ell_2$ 
    є компактними:
    \begin{enumerate}
        \item $A\vec{x} = (0, x_1, x_2, x_3, ...)$;
        \item $A\vec{x} = (x_2, x_3, ..., x_{10}, 0, 0, ...)$;
        \item $A\vec{x} = (x_{100}, \frac{1}{2}x_{101} ,\frac{1}{3}x_{102}, ...)$.
    \end{enumerate}
\end{exercise}

\begin{exercise}
    З'ясувати, які з наведених нижче операторів $A : L_2[0, 1] \rightarrow L_2[0, 1]$ 
    є компактними:
    \begin{enumerate}
        \item $(Ax)(t) = \int\limits_0^1 tsx(s) ds$;
        \item $(Ax)(t) = \int\limits_0^t x(s) ds$;
        \item $(Ax)(t) = t x(t)$;
        \item $(Ax)(t) = x(\sqrt{t})$;
        \item $(Ax)(t) = \int\limits_0^t tsx(s) ds$;
    \end{enumerate}
\end{exercise}

\begin{exercise}
    Нехай $F \in C([a, b] \times [a, b])$; оператори $A$ та $B$ задаються в $X$ формулами:
    $(Ax)(t)=\int\limits_a^b F(t, s)x(s)ds$; $(Bx)(t) = \int\limits_a^t F(t, s)x(s)ds$.
    Довести їх компактність у випадках:
    \begin{enumerate}
        \item $X = C[a, b]$;
        \item $X = L_2[a, b]$.
    \end{enumerate}
\end{exercise}

\begin{exercise}
    Нехай $H$ --- сепарабельний гільбертів простір; $A \in K(H)$. Довести: $A$ 
    досягає своєї норми на замкненій одиничній сфері (тобто $\norm{A} = 
    \underset{\norm{x} = 1}{\max}\norm{Ax}$).
\end{exercise}

\begin{exercise}
    Нехай $Z$ --- замкнений підпростір $C[a, b]$, який є підмножиною в $C^1[a, b]$. 
    Довести: $\mathrm{dim} Z < \infty$.
\end{exercise}

\begin{exercise}
    Нехай $X$ --- сепарабельний гільбертів простір; $A \in K(X)$. Довести: образ замкненої 
    одиничної кулі $B[0; 1]$ є компактом.
\end{exercise}

\begin{exercise}
    Довести, що будь-який компактний оператор в гільбертовому просторі є рівномірною 
    границею послідовності скінченновимірних операторів.
\end{exercise}

\begin{exercise}
    Довести, що будь-який обмежений оператор в сепарабельному гільбертовому просторі $H$ є 
    сильною границею послідовності:
    \begin{enumerate}
        \item компактних операторів;
        \item скінченновимірних операторів.
    \end{enumerate}
    А в несепарабельному просторі $H$ це вірно?
\end{exercise}

\begin{exercise}
    Довести, що оператор диференціювання $Ax = x^\prime$, що діє з $C^1[a, b]$ в $C[a, b]$ 
    не є компактним.
\end{exercise}
            % !TEX root = ../main.tex

\begin{exercise}
    Чи є компактним оператор $(Ax)(t) = \frac{1}{2}\left(x(t)+x(-t)\right)$,
    що діє в просторі $C[-1;1]$?
\end{exercise}

\begin{exercise}\label{N:1_5_26}
    Нехай $H_1$, $H_2$ сепарабельні гільбертові простори, $A: H_1 \to H_2$ ---
    лінійний оператор, $x_n, x \in H_1$, $y_n, y \in H_2$.
    Довести еквівалентність наступних умов:
    \begin{enumerate}
        \item $A$  --- компактний;
        \item $(x_n \underset{\text{сл.}}{\to} x; \; y_n \underset{\text{сл.}}{\to} y)
        \Rightarrow \left(\dotprod{Ax_n}{y_n} \to \dotprod{Ax}{y}\right)$;
        \item $(x_n \underset{\text{сл.}}{\to} x) \Rightarrow (Ax_n \to Ax \;\text{(сильно)})$.
    \end{enumerate}
\end{exercise}

\begin{exercise*}
    Довести, що будь-який обмежений оператор $A: \ell_2 \to \ell_1$ є компактним.
\end{exercise*}

\begin{exercise}
    Нехай оператор $A: \ell_1 \to \ell_2$ є вкладенням, тобто $A\vec{x} = \vec{x}$.
    Чи буде $A$ компактним?
\end{exercise}

\begin{exercise}
    Чи може ненульовий компактний оператор $A$ в нескінченновимірному нормованому
    просторі задовольняти рівняння $\sum\limits^n_{k=0} c_k A^k = 0$, де $A^0 \coloneqq I$?
\end{exercise}

\begin{exercise}
    Нехай $H$ --- гільбертів простір, $A \in L(H)$. Довести, що оператори $A$, $A^*$,
    $A^* A$ одночасно компактні або некомпактні.
\end{exercise}

\begin{exercise}\label{N:1_5_31}
    Нехай $A$, $B$, $C$ --- самоспряжені обмежені оператори в гільбертовому просторі $H$;
    $A$, $C$ --- компактні, $A \leq B \leq C$. Довести $B$ --- компактний.
\end{exercise}

\begin{exercise}
    Нехай $A$ --- оператор правого зсуву в $\ell_2$: $A\vec{x} = (0,x_1,x_2,\dots)$.
    Довести $(B \in K(\ell_2), AB=BA) \Rightarrow (B=0)$.
\end{exercise}

\begin{exercise}\label{N:1_5_33}
    Нехай $H$ --- гільбертів простір; $A \in K(H)$; $x \in H$, $\norm{x} = 1$;
    $\norm{Ax} = \norm{A}$. Довести, що оператор $A$ переводить $\{x\}^\perp$
    в $\{Ax\}^\perp$
\end{exercise}

\begin{exercise}
    Довести, що кожний компактний оператор $A$ в $\ell_2$ дорівнює сумі $A_1 + A_2$,
    де $A_1$ --- оператор скінченного рангу, а $\norm{A_2}<1$.
\end{exercise}

\begin{exercise}
    Нехай $A$ --- компактний оператор в гільбертовому просторі $H$, $M$ ---
    замкнена опукла обмежена множина в $H$. Довести:
    \begin{enumerate}
        \item $A(M)$ --- замкнена множина в $H$;
        \item $\forall y \in H$ $\exists x_0 \in M$: $\rho(A(M),y) = \norm{Ax_0 - y}$.
    \end{enumerate}
\end{exercise}

\begin{exercise}
    Нехай $X$ --- банахів простір, $A \in L(X)$ та існує таке $m>0$, що для кожного
    $x \in X$ виконується нерівність $\norm{Ax} \geq m \norm{x}$. Чи може $A$ бути
    компактним оператором?
\end{exercise}

\begin{exercise}
    Чи може образ компактного оператора бути замкненим?
\end{exercise}

\begin{exercise}
    Навести приклад оператора $A$, який не є компактним, але $A^2$ --- компактний.
\end{exercise}

\begin{theory}
    Нехай $H$ --- сепарабельний нескінченновимірний гільбертів простір, $\{e_n\}$ ---
    ортонормований базис в $H$; $A \in L(H)$. $A$ називається \ul{оператором
    Гільберта-Шмідта}, якщо збігається ряд $\sum\limits^\infty_{n=1} \norm{A e_n}^2$.

    Множину всіх операторів Гільберта-Шмідта в $H$ позначимо через $S_2(H)$. 
\end{theory}
            % !TEX root = ../main.tex

\begin{exercise}
    Нехай $\{e_n\}$ та $\{f_n\}$ --- два ортонормованих базиси в гільбертовому просторі $H$, 
    $A \in L(H)$. Довести рівність: 
    $\sum\limits_{n = 1}^\infty \norm{A e_n}^2 = \sum\limits_{n = 1}^\infty \norm{A f_n}^2$ (точніше: якщо
    один ряд збігається, то збігається інший та вони мають однакові суми).
\end{exercise}

\begin{theory}
    Число $\norm{A}_2 = \left(\sum\limits_{n = 1}^\infty \norm{A e_n}^2\right)^{\frac{1}{2}}$ називається
    \underline{абсолютною нормою} оператора $A$.
\end{theory}

\begin{exercise}
    Нехай $A \in S_2 (H)$; $B \in L(H)$. Довести наступні твердження:
    \begin{enumerate}
        \item $A^* \in S_2 (H)$; $\norm{A^*}_2 = \norm{A}_2$;
        \item $\norm{A} \leq \norm{A}_2$;
        \item $AB \in S_2 (H)$; $BA \in S_2 (H)$; $\norm{AB}_2 \leq \norm{A}_2 \cdot \norm{B}$; 
        $\norm{BA}_2 \leq \norm{A}_2 \cdot \norm{B}$;
        \item $A \in K(H)$.
    \end{enumerate}
\end{exercise}

\begin{exercise}
    Довести наступні твердження:
    \begin{enumerate}
        \item $S_2(H)$ є лінійним простором за стандартними операціями над лінійними операторами; 
        \item $S_2(H)$ є гільбертовим простором із скалярним добутком: 
        $(A, B) = \sum\limits_{n = 1}^\infty \left(A e_n, B e_n\right)$, причому сума цього ряду від вибору
        ортонормованого базису не залежить. Зокрема, $S_2(H)$ --- банахів за нормою $\norm{\cdot}_2$;
        \item $S_2(H)$ є незамкненою множиною в $L(H)$ (за нормою в $L(H)$).
    \end{enumerate}
\end{exercise}

\begin{exercise}\label{N:1_5_42}
    Нехай $\{a_n\}$ --- числова послідовність; оператор $A:\ell_2 \rightarrow \ell_2$ визначений формулою:
    $A \vec{x} = \left(a_1 x_1, a_2 x_2, \dots \right)$. З'ясувати:
    \begin{enumerate}
        \item за якою умовою на послідовність $\{a_n\}$ оператор $A$ буде оператором Гільберта-Шмідта;
        \item за якою умовою на послідовність $\{a_n\}$ оператор $A$ буде компактним, але не оператором Гільберта-Шмідта?
    \end{enumerate}
\end{exercise}

\begin{theory}
    Лінійний оператор $A$ на сепарабельному гільбертовому просторі називається \underline{ядерним}, якщо він
    може бути представлений як добуток: $A = BC$, де $B, C \in S_2(H)$. Множину всіх ядерних операторів позначимо
    через $S_1(H)$.
\end{theory}

\begin{exercise}
    Довести: $S_1(H) \subset S_2(H)$.
\end{exercise}

\begin{exercise}
    За якої умови на $\{a_n\}$ оператор в $\ell_2$ із задачі \ref{N:1_5_42} буде:
    \begin{enumerate}
        \item ядерним;
        \item оператором Гільберта-Шмідта, але не ядерним.
    \end{enumerate}
\end{exercise}

\begin{exercise}
    Довести:
    \begin{enumerate}
        \item $(A \in S_1(H)) \Rightarrow (A^* \in S_1(H))$;
        \item $(A \in S_1(H)) \Rightarrow$ (число $Tr A = \sum\limits_{n = 1}^\infty (A e_n, e_n)$ є скінченним
        і не залежить від вибору ортонормованого базису $\{e_n\}$);
        \item $(A \in S_1(H); B \in L(H)) \Rightarrow (AB \in S_1(H); BA \in S_1(H))$; 
        \item[г)*] $(A \in S_1(H)) \Rightarrow$ (ряд $\sum\limits_{n = 1}^\infty |(A e_n, e_n)|$ збіжний для будь-якого ортонормованого базису).
    \end{enumerate}
\end{exercise}

\begin{exercise}
    Нехай $A, B \in S_2(H)$. Довести: $Tr (AB) = Tr (BA)$.
\end{exercise}
  
        \section{Обернені оператори}
            % !TEX root = ../main.tex

\begin{theory}
    Нехай $X$, $Y$ --- нормовані простори з нормами $\norm{\cdot}_X$ та 
    $\norm{\cdot}_Y$ відповідно; $I_X$ та $I_Y$ --- відповідні тотожні 
    оператори; $A \in L(X, Y)$. Оператор $B \in L(Y, X)$ називається 
    \ul{лівим оберненим до $A$}, якщо $BA = I_X$; $C \in L(Y, X)$ 
    називається \ul{правим оберненим до $A$}, якщо $AC = I_Y$. В разі якщо оператор 
    $B \in L(Y, X)$ є одночасно лівим і правим оберненим до $A$, він називається 
    (\ul{неперервно}) \ul{оберненим} до $A$ (позначення: $B = \inv{A}$), 
    а сам оператор $A$ --- \ul{неперервно оборотним}.
\end{theory}

\begin{exercise}
    Нехай оператор $A \in L(X, Y)$ має лівий обернений $B \in L(Y, X)$ та правий обернений 
    $C \in L(Y, X)$. Доведіть: 
    \begin{enumerate}
        \item $\mathrm{Im} A = Y$, $\mathrm{Im} B = X$;
        \item $\mathrm{Ker} A = \{0\}$, $\mathrm{Ker} C = \{0\}$;
        \item $B = C$ --- обернений оператор до $A$.
    \end{enumerate}
\end{exercise}

\begin{exercise}
    Нехай $A \in L(X, Y)$. Довести, що наступні дві умови еквівалентні:
    \begin{enumerate}
        \item $A$ --- неперервно оборотний;
        \item $\mathrm{Im} A = Y$ та $\exists \; m > 0$ таке, що $\forall x \in X:\norm{Ax} \geq m \norm{x}$. 
    \end{enumerate}
\end{exercise}

\begin{exercise}
    Нехай $\mathrm{dim} X < \infty$, $A$ --- лінійний оператор в $X$. Тоді $A$ --- неперервно оборотний 
    тоді й тільки тоді, коли $\mathrm{det} A \neq 0$.
\end{exercise}

\begin{exercise}
    Нехай $X$ --- банахів простір; $A \in L(X)$, $\norm{A} < 1$. Тоді оператор $I-A$ --- 
    неперервно оборотний і при цьому $\inv{(I-A)} = I + \sum\limits_{n=1}^\infty A^n$.
\end{exercise}

\begin{exercise}
    Нехай оператор $A: \ell_2 \rightarrow \ell_2$ визначено формулою: 
    $A\vec{x} = (a_1x_1, a_2x_2, ...)$, де $\underset{n}{\sup}|a_n| < \infty$. Довести: 
    $A$ --- неперервно оборотний тоді й тільки тоді, коли $\underset{n}{\inf}|a_n| > 0$. 
    Знайти $\inv{A}$.
\end{exercise}

\begin{exercise}
    Нехай $X$ --- банахів простір, $A, B \in L(X)$. Нехай $A$ --- неперервно оборотний, 
    $\norm{B} < \inv{\norm{\inv{A}}}$. Довести: $A + B$ неперервно оборотний і знайти 
    $\inv{(A+B)}$.
\end{exercise}

\begin{exercise}
    Доведіть, що множина неперервно оборотних операторів в банаховому просторі $X$ є відкритою 
    в $L(X)$. 
\end{exercise}

\begin{exercise}
    Нехай $X$ --- нормований простір, $A \in L(X)$, $n \in \mathbb{N}$. Довести: оператори 
    $A$ та $A^n$ одночасно неперервно оборотні чи ні.
\end{exercise}

\begin{exercise}
    Нехай $A, B \in L(X)$, оператори $A$, $BA$ --- неперервно оборотні. Довести: оператор 
    $B$ також неперервно оборотний.
\end{exercise}

\begin{exercise}
    Нехай $X$ --- нормований простір; $A, B \in L(X)$; оператор $(I - AB)$ --- неперервно 
    оборотний; $\inv{(I - AB)} = C$. Довести: $(I - BA)$ --- неперервно оборотний, і знайти 
    $\inv{(I - BA)}$.
\end{exercise}

\begin{exercise}
    Нехай $A: C[0, 1] \rightarrow C[0, 1]$ визначений формулою: $(Ax)(t) = \alpha(t)x(t)$, 
    де $\alpha \in C[0, 1]$ --- фіксована функція. Довести: $A$ -- неперервно оборотний 
    тоді й тільки тоді, коли $\alpha(t) \neq 0$ для всіх $t \in [0, 1]$. 
\end{exercise}

\begin{exercise}
    Нехай $A, B : C[-1, 1] \rightarrow C[-1, 1]$ визначено формулами: $(Ax)(t) = 
    x(t^2)$; $(Bx)(t) = x(t^3)$. Доведіть, що $\inv{A}$ не існує, а $B$ --- неперервно 
    оборотний і знайти $\inv{B}$.
\end{exercise}
            % !TEX root = ../main.tex

\begin{exercise}
    Оператори $A: \ell_2 \to \ell_2$ дослідити на неперервну оборотність:
    \begin{enumerate}
        \item $A\vec{x} = (0, x_1, x_2, ...)$;
        \item $A\vec{x} = (x_2, x_3, x_4, ...)$;
        \item $A\vec{x} = (x_1+x_2, x_2, x_3, ...)$;
        \item $A\vec{x} = (x_3, x_4, x_2, x_1, x_5, x_6, x_7, ...)$;
        \item $A\vec{x} = (x_1+x_2, x_1 - x_2 + x_3, x_2 + x_3 + x_4, x_4, x_5, ...)$.
    \end{enumerate}
\end{exercise}

\begin{exercise}
    $A \in L(X, Y)$. Довести, що якщо обернений оператор $\inv{A}$
    існує, то він єдиний.
\end{exercise}

\begin{exercise}
    Нехай $A, B \in L(X)$ і є неперервно оборотними.
    Довести: $AB$ --- неперервно оборотний і при цьому $\inv{(AB)} = \inv{B} \inv{A}$.
\end{exercise}

\begin{exercise}
    $A: C[0;1] \to C[0;1]$. Дослідити $A$ на неперервну оборотність та знайти $\inv{A}$ (якщо існує):
    \begin{enumerate}
        \item $(Ax)(t) = \int\limits_0^t x(s)ds$;
        \item $(Ax)(t) = x(t) - \int\limits_0^t x(s)ds$;
        \item $(Ax)(t) = x(t) - \int\limits_0^1 t s x(s) ds$.
    \end{enumerate}
\end{exercise}

\begin{exercise}
    При яких $\lambda \in \real$ оператор $(A_\lambda x)(t) = x(t) + \lambda \int\limits_0^1 (t+s) x(s) ds$,
    що діє в просторі $C[0;1]$, має неперервний обернений? Знайти $\inv{A_\lambda}$.
\end{exercise}

\begin{exercise}
    Оператор $A$ в просторі $L_2 [0;1]$ визначено формулою $(Ax)(t) = x(t) - \int\limits_0^t x(s) ds$.
    Дослідити $A$ на неперервну оборотність та знайти $\inv{A}$, якщо він існує.
\end{exercise}

\begin{exercise}
    Нехай $A, B \in L(X)$, $B$ --- неперервно оборотний.
    Довести $\norm{AB} \geq \norm{A} \cdot \norm{\inv{B}}^{-1}$.
\end{exercise}

\begin{exercise}
    $X$ --- банахів простір, $A \in L(X)$, $\left| \lambda\right| > \norm{A}$.
    Довести: $A - \lambda I$ --- неперервно оборотний оператор.
\end{exercise}

\begin{exercise}
    Нехай $H$ --- гільбертів простір, $A \in L(H)$, $\underset{\norm{x} = 1}{\inf} \norm{Ax} > 0$, $\mathrm{Ker} A^* = \{0\}$.
    Довести: $A$ --- неперервно оборотний.
\end{exercise}

\begin{exercise}
    Розглянемо в просторах $C[0;1]$ та $L_2 [0;1]$ оператор взяття первісної \\
    $(Ax)(t) = \int\limits_0^t x(s) ds$. Довести, що у цього оператора немає
    ані лівого, ані правого оберненого.
\end{exercise}

\begin{exercise}
    Нехай оператор $A: C^1 [0;1] \to C[0;1]$ визначено формулою $(Ax)(t) = x^\prime(t)$.
    Довести, що він не є оборотним. Знайти правий обернений оператор.
\end{exercise}

\begin{exercise}
    Оператор $A$ в просторі $C[0;1]$ визначено формулою $(Ax)(t) = x(t) + \int\limits_0^1 e^{t+s}x(s) ds$.
    Довести його неперервну оборотність та знайти $\inv{A}$.
\end{exercise}

\begin{exercise}
    Нехай $A: \vec{x} \mapsto (0, x_1, x_2, ...)$ --- оператор правого зсуву в $\ell_2$.
    Нехай $B \in L(\ell_2)$, $\norm{B} < 1$. Довести, що оператор $A+B$ не є оборотним.
\end{exercise}

\begin{theory}
    \ul{Наслідок:} множина оборотних операторів не щільна в $L(\ell_2)$.
\end{theory}
            % !TEX root = ../main.tex

\begin{theory}
    \begin{theorem*}[Банаха про обернений оператор]
        Нехай $X$, $Y$ --- банахові простори; $A \in L(X,Y)$,
        $A$ --- бієктивне відображення. Тоді $A$ --- неперервно оборотний
        $\left(\exists \; A^{-1} \in L(Y, X)\right)$.
    \end{theorem*}
\end{theory}

\begin{exercise}
    Нехай $X$, $Y$ --- банахові простори, $A \in L(X,Y)$.
    Доведіть, що наступні три умови еквівалентні:
    \begin{enumerate}
        \item $A$ --- неперервно оборотний;
        \item $A$ має і при тому єдиний правий обернений оператор;
        \item[в)*] $A$ має і при тому єдиний лівий обернений оператор.
    \end{enumerate}
\end{exercise}

\begin{exercise}
    Навести приклад оператора $A \in L(X,Y)$ в банахових просторах, для якого:
    \begin{enumerate}
        \item існує правий обернений $B\in L(Y,X)$, але немає лівого оберненого;
        \item існує лівий обернений $C\in L(Y,X)$, але немає правого оберненого.
    \end{enumerate}
\end{exercise}

\begin{exercise}\label{N:1_6_28}
    Нехай $X$, $Y$, $Z$ --- банахові простори; $B \in L(Y,Z)$ --- ін'єктивний
    оператор ($\mathrm{Ker}B = \{0\}$); $A$ --- лінійний оператор з $X$ в $Y$;
    $BA \in L(X,Z)$, $\mathrm{Im}(BA) = Z$. Довести $A \in L(X,Y)$.
\end{exercise}

\begin{exercise}
    Нехай $\norm{\cdot}_1$, $\norm{\cdot}_2$  --- дві норми в лінійному просторі
    $X$, причому обидва простори $(X, \norm{\cdot}_1)$ та $(X, \norm{\cdot}_2)$ 
    повні. Відомо, що існує константа $C > 0$ така, що для $\forall \; x \in X$:
    $\norm{x}_1 \leq C \norm{x}_2$. Доведіть еквівалентність цих норм.
\end{exercise}

\begin{exercise}
    Нехай $X$ --- нормований простір, $A \in L(X)$ і існують такі комплексні числа
    $\lambda_1, \lambda_2, ..., \lambda_n$, що $I + \lambda_1 A + \lambda_2 A^2
    + ... + \lambda_n A^n = 0$. Довести, що $A$ --- неперервно оборотний.
\end{exercise}

\begin{exercise}
    $X$ --- нормований простір; $A, B \in L(X)$, $AB+A+I=BA+A+I=0$.
    Довести, що $A$ --- неперервно оборотний.
\end{exercise}

\begin{exercise}
    $X$ --- нормований простір; $A, B \in L(X)$, $A$ --- оборотний, $AB=BA$.
    Довести: $A^{-1}B = BA^{-1}$.
\end{exercise}

\begin{exercise}
    Нехай $H$ --- гільбертів простір, $A \in L(X)$, $A$ --- самоспряжений
    і існує таке число $c > 0$, для якого $A \geq cI$.
    Довести, що $A$ --- неперервно оборотний.
\end{exercise}

\begin{exercise}
    Навести приклад банахових просторів $X$, $Y$ і послідовності неперервно
    оборотних операторі $A_n \in L(X,Y)$, для яких $A_n \rightrightarrows A$,
    але $A$ не є оборотним.
\end{exercise}

\begin{exercise}
    Навести приклад банахового простору $X$ і неперервно оборотного оператора
    $A \in L(X)$, для якого існує послідовність необоротних операторів
     $A_n \in L(X)$, що $A_n \overset{s}{\to} A$.
\end{exercise}

\begin{exercise}\label{N:1_6_36}
    Нехай $A, A_n \in L(X, Y)$, $A_n \rightrightarrows A$.
    Довести: $A$ --- неперервно оборотний тоді й тільки тоді, коли
    виконуються дві наступні умови:
    \begin{enumerate}
        \item $\exists \; N$ $\forall n \geq N$: $A_n$ -- неперервно оборотні;
        \item $\underset{n \geq N}{\sup} \norm{A_n^{-1}} < \infty$.
    \end{enumerate}
\end{exercise}
        \section{Спектр та резольвента обмеженого лінійного оператора. 
                 Спектр та резольвента самоспряженого оператора.}
            % !TEX root = ../main.tex

\begin{theory}
    Нехай $X$ --- комплексний банахів простір; $A \in L(X)$; $\lambda \in \complex$.
    Для оператора $\lambda - A$ є наступні можливості:
    \begin{enumerate}[label=\arabic*)]
        \item ${\mathrm{Ker}(\lambda - A) = {0}}$, ${\mathrm{Im}(\lambda - A)} = X$. В цьому разі 
        за теоремою Банаха про обернений оператор, $\lambda - A$ є неперервно оборотним 
        оператором, а тому $\exists \; \inv{(\lambda - A)} \in L(X)$. Такі числа $\lambda$ 
        називаються \ul{регулярними значеннями}  оператора $A$, а множина 
        \ul{$\rho(A) = \rho_A$} всіх регулярних значень називається 
        \ul{резольвентною множиною} оператора $A$.
    \end{enumerate}
    Операторнозначна функція $\rho_A : \lambda \mapsto \inv{(\lambda - A)} \in L(X)$ 
    називається \ul{резольвентою} оператора $A$ і позначається так:
    $R_A(\lambda) = R(\lambda; A) = R_\lambda(A) = \inv{(\lambda - A)}$
    \begin{enumerate}[label=\arabic*), resume]
        \item $\mathrm{Ker}(\lambda - A) \neq {0}$. В цьому разі існує ненульовий вектор $x_0$, 
        для якого $Ax_0 = \lambda x_0$. $\lambda$ називається 
        \ul{власним числом} оператора $A$; $x_0$ --- відповідним \ul{власним вектором}.
        Множина $\sigma_\text{т}(A)$ всіх власних чисел A утворює 
        \uline{<<точковий спектр>>} оператора $A$.
        \item $\mathrm{Ker}(\lambda - A) = {0}$, $\mathrm{Im}(\lambda - A) \neq X$. 
        У випадку нескінченновимірного простору X такі числа можуть існувати,
        найчастіше їх поділяють на дві частини:
        \begin{enumerate}[label = \ukr*)]
            \item $\overline{\mathrm{Im}(\lambda - A)} = X$;
            \item $\overline{\mathrm{Im}(\lambda - A)} \neq X$;
        \end{enumerate}
        У випадку 3а) такі числа утворюють \uline{<<неперервний спектр>>} оператора $A$ \ul{$\sigma_\text{н}(A)$}; у випадку 3б) такі $\lambda$ 
        утворюють \uline{<<залишковий спектр>>} оператора $A$ \ul{$\sigma_\text{з}(A)$}. 
    \end{enumerate}
    Об'єднання $\sigma_\text{н}(A) \bigvee \sigma_\text{з}(A) \bigvee \sigma_\text{т}(A) = \uline{\sigma(A) = \sigma_A}$ називається 
    \ul{спектром} оператора $A$. Тож $\complex = \rho(A) \bigvee \sigma(A)$.
\end{theory}

\begin{exercise}
    \begin{enumerate}
        \item Доведіть, що у випадку $\dim X < \infty$ має місце рівність: 
        $\sigma_\text{т}(A) = \sigma(A)$ для будь-якого лінійного оператора 
        $A: X \rightarrow X$.
        \item Наведіть приклад комплексного банахового простору $X$ і 
        оператора $A \in L(X)$, для якого: $\mathrm{Ker} A = {0}$ та $\mathrm{Im} A \neq X$.
    \end{enumerate}
\end{exercise}

\begin{exercise}\label{N:1_7_2}
    Нехай $A \in L(X)$, $|\lambda| > \norm{A}$. Доведіть, що $\lambda \in 
    \rho(A)$.
\end{exercise}

\begin{theory}
    Число $r(A) = \sup\{|\lambda| : \lambda \in \sigma(A)\}$ називається 
    \ul{спектральним радіусом} оператора $A$. За результатом задачі \ref{N:1_7_2} 
    маємо нерівність: $r(A) < \norm{A}$.
\end{theory}

\begin{exercise}
    Нехай $A \in L(X)$, де $X$ --- комплексний банахів простір. Доведіть: 
    $\sigma(A)$ --- компакт в $\complex$.
\end{exercise}

\begin{exercise}
    Нехай $A \in L(X)$. Довести: $r(A) = \max\{|\lambda| : \lambda \in \sigma(A)\}$.
\end{exercise}

\begin{exercise}
    Нехай $\{a_n\}$ --- обмежена числова послідовність в $\complex$; оператор $A: \ell_2 
    \rightarrow \ell_2$ визначено формулою: $A\vec{x} = (a_1 x_1, a_2 x_2, ...)$. 
    Знайти $\sigma_\text{т}(A)$, $\sigma_\text{н}(A)$, $\sigma_\text{з}(A)$, $r(A)$ та 
    побудувати резольвенту $R_\lambda (A)$.
\end{exercise}

\begin{exercise}
    Нехай $P$ --- ортопроектор в гільбертовому просторі $H$. Знайти 
    $\sigma_\text{т}(P)$, $\sigma_\text{н}(P)$, $\sigma_\text{з}(P)$, $R_\lambda(P)$.
\end{exercise}

\begin{exercise}
    Для оператора $A \in L(C[0, 1])$ знайти 
    $\sigma_\text{т}(A)$, $\sigma_\text{н}(A)$, $\sigma_\text{з}(A)$; $r(A)$, $R_\lambda(A)$, 
    якщо:
    \begin{enumerate}
        \item $(Ax)(t) = tx(t)$;
        \item $(Ax)(t) = a(t)x(t)$, де $a \in C[0, 1]$;
        \item $(Ax)(t) = x(0) + tx(1)$;
        \item $(Ax)(t) = \int\limits_0^t x(s) ds$.
    \end{enumerate}
\end{exercise}

\begin{exercise}
    Для оператора $A \in L(\ell_2)$ знайти 
    $\sigma_\text{т}(A)$, $\sigma_\text{н}(A)$, $\sigma_\text{з}(A)$, $r(A)$, якщо:
    \begin{enumerate}
        \item $A\vec{x} = (x_1 + x_2, x_2, x_3, ...)$;
        \item $A\vec{x} = (x_3 , x_1, x_2, x_4, x_5, ...)$;
        \item $A\vec{x} = (x_1 , x_2, x_3, 0, 0, ...)$;
        \item $A\vec{x} = (-x_1, x_2, -x_3, ..., (-1)^n x_n, ...)$.
    \end{enumerate}
\end{exercise}

\begin{exercise}
    Нехай $H$ --- гільбертів простір. оператор $A \in L(H)$ визначений формулою: 
    $Ax = (x, x_0)x_1$, де $x_0$, $x_1$ --- фіксовані вектори. Знайти
    $\sigma_\text{т}(A)$, $\sigma_\text{н}(A)$, $\sigma_\text{з}(A)$, $r(A)$, $R_\lambda(A)$.
\end{exercise}

\begin{exercise}
    Довести, що для $\lambda, \mu \in \rho(A)$ виконуються рівності: 
    \begin{enumerate}
        \item $AR_\lambda(A) = R_\lambda(A)A$;
        \item $R_\lambda(A) - R_\mu(A) = (\mu - \lambda)R_\lambda R_\mu = (\mu - \lambda) 
        R_\mu R_\lambda$ (\ul{тотожність Гільберта}).
    \end{enumerate}
\end{exercise}
            % !TEX root = ../main.tex

\begin{exercise}
    Довести, що для обмеженого оператора в гільбертовому просторі виконується
    рівність $\sigma(A^{*}) = \set{\overline{\lambda} \mid \lambda \in \sigma (A)}$.
\end{exercise}

\begin{exercise}
    Нехай $X$ --- комплексний банахів простір, $A \in L(X)$. Довести:
    \begin{enumerate}
        \item $\left( \lambda \in \sigma(A)\right) \Rightarrow \left( \lambda^n \in \sigma(A^n)\right)$;
        \item $\sigma(A^n) = \set{\lambda^n \mid \lambda \in \sigma(A)}$ ($n\in \natur$);
        \item $\sigma(p(A)) = \set{p(\lambda) \mid \lambda \in \sigma(A)}$, де $p(A) = a_0 A^n + a_1 A^{n-1} + ... + a_{n-1} A + a_n I$ ($a_k \in \complex$);
        \item якщо $A$ --- неперервно оборотний, то $\left( \lambda \in \sigma(A)\right) \Rightarrow \left( \lambda^{-1} \in \sigma(\inv{A})\right)$.
    \end{enumerate}
    Які з цих тверджень мають місце і для дійсного банахового простору?
\end{exercise}

\begin{theory}
    \begin{theorem*}
        $r(A) = \underset{n \to \infty}{\lim} \sqrt[\leftroot{-3}\uproot{3}n]{\norm{A^n}}$
    \end{theorem*}
    \begin{theorem*}
        $\left( A \in L(X)\right) \Rightarrow (\sigma(A) \neq \varnothing)$
    \end{theorem*}
\end{theory}

\begin{exercise}
    Нехай $A, B \in L(\ell_2)$ є відповідно операторами лівого та правого зсуву:
    $A \vec{x} = (x_2, x_3, ...)$, $B\vec{x} = (0, x_1, x_2, x_3, ...)$.
    Знайти $\sigma_\text{т}(A)$, $\sigma_\text{н}(A)$, $\sigma_\text{з}(A)$,
    $\sigma_\text{т}(B)$, $\sigma_\text{н}(B)$, $\sigma_\text{з}(B)$.
\end{exercise}

\begin{exercise}
    Для операторів $A \in L(L_2 [0;1])$ знайти $\sigma_\text{т}(A)$, $\sigma_\text{н}(A)$, $\sigma_\text{з}(A)$,
    $r(A)$, $R_\Lambda(A)$, якщо:
    \begin{enumerate}
        \item $(Ax)(t) = t x(t)$;
        \item $(Ax)(t) = a(t) x(t)$, де $a(\cdot) \in C[0;1]$;
        \item $(Ax)(t) = \int\limits_0^t x(s) ds$.
    \end{enumerate}
\end{exercise}

\begin{exercise}
    Оператор $A \in L(C[0; \frac{1}{2}])$ визначено формулою $(Ax)(t) = t x(t^2)$. Довести: $r(A) = 0$.
\end{exercise}

\begin{exercise}
    Знайти спектральний радіус оператора $A \in L(X)$, визначеного формулою
    $(Ax)(t) = \int\limits_0^t K(t, s) x(s) ds$, де $K \in C([0;1] \times [0;1])$ в разі, якщо:
    \begin{enumerate}
        \item $X = C[0;1]$;
        \item $X = L_2 [0;1]$.
    \end{enumerate}
\end{exercise}

\begin{exercise}
    Оператор $A \in L(C[0;1])$ визначено як $(Ax)(t) = x(t) + \int\limits_0^t e^{t-s} x(s) ds$.
    Обчислити $r(A)$.
\end{exercise}

\begin{exercise}
    Нехай $X$ --- банахів простір, $A \in L(X)$. Довести, що спектральний радіус $A$ не зміниться, якщо
    в $X$ перейти до еквівалентної норми.
\end{exercise}

\begin{exercise}
    Довести, що будь-який непорожній компакт в $\complex$ є спектром деякого обмеженого оператора.
\end{exercise}

\begin{exercise}
    Нехай $X$ --- банахів простір, $A \in L(X)$, $\lambda \in \complex$. Довести,
    що якщо існує така послідовність $\{ x_n\} \subset X$, для якої $\norm{x_n} = 1$ ($\forall n \in \natur$)
    і $\underset{n \to \infty}{\lim} (A x_n - \lambda x_n) = 0$, то $\lambda \in \sigma (A)$.
\end{exercise}

\begin{exercise}
    Нехай $X$ --- банахів простір, $A, B \in L(X)$, $AB = BA$. Довести:
    \begin{enumerate}
        \item $r(A B) \leq r(A) \cdot r(B)$;
        \item[б)*] $r(A+B) \leq r(A) + r(B)$.
    \end{enumerate}
\end{exercise}

\begin{exercise}
    Нехай $A, B \in L(X)$. Довести:
    \begin{enumerate}
        \item $\left( 0 \notin \sigma(A)\right) \Rightarrow \left( \sigma(AB) = \sigma(BA)\right)$;
        \item $\sigma(AB) \setminus \{ 0 \} = \sigma(BA) \setminus \{ 0 \}$;
        \item $r(AB) = r(BA)$;
        \item $\left( AB - BA = \lambda I \right) \Rightarrow \left( \lambda = 0\right)$.
    \end{enumerate}
\end{exercise}

\begin{exercise}
    Нехай $A \in L(X)$, $A^2$ має власний вектор. Довести: $A$ має власний вектор.
\end{exercise}

\begin{exercise}
    $A \in L(X)$, $| \lambda | > r(A)$. 
    Довести: $\norm{R_\lambda (A)} \leq \left( | \lambda | - \norm{A}\right)^{-1}$.
\end{exercise}

\begin{exercise}
    $A \in L(X)$, $\lambda \in \rho (A)$, $\mu \in \complex$, $|\mu| \leq \norm{R_\lambda (A)}^{-1}$.
    Довести: $\lambda - \mu \in \rho (A)$.
\end{exercise}

\begin{exercise}
    $A \in L(X)$. Чи може $R_\lambda (A)$ бути цілком неперервним оператором?
\end{exercise}

\begin{exercise}
    Нехай $A, A_n \in L(X)$, $A_n \rightrightarrows A$. Довести:
    \begin{enumerate}
        \item $\left( \lambda \in \rho(A) \right) \Rightarrow \left( \exists \; N \; \forall \; n \geq N : \lambda \in \rho(A_n) \right)$;
        \item $\forall \; \varepsilon\text{-окола } (\sigma(A))_\varepsilon \text{ спектра } A \; \exists \; N \; \forall \; n \geq N : \sigma(A_n) \subset (\sigma(A))_\varepsilon $.
    \end{enumerate}
\end{exercise}

\begin{exercise}
    $A : C[0;1] \to C[0;1]$, $(Ax)(t) = x(t^2)$. Довести: $\sigma(A) \subset \set{\lambda \mid |\lambda| = 1}$.
\end{exercise}

\begin{exercise}
    Нехай $A$ --- самоспряжений оператор в гільбертовому просторі. Довести:
    \begin{enumerate}
        \item $r(A) = \norm(A)$;
        \item $\sigma_\text{з}(A) = \varnothing$;
        \item $\left( \mathrm{Im} (A - \lambda I) = H\right) \Rightarrow \left( \lambda \in \rho(A)\right)$;
        \item $\sigma_\text{т}(A) \subset \left[-\norm{A}; \norm{A}\right]$;
        \item $\sigma (A) \subset \left[-\norm{A}; \norm{A}\right]$.
    \end{enumerate}
\end{exercise}
            % !TEX root = ../main.tex

\begin{exercise}
    Нехай $A$ --- самоспряжений обмежений оператор в гільбертовому просторі,
    $m = \inf\set{\dotprod{Ax}{x} \mid \norm{x}=1}$,
    $M = \sup\set{\dotprod{Ax}{x} \mid \norm{x}=1}$.
    Довести:
    \begin{enumerate}
        \item $\sigma(A) \subset [m,M]$;
        \item $m,M \in \sigma(A)$.
    \end{enumerate}
\end{exercise}

\begin{exercise}
    Нехай $A$ --- самоспряжений обмежений оператор в $H$. Довести $(A \geq 0)
    \Leftrightarrow (\sigma(A) \subset [0,+\infty])$.
\end{exercise}

\begin{exercise}
    Нехай $A$ --- нормальний оператор в гільбертовому просторі $(AA^* = A^* A)$.
    Довести: $r(A) = \norm{A}$.
\end{exercise}

\begin{exercise}
    $A \in L(H)$ ($H$ --- гільбертів простір). Доведіть: $AA^*$ та $A^* A$ мають
    однакові не нульові власні числа. А нульові?
\end{exercise}

\begin{exercise}
    Нехай $A$ --- самоспряжений обмежений оператор в $H$, $\sigma(A) = \sigma_T(A)
    = \{0,1\}$. Довести: $A$ --- ортогональний проектор.
\end{exercise}

\begin{exercise}
    Нехай $A$ --- нормальний оператор в комплексному гільбертовому просторі;
    $\sigma(A) \subset \real$. Довести: $A$ --- самоспряжений.
\end{exercise}

\begin{exercise}
    Нехай $A \in L(H)$.
    \begin{enumerate}
        \item\label{N:1_7_36_a} Довести: $\norm{A} = (r(A^*A))^{\frac{1}{2}}$;
        \item Користуючись формулою пункту \ref{N:1_7_36_a} знайти операторну
        норму оператора в $\real^2$, якому відповідає матриця $\begin{pmatrix}
        1 & 2 \\ 3 & 4 \end{pmatrix}$ в канонічному базисі.
    \end{enumerate}
\end{exercise}

\begin{exercise}
    Знайти норму оператора $A: L_2[0;1] \to L_2[0;1]$, що визначений формулою
    $(Ax)(t) = \int\limits^t_0 x(s)ds$.
\end{exercise}
        \section{Спектр компактного оператора та теореми Фредгольма.}
            % % !TEX root = ../main.tex

\begin{exercise}
    Нехай $A$ --- компактний оператор в нескінченновимірному нормованому просторі.
    Чи може власний підпростір, що відповідає ненульовому власному числу оператора $A$,
    бути нескінченновимірним? Чи може власний підпростір, що відповідає нульовому власному числу,
    бути нескінченновимірним?
\end{exercise}

\begin{exercise}
    Нехай компактний самоспряжений оператор $A$ в нескінченновимірному
    гільбертовому просторі має скінченну кількість власних чисел. Довести, що $\lambda = 0$
    --- власне число оператора $A$.
\end{exercise}

\begin{theory}
    \begin{theorem*}
        Нехай $A$ --- компактний оператор в гільбертовому просторі. Тоді всі
        ненульові точки $\sigma(A)$ є власними числами і відповідні власні підпростори є скінченновимірними,
        $\sigma(A)$ не більш, ніж зліченний, і єдиною граничною точкою спектра може бути лише $\lambda = 0$.
        Результат має місце і в банаховому просторі.
    \end{theorem*}
\end{theory}

\begin{exercise}
    Нехай $A$ --- компактний оператор в нескінченновимірному банаховому просторі.
    Доведіть, що $0 \in \sigma (A)$.
\end{exercise}

\begin{exercise}
    Чи може бути наступна множина спектром компактного оператора? В разі позитивної відповіді навести приклад.
    \begin{enumerate}
        \item $\{0; 1\}$;
        \item $[0; 1]$;
        \item $\set{\frac{1}{n} \mid n \in \natur}$;
        \item $\set{1-\frac{1}{n} \mid n \in \natur}$;
        \item $\{0\} \cup \set{\frac{1}{n} \mid n \in \natur}$;
        \item $\{0; i\}$.
    \end{enumerate}
\end{exercise}

\begin{exercise}
    Навести приклади компактних операторів $A$ в нескінченновимірному
    банаховому просторі, для яких:
    \begin{enumerate}
        \item $0 \in \sigma_{\text{т}} (A)$;
        \item $0 \in \sigma_{\text{н}} (A)$;
        \item $0 \in \sigma_{\text{з}} (A)$.
    \end{enumerate}
\end{exercise}

\begin{exercise}
    В функціональному просторі $X$ розглядається інтегральний оператор 
    $(Ax)(t) = \int\limits_0^t x(s) ds$. Довести його компактність і дослідити
    $\sigma_{\text{т}} (A)$, $\sigma_{\text{н}} (A)$, $\sigma_{\text{з}} (A)$
    у наступних випадках:
    \begin{enumerate}
        \item $X = C[0;1]$;
        \item $X = L_2 [0;1]$.
    \end{enumerate}
\end{exercise}

\begin{theory}
    Компактний оператор $A$, який діє в банаховому просторі $X$, називається
    \ul{оператором Вольтерра}, якщо $\sigma(A) = 0$.
\end{theory}

\begin{exercise}
    Нехай $K \in C([0;1] \times [0;1])$. Доведіть, що інтегральний оператор
    $(Ax)(t) = \int\limits_0^t K(t,s) x(s)ds$ є оператором Вольтерра
    в просторі $X$ у наступних випадках:
    \begin{enumerate}
        \item $X = C[0;1]$;
        \item $X = L_2 [0;1]$. 
    \end{enumerate}
\end{exercise}

\begin{exercise}
    Довести, що інтегральний оператор в $L_2[a;b]$
    $(Ax)(t) = \int\limits_a^b K(t,s) x(s)ds$ з неперервним
    <<ядром>> $K(t,s)$ є самоспряженим тоді й тільки тоді, коли
    $K(t,s) = \overline{K(s,t)}$ для всіх $(t,s) \in [a;b] \times [a;b]$.
\end{exercise}

\begin{exercise}
    Знайти спектр, власні числа та власні функції оператора $A$,
    що визначений на $L_2[a;b]$, формулою $(Ax)(t) = \int\limits_a^b K(t,s) x(s)ds$, якщо:
    \begin{enumerate}
        \item $K(t,s) = \begin{cases}
            t(1-s), & t \leq s \\
            s(1-t), & s \leq t
        \end{cases}$, $a = 0, b = 1$;
        \item $K(t,s) = \begin{cases}
            \sin(t)\sin(1-s), & t \leq s \\
            \sin(1-t)\sin(s), & s \leq t
        \end{cases}$, $a = 0, b = 1$;
        \item $K(t,s) = \begin{cases}
            \sin(t)\cos(s), & t \leq s \\
            \cos(t)\sin(s), & s \leq t
        \end{cases}$, $a = 0, b = \pi$;
        \item $K(t,s) = \cos(t-s)$, $a = 0, b = 2\pi$;
        \item $K(t,s) = ts + t^2 s^2$, $a = 0, b = 1$.
    \end{enumerate}
\end{exercise}
            % % !TEX root = ../main.tex

\begin{theory}
    Нехай $A$ --- лінійний оператор в гільбертовому просторі $H$. 
    Для оператора $A = I + T$ ($T$ --- компактний оператор) виконуються 
    3 \ul{теореми Фредгольма}:
    \begin{enumerate}
        \item[\ul{т.1}] Рівняння $Ax = y$ має розв'язок для тих і 
        тільки тих $y$, які ортогональні кожному розв'язку рівняння 
        $A^*u = 0$ ($Im A \oplus Ker A^* = H$).
        \item[\ul{т.2}] Виконується \ul{альтернатива}: або рівняння $Ax = y$ 
        має і притому єдиний розв'язок при кожному $y \in H$ або рівняння 
        $Ax = 0$ має ненульовий розв'язок ($0 \in \rho(A) \cup \sigma_t(A)$).
        \item[\ul{т.3}] Рівняння $Ax = 0$ та $A^* u = 0$ мають і притому 
        однакову скінченну кількість лінійно незалежних розв'язків 
        ($dim Ker A = dim Ker A^* < \infty$)
    \end{enumerate}
\end{theory}

\begin{exercise}
    Довести теореми Фредгольма.
\end{exercise}

\begin{exercise}
    При яких $\lambda \in \complex$ рівняння $x(t) = \lambda 
    \intl{a}{b}e^{t-s}x(s)ds$ має ненульовий розв'язок в просторі 
    $L_2[a, b]$.
\end{exercise}

\begin{exercise}
    При яких $f \in L_2 [0, \pi]$ інтегральне рівняння $x(t) = 
    \intl{0}{\pi} sin(t-s)x(s)ds + f(t)$ має розв'язок в просторі 
    $L_2[0, \pi]$?
\end{exercise}

\begin{exercise}
    Для оператора $A \in L(\ell_2)$ перевірити, які з теорем 1-3 
    Фредгольма і за яких умов виконуються для рівняння $Ax = y$ у 
    наступних випадках:
    \begin{enumerate}
        \item $Ax = (\lambda_1 x_1, \lambda_2 x_2, ...)$, де 
        $\{\lambda_n\}_{n=1}^\infty$ --- обмежена послідовність;
        \item $A$ --- оператор правого зсуву в $\ell_2$;
        \item $A$ --- оператор лівого зсуву в $\ell_2$;
        \item[г)*] $A$ --- сума операторів правого та лівого 
        зсуву в $\ell_2$;
    \end{enumerate}
\end{exercise}

\begin{exercise}
    Нехай $\{e_n\}$ --- ортонормований базис гільбертового простору $H$; 
    $\lambda_n \in \real$; $\lambda_n \rightarrow 0$; 
    $A: H \rightarrow H$ --- 
    такий лінійний оператор, що для кожного $x \in H$ має місце рівність:
    $Ax = \suml{n=1}{\infty} \lambda_n \dotprod{x}{e_n}e_n$.

    Довести, що $A$ --- цілком неперервний самоспряжений оператор.
\end{exercise}

\begin{theory}
    \ul{Теорема Гільберта-Шмідта}. Нехай $H$ --- гільбертів простір, 
    $A$ --- компактний самоспряжений оператор в $H$. Тоді існує ортонормована 
    система векторів $\{e_n\}$ ($1 \leq n \leq N \leq \infty$) та 
    числовий набір $\{\lambda_n\} \subset \real \backslash \{0\}$ 
    ($1 \leq n \leq N \leq \infty$) такі, що $e_n$ --- власні вектори 
    оператора $A$, що відповідають власним числам $\lambda_n$; 
    $\lambda_n \rightarrow 0$ за умови $N = \infty$ і при цьому для кожного 
    $x \in H$ виконуються рівності:
    \begin{equation*}
        x = \suml{n=1}{N}\dotprod{x}{e_n}e_n + \tilde{x}, \text{де } 
        \tilde{x} \in KerA;
    \end{equation*}
    \begin{equation*}
        Ax = \suml{n=1}{N}\lambda_n \dotprod{x}{e_n}e_n 
        \text{ (при }
        N = \infty
        \text{ цей ряд називають \ul{рядом Шмідта})}
    \end{equation*}
\end{theory}
            % % !TEX root = ../main.tex

\begin{exercise}
    Нехай $K(t,s) = \sum\limits^\infty_{n=1} \frac{1}{n^2} \sin(n \pi t) \sin(n \pi s)$,
    оператор $A$ діє в $L_2[-1;1]$ за формулою $(Ax)(t) = \int\limits^1_{-1} K(t,s) x(s) ds$.
    \begin{enumerate}
        \item Довести: $A$ --- компактний самоспряжений оператор;
        \item знайти $\sigma(A)$, $r(A)$, $\norm{A}$;
        \item зобразити $A$ у вигляді ряду Шмідта.
    \end{enumerate}
\end{exercise}

\begin{exercise}
    Нехай $K(t) = \sum\limits_{n \in \integers} C_n e^{i n t}$. Розглянемо в $L_2[-\pi; \pi]$
    оператор $A$ за формулою $(Ax)(t) = \int\limits^\pi_{-\pi} K(t-s) x(s) ds$.
    \begin{enumerate}
        \item Довести: $A$ --- самоспряжений оператор Гільберта-Шмідта; функції \\$\varphi_n(t) = 
        \frac{1}{\sqrt{2\pi}} e^{i n t}$, $n \in \integers$ є власними функціями оператора $A$;
        \item знайти $\sigma(A)$, $r(A)$, $\norm{A}$;
        \item зобразити $A$ у вигляді ряду Шмідта.
    \end{enumerate}
\end{exercise}

\begin{exercise}
    Знайти спектри наступних операторів $A \in L(C[a;b])$:
    \begin{enumerate}
        \item $(Ax)(t) = \int\limits^1_0 (t+s)x(s)ds$; ($a=0$, $b=1$);
        \item $(Ax)(t) = x(t) + \int\limits^1_0 (t+s)x(s)ds$; ($a=0$, $b=1$);
        \item $(Ax)(t) = \int\limits^{2\pi}_0 \cos(t+s)x(s)ds$; ($a=0$, $b=2\pi$).
    \end{enumerate}
\end{exercise}

\begin{exercise}
    Довести, що для будь-якої функції $f \in C[a;b]$ (варіант: $f \in L_2[a;b]$)
    наступне рівняння має розв'язок в $L_2[a;b]$:
    \begin{enumerate}
        \item $x(t) = -\int\limits^1_0 x(s) ds + f(t)$; ($a=0$, $b=1$);
        \item $x(t) = -\int\limits^1_0 (ts+2)x(s)ds + f(t)$; ($a=0$, $b=1$);
        \item $x(t) = -\int\limits^1_0 \cos(t-s)x(s)ds + f(t)$; ($a=0$, $b=1$);
        \item $x(t) = -\int\limits^1_0 e^{ts}x(s)ds + f(t)$; ($a=0$, $b=1$);
        \item $x(t) = \int\limits^1_0 (t-s)x(s)ds + f(t)$; ($a=0$, $b=1$);
        \item $x(t) = \int\limits^\pi_0 \sin(t-s)x(s)ds + f(t)$; ($a=0$, $b=\pi$);
        \item $x(t) = \int\limits^1_0 (t-s)\cos(t+s)x(s)ds + f(t)$; ($a=0$, $b=1$);
    \end{enumerate}
\end{exercise}

\begin{theory}
    \textit{Підказ:} якщо для оператора $A\in L(K)$: $\dotprod{Ax}{x} \leq 0$, $\forall x$,
    то $(x=Ax) \Rightarrow (x=0)$.
\end{theory}

\begin{exercise}
    Використовуючи теорему Фредгольма, довести, що наступні рівняння мають розв'язок в $L_2[a;b]$
    при заданих $f$ та довільних $\lambda$:
    \begin{enumerate}
        \item $x(t) = \lambda \int\limits^1_{-1} x(s) ds + f(t)$;
              $f(t) = t, t^3, e^{t^2}\sin t$; ($a=-1$, $b=1$);
        \item $x(t) = \lambda \int\limits^{2\pi}_0 \sin(t+s) x(s) ds + f(t)$; 
              $f(t) = 1, \sin 2t, \cos 2t$; ($a=0$, $b=2\pi$).
    \end{enumerate}
\end{exercise}

\begin{exercise}
    Використовуючи теорему Фредгольма, довести, що наступні рівняння мають розв'язок в $L_2[a;b]$:
    \begin{enumerate}
        \item $x(t) = \int\limits^1_{-1} \left(
                \sum\limits^\infty_{k=1} \frac{t^k}{k!} s^{2k}
            \right) x(s) ds + t^3$; ($a=-1$, $b=1$);
        \item $x(t) = \int\limits^{2\pi}_0 \left(
                \sum\limits^\infty_{k=1} e^{-kt} \frac{\cos(ks)}{k^2}
            \right) x(s) ds + \sin 2t$; ($a=0$, $b=2\pi$).
    \end{enumerate}
\end{exercise}

\begin{exercise}
    Розв'язати рівняння в просторі $L_2[a;b]$ з симетричним ядром, зведенням їх диференціюванням
    до відповідної крайової задачі:
    \begin{enumerate}
        \item $x(t) = \int\limits^1_0 K(t,s) x(s) ds + t^2 - t$,
        $K(t,s) = \begin{cases}
            (t-1)s, & 0 \leq s \leq t \leq 1 \\
            t(s-1), & 0 \leq t < s \leq 1
        \end{cases}$;
        \item $x(t) = \int\limits^\pi_0 K(t,s) x(s) ds + 1$,\
        $K(t,s) = \begin{cases}
            2 \sin t \cos s, & 0 \leq t \leq s \leq \pi \\
            2 \sin s \cos t, & 0 \leq s < t \leq \pi .
        \end{cases}$
    \end{enumerate}
\end{exercise}
        \newpage
        \section*{Підказки}
            % !TEX root = ../main.tex

\noindent\ref{N:1_1_6}. Припустимо, що $|x(t)|\leq 1$, але $|\varphi(x)| = 2$.
Оскільки $|\int\limits^1_0 x(t) dt| \leq 1$, $|x(0)| \leq 1$, то це можливо лише тоді,
якщо $\int\limits^1_0 x(t) dt = 1$ та $x(0) = -1$, або, навпаки: $\int\limits^1_0 x(t) dt = -1$,
$x(0) = 1$. Розглянемо перший варіант. З неперервності функції $x$ виходить існування
$\delta > 0$ такого, що $x(t) < 0$ при $t \in [0,\delta]$. Але тоді 
$\int\limits^1_0 x(t) dt < \int\limits^1_\delta x(t) dt < 1$.

\noindent\ref{N:1_1_9}. Нехай $\bigcap\limits_{k=1}^m \mathrm{Ker} \varphi_k \subset \mathrm{Ker}\varphi$.
Без втрати загальності можна вважати, що $\forall k \in \set{1, ..., m}: \mathrm{Ker} \varphi_k \not\supset \bigcap\limits_{j \neq k} \mathrm{Ker} \varphi_j$.
Звідси зробіть висновок, що
$\forall k \in \set{1, ..., m}$ $\exists \; x_k \in L$
такий, що $\forall j \in \set{1, ..., m} : \varphi_j(x_k) = \delta_{jk}$ (<<символ Кронекера>>)
і розгляньте функціонал $\sum\limits_{k=1}^m \varphi(x_k) \cdot \varphi_k$.

\noindent\ref{N:1_1_12}.
ґ) $\norm{x}_1 = \underset{0\leq t \leq 1}{\max} |x(t)| + \underset{0\leq t \leq 1}{\max} |x'(t)|$, тож $\norm{A} \leq 1$.
Зворотну нерівність можна одержати за допомогою послідовності $x_n(t) = \frac{1}{n} t^n$.

\noindent з) $\norm{A\vec{x}} \leq 2 \cdot \sum\limits_{k=1}^{\infty} |x_k| = 2 \norm{\vec{x}}$, $A \vec{e_1} = (1, 1, 0, ...)$. $\norm{A} = 2$.

\noindent і) $\norm{Ax}^2 = \int\limits_0^1 \varphi^2 (t) x^2(t) dt \leq \left(\underset{[0;1]}{\max} |\varphi(t)|\right)^2 \cdot \norm{x}^2 $, 
тому $\norm{A} \leq \underset{[0;1]}{\max} |\varphi(t)|$.
З іншого боку, якщо $\underset{[0;1]}{\max} |\varphi(t)| = C > 0$ (випадок $C = 0$ очевидний), то
$\forall \varepsilon \in (0, C) \; \exists \; \text{проміжок } \Delta \subset [0;1]$,
на якому $|\varphi(t)| > C - \varepsilon$. Нехай $x = j_\Delta$ ($x(t) = 1$, якщо $t \in \Delta$ та $x(t) = 0$, якщо $t \notin \Delta$).
Тоді $\norm{x}^2 = \mu(\Delta)$, $\norm{Ax}^2 \geq \int\limits_\Delta (C-\varepsilon)^2 dt = (C-\varepsilon)^2 \cdot \mu(\Delta)$ (тут $\mu$ --- міра <<довжина>>).
Звідси отримуємо $\norm{A} = \underset{[0;1]}{\max} |\varphi(t)|$.

\noindent ї) $\left| (Ax)(t) \right|^2 = t^2 \cdot \left( \int\limits_0^1 s x(s) ds\right)^2 \leq t^2 \cdot \left( \int\limits_0^1 s^2 ds\right) \cdot \left( \int\limits_0^1 x^2(s) ds\right) = \frac{1}{3} \cdot t^2 \cdot \norm{x}^2$.
$\norm{Ax}^2 = \int\limits_0^1 \left| (Ax)(t) \right|^2 dt \leq \frac{1}{9} \cdot \norm{x}^2$, звідки $\norm{A} \leq \frac{1}{3}$.
Зворотну нерівність отримуємо підстановкою $x(t) = t$.

\noindent\ref{N:1_1_17}. Нехай $\varphi \neq 0$ та $\mathrm{Ker}\varphi$ --- замкнене,
$x_n \to 0$. Доведемо: $\varphi(x_n) \to 0$. Нехай $\varphi(x_n) \not\to 0$.
\textit{Випадок 1:} послідовність $\{\varphi(x_n)\}$ --- обмежена. Тоді існує підпослідовність
$x_{n_k}$, для якої $\varphi(x_{n_k}) \to C \neq 0$. Візьмемо вектор $z$, для якого
$\varphi(z) = 1$. Тоді $x_{n_k} - \varphi(x_{n_k})\cdot z \in \mathrm{Ker}\varphi$;
$x_{n_k} - \varphi(x_{n_k})\cdot z \to - Cz \notin \mathrm{Ker}\varphi$, суперечність.
\textit{Випадок 2:} послідовність $\{\varphi(x_n)\}$ --- необмежена. Тоді існує
підпослідовність $x_{n_k}$ така, що $\varphi(x_{n_k}) \to \infty$. Тоді 
$y_k = \frac{1}{\varphi(x_{n_k})}x_{n_k} \to 0$, $\varphi(y_k)=1 \; \forall \;k$,
а це суперечить випадку 1.

\noindent\ref{N:1_1_19}. Нехай $X = C^1[a,b]$, але з нормою простору $C[a,b]=Y$.
Оператор $A = \frac{d}{dt}: X \to Y$ необмежений, але його ядро $\mathrm{Ker}A$
замкнене в $X$.

\noindent\ref{N:1_1_20}. Нехай $x_n \to 0$, але $\varphi(x_n) \not\to 0$.
Тоді існує підпослідовність $x_{n_k}$ така, що $|\varphi(x_{n_k})| \geq \delta > 0$.
Тоді $y_k = \frac{1}{\sqrt{\norm{x_{n_k}}}} x_{n_k} \to 0$, але $|\varphi(y_k)|\to\infty$.

\noindent\ref{N:1_1_21}. $\rho(x,\mathrm{Ker}\varphi) \geq \inf
\left\{ \frac{ |\varphi(x-y)| }{ \norm{\varphi} } \mid y \in \mathrm{Ker}\varphi \right\} = 
\frac{|\varphi(x)|}{\norm{\varphi}}$.
\textit{Зворотня нерівність}: $\forall\;\varepsilon > 0$ $\exists\; z \in X$:
$|\varphi(z)| > (\norm{\varphi} - \varepsilon)\norm{z}$;
$z = \alpha x + y$, $y \in \mathrm{Ker}\varphi$, $\alpha \neq 0$.
$|\alpha| \cdot |\varphi(x)|
> (\norm{\varphi} - \varepsilon)\cdot|\alpha|\cdot\norm{x + \frac{1}{\alpha}y}
\geq (\norm{\varphi} - \varepsilon)\cdot|\alpha|\cdot\rho(x,\mathrm{Ker}\varphi)$.
Звідси: $\frac{\varphi(x)}{\norm{\varphi} - \varepsilon} > \rho(x,\mathrm{Ker}\varphi)$
і врахувати довільність $\varepsilon > 0$.

\noindent\ref{N:1_1_22}. Задача рівносильна \ref{N:1_1_21}: якщо
$\varphi(x) = a$, то $\rho(x,\mathrm{Ker}\varphi) = \rho(0,L)$, де
$L = \set{y \mid \varphi(y) = a}$.

\noindent\ref{N:1_1_26}. Якщо ряд $\sum\limits^\infty_{k=0} A^k$ збігається,
то $\norm{A^k} \to 0$, $k \to \infty$. Якщо існує $n \in \mathbb{N}$, для
якого $\norm{A^n} < 1$, то для $m \in \set{0,1,\dots,n-1}$ ряд
$\sum\limits^\infty_{k=0} \norm{A^{m+kn}}$ збіжний, тому що $\norm{A^{m+kn}}
\leq \norm{A^m} \cdot \norm{A^n}^k$. Абсолютна збіжність вихідного ряду є
наслідком рівності $\sum\limits^\infty_{k=0} \norm{A^k} = 
\sum\limits^{n-1}_{m=0} \sum\limits^\infty_{k=0} \norm{A^{m+kn}}$.

\noindent\ref{N:1_1_27}. $\mathrm{Im}A = \text{л.о.}\{y\}$, якщо $\varphi \neq 0$;
$\mathrm{Im}A = \{0\}$, якщо $\varphi = 0$. $\mathrm{Ker}A = X$, якщо $y = 0$;
$\mathrm{Ker}A = \mathrm{Ker}\varphi$, якщо $y \neq 0$.
$\norm{Ax} = |\varphi(x)|\cdot \norm{y} \leq \norm{\varphi}\cdot\norm{y}\cdot\norm{x}$.
Тому $\norm{A} \leq \norm{\varphi}\cdot\norm{y}$. Нехай $\varphi \neq 0$, $y \neq 0$.
Беремо $\varepsilon > 0$, $\exists\; x \in X$: $|\varphi(x)| > (\norm{\varphi}-\varepsilon)
\norm{x}$. Тому $\norm{Ax} > (\norm{\varphi}-\varepsilon)\cdot\norm{y}\cdot\norm{x}$.
Тому $\norm{A} \geq (\norm{\varphi}-\varepsilon)\cdot\norm{y}$ і врахуйте довільність 
$\varepsilon>0$.

\noindent\ref{N:1_1_30}. а) Відображення $A: \ell_\infty \to \ell_1^*$
будуємо наступним чином: $\vec{c} = (c_1,c_2,\dots) \in \ell_\infty$,
$A\vec{c}: \ell_1 \ni \vec{x} \mapsto \sum\limits^\infty_{k=1} c_k x_k$.
При цьому ряд $\sum\limits^\infty_{k=1} |c_k x_k|$ --- збіжний, тому що послідовність
$\{c_k\}$ обмежена, а ряд $\sum\limits^\infty_{k=1} |x_k|$ --- збіжний.
$A\vec{c}$ --- лінійний функціонал на $\ell_1$. Його обмеженість --- наслідок нерівності
$|\sum\limits^\infty_{k=1} c_k x_k| \leq \sup\{|c_k|\} \cdot \sum\limits^\infty_{k=1} |x_k|$.
При цьому $\norm{A \vec{c}} \leq \norm{\vec{c}} = \sup\{|c_k|\}$. Насправді 
$\norm{A \vec{c}} = \norm{\vec{c}}$, тому що для $\forall k: |(A \vec{c})(\vec{e}_k)| = |c_k|$,
де $\vec{e}_k = (\underbrace{0,0,\dots}_{k-1},1,0,\dots) \in \ell_1$.
Наступний крок --- це лінійність відображення $A$ (перевірте!). Залишилось довести
$\mathrm{Im}A = \ell_1^*$. Беремо $\varphi\in \ell_1^*$. Будуємо послідовність $\{c_k\}$
за правилом $c_k = \varphi(\vec{e}_k)$. Тоді $\{c_k\}$ --- обмежена послідовність, 
$\vec{c} \in \ell_\infty$. При цьому $\varphi(\vec{x}) = \sum\limits^\infty_{k=1} c_k x_k$.
Остання формула --- наслідок граничного переходу $\varphi(\vec{x}) = 
\underset{n\to \infty}{\lim} \varphi\left(\sum\limits^n_{k=1} x_k \vec{e}_k\right)$.
Тож $\varphi = A \vec{c}$.

\noindent\ref{N:1_1_42}. а) $\norm{\tilde{x}} = \sup\left\{|\varphi(x)| \mid
\varphi\in X^*; \norm{\varphi}=1\right\} \leq \norm{x}$. За теоремою Гана-Банаха 
$\exists\varphi\in X^*$, $\norm{\varphi} = 1$, для якого досягається рівність
$|\varphi(x)| = \norm{x}$.

\noindent\ref{N:1_1_44}. Для вимірної підмножини $Z \subset [0; 1]$ розглянути оператор
$A_Z: f \mapsto f \cdot j_Z$ ($j_A$ --- індикатор множини $A$). У разі, якщо $\mu(Z) \neq 0$, 
$\norm{A_Z} = 1$ і, якщо $\mu (Z_1 \bigtriangleup Z_2) > 0$, $\norm{A_{Z_1} - A_{Z_2}} = 1$.
Розглянути $\{A_{Z_t}\}$, де $Z_t = [0, t]$.

\noindent\ref{N:1_1_45}. Умова $AB = I_Y$ гарантує, що $\mathrm{Im} A = Y$; умова $BA = I_X$
гарантує, що $\rm KerA = \{0\}$. Тому існує обернений оператор $A^{-1}$, $B = A^{-1}AB = A^{-1}$; $C = CAA^{-1} = A^{-1}$.

\noindent\ref{N:1_1_46}. Розглянути на $\ell_2$ функціонали $\varphi_n (\vec{\beta}) = \sum\limits_{k = 1}^{n} \alpha_k \beta_k$
і застосувати теорему Банаха-Штейнгауза.

\noindent\ref{N:1_1_48}. При кожному фіксованому $x: \varphi_x(y) = \varphi(x, y)$ --- лінійний обмежений функціонал на $Y$. Тому
$|\varphi(x, y)| \leq \norm{\varphi_x} \cdot \norm{y}$. Далі для всіх $y$, що задовольняють нерівність $\norm{y} \leq 1$
розглянемо (обмежені) функціонали $\psi_y (x) = \varphi(x, y)$ на просторі $X$. $\forall x \in X$ значення цієї
сім'ї функціоналів: $\{\psi_y (x) | \norm{y} \leq 1\}$ є обмеженою множиною $(|\psi_y(x)| \leq \norm{\varphi_x})$ і тому за теоремою
Банаха-Штейнгауза: $|\varphi(x, y)| = |\psi_y(x)| \leq K\norm{x}$. Якщо тепер відмовитись від обмеженості $\{\norm{y}\}$, то приходимо до
шуканої нерівності.

\noindent\ref{N:1_2_8}. Нехай $A: H_1 \to H_2$ --- ізоморфізм і $H_1$ --- сепарабельний
гільбертів простір. Доведемо сепарабельність простору $H_2$. Нехай система $\{e_n\}$ ---
ортонормована в $H_1$. Тоді система $\{A e_n\}$ --- ортонормована в $H_2$. Оскільки
лінійна оболонка $\{e_n\}$ щільна в $H_1$, а оператор $A$ --- обмежений і $\mathrm{Im}A = H_2$,
то вектори $f_n = A e_n$ утворюють ортонормований базис в $H_2$. Тому в $H_2$ щільна
множина лінійних комбінацій векторів $f_n$ з раціональними (у випадку поля $\mathbb{R}$)
коефіцієнтами. У випадку поля $\mathbb{C}$ слід брати комплексні коефіцієнти
$z_k = u_k + i v_k$, $u_k, v_k \in \mathbb{Q}$.

\noindent\ref{N:1_2_14}. в) Відповідь: $M^{\perp} = \overline{\{1; \cos nt, n \in \mathbb{N} \}}$.

\noindent\ref{N:1_2_16}. а) $\Rightarrow$ б): 
$(x \in H_2^{\perp}) \iff (\norm{x_2}^2 = (x, x_2) = 0) \iff (x \in H_1)$;

\noindent б) $\Rightarrow$ a): $x - pr_{H_2} x \in H_2^{\perp} = H_1$.

\noindent\ref{N:1_2_18}. ґ) $(x \in M^{\perp}+N^{\perp}) \Rightarrow (x = x_1+x_2; x_1 \perp M; x_2 \perp N)$
$\Rightarrow (x \in (M \cap N)^{\perp})$. Тому $\overline{M^{\perp} + N^{\perp}} \subset (M \cap N)^{\perp}$.
$L = M \cap N$. $M = L \oplus  M_1; N = L \oplus N_1 (M_1 = L^{\perp} \cap M; N_1 = L^{\perp} \cap N); M_1 \cap N_1 = \{0\}$.
Тому $L^{\perp} = M_1 \oplus M^{\perp} = N_1 \oplus N^{\perp}$. 
$M_1 = M_1 \cap L^{\perp} = M_1 \cap (N_1 \oplus N^{\perp}) = M_1 \cap N^{\perp}$. 
Тому $L^{\perp} = (M_1 \cap N^{\perp}) \oplus M^{\perp} \subset \overline{N^{\perp} + M^{\perp}}$.

\noindent\ref{N:1_2_21}. $M = \left\{(1+\frac{1}{n})\vec{e_n}\right\}$ 
(тут $\vec{e_n} = (\underbrace{0, \dots 0}_{n-1}, 1, 0, \dots)$).

\noindent\ref{N:1_2_24}. Нехай $d = \rho(x, M)$; $\rho(x, y_1) = d + \varepsilon_1$; $\rho(x, y_2) = d + \varepsilon_2$;
$\varepsilon_2 \in (0, \varepsilon_1)$. Оскільки $\rho(x, \frac{y_1+y_2}{2}) \geq d$, то з
рівності паралелограма для векторів $y_1 - x$ та $y_2 - x$ одержимо нерівність: 
$\rho(y_1, y_2) \leq 2 \sqrt{2d\varepsilon_1 + \varepsilon_1^2}$. Тому існує послідовність додатних чисел
$\varepsilon_n \searrow 0$, для якої $\rho(y_n, y_{n+1}) \leq \frac{1}{2} \rho(y_{n-1}, y_n)$.
Послідовність $\{y_n\}$ є фундаментальною.

\noindent\ref{N:1_2_26}. б) $\Rightarrow$ в): застосувати теорему Банаха-Штейнгауза до послідовності функціоналів
$\varphi_n (y) = \left(y, \sum\limits_{k = 1}^n x_k\right)$.

\noindent\ref{N:1_2_27}. Якщо $\inf \lambda_k > 0$, то нова норма еквівалентна стандартній в $\ell_2$.
Якщо $\inf \lambda_k = 0$, то в разі повноти простору для функціоналів 
$\varphi_k (y) = (\frac{1}{\lambda_k} \vec{e_k}, \vec{y})$ була б суперечність із теоремою Банаха-Штейнгауза.

\noindent\ref{N:1_2_32}. Якщо послідовність функцій $\{f_n\}$ має обмежені у сукупності похідні, то ця
система функцій одностайно неперервна. Крім того, умови 
$\underset{n}{\sup} \{|f_n (x) - f_n (y)|\mid n \in \mathbb{N}; x, y \in [0, 2\pi]\} < \infty$ та 
$\int\limits_0^{2\pi} f_n^2 (x) dx = 1 \; \forall \;n \in \mathbb{N}$ разом гарантують рівномірну обмеженість системи функцій.
За теоремою Арцела-Асколі: $\{f_n\}$ --- передкомпакт в $C[0; 2\pi]$, а тому й в $L_2 [0; 2\pi]$, що
суперечить її ортонормованості.

\noindent\ref{N:1_3_14}. б) Скористаємось тим, що $A^* A$ --- самоспряжений оператор та
результатом задачі \hyperref[N:1_3_12_h]{\ref*{N:1_3_12} \ref*{N:1_3_12_h}}. $\norm{A^2}^2 = \norm{(A^2)^*A^2} = 
\norm{(A^* A)^2} = \norm{A^* A}^2 = \norm{A}^4$.

\noindent\ref{N:1_3_18}. в) $(P_1 - P_2 \text{ --- ортопроектор}) \Leftrightarrow
\left( \dotprod{(P_1 - P_2)x}{x} = \norm{(P_1 - P_2)x}^2 \geq 0 \right)$;\\
$(P_1 \geq P_2) \Rightarrow \Big( 
(x \in H_2) \Rightarrow \left( \norm{x}^2 = \dotprod{P_2 x}{x} \leq \dotprod{P_1 x}{x} = \norm{P_1 x}^2\right)
\Rightarrow (x \in H_1)\Big) \Rightarrow (H_2 \subset H_1)$;
$(H_2 \subset H_1) \Rightarrow (H_2 \perp (H_1 \ominus H_2)).$
Нехай $H_3 \coloneqq H_1 \ominus H_2$, $P_3$ --- ортопроектор на $H_3$.
Скористаємось \hyperref[N:1_3_18_a]{\ref*{N:1_3_18} \ref*{N:1_3_18_a}}: $P_2 + P_3 = P_1$,
а тому $P_1 - P_2 = P_3$;

\noindent г) Якщо $P_1 P_2 = P_2$, то $P_2 P_1 = (P_1 P_2)^* = P_2$. Тому
$(P_1 - P_2)^2 = P_1 - P_2$ та $P_1 \geq P_2$. Якщо $P_1 \geq P_2$, то $H_2 \subset H_1$,
а тому $\forall x \in H$: $P_1 P_2 x = P_2 x$.

\noindent\ref{N:1_3_22}. б) $((I + A^* A)x = 0) \Rightarrow \left(0 = \dotprod{(I + A^* A)x}{x}
= \norm{x}^2 + \norm{Ax}^2\right) \Rightarrow (x=0) \Rightarrow (\mathrm{Ker}(I + A^* A) = \{0\})$.
Оскільки $B = I + A^* A$ --- самоспряжений, то $\overline{\mathrm{Im}B} =
(\mathrm{Ker}B)^\perp = H$. Замкненість $\mathrm{Im}B$ та обмеженість $B^{-1}$ є наслідком
нерівності $\norm{Bx} \geq \norm{x}$ (обміркуйте!). 

\noindent\ref{N:1_3_23}. а) $\Rightarrow$ б): $\left(\exists \;A^{-1} \in L(H)\right) \Rightarrow
\left(\exists\;(A^*)^{-1} = (A^{-1})^* \in L(H)\right)$;
$\exists \;m>0$: $\norm{Ax} \geq m\norm{x}$. Тому $\dotprod{A^* A x}{x} = \norm{Ax}^2 \geq
m^2 \norm{x}^2$, $\beta = m^2$;

\noindent б) $\Rightarrow$ а): $(A^* A \geq \beta I) \Rightarrow (\norm{Ax} \geq 
\sqrt{\beta}\norm{x})$. Аналогічно $\norm{A^* x} \geq \sqrt{\alpha}\norm{x}$ $(\forall x \in H)$.
Тому $\mathrm{Ker}A = \mathrm{Ker}A^* = \{0\}$, $\mathrm{Im}A = \mathrm{Im}A^* = H$ та
$A^{-1}, (A^*)^{-1} \in L(H)$.

\noindent\ref{N:1_3_25}. Нехай $P_x$ --- ортопроектор на л.о.$\{x\}$. Тоді $(P_x Ax = 
AP_x x = Ax) \Rightarrow (Ax = \lambda(x)x)$. Нехай $L = \text{л.о.}\{x,y\}$;
$Bx = y$, $By = x$ та $Bz = z$ для $z \in L^\perp$. $BAx = \lambda(x)y; ABx = Ay = \lambda(y)y$.
Тож $\lambda(x) = \lambda(y)$.

\noindent\ref{N:1_3_33}. $A^* x = \dotprod{x}{z}y$. Тепер скористайтесь задачею \ref{N:1_1_28}
та теоремою Фреше-Рісса.

\noindent\ref{N:1_3_39}. Нехай $\varphi_y: x \mapsto \dotprod{Ax}{y} = \dotprod{x}{Ay}$.
$\varphi_y \in H^*$ і при цьому $\norm{\varphi_y} = \norm{Ay}$. Якщо 
$\underset{\norm{y}=1}{\sup} \norm{\varphi_y} = \infty$, то існують $y_n$ такі,
що $\norm{y_n} = 1$, $\norm{\varphi_{y_n}} \to \infty$.
Але $\forall x \in H$: $|\varphi_{y_n}(x)| \leq \norm{Ax}$ і треба скористатися теоремою Банаха-Штейнгауза.

\noindent\ref{N:1_4_13}. в) $\Rightarrow$ а). $x_n \rightarrow 0$. Нехай $A x_n \not\rightarrow 0$. Тоді існує
$x_{n_k} \rightarrow 0$ така, що $\norm{A x_{n_k}} \geq \delta > 0$. Тоді 
$y_k = \frac{1}{\sqrt{\norm{x_{n_k}}}} x_{n_k} \rightarrow 0$, але $\{A y_k\}$ --- необмежена.

\noindent\ref{N:1_4_14}. Використовуємо \ref{N:1_4_13}. $(x_n \xrightarrow[\text{сл}]{} x) \Rightarrow (Ax_n \xrightarrow[\text{сл}]{} Ax)$.

\noindent\ref{N:1_4_15}. а) \ul{в)$\Rightarrow$а)} Скористаємось ізометричним вкладенням $Y \subset 
Y^{**}$. Якщо $x_n \rightarrow 0$, але $Ax_n \nrightarrow 0$, то $\exists \delta > 0$ та 
$\{x_{n_k}\}$ такі, що $\norm{Ax_{n_k}} \geq \delta$. Тоді $z_k = 
\frac{x_{n_k}}{\norm{x_{n_k}}} \rightarrow 0$, але послідовність $y_k = Az_k$ --- необмежена 
в $Y^{**}$, при тому, що за умовою в) $\{y_k\}$ --- слабко збіжна послідовность функціоналів на 
банаховому просторі $Y^*$. Повнотою просторів $X$ та $Y$ не користуємось.

\noindent\ref{N:1_4_15}. б) Оскільки для $\forall \varphi \in Y^*$ для $x_1, x_2 \in X$ 
виконується рівність: $\varphi(A(x_1 + x_2)) = \varphi(Ax_1 + Ax_2)$, то, застосовуючи Гана-Банаха, 
$A(x_1 + x_2) = Ax_1 + Ax_2$. Аналогічно $A(\lambda x) = \lambda Ax$. Обмеженість оператора $A$ 
довести з використанням імплікації \ul{в) $\Rightarrow$ а)} пункту а).

\noindent\ref{N:1_4_16}. б) 
$\left(x_n \xrightarrow[\text{сл}]{} x; \norm{x_n} \rightarrow \norm{x}\right) \Rightarrow (\norm{x_n - x}^2 = \norm{x_n}^2 + \norm{x}^2 - (x_n, x) - (x, x_n) \\ \rightarrow 0)$.
$\left(x_n \xrightarrow[\text{сл}]{} x; \overline{\lim\limits_{n \rightarrow \infty}} \norm{x_n} \leq \norm{x}\right)$
$\Rightarrow \left(\overline{\lim\limits_{n \rightarrow 0}} \norm{x_n - x}^2 = \overline{\lim\limits_{n \rightarrow 0}} (\norm{x_n}^2 - \norm{x}^2) \leq 0\right)$
$\Rightarrow (\norm{x_n - x} \rightarrow 0)$.

\noindent\ref{N:1_4_20} в) Нехай $\varphi(y) = \lim\limits_{n \rightarrow \infty} \dotprod{y}{x_n}$. 
$\varphi \in H^*$(теорема Банаха-Штейнгауза). Далі застосувати теорему Рісса.

\noindent\ref{N:1_4_22}. Позначимо через $A: X \rightarrow Y$ відображення $Ax = \lim\limits_{n \rightarrow \infty} A_n x$.
$A(\alpha x + \beta y) = \lim\limits_{n \rightarrow \infty} A_n(\alpha x + \beta y) = \alpha Ax + \beta Ay$.
$\exists C > 0: \forall n: \norm{A_n} \leq C$ (за теоремою Банаха-Штейнгауза). Тому 
$\norm{Ax} = \lim\limits_{n \rightarrow \infty} \norm{A_n x} \leq C\norm{x}$. Тож $A \in L(X, Y)$.

\noindent\ref{N:1_4_28}. Достатньо довести сильну фундаментальність послідовності $\{A_n\}$. Використовуємо задачу \ref{N:1_3_21} б):

$(m \geq n) \Rightarrow (B = A_m - A_n \geq 0) \Rightarrow \left(\norm{A_m x - A_n x}^2 \leq 2C((A_m x, x) - (A_n x, x))\right)$ та
збіжність $\{(A_n x, x)\}$ за класичною теоремою Вейєрштрасса. 

\noindent\ref{N:1_5_14}. $\exists \Delta = [\alpha,\beta] \subset [a,b]$ такий, що
$(t \in [\alpha,\beta]) \Rightarrow (\inf|f(t)| > 0)$. Нехай $|f| \Big|_\Delta
\geq \delta > 0$. Тоді оператор $\tilde{A}: C(\Delta) \to C(\Delta)$, що визначений
формулою: $C(\Delta) \ni x(\cdot) \mapsto f \Big|_\Delta \cdot x(\cdot) \in C(\Delta)$
не є компактним (обміркуйте) (образ кулі містить кулю). Розглянемо оператори 
$B: C[a,b] \ni x \mapsto x \Big|_\Delta \in C(\Delta)$ та $\tilde{B}: C(\Delta) \to C[a,b]$,
що будується так: $(\tilde{B}x)(t) = x(t)$, якщо $t\in \Delta$; $(\tilde{B}x)(t) = x(\beta)$,
для $t\geq \beta$; $(\tilde{B}x)(t) = x(\alpha)$, для $t\leq \alpha$. Тоді
$B \in L(C[a,b], C(\Delta))$; $\tilde{B} \in L(C(\Delta), C[a,b])$; $\tilde{A} = B \cdot A
\cdot \tilde{B}$ і припущення про компактність $A$ приводить до суперечності.

\noindent\ref{N:1_5_15}. г) $x_n(t) = t^{\frac{1}{n}}$ --- обмежена послідовність;
$(Ax_n)(t)$ не має збіжної підпослідовності.

\noindent\ref{N:1_5_17}. б) Нехай $Z$ --- обмежена множина в $L_2[0;1]$. Тоді $Z$
складається з інтегровних на $[0;1]$ функцій і обмежена в $L_1[0;1]$ (Коші-Буняковський).
$A(Z) \subset C[0;1]$; $A(Z)$ --- рівномірно обмежені за нормою $C[0;1]$ та одностайно
неперервні: $\left|(Ax)(t_1) - (Ax)(t_2)\right| \leq \int\limits^1_0 |x| \cdot j_{[t_1,t_2]} ds \leq
\sqrt{t_2 - t_1} \left(\int\limits^1_0 x^2(s) ds\right)^{\frac{1}{2}}$, (тут $t_1 < t_2$).
Тому $A(Z)$ --- передкомпакт в $C[0;1]$. Але $\varepsilon$-сітка в $A(Z)$ за нормою $C[0;1]$ 
також є $\varepsilon$-сіткою за нормою $L_2[0;1]$ $\left(\norm{x}_{L_2[0;1]} \leq 
\norm{x}_{C[0;1]} \right)$. $A$ --- компактний.

\noindent в) $A(B[0;1]) \supset \set{y(t) \cdot j_{[\frac{1}{2};1]} \mid y \in B[0; \frac{1}{2}]}$.

\noindent ґ) $(B_1 x)(t) = \int\limits^t_0 s x(s) ds$; $(B_2 x)(t) = t x(t)$.
$B_1$ --- компактний, $B_2$ --- обмежений, $A = B_2 \circ B_1$.

\noindent\ref{N:1_5_20}. Оператор вкладення $A: C^1[a; b] \rightarrow C[a; b]$ є компактним (т. Арцела-Асколі). Тому
$A\Big|_{Z}$ --- також компактний. Тож $Z$ як підпростір в $C[a; b]$ є зліченним об'єднанням куль, передкомпактних в нормі
$C[a; b]$. Оскільки $Z$ --- повний простір, а передкомпакт в нескінченновимірному просторі ніде не щільний, прийдемо до суперечності
на підставі теореми Бера.

\noindent\ref{N:1_5_23}. Нехай $\{e_n\}$ --- ортонормований базис в сепарабельному гільбертовому просторі $H$; $P_n$ --- ортопроектор
на лінійну оболонку $\{e_1, e_2, \dots, e_n\}$. Тоді $P_n \circ A \xrightarrow{s} A$. Область значень компактного оператора --- сепарабельний
підпростір. Якщо $A_n$ --- підпослідовність компактних операторів; $A_n \xrightarrow{s} A \in L(H)$, то 
$\mathrm{Im} A \subset \overline{\bigcup\limits_{n = 1}^{\infty} \mathrm{Im} A_n}$ --- сепарабельний підпростір в $H$.
Розгляньте оператор $I$ у несепарабельному просторі.

\noindent\ref{N:1_5_26}. б) $\Rightarrow$ в). Нехай $y_n \equiv y$. Тоді
$(x_n \xrightarrow[\text{сл}]{} x) \Rightarrow (Ax_n \xrightarrow[\text{сл}]{} Ax)$.
Тепер з \ref{N:1_4_13} $A\in L(H_1, H_2)$. Тепер нехай $y_n = Ax_n$. Тоді: 
$(x_n \xrightarrow[\text{сл}]{} x, Ax_n \xrightarrow[\text{сл}]{} Ax) \Rightarrow (\norm{A x_n} \rightarrow \norm{A x})$. Далі скористаємося \ref{N:1_4_16} б).
$(Ax_n \xrightarrow[\text{сл}]{} Ax; \norm{A x_n} \rightarrow \norm{A x}) \Rightarrow (A x_n \rightarrow A x)$.

\noindent в) $\Rightarrow$ б). Скористатися \ref{N:1_4_16} а).

\noindent\ref{N:1_5_31}. Без втрати загальності покладемо $A = 0$.
$(x_n \xrightarrow[\text{сл}]{} 0) \Rightarrow \left( (C x_n, x_n) \rightarrow 0 \right)$ (дивись \ref{N:1_5_26}).
$(0 \leq (B x_n , x_n) \leq (C x_n, x_n)) \Rightarrow ((B x_n, x_n) \rightarrow 0)$. Аналогічно:
$(y_n \xrightarrow[\text{сл}]{} 0) \Rightarrow ((B y_n, y_n) \rightarrow 0)$.

\noindent$(B x_n, y_n) = \frac{1}{4}\left( (B(x_n+y_n), x_n+y_n) - (B(x_n - y_n), x_n - y_n)\right)$. 

\noindentТому $(x_n \xrightarrow[\text{сл}]{} 0, y_n \xrightarrow[\text{сл}]{} 0) \Rightarrow ((B x_n, y_n) \rightarrow 0)$. Далі скористатися \ref{N:1_5_26}.

\noindent\ref{N:1_5_33}.
%\begin{minipage}{0.25\textwidth}
%    \begin{tikzpicture}[line cap=round,line join=round,>=triangle 45,x=1.0cm,y=1.0cm]
%        \clip(-2.,-2.) rectangle (2.,2.);
%        \draw [line width=.25pt] (0.,0.) circle [radius = 1.cm];
%        \draw [->,line width=.2pt] (0.,0.) -- (0.705637350341705,0.7085731647492289);
%        \draw [->,line width=.2pt] (0.705637350341705,0.7085731647492289) -- (0.24804622428767287,1.3468866592154136);
%        \begin{scriptsize}
%        \draw[color=black] (0.40565960486893554,0.27) node {$x$};
%       \draw[color=black] (0.5490845250768417,1.101728558923356) node {$z$};
%        \end{scriptsize}
%    \end{tikzpicture}
%\end{minipage}
%\begin{minipage}{0.7\textwidth}
    $y(t)$ --- крива на сфері $\{x \mid \norm{x} = 1\}; y(t_0) = x$. $\frac{d}{dt} \norm{A y(t)}^2 \Big|_{t = t_0} = 0$
    (в т. $t_0$ функція $\norm{A y(t)}$ приймає найбільше значення). Тож $\dotprod{A \dot{y} (t_0)}{A x} = 0$.
    $\dot{y} (t_0)$ --- дотичний вектор до сфери в т. $x$. Тож $\dot{y} (t_0) \perp x$. 
%\end{minipage}
Він переходить у вектор $A \dot{y} (t_0) \in \{Ax\}^{\perp}$. Залишилося зауважити, що кожний вектор $z$ з $\{x\}^{\perp}$ є дотичним вектором
гладкої кривої (кола), що є перетином сфери з л.о. $\{x, z\}$. При розв'язанні задачі використана лише обмеженість оператора $A$.

\noindent \ref{N:1_5_44} a) $\{a_n\}$ задовольняє умову: $\exists \{b_n\}, 
\{c_n\}$ такі, що $\sum\limits_{n=1}^{\infty}|b_n|^2 < \infty$, 
$\sum\limits_{n=1}^{\infty}|c_n|^2 < \infty$ і при цьому $a_n = b_nc_n$.
Доведіть, що ця умова еквівалентна збіжності ряду $\sum\limits_{n=1}^\infty 
|a_n|$.

\noindent\ref{N:1_6_22}. б) $X = L_2 [0; 1]$. Нехай $B: \int\limits_0^t x(s) ds \mapsto x(t)$. Доведіть необмеженість
$B$ за допомогою послідовності $x_n(t) = t^n$.

\noindent\ref{N:1_6_25}. Нехай $\vec{x}$ задовольняє рівняння: $(A+B) \vec{x} = \vec{e_1}$.
Тоді $B \vec{x} = (1, -x_1, -x_2, \dots)$; $\norm{B} > 1$.

\noindent\ref{N:1_6_28}. З умов задачі: $B$ --- неперервна лінійна бієкція $Y$ на $Z$. Тому
$B^{-1} \in L(Z, Y)$. $A = B^{-1} \circ (BA)$.

\noindent\ref{N:1_6_36}. Скористаємося результатом задачі \ref{N:1_6_6}. Позначимо: $C_n = A_n - A$.
Тоді $A_n^{-1} = (I+A^{-1}C_n)^{-1} A^{-1}$, якщо $\norm{C_n} \leq \frac{1}{2} \norm{A^{-1}}^{-1}$, і при цьому має місце
оцінка: $\norm{A_n^{-1}} < 2\norm{A^{-1}}$.
Достатність умови. Якщо $A_n$ --- неперервно оборотний, то $(\norm{C_n}<\frac{1}{M}) \Rightarrow (\norm{C_n} < \norm{A_n^{-1}}^{-1}) \Rightarrow (A$
 --- неперервно оборотний) (тут $M = \sup\limits_{n \geq N} \norm{A_n^{-1}}$).

\noindent\ref{N:1_7_9}. Оператор $A$ компактний. Якщо $\dim H \geq 2$, то $\mathrm{Ker} A = \{0\}$ і $\sigma (A) = \sigma_T (A)$.
 Власні вектори $x$ --- або ортогональні $x_0$ або колінеарні $x_1$. Тому 
 $(\dotprod{x_0}{x_1} = 0) \Rightarrow (\sigma (A) = \{0\})$; 
 $(\dotprod{x_0}{x_1} \neq 0) \Rightarrow (\sigma(A) = \{0; \dotprod{x_1}{x_0}\})$. $r(A) = |\dotprod{x_1}{x_0}|$. 
 Резольвенту $R_{\lambda} (A)$ при $\lambda \notin \{0; (x_1, x_02)\}$ шукаємо як розв'язок рівняння: $\lambda y - \dotprod{y}{x_0} x_1 = x$.
 Покладемо: $x = x' + x''$, де $x' \in$ л. о. $\{x_1\}$; $x'' \perp x_1$; $y = \alpha x_1 + y''$, де $y'' \perp x_{1}$;
 $x'' = x - \dotprod{x}{x_{1}}x_{1}$. Маємо: $y'' = \frac{1}{\lambda} x''$; 
 $\lambda \alpha x_1 - \alpha \dotprod{x_1}{x_0} x_1 - \frac{1}{\lambda}\dotprod{x''}{x_0}x_1 = x' = (x, x_1)x_1$.
 $\alpha(\lambda - (x_1, x_0))x_1 = \frac{1}{\lambda}(x'', x_0)x_1 + (x, x_1)x_1$.
 Якщо $x_1 \neq 0$, то $\alpha = \frac{1}{\lambda - (x_1, x_0)}\left(\frac{1}{\lambda}(x'', x_0)+(x, x_1)\right)$.
 
 Тому $R_{\lambda} (A): x \mapsto \left(\frac{1}{\lambda(\lambda - (x_1, x_0))}(x, x_0)+\frac{\lambda - \norm{x_0}^2}{\lambda}(x, x_1)\right)x_1 + \frac{1}{\lambda}(x - (x, x_1)x_1)$.
 
\noindent\ref{N:1_7_14}. а) Якщо $\lambda \notin [0; 1]$, то рівняння $(\lambda - t)y(t) = x(t)$ має в $L_2 [0; 1]$
 (єдиний) розв'язок: $y(t) = \frac{1}{\lambda - t}x(t)$, тому що $\frac{1}{\lambda - t}$ --- обмежена функція на $[0; 1]$.
 Якщо $\lambda \in [0; 1]$, то рівняння $(\lambda - t)y(t) = \frac{1}{(\lambda - t)^{\frac{1}{3}}} \in L_2[0; 1]$ не має розв'язку в $L_2[0,1]$,
 тому що $y(t) = \frac{1}{(\lambda - t)^{\frac{4}{3}}}$ майже скрізь.
 $\{(\lambda - t)x(t) \big| x(\cdot) \in L_2 [0;1]\}$ --- щільний підпростір в $L_2 [0; 1]$, оскільки він містить всі неперервні функції
 $y \in C[0; 1]$, для яких $y(\lambda) = 0$ та існує $y'(\lambda)$. Тож:
 $\sigma_{\text{н}} (A) = \sigma (A) = [0; 1]$; $r(A) = 1$; $(R_{\lambda} (A)x)(t) = \frac{1}{\lambda - t}x(t), (\lambda \notin [0;1])$.

\noindent в) Оскільки $A$ --- компактний оператор; $\dim L_2  [0; 1] = \infty$, то $\sigma(A) \ni 0$ і всі ненульові точки $\sigma(A)$ --- 
власні числа. Для $\lambda \neq 0$ розглянемо рівняння $(\lambda x)(t) = \int\limits_0^t x(s) ds$. Оскільки $x \in L_2 [0;1]$, то $\int\limits_0^t x(s) ds$ ---
неперервна функція від $t$. Тому $x \in C[0;1]$, а отже $\int\limits_0^t x(s) ds$ --- неперервно диференційовна.
Позначимо: $z(t) = \int\limits_0^t x(s) ds$. $\lambda z' - z = 0$; $z(t) = C \mathrm{e}^{\frac{1}{\lambda}t}$. Але $z(0) = 0$. Тож $C = 0$.
$\sigma(A) = \{0\}$. $\mathrm{Im} A$ --- щільний в $L_2 [0;1]$, тому що $C[0;1] \subset \mathrm{Im}A$. $\sigma_{\text{н}} (A) = \sigma(A) = \{0\}$.
$R_{\lambda} (A)$ знаходимо з рівняння: $(\lambda y)(t) - \int\limits_0^t x(s) ds = x(t) (\lambda \neq 0)$. Спочатку робимо <<халтуру>>: нехай $x(\cdot)$ та
$y(\cdot)$ диференційовні функції. Тоді заміною $z(t) = \int\limits_0^t x(s) ds$ приводимо рівняння до такого: $\lambda z' - z = x$; 
$z(t) = C\mathrm{e}^{-\frac{1}{\lambda} t} + \frac{1}{\lambda} \int\limits_0^t \mathrm{e}^{\frac{1}{\lambda}(t-s)} x(s) ds$;
$z(0) = 0$. Тому $z(t) = \frac{1}{\lambda} \int\limits_0^t \mathrm{e}^{\frac{1}{\lambda}(t-s)} x(s) ds$; 
$y(t) = z'(t) = \frac{1}{\lambda}x(t) + \frac{1}{\lambda^2} \int\limits_0^t \mathrm{e}^{\frac{1}{\lambda}(t-s)} x(s) ds$.
Інтеграл має сенс і для $x \in L_2 [0;1]$. Перевіримо, що $y$ --- шуканий розв'язок.
$(\lambda y)(t) - \int\limits_0^t y(s) ds = x(t) + \frac{1}{\lambda} \int\limits_0^t \mathrm{e}^{\frac{1}{\lambda}(t-s)} x(s) ds - \frac{1}{\lambda} \int\limits_0^t x(s) ds - $
$\frac{1}{\lambda^2} \int\limits_0^t ds \int\limits_0^s \mathrm{e}^{\frac{1}{\lambda}(s-\tau)} x(\tau) d\tau = x(t)$, тому що:
$\int\limits_0^t ds \int\limits_0^s \mathrm{e}^{\frac{1}{\lambda}(s-\tau)} x(\tau) d\tau = \int_0^t d\tau \int_{\tau}^t \mathrm{e}^{\frac{1}{\lambda}(s - \tau)}x(\tau)ds = \lambda \int\limits_0^t (\mathrm{e}^{\frac{t - \tau}{\lambda}} - 1)x(\tau)d\tau$.
\underline{Відповідь}: $(R_{\lambda} x)(t) = \frac{1}{\lambda}x(t)+\frac{1}{\lambda^2}\int_0^t \mathrm{e}^{\frac{t-s}{\lambda}}x(s)ds$.

\noindent\ref{N:1_7_25}. Скористайтесь задачею \ref{N:1_6_6}.

\noindent\ref{N:1_7_27}. а) безпосередній наслідок результату задачі \ref{N:1_6_6}.

\noindent б) Нехай $\varepsilon > 0$ і $F = \complex \setminus (\sigma(A))_{\varepsilon}$. $F$ --- замкнена множина в $\complex$.
$\lambda \mapsto R_{\lambda} (A)$ --- неперервна функція на $F$ та $\lim\limits_{\lambda \rightarrow \infty} R_{\lambda} (A) = 0$.
Тому $\inf\limits_F \{ \norm{R_{\lambda} (A)}^{-1} \} > 0$ і $\exists N, \forall n \geq N: \rho (A_n) \subset F$.

\noindent\ref{N:1_7_28}. $0 \in \rho(A); \norm{A^{-1}} = 1$. Тому $(|\lambda| < 1) \Rightarrow (|\lambda - 0| < \norm{R(0; A)}^{-1}) \Rightarrow (\lambda \in \rho(A))$.

\noindent\ref{N:1_7_30}. б) Припустимо спочатку, що та $M > |m|$. Тоді $M = \sup\limits_{\norm{x} = 1} \dotprod{Ax}{x} = \sup\limits_{\norm{x} = 1} |\dotprod{Ax}{x}| = \norm{A} = r(A)$.
Оскільки $\sigma(A)$ --- замкнена множина в $\mathbb{R}$, то $M \in \sigma(A)$.
Всі інші варіанти розташування $m$ і $M$ в $\mathbb{R}$ паралельним перенесенням зводяться до розглянутого і при цьому
$(\lambda \in \sigma(A)) \Leftrightarrow (\lambda + c \in \sigma (cI + A))$. Перетворення $A \mapsto B = -A$ переводять $m_A$ в 
$M_B$ і $(\lambda \in \sigma(B)) \Leftrightarrow (-\lambda \in \sigma(A))$.

\noindent\ref{N:1_7_33}. $\lambda \neq 0; A A^* x = \lambda x (x \neq 0)$. Тоді
$
    \begin{cases}
        A^* A (A^* x) = \lambda A^* x
        \\A^* x \neq 0,
    \end{cases}
$, $\lambda \in \sigma(A^* A)$. 

\noindent $\lambda = 0$. Контрприклад: $A$ --- оператор правого зсуву в $\ell_2$.

\noindent\ref{N:1_8_6}. $\sigma (A) = \{0\}$; в $X = C[0; 1]: 0 \in \sigma_{\text{з}}(A)$, тому що 
$\mathrm{Ker} A = \{0\}$ та $\mathrm{Im} A \subset \{x \in X \big| x(0) = 0\}$;
в $X = L_2 [0; 1]: 0 \in \sigma_{\text{н}} (A)$, тому що $(\int\limits_0^t x(s) ds = 0 \text{ в } L_2[0; 1]) \Rightarrow$
$(\int\limits_0^t x(s) ds = 0, \forall t \in [0, 1]$, оскільки функція $\int\limits_0^t x(s) ds$ неперервна по $t) \Rightarrow$
$(\forall $ числового проміжку $\Delta \subset [0, t]: \int\limits_{\Delta} x(s)ds = 0) \Rightarrow$
$(\forall $ лебегової множини $A \subset [0, 1): \int\limits_A x(s) ds = 0) \Rightarrow$
$(x(\cdot) = \frac{d(0)}{d \lambda_1} = 0$ (м. с.)).

\noindent Інший шлях: $y(t) = \int\limits_0^t x(s) ds$ --- абсолютно неперервна, а тому $x(t) = y'(t)$ (м. с.).
Крім того, $\mathrm{Im} A$ містить всі поліноми $p (\cdot)$, для яких $p(0) = 0$, а тому $\overline{\mathrm{Im}A} = L_2 [0; 1]$
(обміркуйте!).

\noindent\ref{N:1_8_9}. а) $A$ --- компактний самоспряжений оператор в $L_2 [0; 1]$. Тому крім $0$ всі точки спектра --- власні числа.
Нехай $\lambda \neq 0$. $\lambda x(t) = (1-t)\int\limits_0^t s x(s) ds + t \int\limits_t^1 (1-s) x(s) ds$. Права частина --- неперервна
по $t$ функція. Тоді $x \in C[0;1]$. Але тоді права частина --- неперервно диференційовна функція. Тому $x \in C^1 [0; 1]$. І далі: $x \in C^2 [0; 1]$.
Двічі диференціюємо і приходимо до крайової задачі: $\lambda x'' + x = 0; x(0) = x(1) = 0$.

\noindent\ref{N:1_8_12}. Скористаємось 1-ою теоремою Фредгольма і розглянемо супутнє рівняння: $x(t) - \int\limits_0^{\pi} \sin (s-t)x(s) ds = 0$.
$x(t) = \cos t \cdot \int\limits_0^{\pi} \sin s \cdot x(s) ds - \sin t \int\limits_0^{\pi} \cos s \cdot x(s) ds$. $x(t) = A \cos t + B \sin t$.
Одержимо: $A = B = 0$. Тому вихідне рівняння має розв'язок при кожному $f \in L_2 [0; \pi]$.

\noindent\ref{N:2_1_10}. Для перевірки необмеженості $A$ розгляньте послідовність $x_n (t) = \sin (nt)$.
Якщо $x_n \rightrightarrows x$; $y_n = A x_n \rightrightarrows y$ на $[0; 1]$, то $x_n (t) = t y_n(t) \rightrightarrows t y(t)$
на $[0; 1]$; $\lim\limits_{t \rightarrow 0+} \frac{1}{t} x(t) = y(0)$; $(t \in (0; 1]) \Rightarrow (y(t) = \frac{1}{t} x(t))$.

\noindent\ref{N:2_1_27}. $H = L_2 [0; 1]$. $(A u)(t) = j_{\Delta} \cdot u'(t)$; 
$D(A) = \{u$ --- поліном; $u(0) = 0\}$; $\Delta$ --- числовий відрізок $[a; b] \subset [0; 1]$, $[a; b] \neq [0; 1]$ ; $j_{\Delta}$ --- індикатор $\Delta$.

\noindent\ref{N:2_2_20} \ul{а) $\Rightarrow$ б)} Скористатись \ref{N:2_2_8} і 
довести: $\exists \lim\limits_{t\rightarrow 0+0}\frac{1}{t}\norm{T(t) - I}$.

\noindent \ul{а) $\Rightarrow$ в)} Скористатись \ref{N:2_2_8}. 
$\norm{\lambda A \inv{(\lambda - A)}} = \norm{\sum\limits_{n=0}^\infty 
(\frac{1}{\lambda}A)^{n+1}} \leq \frac{\lambda \norm{A}}{\lambda - \norm{A}}$.

\noindent \ul{в) $\Rightarrow$ а)} Оператори $A_\lambda = \lambda A \inv{(\lambda - A)} = 
\lambda^2 \inv{(\lambda - A)} - \lambda I \in L(X)$ рівномірно обмежені за 
нормою на $(\omega, +\infty)$; для $x \in D(A)$: $A_\lambda x \rightarrow 
Ax$, $\lambda \rightarrow +\infty$. Тому $\forall x \in X$ $\exists Bx 
= \lim\limits_{\lambda \rightarrow +\infty} A_\lambda x$. 
Тож $(A \subset B;$ $B \in L(X))$ $\Rightarrow$ ($A = B$).

\noindent\ref{N:2_2_21} \ul{a) $\Rightarrow$ б)} $\left(f, g \in D(A)\right)$ 
$\Rightarrow$ $\left(\exists \left.\frac{d}{ds}\right|_{s=0}T(s)f, 
\left.\frac{d}{ds}\right|_{s=0}T(s)g\right)$ $\Rightarrow$ 
$(\exists \left.\frac{d}{ds}\right|_{s=0}T(s)(f\cdot g) = Af \cdot g + f \cdot Ag)$

\ul{б) $\Rightarrow$ а)} $(f, g\in D(A))$ $\Rightarrow$ $\frac{d}{dt}(T(t)f \cdot T(t)g) = 
T(t)f AT(t)g + AT(t)f\cdot T(t)g =$ $=A(T(t)f\cdot T(t)g)$.
З корректності відповідної задачі Коші робимо висновок:
$T(t)f \cdot T(t)g = T(t)(f \cdot g)$.

Для довільних $f, g \in X$: $\exists f_n, g_n \in D(A)$ такі, що $f_n \rightrightarrows f$, 
$g_n \rightrightarrows g$. Тоді $f_n g_n \rightrightarrows fg$ і застосувати граничний перехід.

\end{document}