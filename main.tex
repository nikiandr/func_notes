\documentclass[12pt]{report}

%%% Работа с языком
%\usepackage{cmap}					    % поиск в PDF				    % русские буквы в формулах
\usepackage[T2A]{fontenc}			% кодировка
\usepackage[utf8]{inputenc}             % кодировка исходного текста
\usepackage[english,ukrainian]{babel}	% локализация и переносы	    
\usepackage{indentfirst}
\usepackage{mdframed}

\usepackage[a4paper, top=25mm, bottom=25mm, left=30mm, right=30mm]{geometry}

\usepackage{amsmath,amsfonts,amssymb,amsthm,mathtools} % AMS
\usepackage{dsfont} %для цифр в шрифте типа mathbb
\usepackage{braket} % для команды \set

\usepackage{makecell}% Прикольная настройка ячеек в tabular 
%http://ctan.math.utah.edu/ctan/tex-archive/macros/latex/contrib/makecell/makecell-rus.pdf

\usepackage{diagbox}
% Разделение клеток tabular по диагонали

\usepackage{multicol} % Несколько колонок
\usepackage{wrapfig} % Картинки посреди текста
\usepackage{chngcntr} % для нумерации формул
\usepackage{enumitem} % чтоб можно было делать римскую нумерацию
\usepackage{mdwlist}
\usepackage{adjustbox}
\usepackage{xcolor} % градации цветов
\usepackage[psdextra]{hyperref}
\hypersetup{unicode=true}
\hypersetup{
    colorlinks=true,
    linkcolor=blue,
}
\setlist[enumerate]{nosep}

%%%% Теоремы, определения и т.д. и т.п.
\theoremstyle{plain} % Это стиль по умолчанию, его можно не переопределять.
\newtheorem{theorem}{Теорема}
\newtheorem*{theorem*}{Теорема}
\newtheorem{proposition}{Твердження}
%\renewcommand{\qedsymbol}{$\blacktriangle$}
 
% \theoremstyle{definition} % "Определение"
% \newtheorem{definition}{Означення}[section]
% \newtheorem*{definition*}{Означення}
% \newtheorem*{example}{Приклад}

% \theoremstyle{remark}
% \newtheorem*{remark}{Зауваження}
\theoremstyle{remark}
\newtheorem{exercise}{}[section]
\newenvironment{exercise*}
  {\renewcommand\theexercise{\thesection.\arabic{exercise}\rlap{$^*$}}%
   \exercise\edef\@currentlabel{\thesection.\arabic{exercise}}}
  {\endexercise}

%\counterwithout{equation}{chapter} % нумерация формул
%\counterwithin*{equation}{section}

%%%% Картинки
\usepackage{graphicx}
\usepackage{tikz}
\usepackage{pgfplots}
\pgfplotsset{compat=1.16}

\usetikzlibrary{patterns}
\usetikzlibrary{shapes.geometric}
\usetikzlibrary{arrows.meta}
\usetikzlibrary{shapes.geometric}
\graphicspath{{pictures/}}
\DeclareGraphicsExtensions{.pdf,.png,.jpg}

\newmdenv[
  topline=false,
  bottomline=false,
  rightline=false,
  skipabove=\topsep,
  skipbelow=\topsep,
  linecolor=black,
  outerlinecolor=black,
  innerlinecolor=black
]{theory}


\newcommand{\norm}[1]{\left\lVert#1\right\rVert} % command for norm
\newcommand{\dotprod}[2]{\left(#1, #2\right)} % command for scalar
\DeclareMathOperator{\sgn}{sgn} %

\makeatletter
\def\ukr#1{\expandafter\@ukr\csname c@#1\endcsname}
\def\@ukr#1{\ifcase#1\or а\or б\or в\or
г\or ґ\or д\or е\or є\or ж\or з\or и\or і\or ї\or й\or к\or л\or м\or н
\or о\or п\or р\or с\or т\or у\or ф\or х\or ц\or ч\or ш\or щ\fi}
\makeatother
\AddEnumerateCounter{\ukr}{\@ukr}{Українська}

\author{\Huge Богданський Ю.В.}
\title{\textbf{
    \fontsize{40}{48}\selectfont{Задачі з функціонального аналізу}}
    \\
    \vspace{0.5em}
    \fontsize{16}{20}\selectfontОбмежені лінійні оператори в нормованих просторах.
    Необмежені лінійні оператори. Операторні півгрупи.
    Простори Соболєва та узагальнені функції.
    }
\date{}

\begin{document}
    \maketitle
    \tableofcontents
    \chapter{Обмежені лінійні оператори в нормованих просторах}
        \section{Основні положення. Властивості.}
            % !TEX root = ../main.tex

\begin{theory}
    Нехай $(X, \norm{\cdot}_X)$, $(Y, \norm{\cdot}_Y)$ - нормовані простори над полем K ($K = 
    \mathbb{R}$ або $\mathbb{C}$, 
    але обов'язково однакове для обох просторів). 

    Лінійний оператор $A: X \rightarrow Y$ (визначений на всьому X) називається 
    \emph{обмеженим}, якщо існує число $C > 0$, для якого $\forall x \in X$: 
    $\norm{Ax}_Y \leq C \norm{x}_X$. Надалі найчастіше нижні індекси при позначенні норми 
    не будемо ставити; за змістом формул зрозуміло, до якого простору належить відповідна 
    норма. Також при позначенні нормованого простору $(X, \norm{\cdot})$ будемо писати лише 
    літеру X, якщо за змістом задачі норма в X не викликає сумніву. 

    В разі якщо Y є основне поле ($Y=\mathbb{R}$ або $Y = \mathbb{C}$), то лінійний оператор 
    $\varphi: X \rightarrow Y$ прийнято називати \emph{(лінійним) функціоналом}.

    Норма лінійного обмеженого оператора (функціонала) задається формулою: 
    $\norm{A} = \inf\left\{C | \forall x \in X : \norm{Ax}\leq C\norm{x} \right\}$
    ($\norm{\varphi} = \inf\left\{C | \forall x \in X : |\varphi(x)|\leq C\norm{x} \right\}$)
\end{theory}

\begin{exercise}
    Нехай A - обмежений лінійний оператор з X в Y ($A: X \rightarrow Y$).
    Довести: $\norm{A} = \min\left\{C|\forall x \in X : \norm{Ax} \leq C\norm{x}\right\}$.
\end{exercise}

\begin{exercise}
    Нехай $A: X \rightarrow Y$ - обмежений лінійний оператор.
    Довести: $\norm{A} = \sup\limits_{x \neq 0}\frac{\norm{Ax}}{\norm{x}} = 
    \sup\limits_{\norm{x} \leq 1}\norm{Ax} = \sup\limits_{\norm{x} = 1}\norm{Ax}$.
\end{exercise}

\begin{theory}
    Лінійний оператор $A: X \rightarrow Y$ називається \emph{неперервним}, якщо відображення 
    A є неперервним в кожній точці $x \in X$.
\end{theory}

\begin{exercise}
    Нехай А - лінійний оператор з X в Y. Довести: 
    \begin{enumerate}[label=\alph*)]
        \item якщо A - обмежений оператор, то A - неперервний
        \item якщо існує точка $x_0 \in X$, в якій А - неперервний, то А - обмежений
    \end{enumerate}
\end{exercise}

\begin{theory}
    \underline{\emph{Висновок}} Для перевірки неперервності лінійного оператору достатньо 
    довести його неперервність лише в одній точці простору аргументу.
\end{theory}

\begin{exercise}
    З'ясувати, чи є наведені нижче функціонали в просторі $C{\left[-1, 1\right]}$ лінійними; 
    неперервними. В разі позитивної відповіді знайти їх норми. 
    \begin{enumerate}[label=\ukr*)]
        \item $\varphi(x) = x(0)$
        \item $\varphi(x) = \int\limits_0^1x(t)dt$
        \item $\varphi(x) = \frac{1}{2}(x(1) - x(-1))$
        \item $\varphi(x) = \int\limits_0^1tx(t)dt$
        \item $\varphi(x) = \int\limits_{-1}^1tx(t)dt$
        \item $\varphi(x) = \norm{x} = \max\limits_{-1 \leq t \leq 1} |x(t)|$
        \item $\varphi(x) = \int\limits_0^1|x(t)|dt$
        \item $\varphi(x) = \int\limits_{-1}^1x(t)cos(\pi t)dt$
        \item $\varphi(x) = \int\limits_0^1tx^2(t)dt$
        \item $\varphi(x) = \int\limits_{-1}^0x(t)dt - \int\limits_0^1x(t)dt$
        \item $\varphi(x) = \int\limits_{-1}^1x(t^2)dt$
    \end{enumerate}
\end{exercise}

\begin{exercise}
    З'ясувати, чи є наведені нижче функціонали на X лінійними; 
    неперервними. В разі позитивної відповіді знайти їх норми.
    \begin{enumerate}[label=\ukr*)]
        \item $X = \ell_1$; $\varphi(\vec{x}) = \sum\limits_{n=1}^{\infty} x_n $
        \item $X = \ell_1$; $\varphi(\vec{x}) = \sum\limits_{n=1}^{\infty} |x_n| $
        \item $X = \ell_2$; $\varphi(\vec{x}) = x_1 + x_2 $
        \item $X = \ell_2$; $\varphi(\vec{x}) = x_1 - x_2 + x_3$
        \item $X = \ell_2$; $\varphi(\vec{x}) = \sum\limits_{n=1}^{\infty} \frac{x_n}{n}$
        \item $X = \ell_p$ ($1 \leq p \leq +\infty$); $\varphi(\vec{x}) = 
        \sum\limits_{n=1}^{\infty} \frac{x_n}{n}$
        \item $X = L_2[0,1]$; $\varphi(x) = \int\limits_0^1 tx(t)dt$
        \item $X = L_2[0,1]$; $\varphi(x) = \int\limits_0^1 |x(t)|^{\frac{1}{2}}dt$
        \item $X = L_2[0,\frac{\pi}{2}]$; 
        $\varphi(x) = \int\limits_0^{\frac{\pi}{2}} x(t)sin(t)dt$
        \item $X = C^1[0,1]$ $(\norm{x}_1 = \max\limits_{[0,1]}|x(t)| + 
        \max\limits_{[0,1]}|x^{\prime}(t)|)$; $\varphi(x) = x^{\prime}(0) + x(1)$
    \end{enumerate}
\end{exercise}

\begin{exercise}
    Лінійний функціонал $\varphi$ визначено на просторі $C[0, 1]$ формулою 
    $\varphi(x) = \int\limits_0^1 x(t)dt - x(0)$. Довести: $\norm{\varphi} = 2$, 
    але не існує такої функції $x_0 \in C[0, 1]$, що $\norm{x_0} = 1$, $|\varphi(x_0)| = 2$
\end{exercise}

\begin{exercise}
    Нехай $X = C^1[0, 1]$ з нормою $\norm{x} = \max\limits_{[0, 1]}|x(t)|$, 
    $\varphi(x) = x^\prime(0)$. Доведіть необмеженість функціонала $\varphi$.
\end{exercise}

\begin{exercise}
    Нехай $f$, $g$ - лінійні функціонали на лінійному просторі $L$; $\ker f = \ker g$. 
    Довести: існує $\alpha \in K$, для якого $f = \alpha g$.
\end{exercise}

\begin{exercise}
    Нехай $f$, $f_1$, ..., $f_n$ - лінійні функціонали на лінійному просторі $L$. 
    Довести, що $f$ є лінійною комбінацією $f_1$, ..., $f_n$ тоді й тільки тоді, 
    коли $\bigcap\limits_{k=1}^n \ker f_k \subset \ker f$.
\end{exercise}

\begin{exercise}
    Довести, що будь-який оператор в нормованих просторах із скінченновимірною областю 
    визначення $X$ є неперервним і при цьому існує такий ненульовий вектор $x_0 \in X$, 
    для якого $\norm{Ax_0} = \norm{A}\norm{x_0}$.
\end{exercise}

\begin{exercise}
    Довести, що лінійний неперервний оператор $A: X \rightarrow Y$ залишається неперервним, 
    якщо в $X$ та $Y$ замінити норми на еквівалентні.
\end{exercise}

\begin{exercise}
    З'ясувати, чи є наведені нижче функціонали на X лінійними; 
    неперервними. В разі позитивної відповіді знайти їх норми.
    \begin{enumerate}[label=\ukr*)]
        \item $A: C[0, 1] \rightarrow C[0, 1]$; $(Ax)(t) = \int\limits_0^t x(\tau) d\tau$
        \item $A: C[0, 1] \rightarrow C[0, 1]$; $(Ax)(t) = \int\limits_0^t \tau x(\tau) d\tau$
        \item $A: C[0, 1] \rightarrow C[0, 1]$; $(Ax)(t) = \int\limits_0^t \tau x^2(\tau) d\tau$
        \item $A: C[0, 1] \rightarrow C[0, 1]$; $(Ax)(t) = x(t^2)$
        \item $A: C^1[0, 1] \rightarrow C[0, 1]$; $(Ax)(t) = x^\prime (t)$ 
        \item $A: \ell_2 \rightarrow \ell_2$; $A\vec{x} = (\underbrace{0,0,...,0}_n,
        x_1,x_2,...)$
        \item $A: \ell_2 \rightarrow \ell_2$; $A\vec{x} = (x_2,x_3,x_4,...)$
        \item $A: \ell_2 \rightarrow \ell_2$; $A\vec{x} = (x_1,0,x_3,0,x_5,...)$
        \item $A: \ell_2 \rightarrow \ell_1$; $A\vec{x} = \vec{x}$
        \item $A: \ell_1 \rightarrow \ell_1$; $A\vec{x} = (x_1+x_2, x_1-x_2, x_3, x_4, ...)$
        \item $A: L_2[0, 1] \rightarrow L_2[0, 1]$; $(Ax)(t) = t \int\limits_0^1 x(s)ds$
        \item $A: L_2[0, 1] \rightarrow L_2[0, 1]$; $(Ax)(t) = \varphi(t)x(t)$ 
        ($\varphi \in C[0,1]$)
        \item $A: L_2[0, 1] \rightarrow L_2[0, 1]$; $(Ax)(t) = \int\limits_0^1 tsx(s)ds$
    \end{enumerate}
\end{exercise}
            % !TEX root = ../main.tex
\begin{exercise}
    Нехай $\left\{c_n\right\}$ --- числова послідовність. Довести, що оператор
    $A : \ell_2 \ni (x_1, x_2, ...) = \vec{x} \mapsto \vec{y} = (c_1x_1, c_2x_2, ...) \in \ell_2$
    є обмеженим (та визначеним на всьому $\ell_2$) тоді й тільки тоді, коли $c = \underset{n\in\mathbb{N}}{\sup} |c_n| < +\infty$, 
    і при цьому $\norm{A} = c$.
\end{exercise}

\begin{exercise}
    Нехай $K \in C\left( [a;b] \times [a;b]\right)$ і оператор $A: C\left( [a;b]\right) \rightarrow C\left( [a;b]\right)$
    визначено формулою $(Ax)(t) = \int\limits_a^b K(t, s) x(s) ds$. Довести лінійність і обмеженість оператора $A$.
\end{exercise}

\begin{exercise}
    Довести, що ядро лінійного неперервного оператора замкнене. Чи завжди замкнена його область значень?
\end{exercise}

\begin{exercise}
    Нехай $A: X \rightarrow Y$ --- обмежений лінійний оператор, $Z$ --- передкомпакт в $X$.
    Довести $A(Z)$ --- передкомпакт в $Y$. Чи завжди образ компакта під дією обмеженого оператора буде компактом?
\end{exercise}

\begin{exercise}
    Нехай $\varphi$ --- лінійний функціонал на нормованому просторі $X$. 
    Довести, що $\varphi$ --- обмежений тоді й тільки тоді, коли його ядро замкнене. 
\end{exercise}

\begin{exercise}
    Нехай $A$ --- лінійний оператор з $X$ в $Y$, $\mathrm{dim} Y < \infty$.
    Доведіть, що $A$ --- обмежений тоді й тільки тоді, коли його ядро замкнене.
\end{exercise}

\begin{exercise}
    Нехай $A$ --- лінійний оператор з $X$ в $Y$. Чи завжди з умови замкненості ядра $A$ випливає його обмеженість?
\end{exercise}

\begin{exercise}
    Нехай $\varphi$ --- лінійний функціонал на нормованому просторі. Довести $\varphi$ --- неперервний тоді й тільки тоді, коли
    для будь-якої послідовності $x_n \rightarrow 0$ числова послідовність $\left\{\varphi(x_n)\right\}$ --- обмежена.
\end{exercise}

\begin{exercise}
    Нехай $\varphi$ --- ненульовий лінійний функціонал на нормованому просторі $X$. Довести, що для будь-якого $x \in X$ виконується рівність
    $\rho \left( x, \mathrm{Ker}\varphi\right) = \inf \left\{\norm{x - y} \; | \; y \in \mathrm{Ker} \varphi\right\} = \frac{|\varphi(x)|}{\norm{\varphi}}$.
\end{exercise}

\begin{exercise}
    Нехай $\varphi$ --- ненульовий лінійний функціонал на нормованому просторі $X$, $a \in K$ ($\mathbb{R}$ або $\mathbb{C}$).
    Позначимо $L = \left\{x\in X \; | \; \varphi(x) = a\right\}$. Довести $\rho(0, L) = \frac{|a|}{\norm{\varphi}}$.
\end{exercise}

\begin{theory}
    Обмежені лінійні оператори з нормованого простору $X$ в нормований простір $Y$ утворюють лінійний простір за поточковими операціями:
    $(A+B)x = Ax + Bx$, $(\lambda\cdot A)x = \lambda \cdot Ax$. Цей простір є нормований із стандартною операторною нормою.
    Найчастіше він позначається так: $L\left( X, Y\right)$ або $\left\{X \rightarrow Y\right\}$. В разі, якщо $X = Y$, застосовується
    скорочене позначення $L(X)$. Якщо $Y = K$ (основне поле), то $L\left( X, K\right)$ позначається $X^*$ і називається 
    \underline{простором, спряженим} до $X$.
\end{theory}

\begin{exercise}
    Довести, що в разі, якщо простір $Y$ є повним, простір $L\left( X, Y\right)$ також є повним.
    Зокрема, для кожного нормованого простору $X$ спряжений простір $X^*$ є банаховим.
\end{exercise}

\begin{exercise}
    Нехай $X$, $Y$, $Z$ --- нормовані простори. Довести:
    \begin{enumerate}[label=\ukr*)]
        \item $\left( A \in L\left( X, Y\right); B \in L\left( Y, Z\right)\right) \Rightarrow \left( B \circ A \in L\left( X, Z\right); \norm{B\circ A} \leq \norm{B} \cdot \norm{A}\right)$;
        \item $\left( A \in L(X), n\in\mathbb{N}\right) \Rightarrow \left( A^n \in L(X) ; \norm{A^n} \leq \norm{A}^n\right)$;
        \item $\left( X \text{ --- банахів}; A_n \in L(X), n \in \mathbb{N} ; \text{ ряд } \sum\limits_{n=1}^{\infty} \norm{A_n} \text{ --- збіжний}\right) \Rightarrow$ 

        $\Rightarrow \left( \exists A \in L(X): \norm{A - \sum\limits_{k=1}^n A_k} \rightarrow 0, n\rightarrow \infty \right)$.
    \end{enumerate}
\end{exercise}

\begin{exercise}
    Нехай $X$ --- банахів простір, $A \in L(X)$. $e^A = \exp{A} := I + \sum\limits_{n=1}^{\infty} \frac{1}{n!}A^n$, $I$ --- тотожний оператор.
    Довести: існування, $\exp{A} \in L(X)$, $\norm{e^A} \leq e^{\norm{A}}$.
\end{exercise}

\begin{exercise}
    Нехай $X$ --- банахів простір, $A\in L(X)$. Довести, що ряд $\sum_{k=0}^{\infty} A^k$ ($A^0 := I$)
    збігається тоді й тільки тоді, коли існує натуральне число $n$, для якого виконується нерівність $\norm{A^n} < 1$.
\end{exercise}

\begin{exercise}
    Знайти образ та ядро оператора $A \in L\left( X, Y\right)$, визначеного формулою $Ax = \varphi(x) y$,
    де $\varphi \in X^*$ --- фіксований обмежений лінійний функціонал на $X$, $y$ --- фіксований вектор з $Y$.
    Знайти норму $\norm{A}$.
\end{exercise}

\begin{theory}
    \underline{Рангом} оператора $A$ називається число $\mathrm{rank}A := \mathrm{dim} \mathrm{Im}A$.
    В разі, якщо $\mathrm{rank} A < \infty$, оператор $A$ називається
    \underline{оператором скінченного рангу} або \underline{скінченновимірним оператором}.
\end{theory}

\begin{exercise}
    Нехай $A \in L\left( X, Y\right)$. Довести: $A$ --- оператор скінченного рангу тоді й тільки тоді, коли він допускає представлення
    $Ax = \sum\limits_{k=1}^n \varphi_k(x) y_k$, де $\varphi_k \in X^*$, $y_k \in Y$.
\end{exercise}

\begin{theory}
    Оператори $A, B: \ell_2 \rightarrow \ell_2$, $A : \vec{x} \mapsto (0, x_1, x_2, ...)$, $B : \vec{x} \mapsto (x_2, x_3, x_4, ...)$
    називаються відповідно операторами \underline{правого} та \underline{лівого зсуву}.
\end{theory}

\begin{exercise}
    Знайти норми $\norm{A}$ та $\norm{B}$ операторів зсуву.
\end{exercise}

\begin{theory}
    Нормовані простори $(X, \norm{\cdot}_X)$, $(Y, \norm{\cdot}_Y)$ називаються 
    \underline{ізоморфними}, якщо існує лінійний оператор $A: X \rightarrow Y$, для якого $\mathrm{Ker}A = {0}$,
    $\mathrm{Im}A = Y$ і для кожного $x \in X$ має місце $\norm{Ax}_Y = \norm{x}_X$. Такий оператор
    називається \underline{ізоморфізмом}. Позначення: $X \cong Y$.
\end{theory}

\begin{exercise}
    Довести наступні ізоморфізми:
    \begin{enumerate}[label=\ukr*)]
        \item $\ell_1^* \cong \ell_\infty$;
        \item $c_0^* \cong \ell_1$;
        \item $\left( 1<p<\infty\right) \Rightarrow \left( \ell_p^* \cong \ell_q, \text{ де } \frac{1}{p} + \frac{1}{q} = 1\right)$.
    \end{enumerate}
    Тут $c_0 = \left\{ \vec{x} = (x_1, x_2, ...) \; | \; \underset{n\rightarrow\infty}{\lim} x_n = 0\right\}$ 
    з нормою $\norm{\vec{x}} = \sup\left\{|x_n| \; | \; n \in \mathbb{N}\right\}$.
\end{exercise}

\begin{exercise*}
    Нехай $X$ --- нормований простір, $X^*$ --- сепарабельний простір. Довести сепарабельність простору $X$.
    Чи можна стверджувати, що спряжений простір до сепарабельного простору $X$ завжди є сепарабельним?
\end{exercise*}
            % !TEX root = ../main.tex

\begin{exercise}
    Лінійний функціонал $\varphi: C[a; b] \rightarrow \mathbb{R}$ називається \underline{невід'ємним},
    якщо для кожної невід'ємної на $[a; b]$ функції $x \in C[a, b]$ має місце нерівність 
    $\varphi(x) \geq 0$. Довести, що невід'ємний функціонал $\varphi$ є неперервним і при цьому
    $\norm{\varphi} = \varphi(\mathds{1})$, де $\mathds{1}$ --- тотожно одинична функція.
\end{exercise}

\begin{exercise}
    Довести, що лінійний оператор $A$, який діє в нормованому просторі $X$, є неперервним тоді
    й тільки тоді, коли множина $\set{x \in X \mid \norm{Ax} < 1}$ має внутрішні точки.
\end{exercise}

\begin{exercise}
    Нехай $X$ --- банахів простір; $A \in L(X)$ та існує $c > 0$ таке, що для кожного $x \in X$
    виконується нерівність $\norm{Ax} \geq c \norm{x}$. Довести, що $\mathrm{Im} A$ є замкненим підпростором.
\end{exercise}

\begin{exercise}
    Довести, що після заміни норм в нормованих просторах $X$ і $Y$ на еквівалентні,
    нова норма в $L\left( X, Y\right)$ буде еквівалентна старій.
\end{exercise}

\begin{exercise}
    Нехай $X$ --- комплексний нормований простір, $\varphi \in X^*$. Довести $\norm{\varphi} = \underset{\norm{x}=1}{\sup}\mathrm{Re}\varphi(x)$.
\end{exercise}

\begin{exercise}
    Нехай лінійний функціонал, визначений на нормованому просторі  $X$ --- необмежений.
    Довести, що в будь-якому околі нуля він приймає всі дійсні значення.
\end{exercise}

\begin{exercise}
    Нехай $X,Y$ --- нормовані простори, $Z$ --- замкнений підпростір в $X$. Покладемо 
    $M = \set{A \in L(X,Y) \mid \mathrm{Ker} A = Z}$. Чи буде $M$ замкненим підпростором в $L\left( X, Y\right)$?
\end{exercise}

\begin{exercise}
    Нехай $X,Y$ --- нормовані простори; $Z$ --- замкнений підпростір в $X$. Покладемо 
    $M = \set{A \in L(X,Y) \mid \mathrm{Ker} A \supset Z}$. Чи буде $M$ замкненим підпростором в $L\left( X, Y\right)$?
\end{exercise}

\begin{exercise}
    Нехай $X$ --- нормований простір; $A \in L(X)$. Покладемо $M = \{B \in L(X) \;|\; AB=0\}$;
    $N = \set{B \in L(X) \mid AB=BA}$. Довести, що $M$ та $N$ --- замкнені підпростори в $L(X)$.
\end{exercise}

\begin{theory}
    \begin{theorem*}[Ган, Банах]
        Для кожного ненульового вектора $x$ нормованого простору $X$ існує $\varphi \in X^*$,
        для якого $\norm{\varphi} = 1$ та $\varphi(x) = \norm{x}$. 
    \end{theorem*}
\end{theory}

\begin{exercise} 
    \begin{enumerate}[label=\ukr*)]
        \item Нехай $X$ --- нормований простір; $x,y \in X$. Довести $(x=y) \Leftrightarrow 
        \big(\forall\varphi \in X^*: \varphi(x) = \varphi(y)\big)$;
        \item Нехай $X$ --- сепарабельний нормований простір. Довести існування зліченної сім'ї
        функціоналів $\varphi_n \in X^*$, що розділяють точки $X$, тобто 
        $(x\neq y) \Leftrightarrow \big(\exists n \in \mathbb{N}: \varphi_n(x) \neq \varphi_n(y)\big)$.
    \end{enumerate}
\end{exercise}

\begin{theory}
    Кожен вектор $x$ нормованого простору $X$ можна розглядати як функцію $\tilde{x}$, яка 
    діє на $X^*$ за правилом $\tilde{x}(\varphi)=\varphi(x)$.
\end{theory}

\begin{exercise}
    Доведіть:~
    \begin{enumerate}[label=\ukr*)]
        \item Для будь-якого $x \in X$ функція $\tilde{x}$ є лінійним обмеженим функціонал на просторі $X^*$
        і його норма, як елемента $X^{**}$, співпадає з нормою $x \in X$;
        \item відображення $\alpha: X \ni x \mapsto \tilde{x} \in X^{**}$ є ізометричним вкладенням
        простору $X$ в $X^{**}$, тобто $\alpha$ --- лінійний оператор; $\mathrm{Ker}\alpha = \{0\}$; 
        $\norm{\alpha(x)} = \norm{\tilde{x}} = \norm{x}$.
    \end{enumerate}
\end{exercise}

\begin{theory}
    В разі, якщо $\mathrm{Im}\alpha = X^{**}$, тобто $\alpha$ --- ізоморфізм просторів $X$ та $X^{**}$,
    вихідний простір $X$ називається \underline{рефлексивним}.
\end{theory}

\begin{exercise}
    Доведіть, що простори  $\ell_p$, де $1<p<\infty$, є рефлексивними, а простори $\ell_1$ та $\ell_\infty$ --- ні.
\end{exercise}

\begin{exercise}
    Чи буде простір $L(X)$, де $X=L_2[0;1]$, сепарабельним?
\end{exercise}

\begin{exercise}
    Нехай $X$, $Y$ --- нормовані простори; $A \in L\left( X, Y\right)$; $B, C \in L\left( Y, X\right)$;
    $AB = I_Y$; $CA = I_X$ ($I_X$ --- тотожній оператор в $X$).
    Довести $B = C = A^{-1} \in L\left( Y, X\right)$ --- обернений оператор до $A$.
\end{exercise}

\begin{theory}
    \begin{theorem*}[Банах, Штейнгауз; принцип рівномірної обмеженості]
        Нехай $X, Y$ --- нормовані простори; $X$ --- повний; $A_n \in L\left( X, Y\right),\; n\in \mathbb{N}$;
        $\forall x \in X, \; \exists C(x) > 0$ таке, що $\forall n \in \mathbb{N}: \norm{A_n x} \leq C(x)$.
        Тоді $\exists \; C>0: \forall n \in \mathbb{N}: \norm{A_n} \leq C$.
    \end{theorem*}
\end{theory}

\begin{exercise}
    Нехай $\alpha_n$ --- числова послідовність в $\mathbb{R}$, для якої 
    $\sum\limits^\infty_{k=1} \alpha^2_k =\infty$. Довести, що існує така послідовність 
    $\beta_n$ в $\mathbb{R}$, для якої $\sum\limits^\infty_{k=1} \beta^2_k < \infty$, 
    але $\sum\limits^\infty_{k=1} \alpha_k\beta_k$ --- розбіжний ряд.
\end{exercise}

\begin{exercise}
    Нехай $X$ --- нормований простір, $L$ --- замкнений підпростір $X$; $\mathrm{codim}L \geq 1$.
    Тоді $\exists \; \varphi \in X^*$; $\varphi \neq 0$, для деякого $L \subset \mathrm{Ker}\varphi$.
\end{exercise}

\begin{exercise}
    Нехай $X, Y$ --- банахові простори над полем $K$ ($K$ = $\mathbb{R}$ або $\mathbb{C}$).
    $\varphi: X \times Y \rightarrow K$ --- білінійний функціонал, що має наступні властивості:
    \begin{enumerate}[label=\ukr*)]
        \item $\forall y \in Y$: $\varphi(\,\cdot\,, y)$ --- неперервний по $x$;
        \item $\forall x \in X$: $\varphi(x, \,\cdot\,)$ --- неперервний по $y$.
    \end{enumerate}
    Довести існування константи $k$ такої, що для кожного $(x,y) \in X \times Y$ має місце нерівність
    $|\varphi(x,y)| \leq k \norm{x}\cdot\norm{y}$.
\end{exercise}
            \newpage
        \section{Геометрія гільбертового простору.}
            % !TEX root = ../main.tex

\begin{theory}
    Нехай $H$ --- нескінченновимірний гільбертів простір (дійсний або комплексний); 
    $\{e_1, e_2, ...\}$ --- зліченна ортонормована система векторів в $H$ (тобто $\dotprod{e_i}{e_j} 
    = \delta_{ij}$). Для $x \in H$ покладемо $c_n = \dotprod{x}{e_n}$. Ряд $\sum\limits_{n=1}^\infty 
    c_n e_n$ називається \underline{рядом Фур'є} вектора $x$ за ортонормованою системою 
    $\{e_n\}$, $c_n$ --- \underline{коефіцієнтами Фур'є}.
\end{theory}

\begin{exercise}
    Довести \underline{нерівність Бесселя}: $\sum\limits_{n=1}^\infty |\dotprod{x}{e_n}|^2 \leq 
    \norm{x}^2$ ($\norm{x} = \sqrt{\dotprod{x}{x}}$).
\end{exercise}

\begin{theory}
    
    Ортонормована система $\{e_n\}$ в $H$ називається \underline{повною}, якщо її лінійна 
    оболонка (л.о.) щільна в $H$:
     $\forall x \in H, \;\forall \varepsilon > 0 \; \exists \{\alpha_1, ..., 
    \alpha_m\}$: $\norm{\sum\limits_{k=1}^m \alpha_k e_k - x} < \varepsilon$.

    Ортонормована система $\{e_n\}$ називається \underline{замкненою}, якщо для кожного 
    $x \in H$ має місце рівність: $\norm{x}^2 = \sum\limits_{n=1}^\infty |\dotprod{x}{e_n}|^2$ 
    (\underline{рівність Парсеваля}).

    Повна ортонормована система $\{e_n\}$ називається \uline{ортонормованим 
    базисом} в $H$.
\end{theory}

\begin{exercise}
    Нехай $\{e_n\}$ --- ортонормована система векторів в гільбертовому просторі $H$. 
    Довести еквівалентність трьох умов:
    \begin{enumerate}[label=\ukr*)]
        \item система векторів $\{e_n\}$ повна в $H$;
        \item система векторів $\{e_n\}$ замкнена в $H$;
        \item $\forall x \in H$ ряд $\sum\limits_{n=1}^\infty \dotprod{x}{e_n}e_n$ збігається до $x$
        (тобто $\norm{x - \sum\limits_{n=1}^m \dotprod{x}{e_n}e_n}
        \underset{m\rightarrow\infty}{\rightarrow}0$)
    \end{enumerate}
\end{exercise}

\begin{exercise}
    Довести еквівалентність двох умов:
    \begin{enumerate}[label=\ukr*)]
        \item $H$ --- сепарабельний гільбертів простір;
        \item в просторі $H$ існує ортонормований базис.
    \end{enumerate}
\end{exercise}

\begin{theory}
    Ортонормована система векторів $\{e_n\}$ в $H$ називається \underline{тотальною}, 
    якщо умова <<$\dotprod{x}{e_n} = 0$ $\forall n \in N$>> виконується лише для вектора $x=0$.
\end{theory}

\begin{exercise}[лема Ріса-Фішера]
    Нехай $\{e_n\}$ --- ортонормована система векторів в $H$, $\{c_n\}$ --- числова послідовність,
    для якої ряд $\sum\limits_{n=1}^\infty |c_n|^2$ --- збіжний. Тоді існує $x \in H$, 
    для якого при всіх $n \in \mathbb{N}$ має місце рівність: $c_n = \dotprod{x}{e_n}$ і при цьому 
    $\norm{x}^2 = \sum\limits_{n=1}^\infty |c_n|^2$.
\end{exercise}

\begin{exercise}
    Нехай $\{e_n\}$ --- ортонормована система в $H$. Тоді дві умови еквівалентні:
    \begin{enumerate}[label=\ukr*)]
        \item $\{e_n\}$ --- ортонормований базис в $H$;
        \item $\{e_n\}$ --- тотальна система векторів в $H$.
    \end{enumerate}
\end{exercise}

\begin{theory}
    Гільбертові простори $H_1$ і $H_2$ називають \underline{ізоморфними}, якщо існує 
    лінійний оператор $A: H_1 \rightarrow H_2$, для якого виконуються рівності: 
    $\ker A = \{0\}$; $ImA = H_2$; $\dotprod{Ax}{Ay}_2 = \dotprod{x}{y}_1$ (тут $x$, $y$ --- довільні 
    вектори в $H_1$; $\dotprod{\cdot}{\cdot}_k$ --- скалярний добуток в $H_k$). 
    Такий оператор $A$ називається \underline{ізоморфізмом}. Позначення: $H_1 \cong H_2 $.
\end{theory}

\begin{exercise}
    Нехай $A: H_1 \rightarrow H_2$ --- ізоморфізм. Доведіть, що в означенні ізоморфізма:
    \begin{enumerate}[label=\ukr*)]
        \item умова <<$\ker A = \{0\}$>> є зайвою;
        \item умова <<$\dotprod{Ax}{Ay} = \dotprod{x}{y}$ для всіх $x, y \in H_1$>> може бути 
        замінена на таку: <<$\norm{Ax} = \norm{x}$ для всіх $x \in H_1$>>.
    \end{enumerate}
\end{exercise}
            % !TEX root = ../main.tex
\begin{exercise}
    Довести, що на множині гільбертових просторів відношення <<бути ізоморфним>>
    є відношенням еквівалентності, тобто виконуються наступні властивості:
    \begin{enumerate}[label=\ukr*)]
        \item $H \cong H$ (рефлексивність);
        \item $(H_1 \cong H_2) \Leftrightarrow (H_2 \cong H_1)$ (симетричність);
        \item $(H_1 \cong H_2, H_2 \cong H_3) \Rightarrow (H_1 \cong H_3)$ (транзитивність).
    \end{enumerate}
\end{exercise}

\begin{exercise}\label{N:1_2_8}
    Нехай $H_1$, $H_2$ --- два нескінченновимірні гільбертові простори над однаковим полем, $H_1$ --- сепарабельний.
    Довести $H_2$ --- сепарабельний простір тоді й тільки тоді, коли $H_1 \cong H_2$.
\end{exercise}

\begin{exercise}
    Нехай $H$ --- гільбертів простір, $x_n, y_n \in H \; (n \in \mathbb{N})$, $\norm{x_n} = \norm{y_n} = 1 \; \forall n \in \mathbb{N}$.
    Довести наступні твердження:
    \begin{enumerate}[label=\ukr*)]
        \item $(\dotprod{x_n}{y_n} \rightarrow 1) \Rightarrow (\norm{x_n - y_n} \rightarrow 0)$;
        \item $(\norm{x_n + y_n} \rightarrow 2) \Rightarrow (\norm{x_n - y_n} \rightarrow 0)$.
    \end{enumerate}
\end{exercise}

\begin{theory}
    Нехай $x \in H$, $M$ --- підмножина $H$. Вектор $x$ називається \uline{ортогональним до $M$},
    якщо $\dotprod{x}{y} = 0$ для кожного $y \in M$. Позначення: $x \perp M$.
    Множина всіх векторів, ортогональних заданій множині $M$, називається \uline{ортогональним доповненням до $M$} 
    та позначається $M^\perp$.
\end{theory}

\begin{exercise}
    Нехай $H$ --- гільбертів простір, $x\in H$, $M \subset H$, $x \perp M$.
    Довести:
    \begin{enumerate}[label=\ukr*)]
        \item $x \perp \overline{M}$ ($\overline{M}$ --- замикання $M$);
        \item $M^\perp$ --- замкнений підпростір в $H$;
        \item $\overline{M}^\perp = M^\perp$;
        \item $(\text{л.о. } M)^\perp = M^\perp$.
    \end{enumerate}
\end{exercise}

\begin{exercise}
    Нехай $L$ --- замкнений підпростір гільбертового простору $H$, $x \in H$. Довести:
    $x \perp L$ тоді й тільки тоді, коли для кожного $y \in L$ виконується нерівність $\norm{x} \leq \norm{x-y}$.
\end{exercise}

\begin{exercise}
    Нехай $M \subset H$. Довести: умова <<$M^\perp = \{0\}$>> еквівалентна <<$\text{л.о. } M$ щільна в $H$>>.
\end{exercise}

\begin{exercise}
    Перевірити взаємну ортогональність наступних систем векторів:
    \begin{enumerate}[label=\ukr*)]
        \item $\set{1, \cos{nt}, \sin{nt} \mid n \in \mathbb{N}}$, $H = L_2 [-\pi; \pi]$;
        \item $\set{\frac{d^n}{dt^n} ((t^2-1)^n) \mid n \in \{0 \} \cup \mathbb{N}}$, $H = L_2 [-1; 1]$.
    \end{enumerate}
\end{exercise}

\begin{exercise}\label{N:1_2_14}
    В комплексному просторі $H = L_2 [-\pi; \pi]$ знайти $M^\perp$ для наступних множин:
    \begin{enumerate}[label=\ukr*)]
        \item $M = \set{e^{i n t} \mid n \in \mathbb{Z}}$;
        \item $M = \set{e^{i n t} \mid n \in \mathbb{N}}$;
        \item $M = \set{\sin{n t} \mid n \in \mathbb{N}}$.
    \end{enumerate}
\end{exercise}

\begin{exercise}
    В просторі $L_2 [0; 1]$ знайти ортогональне доповнення до наступних множин:
    \begin{enumerate}[label=\ukr*)]
        \item $M = C[0; 1]$;
        \item $M = P[0;1]$ --- множина всіх многочленів, що визначені на $[0; 1]$;
        \item $M = \set{ x \in C[0;1] \mid x(0) = 0}$.
    \end{enumerate}
\end{exercise}

\begin{theory}
    Нехай $H_1$, $H_2$ --- замкнені підпростори в гільбертовому просторі $H$.
    $H$ називається \uline{сумою $H_1$ та $H_2$} ($H = H_1 + H_2$), якщо виконується умова
    $(x \in H) \Rightarrow (\exists \; x_1 \in H_1, x_2 \in H_2 : x = x_1 + x_2)$;
    \uline{ортогональною сумою $H_1$ та $H_2$} ($H = H_1 \oplus H_2$), якщо додатково виконується умова
    $(x \in H_1, y \in H_2) \Rightarrow ( \dotprod{x}{y} = 0)$.
\end{theory}
            % !TEX root = ../main.tex

\begin{exercise}
    Нехай $H_1$, $H_2$ --- замкнені підпростори в $H$.
    Довести еквівалентність наступних тверджень:
    \begin{enumerate}[label=\ukr*)]
        \item $H = H_1 \oplus H_2$;
        \item $H_1 = H_2^\perp $;
        \item $H_2 = H_1^\perp $.
    \end{enumerate}
\end{exercise}

\begin{exercise}
    Довести, що при фіксованому $n$ множина $M = \set{ \vec{x}=(x_1,x_2,\dots) \in l_2
    \mid \sum\limits^n_{k=1} x_k=0 }$ є замкненим підпростором в $l_2$.
    Знайти такий замкнений підпростір $N$, що виконується рівність $l_2 = M \oplus N$.
\end{exercise}

\begin{exercise}
    Нехай $M$ та $N$ --- підмножини гільбертового простору $H$. Довести наступні твердження:
    \begin{enumerate}[label=\ukr*)]
        \item $(M \subset N) \Rightarrow (N^\perp \subset M^\perp)$;
        \item $M \subset M^{\perp\perp}$, $\left(M^{\perp\perp}={(M^\perp)}^\perp\right)$;
        \item $M = M^{\perp\perp}$ тоді й тільки тоді, коли $M$ ---
              замкнений підпростір в $H$;
        \item $M^\perp \subset M^{\perp\perp\perp}$;
        \item $(M \cap N)^\perp = \overline{M^\perp + N^\perp}$.
    \end{enumerate}
\end{exercise}

\begin{exercise}
    Нехай $M_\alpha$ --- сім'я підмножин гільбертового простору $H$. Довести:
    \begin{enumerate}[label=\ukr*)]
        \item $\Big( \bigcup\limits_\alpha M_\alpha \Big)^\perp = 
              \bigcap\limits_\alpha M_\alpha^\perp$;
        \item $\Big( \bigcap\limits_\alpha M_\alpha \Big)^\perp = 
              \overline{\text{л.о.}\Big(\bigcup\limits_\alpha M_\alpha^\perp\Big)}$.
    \end{enumerate}
\end{exercise}

\begin{exercise}
    Нехай $M$ --- замкнена опукла множина в дійсному гільбертовому просторі $H$.
    Довести, що вектор $y \in M$ задовольняє умову $\rho(x, M) = \norm{x-y}$,
    де $x \in H$ тоді й тільки тоді, коли для будь-якого $z \in H$ виконується
    нерівність $\dotprod{x-y}{y-z} \geq 0$.
\end{exercise}

\begin{exercise}
    В просторі $l_2$ знайти замкнену підмножину в якій немає вектора з найменшою нормою.
\end{exercise}

\begin{exercise}
    В просторі $L_2[0;1]$ знайти відстань від елемента $x_0(t) = t^2$ до підпростору
    $L = \set{x \in L_2[0;1] \mid \int\limits_0^1x(t)dt = 0}$.
\end{exercise}

\begin{exercise}
    В просторі $l_2$ знайти відстань $\rho_n(\vec{x}_0, L_n)$ від вектора $\vec{x}_0$ до
    підпростору $L_n = \set{ \vec{x}\in l_2 \mid \sum\limits^n_{k=1} x_k = 0}$.
    Чому дорівнює $\lim\limits_{n \to \infty} \rho_n(x_0, L_n)$?
\end{exercise}

\begin{exercise}
    Нехай $M$ --- замкнена опукла множина в гільбертовому просторі $H$; $x \in H$.
    Довести, що існує, і при тому єдиний, вектор $y \in M$ для якого $\norm{x-y} =
    \rho(x, M)$.
\end{exercise}

\begin{exercise}
    Довести, що \underline{система Радемахера} $f_n(t) = \sgn\sin(2^n \pi t)$,
    $n=0,1,2,\dots$ ортонормована, але не є повною в $L_2[0;1]$
\end{exercise}

\begin{exercise}
    Нехай $\{x_n\}$ --- ортогональна система векторів гільбертового простору $H$.
    Довести еквівалентність наступних трьох умов:
    \begin{enumerate}[label=\ukr*)]
        \item ряд $\sum\limits^\infty_{n=1} x_n$ збігається;
        \item для кожного $y=H$ ряд $\sum\limits^\infty_{n=1} \dotprod{x_n}{y}$ збігається;
        \item ряд $\sum\limits^\infty_{n=1} \norm{x_n}^2$ збігається.
    \end{enumerate}
\end{exercise}

\begin{exercise}
    В лінійному просторі послідовностей $\vec{x} = (x_1, x_2, \dots)$
    $(x_n \in \mathbb{R})$ таких, що $\sum\limits^\infty_{n=1} x_n^2 < \infty$,
    покладемо $\dotprod{x}{y} \coloneqq \sum\limits^\infty_{k=1} \lambda_k x_k y_k$,
    де $\lambda_k \in \mathbb{R}$; $0 < \lambda_k < 1$. Перевірте, що остання формула
    коректно визначає скалярний добуток. Чи буде одержаний евклідів простір гільбертовим?
\end{exercise}

\begin{exercise}
    Нехай $\{e_n\}$ --- ортонормована система в гільбертовому просторі $H$;
    $\{\lambda_n\}$ --- числова послідовність. Доведіть, що ряд $\sum\limits^\infty_{n=1}
    \lambda_n e_n$ збігається в $H$ тоді й тільки тоді, коли $\sum\limits^\infty_{n=1}
    |\lambda_n|^2 < \infty$
\end{exercise}
            % !TEX root = ../main.tex

\begin{exercise}
    Довести, що множина $M = \{ \vec{x} \in \ell_2 | \sum\limits_{n = 1}^{\infty}x_n = 0\}$
    щільна в $\ell_2$.
\end{exercise}

\begin{exercise}\label{N:1_2_30}
    Нехай $M$ --- замкнена опукла множина в гільбертовому просторі $H$. Довести, що в $M$
    існує і притому єдиний вектор із найменшою нормою.
\end{exercise}

\begin{exercise}
    На просторі $C[0;1]$ розглянемо функціонал $\varphi$, який визначимо формулою: 
    $\varphi(x) = \int\limits_{0}^{\frac{1}{2}}x(t)dt - \int\limits_{\frac{1}{2}}^{1}x(t)dt$. Тоді
    $M = \set{x \in C[0; 1] \mid \varphi(x) = 1}$ опукла замкнена множина, яка не містить найближчого до 
    нуля елемента. Довести. Порівняйте із задачею \ref{N:1_2_30}.
\end{exercise}

\begin{exercise}\label{N:1_2_32}
    Нехай послідовність неперервно диференційовних функцій утворює ортонормовану систему
    в $L_2[0; 2\pi]$. Доведіть, що похідні цих функцій не можуть бути обмеженими у сукупності.
\end{exercise}

\begin{exercise}
    Побудувати конкретний ізоморфізм просторів $L_2[0;1]$ та $\ell_2$.
\end{exercise}

\begin{theory}
    Нехай $L$ --- замкнений підпростір гільбертового простору $H$; $x \in H$. Вектор
    $y \in L$ називається \uline{(ортогональною) проекцією} вектора $x$ на $L$, якщо
    $x - y \bot L$. Вектор $x - y$ називається \uline{ортогональною складовою} при
    проектуванні $x$ на $L$. Позначення: $y = pr_L x$; $x - y = ort_L x$.
\end{theory}

\begin{exercise}
    Нехай $L$ --- замкнений підпростір $H$, $x \in H$, $y = pr_L x$, $z \in L, z \neq y$.
    Доведіть: $\norm{x - y} < \norm{x - z}$ (екстремальна властивість ортогональної проекції).
\end{exercise}

\begin{exercise}
    Нехай $L$ --- замкнений підпростір $H$; $x \in H$. Довести існування та єдиність ${pr}_L x$:
    \begin{enumerate}[label=\ukr*)]
        \item в разі, якщо $L$ --- сепарабельний підпростір;
        \item для загального випадку.
    \end{enumerate}
\end{exercise}

\begin{exercise}
    Нехай $L$ --- замкнений підпростір гільбертового простору $H$, $M$ --- замкнений підпростір $L$, $x \in H$.
    Доведіть:
    \begin{enumerate}[label=\ukr*)]
        \item $L$ --- гільбертів простір, що успадковує скалярний добуток простору $H$;
        \item $M$ є замкнений підпростором $H$;
        \item ${pr}_M({pr}_L x) = {pr}_M x$.
    \end{enumerate}
\end{exercise}
            \newpage
        \section{Обмежені лінійні функціонали та оператори в гільбертових просторах.}
            % !TEX root = ../main.tex

\begin{exercise}
    Нехай $L$ --- замкнений підпростір в $H$, $L \neq \{0\}$. 
    Відображення $P: H \rightarrow H$ визначене формулою 
    $P: x \rightarrow {pr}_L x$. Довести:
    \begin{enumerate}[label=\ukr*)]
        \item $P$ --- лінійний обмежений оператор в $H$;
        \item $\norm{P} = 1$, $\mathrm{Im}P = L$, $\mathrm{Ker} P = L^\bot$;
        \item $P^2 = P$, $L = \set{x \mid x=Px}$, $\dotprod{Px}{y}=\dotprod{x}{Py}$,
        $\dotprod{Px}{x} = \norm{Px}^2$ ($\forall x,y \in H$);
        \item оператор $Q: x \rightarrow {pr}_{L^\bot}x$ пов'язаний з $P$ 
        співвідношенням: $Q = I-P$ (тут $I$ --- тотожній оператор в $H$);
        \item $PQ = QP = 0$, $\mathrm{KerP} = \mathrm{Im}Q$, $\mathrm{Im}P = \mathrm{Ker}Q$;
        \item якщо $M$ --- замкнений підпростір в $L$ і для $x \in H$:
        $P_1 x = {pr}_M x$, то $P_1 P = P P_1 = P_1$,
        $\dotprod{P_1x}{x} = \dotprod{Px}{x}$ ($\forall x \in H$).
    \end{enumerate}
\end{exercise}

\begin{theory}
    Лінійний оператор $P$ із задачі 1.3.1 називається \uline{ортопроектором}.
\end{theory}

\begin{exercise}
    Нехай $X$ --- нормований простір, $P \in L(X)$.
    $P$ називається \uline{проектором}, якщо $P^2=P$. Доведіть наступні 
    властивості (обмеженого) проектора:
    \begin{enumerate}[label=\ukr*)]
        \item $M = KerP$ --- замкнений підпростір;
        \item оператор $Q = I - P$ також є (обмеженим) проектором;
        \item $L=ImP=KerQ=\set{x \mid Px=x}$ --- замкнений підпростір в $X$;
        \item $X = L \dotplus M$ (тобто $\forall x \in X$ має, і притому 
        єдиний розклад: $x=x_1+x_2$, де $x_1 \in L$, $x_2 \in M$).
    \end{enumerate}
\end{exercise}

\begin{theory}
    Оператор $P$ називається \uline{проектором на $L$ паралельно $M$}.
\end{theory}

\begin{exercise}[теорема Фреше-Ріса]
    Нехай $H$ --- гільбертів простір над полем $K$ ($K=\mathbb{R}$ 
    або $\mathbb{C}$), $y \in H$, $\varphi_y : H \rightarrow K$ визначено 
    формулою $\phi_y(x) = \dotprod{x}{y}$. Довести:
    \begin{enumerate}[label=\ukr*)]
        \item $\varphi_y \in H^*$, $\norm{\phi_y} = \norm{y}$;
        \item $\forall \varphi \in H^*$ $\exists! \; y\in H$, для якого 
        $\varphi = \varphi_y$ (при цьому $y\in(Ker\varphi)^\bot$).
    \end{enumerate}
\end{exercise}

\begin{exercise}
    Застосувати теорему Фреше-Ріса для розв'язання задачі \ref{N:1_1_5}
    (в, г, ґ, е, ж).
\end{exercise}

\begin{exercise}
    Нехай $H$ --- гільбертів простір; $A$ --- лінійний оператор. 
    Довести, що $A \in L(H)$ тоді й тільки тоді, коли існує 
    $C > 0$ таке, що для кожного $x, y \in H$ має місце нерівність:
    $|\dotprod{Ax}{y}| \leq C\norm{x}\norm{y}$. При цьому $\norm{A} = 
    \underset{x,y \neq 0}{\sup} \frac{|\dotprod{Ax}{y}|}{\norm{x}\norm{y}}
    =\underset{\norm{x},\norm{y} \leq 1}{\sup} |\dotprod{Ax}{y}| = 
    \underset{\norm{x} = \norm{y} = 1}{\sup} |\dotprod{Ax}{y}|$.
\end{exercise}

\begin{theory}
    Нехай $A \in L(H)$. \uline{Спряженим оператором} до $A$ називається 
    такий оператор $B \in L(H)$, для якого при всіх $x, y \in H$ 
    виконується рівність $\dotprod{Ax}{y} = \dotprod{x}{By}$. 
    Позначення: $B = A^*$. 
\end{theory}

\begin{exercise}
    Довести коректність означення $A^*$ (тобто для кожного $A\in L(H)$ 
    $\exists! \; A^* \in L(H)$).
\end{exercise}

\begin{exercise}
    Нехай $A, B \in L(H)$, $\alpha \in K$. Доведіть наступні властивості:
    \begin{enumerate}[label=\ukr*)]
        \item $(A+B)^* = A^* + B^*$;
        \item $0^*=0$; $I^* = I$;
        \item $(\alpha A)^* = \overline{\alpha}A^*$
        \item $(A^*)^* = A^{**} = A$;
        \item якщо існує $A^{-1} \in L(H)$, то існує $(A^*)^{-1} \in 
        L(H)$ і при цьому $(A^*)^{-1} = (A^{-1})^*$;
        \item $\norm{A^*} = \norm{A}$.
    \end{enumerate}
\end{exercise}

\begin{theory}
    Оператор $A \in L(H)$ називається \uline{самоспряженим}, якщо 
    $A = A^*$.
\end{theory}

\begin{exercise}
    Нехай $A, B \in L(H)$; $A, B$ --- самоспряжені. Довести:
    \begin{enumerate}[label=\ukr*)]
        \item $A+B$ --- самоспряжений;
        \item $\alpha \in \mathbb{R}, \alpha A $ 
        --- самоспряжений;
        \item $0, I$ --- самоспряжені;
        \item $AB$ -- самоспряжений $\Rightarrow AB = BA$;
        \item якщо існує $A^{-1} \in L(H)$, то $A^{-1}$ --- 
        самоспряжений.
    \end{enumerate}
\end{exercise}

\begin{exercise}
    Знайти спряжений оператор до $A: H\rightarrow H$ в наступних 
    прикладах:
    \begin{enumerate}[label=\ukr*)]
        \item $H = \ell_2$; $A\vec{x} = (\alpha_1x_1, \alpha_2x_2, ...)$
        ($\alpha_k \in \mathbb{C}$, $\underset{k \in \mathbb{N}}{\sup}|\alpha_k| < \infty$);
        \item $H = \ell_2$; $A\vec{x} = (0, x_1, x_1, ...)$;
        \item $H = \ell_2$; $A\vec{x} = (x_5, x_6, x_7, ...)$;
        \item $H = \ell_2$; $A\vec{x} = (x_1, x_2, x_3,0,0,...)$;
        \item $H = \ell_2$; $A\vec{x} = (\underbrace{0,...,0}_m, 
        x_1, 0, 0, ...)$;
        \item $H = \ell_2$; $A\vec{x} = (2x_1 + 5x_2, x_2, x_3, ...)$;
        \item $H = \ell_2$; $A\vec{x} = (0, 0, x_1 + x_2, x_1 - x_2, 0,  
        0, ...)$;
        \item $H = L_2[0,1]$; $(Ax)(t) = \int\limits_0^t x(\tau) d\tau$;
        \item $H = L_2[0,1]$; $(Ax)(t) = x(t^\alpha)$, $\alpha \in (0,1)$;
        \item $H = L_2[0,1]$; $(Ax)(t) = \int\limits_0^1 e^{t+\tau}x(\tau) 
        d\tau$;
        \item $H = L_2[0,1]$; $(Ax)(t) = \int\limits_0^1 
        sin(t+2\tau) x(\tau) d\tau$;
    \end{enumerate}
\end{exercise}
            % !TEX root = ../main.tex

\begin{exercise}
    Нехай $A \in L(H)$ --- самоспряжений оператор.
    Довести: $\norm{A} = \underset{\norm{x} = 1}{\sup}|(Ax, x)|$.
\end{exercise}

\begin{exercise}
    Нехай $A \in L(H)$.
    \begin{enumerate}[label=\ukr*)]
        \item довести рівності: $(\mathrm{Im}A)^\perp = \mathrm{Ker}A^*$, $(\mathrm{Ker}A)^\perp = \overline{\mathrm{Im}A^*}$;
        \item навести приклад, коли $(\mathrm{Ker}A)^\perp \neq \mathrm{Im}A^*$.
    \end{enumerate}
\end{exercise}

\begin{exercise}\label{N:1_3_12}
    Нехай $A \in L(H)$. Довести наступні твердження:
    \begin{enumerate}[label=\ukr*)]
        \item $\mathrm{Ker}(A A^*) = \mathrm{Ker}(A^*)$;
        \item $\mathrm{Ker}(A^* A) = \mathrm{Ker}(A)$;
        \item $\overline{\mathrm{Im} (A A^*)} = \overline{\mathrm{Im} (A)}$;
        \item\label{N:1_3_12_h} $\norm{A^* A} = \norm{A}^2$.
    \end{enumerate}
\end{exercise}

\begin{exercise}
    Нехай $A \in L(H)$. Довести еквівалентність двох умов:
    \begin{enumerate}[label=\ukr*)]
        \item $A A^* = A^* A$;
        \item $\forall x \in H : \norm{Ax} = \norm{A^{*}x}$.
    \end{enumerate}
\end{exercise}

\begin{theory}
    Оператор $A \in L(H)$, для якого $A A^* = A^* A$, називається \uline{нормальним}.
\end{theory}

\begin{exercise}\label{N:1_3_14}
    Нехай $A$ --- нормальний оператор в $H$. Довести:
    \begin{enumerate}[label=\ukr*)]
        \item $\mathrm{Ker} A = \mathrm{Ker} A^* = (\mathrm{Im} A)^\perp$;
        \item $\norm{A^2} = \norm{A}^2$;
        \item $(A^2 = 0) \Leftrightarrow (A = 0)$;
        \item $\left( \exists \; A^{-1} \in L(H)\right) \Rightarrow (A^{-1} \text{ --- нормальний оператор})$.
    \end{enumerate}
\end{exercise}

\begin{exercise}
    Нехай $H$ --- комплексний гільбертів простір, $A \in L(H)$. Довести: 
    $A$ --- самоспряжений оператор тоді й тільки тоді, коли 
    \uline{квадратична форма} $(Ax, x)$ набуває лише дійсних значень.
\end{exercise}

\begin{theory}
    Самоспряжений оператор $A \in L(H)$ називається \uline{невід'ємним} ($A \geq 0$),
    якщо квадратична форма $(Ax, x)$ на $H$ набуває лише невід'ємних значень.

    Самоспряжені оператори $A, B \in L(H)$ задовольняють нерівність $A \geq B$, якщо $A - B \geq 0$.
\end{theory}

\begin{exercise}
    Довести, що ортопроектор --- невід'ємний оператор.
\end{exercise}

\begin{exercise}
    Нехай $P \in L(H)$, $P^2 = P$. Довести еквівалентність наступних умов:
    \begin{enumerate}[label=\ukr*)]
        \item $P$ --- ортопроектор;
        \item $P = P^*$;
        \item $P P^* = P^* P$;
        \item $\mathrm{Im} P = (\mathrm{Ker} P)^\perp$;
        \item $\forall x \in H: (Px, x) = \norm{Px}^2$.
    \end{enumerate}
\end{exercise}

\begin{exercise}\label{N:1_3_18}
    Нехай $P_1$, $P_2$ --- ортопроектори в $H$, $H_k = \mathrm{Im} P_k$ ($k = 1, 2$).
    Довести:
    \begin{enumerate}[label=\ukr*)]
        \item\label{N:1_3_18_a} $(P_1 + P_2 \text{ --- ортопроектор}) \Leftrightarrow (P_1 P_2 = 0) \Leftrightarrow (H_1 \perp H_2)$.
        При цьому $P_1 + P_2$ --- ортопроектор на $H_1 \oplus H_2$;
        \item $(P_1 P_2 \text{ --- ортопроектор}) \Leftrightarrow (P_1 P_2 = P_2 P_1)$.
        При цьому $P_1 P_2$ --- ортопроектор на $H_1 \cap H_2$;
        \item $(P_1 - P_2 \text{ --- ортопроектор}) \Leftrightarrow (P_1 \geq P_2) \Leftrightarrow (H_1 \supset H_2)$.
        При цьому $P_1 - P_2$ --- ортопроектор на $H_1 \ominus H_2 = H_1 \cap H_2^\perp$;
        \item $(P_1 P_2 = P_2) \Leftrightarrow (P_2 P_1 = P_2) \Leftrightarrow (P_1 \geq P_2) \Leftrightarrow (H_1 \supset H_2)$.
    \end{enumerate}
\end{exercise}

\begin{exercise}\label{N:1_3_19}
    Нехай $P_1$, $P_2$ --- ортопроектори в гільбертовому просторі, $H_k = \mathrm{Im} P_k$ ($k = 1, 2$). Довести:
    \begin{enumerate}[label=\ukr*)]
        \item $(H_1 \subset H_2; \norm{P_1 - P_2} < 1) \Rightarrow (P_1 = P_2)$;
        \item навести приклад ортопроекторів $P_1 \neq P_2$, для яких $\norm{P_1 - P_2} < 1$.
    \end{enumerate}
\end{exercise}

\begin{exercise}\label{N:1_3_20}
    В позначеннях задачі \ref{N:1_3_19}:
    \begin{enumerate}[label=\ukr*)]
        \item довести: $(\norm{P_2 - P_1} < 1) \Rightarrow (\dim H_1 = \dim H_2)$;
        \item навести приклад таких ортопроекторів $P_1$, $P_2$, для яких $\norm{P_2 - P_1} = 1$ та $\dim H_1 \neq \dim H_2$.
    \end{enumerate}
\end{exercise}

\begin{exercise}\label{N:1_3_21}
    Нехай $A \in L(H), A \geq 0, x, y \in H$. Довести:
    \begin{enumerate}[label=\ukr*)]
        \item $|(Ax, y)|^2 \leq (Ax, x) \cdot (Ay, y)$;
        \item $\norm{Ax}^2 \leq \norm{A} \cdot (Ax, x)$.
    \end{enumerate}
\end{exercise}

\begin{exercise}\label{N:1_3_22}
    Нехай $A \in L(H)$. Довести:
    \begin{enumerate}[label=\ukr*)]
        \item $(A \geq 0, \exists \; A^{-1} \in L(H)) \Rightarrow (\exists \; \lambda > 0 : A \geq \lambda I)$;
        \item $\exists \; (I + A^* A)^{-1} \in L(H)$.
    \end{enumerate}
\end{exercise}

\begin{exercise}\label{N:1_3_23}
    Нехай $A \in L(H)$. Довести еквівалентність умов:
    \begin{enumerate}[label=\ukr*)]
        \item $\exists \; A^{-1} \in L(H)$;
        \item $\exists \; \alpha, \beta > 0 : A A^{*} \geq \alpha I, A^{*} A \geq \beta I$.
    \end{enumerate}
\end{exercise}
            % !TEX root = ../main.tex

\begin{exercise}
    Нехай $A \in L(H)$ --- самоспряжений оператор. Довести, що $A = 0$ тоді,
    й тільки тоді, коли для кожного $x \in H$ виконується рівність $\dotprod{Ax}{x}=0$.
    Навести приклад дійсного гільбертового простору $H$ і ненульового оператора
    $A \in L(H)$, для якого $\dotprod{Ax}{x}=0$ для всіх $x \in H$.
\end{exercise}

\begin{exercise}
    Нехай $A \in L(H)$ і для кожного $B \in L(H)$ має місце рівність $AB = BA$.
    Довести, що існує число $\alpha$ таке, що $A = \alpha I$.
\end{exercise}

\begin{exercise}
    Нехай $H$ --- комплексний гільбертів простір. Для оператора $A \in L(H)$ позначимо
    \uline{дійсну} та \uline{уявну} частину формулами $\mathfrak{Re}A = \frac{1}{2} (A + A^*)$;
    $\mathfrak{Im}A = \frac{1}{2i} (A - A^*)$. Доведіть:
    \begin{enumerate}[label=\ukr*)]
        \item $\mathfrak{Re}A$ та $\mathfrak{Im}A$ --- самоспряжені оператори;
        \item якщо $A$ --- нормальний оператор, то $\norm{A} = 
        \sqrt{\norm{ (\mathfrak{Re}A)^2 + (\mathfrak{Im}A)^2 }}$.
    \end{enumerate}
\end{exercise}

\begin{exercise}
    Як повинні бути пов'язані між собою замкнені підпростори $H_1, H_2 \subset H$,
    щоб ортопроектори $P_1$ і $P_2$ на ці підпростори комутували?
\end{exercise}

\begin{exercise}
    Нехай $A, B \in L(H)$ --- самоспряжені оператори; $A \geq 0$; $B \geq 0$.
    Доведіть:
    \begin{enumerate}
        \item $A + B \geq 0$;
        \item[б)*] $(AB = BA) \Rightarrow (AB \geq 0)$. %тут костыль
    \end{enumerate}
\end{exercise}

\begin{exercise}
    Нехай $A \in L(H)$, $A$ --- самоспряжений оператор; $n \in \mathbb{N}$.
    Довести $\mathrm{Ker}A^n = \mathrm{Ker}A$. 
\end{exercise}

\begin{theory}
    Лінійний оператор $U$ в $H$ називається \uline{унітарним}, якщо
    $U$ --- лінійний ізоморфізм $H$ на $H$, і при цьому $(x, y \in H) \Rightarrow 
    \left(\dotprod{Ux}{Uy} = \dotprod{x}{y}\right)$. За іншою термінологією, оператор
    $U$ у випадку дійсного простору $H$ називають \uline{ортогональним}.
\end{theory}

\begin{exercise}
    Нехай лінійний оператор $U:H \to H$ задовольняє умови:
    \begin{enumerate}[label=\ukr*)]
        \item $\mathrm{Im}A = H$;
        \item $\norm{Ux} = \norm{x}$ для кожного $x \in H$.
    \end{enumerate}
    Доведіть: $U$ --- унітарний оператор.
\end{exercise}

\begin{exercise}
    Довести, що оператор $U \in L(H)$ ($H$ --- комплексний) є унітарним в тому,
    й тільки в тому разі, якщо виконуються дві умови:
    \begin{enumerate}[label=\ukr*)]
        \item $U$ --- нормальний;
        \item $(\mathfrak{Re}U)^2 + (\mathfrak{Im}U)^2 = I$.
    \end{enumerate}
\end{exercise}

\begin{exercise}
    Нехай $y, z \in H$; оператор $A: H \to H$ визначено формулою: $Ax = \dotprod{x}{y} \cdot z$.
    Знайти $A^*$ та з'ясувати за яких умов на вектори $y$ та $z$ оператор $A$ буде:
    \begin{enumerate}[label=\ukr*)]
        \item нормальним;
        \item самоспряженим;
        \item додатним;
        \item унітарним.
    \end{enumerate}
\end{exercise}

\begin{exercise}
    Довести: спряжений оператор до скінченновимірного також має скінченний ранг.
\end{exercise}

\begin{exercise}
    Нехай $A$, $B$ --- самоспряжені оператори в $H$; $A \geq 0$.
    Довести $BAB \geq 0$.
\end{exercise}
            % !TEX root = ../main.tex

\begin{exercise}
    Нехай $A, B$ --- самоспряжені оператори в $H$; $A \geq B$; $B \geq A$.
    Довести: $A = B$.
\end{exercise}

\begin{exercise}
    Нехай $A$ --- самоспряжений оператор в $H$; $\lambda \geq 0$; $0 \leq A \leq \lambda I$.
    Довести: $\norm{A} \leq \lambda$.
\end{exercise}

\begin{exercise}
    Нехай $A$ --- самоспряжений оператор в $H$. Довести:
    $ (A \geq 0; \exists A^{-1} \in L(H)) \Rightarrow (A^{-1} \geq 0)$.
\end{exercise}

\begin{exercise}
    Нехай $A$ --- самоспряжений оператор в $H$; $A \geq 0$. Довести еквівалентність наступних умов:
    \begin{enumerate}[label=\ukr*)]
        \item $\overline{\rm \mathrm{Im}A} = H$;
        \item $\mathrm{Ker}A = \{0\}$;
        \item $(Ax, x) > 0$ для $\forall x \in H \setminus \{0\}$.
    \end{enumerate}
\end{exercise}

\begin{exercise}\label{N:1_3_39}
    Нехай $A$ --- лінійний оператор в гільбертовому просторі $H$ і для $\forall x, y \in H$ виконується
    рівність: $(Ax, y) = (x, Ay)$. Довести: $A \in L(H)$ (а тому $A$ --- самоспряжений).
\end{exercise}

\begin{exercise}
    Нехай числова послідовність $\vec{a} = (a_1, a_2, \dots)$ така, що для кожного $\vec{x} \in \ell_2$
    ряд $\sum\limits_{n = 1}^\infty a_n x_n$ збігається. Тоді $\vec{a} \in \ell_2$ і формула 
    $\varphi(\vec{x}) = \sum\limits_{n = 1}^\infty a_n x_n$ задає неперервний лінійний функціонал на $\ell_2$.
    Довести.
\end{exercise}

\begin{exercise}\label{N:1_3_41}
    Нехай $A$ --- самоспряжений оператор в $H$; $\exists m > 0: (Ax, x) \geq m \norm{x}^2$ для $\forall x \in H$.
    Довести: $\forall f \in H$ рівняння $Ax = f$ має і при тому єдиний розв'язок.
\end{exercise}

\begin{exercise}
    Використовуючи результат задачі \ref{N:1_3_41}, довести розв'язність в дійсному просторі $L_2[0; 1]$ рівняння
    $x(t) = \int\limits_{0}^1 K(t, s)x(s)ds + f(t)$ для таких функцій $K$:
    \begin{enumerate}[label=\ukr*)]
        \item $K(t, s) = -e^{ts}$;
        \item $K(t, s) = \sin(ts)$;
        \item $K(t, s) = -\sum\limits_{n = 1}^\infty \frac{\sin(nt)\sin(ns)}{n^2}$;
        \item $K(t, s) = -\sum\limits_{n = 1}^\infty t^n(1-t)s^n(1-s)$.
    \end{enumerate}
\end{exercise}

\begin{exercise}
    Для функціоналів на $L_2[0; 1]$ вказати такий вектор $h \in L_2[0; 1]$, що $\varphi(x) = (x, h)$ для $\forall x$:
    \begin{enumerate}[label=\ukr*)]
        \item $\varphi(x) = \int\limits_{0}^{\frac{1}{2}}x(s)ds$;
        \item $\varphi(x) = \int\limits_{0}^{\frac{1}{3}}x(s)ds - \int\limits_{\frac{1}{2}}^{1}x(s)ds$;
        \item $\varphi(x) = \int_A x(s)ds$, де $A$ --- вимірна множина на $[0; 1]$.
    \end{enumerate}
\end{exercise}

\begin{exercise}
    Нехай $H$ --- нескінченновимірний гільбертів простір. Довести, що простір $L(H)$ не є сепарабельним.
\end{exercise}

\begin{exercise}
    Нехай $A \geq B \geq 0$. Довести: $\norm{A} \geq \norm{B}$.
\end{exercise}
            \newpage
        \section{Збіжність послідовностей векторів, функціоналів та операторів.}
            % !TEX root = ../main.tex

\begin{theory}
    Нехай $X$ --- нормований простір; $X^*$ --- його спряжений. Послідовність $\{x_n\}$ 
    векторів простору $X$ називається \ul{сильно збіжною} до вектора $x \in X$, якщо 
    $\norm{x_n - x} \underset{n \rightarrow \infty}{\rightarrow} 0$. Це є звичайна збіжність 
    за нормою. Позначення: $x_n \rightarrow x$.

    Послідовність $x_n \in X$ називається \ul{слабо} (або \ul{слабко}) збіжною до $x \in X$, 
    якщо $\forall \varphi \in X^* : \varphi(x_n) \underset{n \rightarrow \infty}{\rightarrow} 
    \varphi(x)$. Позначення: $x_n \underset{\text{сл.}}{\rightarrow} x$ 
    (або $x_n \rightharpoonup x$).

    Послідовність функціоналів $\varphi_n \in X$ \ul{сильно збігається} до $\varphi \in X^*$, 
    якщо $\norm{\varphi_n - \varphi} \underset{n \rightarrow \infty}{\rightarrow} 0$ 
    (збіжність за нормою в $X^*$). \ul{Слабка збіжність} $\varphi_n$ до $\varphi$ --- 
    це просто поточкова збіжність  $\forall x \in X : \varphi_n(x) \underset{n \rightarrow \infty}{\rightarrow} \varphi(x)$.
    Позначаємо її так: $\varphi_n \overset{*}{\rightarrow} \varphi$. 
    Її також називають $*$-слабкою, бо в $X^*$ є і інша слабка збіжність: $\forall \alpha 
    \in X^{**} : \alpha(\varphi_n) \rightarrow \alpha(\varphi)$ (як в будь-якому 
    нормованому просторі).

    Для послідовності операторів $A_n \in L(X, Y)$ будемо розглядати \ul{збіжність за нормою} 
    (або \ul{рівномірну}): $\norm{A_n - A} \underset{n \rightarrow \infty}{\rightarrow} 0$. 
    Позначення: $A_n \rightrightarrows A$.

    Також нам важлива \ul{сильна збіжність}, яка визначена умовою: 
    $A_n x \rightarrow Ax$ для $\forall x \in X$. Позначення: $A_n \overset{s}{\rightarrow} A$ 
    (або інакше: <<$A_n \rightarrow A$ сильно>>).

    Є ще <<слабка операторна збіжність>> $A_n \underset{\text{сл.}}{\rightarrow} A$ 
    (або $A_n \rightharpoonup A$). Слабка збіжність за означенням --- це умова: 
    $\forall x \in X$, $\forall \varphi \in X^*$: $\varphi(A_nx) \rightarrow \varphi(Ax)$.
\end{theory}

\begin{exercise}
    Доведіть, що у випадку гільбертового простору $H$ слабка збіжність $x_n 
    \underset{\text{сл.}}{\rightarrow} x$ ($x, x_n \in H$) рівносильна умові: 
    $\forall y \in H : \dotprod{x_n}{y} \rightarrow \dotprod{x}{y}$.
\end{exercise}

\begin{exercise}
    Доведіть, що сильна збіжність послідовності $x_n$ нормованого простору $X$ гарантує 
    її слабку збіжність. Доведіть, що у випадку скінченновимірного $X$ має місце і зворотній 
    факт. Наведіть приклад нескінченновимірного простору $X$ і слабко збіжної послідовності 
    $x_n$, яка не має сильної границі.
\end{exercise}

\begin{exercise}
    Нехай $\{x_n\}$ --- слабо збіжна послідовність векторів гільбертового простору $H$. 
    Доведіть : 
    \begin{enumerate}[label=\ukr*)]
        \item послідовність $\{x_n\}$ --- обмежена;
        \item послідовність $\{x_n\}$ не може мати двох різних слабких 
        границь.
    \end{enumerate}
\end{exercise}

\begin{exercise}
    Нехай $\{x_n\}$ --- слабо збіжна послідовність векторів в нормованому 
    просторі $X$. Доведіть :
    \begin{enumerate}[label=\ukr*)]
        \item послідовність $\{x_n\}$ --- обмежена;
        \item послідовність $\{x_n\}$ не може мати двох різних 
        слабких границь.
    \end{enumerate}
\end{exercise}

\begin{exercise}
    Нехай $X$ - банахів простір; $\varphi_n \in X^*$, $\varphi_n$ --- 
    слабко збіжна. Доведіть:
    \begin{enumerate}[label=\ukr*)]
        \item послідовність $\{\varphi_n\}$ обмежена (за нормою в $X^*$);
        \item послідовність $\{\varphi_n\}$ не може мати двох різних слабких 
        границь.
    \end{enumerate}
\end{exercise}

\begin{exercise}
    Нехай $A_n$, $A \in L(X, Y)$. Доведіть: 
    \begin{enumerate}
        \item $(A_n \rightrightarrows A) \Rightarrow (A_n 
        \overset{s}{\rightarrow} A)$;
        \item взагалі кажучи: $(A_n 
        \overset{s}{\rightarrow} A) \nRightarrow (A_n \rightrightarrows A)$;

        Розгляньте приклад: $H = \ell_2$, $P_nx = (x_1, x_2, ..., x_n, 0, 0, ...)$;

        \item $(A_n \overset{s}{\rightarrow} A) \Rightarrow (\underset{n}{\sup}\norm{A_n} 
        < \infty$), $X$ --- повний простір;

        \item $A_n$ не може мати двох різних сильних границь.
    \end{enumerate}
\end{exercise}

\begin{exercise}
    Наступні послідовності $x_n \in X$ дослідити на сильну і слабку збіжність: 
    \begin{enumerate}
        \item $X = \ell_2$; $\vec{x_n} = (1, \frac{1}{2}, \frac{1}{3}, ..., \frac{1}{n}, 0, 0
        , ...)$;
        \item $X = \ell_2$; $\vec{x_n} = ( \underbrace{0, ..., 0}_{n-1} ,
        1, \frac{1}{2}, \frac{1}{3}, ...)$;
        \item $X = \ell_2$; $\vec{x_n} = ( \underbrace{1, ..., 1}_{n-1} ,
        \frac{1}{n}, \frac{1}{n+1}, ...)$;
        \item $X = L_2[0, 1]$; $x_n(t) = t^n$;
        \item $X = L_2[0, 1]$; $x_n(t) =  \begin{cases}
            \sqrt{n} & t \in [0; \frac{1}{n}] \\
            0 & t \in (\frac{1}{n}; 1]
        \end{cases}$;
        \item $X = L_2[0, 1]$; $x_n(t) = e^{int}$;
        \item $X = L_2[0, 1]$; $x_n(t) = sin(2^nt)$;
        \item $X = \ell_p \; (1 < p < \infty)$; $x_n = ( \underbrace{0, ..., 0}_{n-1} ,
        1, \frac{1}{2}, \frac{1}{3}, ...)$;
        \item $X = c_0$; $x_n = ( \underbrace{0, ..., 0}_{n-1} ,
        1, \frac{1}{2}, \frac{1}{3}, ...)$.
    \end{enumerate}
\end{exercise}
            % !TEX root = ../main.tex

\begin{exercise}
    Нехай $X$ --- банахів простір, $\varphi, \varphi_n \in X^{*}$.
    Довести еквівалентність двох умов:
    \begin{enumerate}
        \item $\varphi_n \underset{\text{сл.}}{\rightarrow} \varphi$;
        \item $\underset{n}{\sup}\norm{\varphi_n} < \infty$, $\exists$ щільна
        в $X$ множина $Z$ така, що $(x \in Z) \Rightarrow (\varphi_n (x) \rightarrow \varphi(x))$.
    \end{enumerate}
\end{exercise}

\begin{exercise}
    Нехай $X$ --- нормований простір, $x, x_n \in X$.
    Довести еквівалентність двох умов:
    \begin{enumerate}
        \item $x_n \underset{\text{сл.}}{\rightarrow} x$;
        \item $\underset{n}{\sup}\norm{x_n} < \infty$, $\exists$ щільна
        в $X^{*}$ множина $Z$ така, що $(\varphi \in Z) \Rightarrow (\varphi (x_n) \rightarrow \varphi(x))$.
    \end{enumerate}
\end{exercise}

\begin{exercise}
    Нехай $X, Y$ --- нормовані простори, $\mathrm{dim}X < \infty$. Довести:
    \begin{enumerate}
        \item $\left(x_n, x \in X, \; x_n \underset{\text{сл.}}{\rightarrow} x\right) \Rightarrow \left(x_n \rightarrow x\right)$;
        \item $\left(\varphi_n, \varphi \in X^{*}, \; \varphi_n \underset{\text{сл.}}{\rightarrow} \varphi\right) \Rightarrow \left(\varphi_n \rightarrow \varphi\right)$;
        \item $\left(A_n, A \in L(X, Y), \; A_n \overset{s}{\rightarrow} A\right) \Rightarrow \left(A_n \rightrightarrows A\right)$.
    \end{enumerate}
\end{exercise}

\begin{exercise}
    Наступні послідовності операторів $A_n \in L(X)$ дослідити на сильну та рівномірну збіжність:
    \begin{enumerate}
        \item $X = \ell_2$; $A_n \vec{x} = (x_1, x_2, ..., x_n, 0, 0, ...)$;
        \item $X = \ell_2$; $A_n \vec{x} = (\underbrace{0, ..., 0}_{n}, x_1, x_2, ...)$;
        \item $X = \ell_2$; $A_n \vec{x} = (\underbrace{0, ..., 0}_{n}, x_1, 0, 0, ...)$;
        \item $X = \ell_2$; $A_n \vec{x} = (x_1, \frac{1}{2} x_2, \frac{1}{3} x_3, ...)$;
        \item $X = \ell_p \; (1 \leq p \leq \infty)$; $A_n \vec{x} = (\underbrace{0, ..., 0}_{n-1}, x_n, x_{n+1}, ...)$;
        \item $X = \ell_p \; (1 \leq p \leq \infty)$; $A_n \vec{x} = (\underbrace{0, ..., 0}_{n-1}, x_n, 0, 0, ...)$;
        \item $X = \ell_p \; (1 \leq p \leq \infty)$; $A_n \vec{x} = (x_n, x_{n-1}, ..., x_1, 0, 0, ...)$;
        \item $X = L_2 [0; 1]$; $(A_n x)(t) = x(t) \cos(nt)$;
        \item $X = C [0; 1]$; $(A_n x)(t) = t^n x(t)$;
        \item $X = C [0; 1]$; $(A_n x)(t) = e^{-nt} x(t)$;
        \item $X = C [0; 1]$; $(A_n x)(t) = t \cdot \int\limits_0^1 x(s) \sin^n (\frac{\pi s}{2}) ds$.
    \end{enumerate}
\end{exercise}

\begin{exercise}
    Нехай $X, Y$ --- банахові простори, $A, A_n \in L(X, Y)$.
    Довести еквівалентність двох умов:
    \begin{enumerate}
        \item $A_n \overset{s}{\rightarrow} A$;
        \item $\forall x \in X$ послідовність $\left\{ A_n x\right\}$ обмежена в $Y$ та $\exists$ щільна
        в $X$ множина $Z$, для якої $\forall z \in Z$ $A_n z \rightarrow A z$.
    \end{enumerate} 
\end{exercise}

\begin{exercise}\label{N:1_4_13}
    Нехай $X, Y$ --- гільбертові простори, $A : X \rightarrow Y$ --- лінійний оператор. 
    Довести еквівалентність трьох умов:
    \begin{enumerate}
        \item $( x_n \rightarrow x) \Rightarrow (A x_n \rightarrow A x)$;
        \item $(x_n \underset{\text{сл.}}{\rightarrow} x ) \Rightarrow (A x_n \underset{\text{сл.}}{\rightarrow} A x )$;
        \item $( x_n \rightarrow x) \Rightarrow (A x_n \underset{\text{сл.}}{\rightarrow} A x )$.
    \end{enumerate}
\end{exercise}

\begin{exercise}\label{N:1_4_14}
    Нехай $H_1, H_2$ --- гільбертові простори, $A: H_1 \rightarrow H_2$ та $B: H_2 \rightarrow H_1$
    --- лінійні оператори, для яких $\forall x \in H_1, y \in H_2$ виконується $\dotprod{Ax}{y}_2 = \dotprod{x}{By}_1$.
    Доведіть обмеженість операторів $A$ і $B$.
    Зокрема, якщо $H_1 = H_2 = H$ та $A = B$, з рівності $\dotprod{Ax}{y} = \dotprod{x}{Ay}$,
    що виконується для всіх $x, y \in H$, випливає обмеженість (та самоспряженість) оператора $A$ (\ul{теорема Хелінгера-Тепліца}).
\end{exercise}

\begin{exercise*}
    \begin{enumerate}
        \item Доведіть твердження задачі \ref{N:1_4_13} для випадку довільних банахових просторів $X$ та $Y$.
        \item Нехай відображення $A: X \rightarrow Y$ нормованих просторів $X$ та $Y$ задовольняє умову
        $(\varphi \in Y^{*}) \Rightarrow (\varphi \circ A \in X^{*})$. Довести: $A \in L(X, Y)$.
    \end{enumerate}
\end{exercise*}

\begin{exercise}\label{N:1_4_16}
    Нехай $H$ --- гільбертів простір. Довести:
    \begin{enumerate}
        \item $(x_n \underset{\text{сл.}}{\rightarrow} x, y_n \rightarrow y) \Rightarrow (\dotprod{x_n}{y_n} \rightarrow \dotprod{x}{y})$;
        \item $(x_n \rightarrow x) \Leftrightarrow (x_n \underset{\text{сл.}}{\rightarrow} x, \norm{x_n} \rightarrow \norm{x}) 
        \Leftrightarrow (x_n \underset{\text{сл.}}{\rightarrow} x,\; \underset{n \rightarrow \infty}{\overline{\lim}} \norm{x_n} \leq \norm{x})$.
    \end{enumerate}
\end{exercise}

\begin{exercise}
    Нехай $H$ --- гільбертів простір. Побудувати приклади послідовностей $\{x_n\}$ та $\{y_n\}$ в $H$, для яких виконується одна з наступних умов:
    \begin{enumerate}
        \item $x_n \underset{\text{сл.}}{\rightarrow} x$, $y_n \underset{\text{сл.}}{\rightarrow} y$, $\dotprod{x_n}{y_n} \not\rightarrow \dotprod{x}{y}$;
        \item $x_n \underset{\text{сл.}}{\rightarrow} x$, $y_n \underset{\text{сл.}}{\rightarrow} y$, $x_n \not\to x$, $y_n \not \to y$, $\dotprod{x_n}{y_n} \to \dotprod{x}{y}$.
    \end{enumerate}
\end{exercise}

\begin{exercise}
    Нехай $X$ --- банахів простір, $x, x_n \in X$, $\varphi, \varphi_n \in X^{*}$.
    Довести збіжність $\varphi_n(x_n) \to \varphi(x)$, якщо виконується одна з наступних умов:
    \begin{enumerate}
        \item $x_n \to x$, $\varphi_n \to \varphi$;
        \item $x_n \underset{\text{сл.}}{\rightarrow} x$, $\varphi_n \to \varphi$;
        \item $x_n \to x$, $\varphi_n \underset{\text{сл.}}\rightarrow \varphi$.
    \end{enumerate}
\end{exercise}
            % !TEX root = ../main.tex

\begin{exercise}
    Нехай $\{x_n\}$ --- ортогональна система векторів в гільбертовому просторі $H$.
    Довести еквівалентність наступних тверджень:
    \begin{enumerate}
        \item ряд $\sum\limits^\infty_{n=1} x_n$ збігається сильно;
        \item ряд $\sum\limits^\infty_{n=1} x_n$ збігається слабо (тобто слабо збігається
        послідовність його часткових сум);
        \item числовий ряд $\sum\limits^\infty_{n=1} \norm{x_n}^2$ --- збіжний.
    \end{enumerate}
\end{exercise}

\begin{theory}
    Послідовність $\{x_n\}$ в нормованому просторі $X$ називається \ul{слабо фундаментальною},
    якщо для кожного $\varphi \in X^*$ числова послідовність $\{\varphi(x_n)\}$ є фундаментальною.
    
    Послідовність операторів $A_n \in L(X,Y)$ називається \ul{сильно фундаментальною},
    якщо для кожного $x \in X$ послідовність векторів $\{A_n x\} \subset Y$ є фундаментальною
    за нормою.
\end{theory}

\begin{exercise}
    \begin{enumerate}
        \item Довести, що слабо збіжна послідовність векторів в банаховому просторі 
        є слабо фундаментальною;
        \item Довести, що послідовність векторів $\vec{x}_n = 
        (\underbrace{1,\dots,1}_{n},0,0,\dots)$ є слабо фундаментальною, але не слабо 
        збіжною в банаховому просторі $c_0$;
        \item Довести, що слабо фундаментальна послідовність в гільбертовому просторі 
        є слабо збіжною.
    \end{enumerate}
\end{exercise}

\begin{exercise}
    Довести, що послідовність $\vec{x}_n$ векторів простору $\ell_2$ слабо збігається тоді
    й тільки тоді, коли виконуються дві умови:
    \begin{enumerate}
        \item послідовність $\vec{x}_n = (x_n^1,x_n^2,\dots)$ рівномірно обмежена, тобто
        $\exists C > 0 \; \forall n$: $\norm{\vec{x}_n} \leq C$;
        \item $\forall k \in \mathbb{N}$ числова послідовність $\{x_n^k\}$ збігається при
        $n \to \infty$ (<<\ul{покоординатна збіжність}>>).
    \end{enumerate}
\end{exercise}

\begin{exercise}
    Нехай $X$, $Y$ --- банахові простори, послідовність операторів $A_n \in L(X,Y)$ сильно
    фундаментальна. Довести, що існує оператор $A \in L(X,Y)$, до якого послідовність $A_n$
    збігається сильно (<<\ul{повнота $L(X,Y)$ відносно сильної операторної збіжності}>>).
\end{exercise}

\begin{exercise}
    Нехай $X$, $Y$ --- банахові простори, $A, A_n \in L(X,Y)$, $x, x_n \in X$,
    $A_n \overset{s}{\to} A$, $x_n \to x$ (за нормою). Доведіть $A_n x_n \to Ax$ (за нормою).
\end{exercise}

\begin{exercise}
    Нехай $X$, $Y$, $Z$ --- нормовані простори, $A, A_n \in L(X,Y)$, $B, B_n \in L(Y,Z)$,
    $A_n \rightrightarrows A$, $B_n \rightrightarrows B$.
    Довести $B_n A_n \rightrightarrows BA$.
\end{exercise}

\begin{exercise}
    Нехай $X$, $Y$, $Z$ --- банахові простори; $A, A_n \in L(X,Y)$, $B, B_n \in L(Y,Z)$,
    $A_n \overset{s}{\to} A$, $B_n \overset{s}{\to} B$.
    Довести $B_n A_n \overset{s}{\to} BA$.
\end{exercise}

\begin{exercise}\label{N:1_4_26}
    Нехай $H$ --- сепарабельний гільбертів простір, $Z$ --- підмножина в $H$.
    Довести еквівалентність двох умов:
    \begin{enumerate}
        \item $Z$ --- обмежена множина в $H$;
        \item кожна послідовність точок $x_n \in Z$ містить слабо збіжну (в $H$) 
        підпослідовність.
    \end{enumerate}
    Умова б) називається умовою <<\uline{слабкої компактності}>> множини $Z$,
    а твердження задачі є спрощеним варіантом теореми Банаха-Алаоглу.
\end{exercise}

\begin{exercise}
    $A_n$ --- самоспряжені обмежені оператори в гільбертовому просторі $H$,
    $A_n \overset{s}{\to} A$. Довести, що $A$ --- самоспряжений оператор.
    Якщо, на додачу, $A_n \geq 0$ при всіх $n$, то $A \geq 0$.
\end{exercise}

\begin{exercise}
    $A_n$ --- самоспряжені обмежені оператори в гільбертовому просторі $H$,
    $A_1 \leq A_2 \leq \dots$ та $\exists \; C > 0 \; \forall n \in \mathbb{N}$: $\norm{A_n}\leq C$.
    Тоді існує самоспряжений оператор $A$ такий, що $A_n \overset{s}{\to} A$
    (аналог теореми Вейєрштрасса).
\end{exercise}

\begin{exercise}
    Оператори $A_n \in L(L_2[0;1])$ визначено формулою $(A_n x)(t) = a_n(t) x(t)$.
    Дослідити послідовність $\{A_n\}$ на сильну та рівномірну збіжність, якщо:
    \begin{enumerate}
        \item $a_n(t) = t^n$;
        \item $a_n(t) = t^n (1-t)$.
    \end{enumerate}
\end{exercise}

\begin{exercise*}
    Довести, що в просторі $\ell_1$ сильна збіжність послідовності векторів співпадає зі слабкою.
\end{exercise*}
            \newpage
        \section{Цілком неперервні оператори.}
            % !TEX root = ../main.tex

\begin{theory}
    Нехай $X$, $Y$ --- нормовані простори, $A: X \to Y$ --- лінійний оператор.
    Оператор $A$ називається \ul{цілком неперервним} (або \ul{компактним}),
    якщо для кожної обмеженої множини $Z \subset X$ її образ $A(Z)$ є передкомпактною
    множиною в $Y$. 
    Множина всіх компактних операторів з $X$ в $Y$ позначається через 
    $K(X,Y)$, а якщо $X = Y$, то через $K(X)$.
\end{theory}

\begin{exercise}
    Нехай $X$, $Y$ --- нормовані простори, $A: X \to Y$ --- лінійний оператор. Доведіть:
    \begin{enumerate}
        \item якщо образ кулі $B(0;1) = \set{x \in X \mid \norm{x} < 1}$ є передкомпактом в $Y$, то $A$ --- компактний оператор;
        \item якщо $A$ --- компактний оператор, то він обмежений, тобто $K(X,Y) \subset L(X,Y)$.
    \end{enumerate}
\end{exercise}

\begin{exercise}
    Нехай $X$, $Y$ --- нормовані простори, $A, B, A_n \in K(X,Y)$. Довести:
    \begin{enumerate}
        \item $A+B \in K(X, Y)$;
        \item $\lambda A \in K(X, Y)$ (тут $\lambda$ --- число);
        \item $\left( A_n \rightrightarrows C, C\in L(X, Y)\right) \Rightarrow \left( C \in K(X, Y)\right)$.
    \end{enumerate}
\end{exercise}

\begin{exercise}
    Нехай $X$, $Y$, $Z$ --- нормовані простори, $A \in L(X, Y)$, $B \in L(Y, Z)$.
    Довести, що якщо принаймні один з операторів $A$ або $B$ є компактним, то й оператор $BA$ --- компактний.
\end{exercise}

\begin{theory}
    Результати трьох задач вище приводять до висновку: для нормованого простору $X$ $K(X)$ 
    є замкненим двобічним ідеалом в операторній алгебрі $L(X)$.
\end{theory}

\begin{exercise}
    Нехай $X$, $Y$ --- нормовані простори, $A: X \to Y$ --- обмежений лінійний оператор
    скінченного рангу. Довести, що тоді $A \in K(X, Y)$.
\end{exercise}

\begin{exercise}
    Довести, що тотожній оператор в нормованому просторі $X$ є компактним
    тоді й тільки тоді, коли $\mathrm{dim} X < \infty$.
\end{exercise}

\begin{exercise}
    Нехай $A \in K(X)$, $\mathrm{dim} X = \infty$. Довести, що не існує оператора
    $B \in L(X)$, для якого виконується або рівність $AB = I$, або рівність $BA = I$ 
    ($I$ --- тотожний оператор в $X$).
\end{exercise}

\begin{exercise}
    Нехай $X$, $Y$ --- нормовані простори, $A \in K(X, Y)$. Довести, що простір $\left( \mathrm{Im} A, \norm{\cdot}_Y\right)$ --- сепарабельний.
\end{exercise}

\begin{exercise}
    Нехай $X$ та $Y$ --- банахові простори, $A \in K(X, Y)$. $Z \subset \mathrm{Im} A$, 
    $Z$ --- замкнений підпростір в $Y$. Довести: $\mathrm{dim} Z < \infty$.
\end{exercise}

\begin{exercise}
    Довести, що обмежений оператор проектування в банаховому просторі $X$
    є компактним тоді й тільки тоді, коли він скінченновимірний.
\end{exercise}

\begin{exercise}
    Чи вірно, що якщо в нескінченновимірному нормованому просторі $X$ для оператора $A \in L(X)$
    виконується рівність $A^2 = 0$, то $A$ --- компактний оператор?
\end{exercise}

\begin{exercise}
    Нехай $\{e_n\}$ --- ортонормована система в гільбертовому просторі $H$, $A \in K(H)$.
    Довести: $\norm{A e_n} \to 0$, $n \to \infty$.
\end{exercise}

\begin{exercise}
    Нехай $\{a_n\}$ --- числова послідовність. Довести, що оператор $A: \ell_2 \to \ell_2$,
    визначений формулою $A\vec{x} = (a_1 x_1, a_2 x_2, ...)$ є компактним тоді й тільки тоді,
    коли $a_n \to 0$.
\end{exercise}

\begin{exercise}
    Довести, що оператор вкладення $A: C^1 [a;b] \to C[a;b]$, $(Ax)(t) = x(t)$ є компактним.
\end{exercise}
            % !TEX root = ../main.tex

\begin{exercise}
    Довести, що оператор $A: C[a, b] \rightarrow C[a, b]$, визначений формулою : $(Ax)(t) = 
    f(t)x(t)$ ($f \in C[a, b]$; $\exists t \in [a, b] : f(t) \neq 0$) не є компактним.
\end{exercise}

\begin{exercise}
    З'ясувати, які з наведених нижче операторів $A : C[0, 1] \rightarrow C[0, 1]$ 
    є компактними:
    \begin{enumerate}
        \item $(Ax)(t) = x(0) + tx(\frac{1}{2}) + t^2 x(1)$;
        \item $(Ax)(t) = \int\limits_0^t sx(s) ds$
        \item $(Ax)(t) = x(t^2)$
        \item $(Ax)(t) = \int\limits_0^1 x(ts) ds$
    \end{enumerate}
\end{exercise}

\begin{exercise}
    З'ясувати, які з наведених нижче операторів $A : \ell_2 \rightarrow \ell_2$ 
    є компактними:
    \begin{enumerate}
        \item $A\vec{x} = (0, x_1, x_2, x_3, ...)$;
        \item $A\vec{x} = (x_2, x_3, ..., x_{10}, 0, 0, ...)$;
        \item $A\vec{x} = (x_{100}, \frac{1}{2}x_{101} ,\frac{1}{3}x_{102}, ...)$.
    \end{enumerate}
\end{exercise}

\begin{exercise}
    З'ясувати, які з наведених нижче операторів $A : L_2[0, 1] \rightarrow L_2[0, 1]$ 
    є компактними:
    \begin{enumerate}
        \item $(Ax)(t) = \int\limits_0^1 tsx(s) ds$;
        \item $(Ax)(t) = \int\limits_0^t x(s) ds$;
        \item $(Ax)(t) = t x(t)$;
        \item $(Ax)(t) = x(\sqrt{t})$;
        \item $(Ax)(t) = \int\limits_0^t tsx(s) ds$;
    \end{enumerate}
\end{exercise}

\begin{exercise}
    Нехай $F \in C([a, b] \times [a, b])$; оператори $A$ та $B$ задаються в $X$ формулами:
    $(Ax)(t)=\int\limits_a^b F(t, s)x(s)ds$; $(Bx)(t) = \int\limits_a^t F(t, s)x(s)ds$.
    Довести їх компактність у випадках:
    \begin{enumerate}
        \item $X = C[a, b]$;
        \item $X = L_2[a, b]$.
    \end{enumerate}
\end{exercise}

\begin{exercise}
    Нехай $H$ --- сепарабельний гільбертів простір; $A \in K(H)$. Довести: $A$ 
    досягає своєї норми на замкненій одиничній сфері (тобто $\norm{A} = 
    \underset{\norm{x} = 1}{\max}\norm{Ax}$).
\end{exercise}

\begin{exercise}
    Нехай $Z$ --- замкнений підпростір $C[a, b]$, який є підмножиною в $C^1[a, b]$. 
    Довести: $\mathrm{dim} Z < \infty$.
\end{exercise}

\begin{exercise}
    Нехай $X$ --- сепарабельний гільбертів простір; $A \in K(X)$. Довести: образ замкненої 
    одиничної кулі $B[0; 1]$ є компактом.
\end{exercise}

\begin{exercise}
    Довести, що будь-який компактний оператор в гільбертовому просторі є рівномірною 
    границею послідовності скінченновимірних операторів.
\end{exercise}

\begin{exercise}
    Довести, що будь-який обмежений оператор в сепарабельному гільбертовому просторі $H$ є 
    сильною границею послідовності:
    \begin{enumerate}
        \item компактних операторів;
        \item скінченновимірних операторів.
    \end{enumerate}
    А в несепарабельному просторі $H$ це вірно?
\end{exercise}

\begin{exercise}
    Довести, що оператор диференціювання $Ax = x^\prime$, що діє з $C^1[a, b]$ в $C[a, b]$ 
    не є компактним.
\end{exercise}
            % !TEX root = ../main.tex

\begin{exercise}
    Чи є компактним оператор $(Ax)(t) = \frac{1}{2}\left(x(t)+x(-t)\right)$,
    що діє в просторі $C[-1;1]$?
\end{exercise}

\begin{exercise}\label{N:1_5_26}
    Нехай $H_1$, $H_2$ сепарабельні гільбертові простори, $A: H_1 \to H_2$ ---
    лінійний оператор. Довести еквівалентність наступних умов:
    \begin{enumerate}
        \item $A$  --- компактний;
        \item $(x_n \underset{\text{сл.}}{\to} x; \; y_n \underset{\text{сл.}}{\to} y)
        \Rightarrow \left(\dotprod{Ax_n}{y_n} \to \dotprod{Ax}{y}\right)$;
        \item $(x_n \underset{\text{сл.}}{\to} x) \Rightarrow (Ax_n \to Ax \;\text{(сильно)})$.
    \end{enumerate}
\end{exercise}

\begin{exercise*}
    Довести, що будь-який обмежений оператор $A: \ell_2 \to \ell_1$ є компактним.
\end{exercise*}

\begin{exercise}
    Нехай оператор $A: \ell_1 \to \ell_2$ є вкладенням, тобто $A\vec{x} = \vec{x}$.
    Чи буде $A$ компактним?
\end{exercise}

\begin{exercise}
    Чи може ненульовий компактний оператор $A$ в нескінченновимірному нормованому
    просторі задовольняти рівняння $\sum\limits^n_{k=0} c_k A^k = 0$, де $A^0 \coloneqq I$?
\end{exercise}

\begin{exercise}
    Нехай $H$ --- гільбертів простір, $A \in L(H)$. Довести, що оператори $A$, $A^*$,
    $A^* A$ одночасно компактні або некомпактні.
\end{exercise}

\begin{exercise}\label{N:1_5_31}
    Нехай $A$, $B$, $C$ --- самоспряжені обмежені оператори в гільбертовому просторі $H$;
    $A$, $C$ --- компактні, $A \leq B \leq C$. Довести $B$ --- компактний.
\end{exercise}

\begin{exercise}
    Нехай $A$ --- оператор правого зсуву в $\ell_2$: $A\vec{x} = (0,x_1,x_2,\dots)$.
    Довести $(B \in K(\ell_2), AB=BA) \Rightarrow (B=0)$.
\end{exercise}

\begin{exercise}\label{N:1_5_33}
    Нехай $H$ --- гільбертів простір; $A \in K(H)$; $x \in H$, $\norm{x} = 1$;
    $\norm{Ax} = \norm{A}$. Довести, що оператор $A$ переводить $\{x\}^\perp$
    в $\{Ax\}^\perp$
\end{exercise}

\begin{exercise}
    Довести, що кожний компактний оператор $A$ в $\ell_2$ дорівнює сумі $A_1 + A_2$,
    де $A_1$ --- оператор скінченного рангу, а $\norm{A_2}<1$.
\end{exercise}

\begin{exercise}
    Нехай $A$ --- компактний оператор в гільбертовому просторі $H$, $M$ ---
    замкнена опукла обмежена множина в $H$. Довести:
    \begin{enumerate}
        \item $A(M)$ --- замкнена множина в $H$;
        \item $\forall y \in H$ $\exists x_0 \in M$: $\rho(A(M),y) = \norm{Ax_0 - y}$.
    \end{enumerate}
\end{exercise}

\begin{exercise}
    Нехай $X$ --- банахів простір, $A \in L(X)$ та існує таке $m>0$, що для кожного
    $x \in X$ виконується нерівність $\norm{Ax} \geq m \norm{x}$. Чи може $A$ бути
    компактним оператором?
\end{exercise}

\begin{exercise}
    Чи може образ компактного оператора бути замкненим?
\end{exercise}

\begin{exercise}
    Навести приклад оператора $A$, який не є компактним, але $A^2$ --- компактний.
\end{exercise}

\begin{theory}
    Нехай $H$ --- сепарабельний нескінченновимірний гільбертів простір, $\{e_n\}$ ---
    ортонормований базис в $H$; $A \in L(H)$. $A$ називається \ul{оператором
    Гільберта-Шмідта}, якщо зібгається ряд $\sum\limits^\infty_{n=1} \norm{A e_n}^2$.

    Множину всіх операторів Гільберта-Шмідта в $H$ позначимо через $S_2(H)$. 
\end{theory}
            % !TEX root = ../main.tex

\begin{exercise}
    Нехай $\{e_n\}$ та $\{f_n\}$ --- два ортонормованих базиси в гільбертовому просторі $H$, 
    $A \in L(H)$. Довести рівність: 
    $\sum\limits_{n = 1}^\infty \norm{A e_n}^2 = \sum\limits_{n = 1}^\infty \norm{A f_n}^2$ (точніше: якщо
    один ряд збігається, то збігається інший та вони мають однакові суми).
\end{exercise}

\begin{theory}
    Число $\norm{A}_2 = \left(\sum\limits_{n = 1}^\infty \norm{A e_n}^2\right)^{\frac{1}{2}}$ називається
    \underline{абсолютною нормою} оператора $A$.
\end{theory}

\begin{exercise}
    Нехай $A \in S_2 (H)$; $B \in L(H)$. Довести наступні твердження:
    \begin{enumerate}
        \item $A^* \in S_2 (H)$; $\norm{A^*}_2 = \norm{A}_2$;
        \item $\norm{A} \leq \norm{A}_2$;
        \item $AB \in S_2 (H)$; $BA \in S_2 (H)$; $\norm{AB}_2 \leq \norm{A}_2 \cdot \norm{B}$; 
        $\norm{BA}_2 \leq \norm{A}_2 \cdot \norm{B}$;
        \item $A \in K(H)$.
    \end{enumerate}
\end{exercise}

\begin{exercise}
    Довести наступні твердження:
    \begin{enumerate}
        \item $S_2(H)$ є лінійним простором за стандартними операціями над лінійними операторами; 
        \item $S_2(H)$ є гільбертовим простором із скалярним добутком: 
        $(A, B) = \sum\limits_{n = 1}^\infty \left(A e_n, B e_n\right)$, причому сума цього ряду від вибору
        ортонормованого базису не залежить. Зокрема, $S_2(H)$ --- банахів за нормою $\norm{\cdot}_2$;
        \item $S_2(H)$ є незамкненою множиною в $L(H)$ (за нормою в $L(H)$).
    \end{enumerate}
\end{exercise}

\begin{exercise}\label{N:1_5_42}
    Нехай $\{a_n\}$ --- числова послідовність; оператор $A:\ell_2 \rightarrow \ell_2$ визначений формулою:
    $A \vec{x} = \left(a_1 x_1, a_2 x_2, \dots \right)$. З'ясувати:
    \begin{enumerate}
        \item за якою умовою на послідовність $\{a_n\}$ оператор $A$ буде оператором Гільберта-Шмідта;
        \item за якою умовою на послідовність $\{a_n\}$ оператор $A$ буде компактним, але не оператором Гільберта-Шмідта?
    \end{enumerate}
\end{exercise}

\begin{theory}
    Лінійний оператор $A$ на сепарабельному гільбертовому просторі називається \underline{ядерним}, якщо він
    може бути представлений як добуток: $A = BC$, де $B, C \in S_2(H)$. Множину всіх ядерних операторів позначимо
    через $S_1(H)$.
\end{theory}

\begin{exercise}
    Довести: $S_1(H) \subset S_2(H)$.
\end{exercise}

\begin{exercise}
    За якої умови на $\{a_n\}$ оператор в $\ell_2$ із задачі \ref{N:1_5_42} буде:
    \begin{enumerate}
        \item ядерним;
        \item оператором Гільберта-Шмідта, але не ядерним.
    \end{enumerate}
\end{exercise}

\begin{exercise}
    Довести:
    \begin{enumerate}
        \item $(A \in S_1(H)) \Rightarrow (A^* \in S_1(H))$;
        \item $(A \in S_1(H)) \Rightarrow$ (число $Tr A = \sum\limits_{n = 1}^\infty (A e_n, e_n)$ є скінченним
        і не залежить від вибору ортонормованого базису $\{e_n\}$);
        \item $(A \in S_1(H); B \in L(H)) \Rightarrow (AB \in S_1(H); BA \in S_1(H))$; 
        \item[г)*] $(A \in S_1(H)) \Rightarrow$ (ряд $\sum\limits_{n = 1}^\infty |(A e_n, e_n)|$ збіжний для будь-якого ортонормованого базису).
    \end{enumerate}
\end{exercise}

\begin{exercise}
    Нехай $A, B \in S_2(H)$. Довести: $Tr (AB) = Tr (BA)$.
\end{exercise}

            \newpage  
        \section{Обернені оператори.}
            % !TEX root = ../main.tex

\begin{theory}
    Нехай $X$, $Y$ --- нормовані простори з нормами $\norm{\cdot}_X$ та 
    $\norm{\cdot}_Y$ відповідно; $I_X$ та $I_Y$ --- відповідні тотожні 
    оператори; $A \in L(X, Y)$. Оператор $B \in L(Y, X)$ називається 
    \ul{лівим оберненим до $A$}, якщо $BA = I_X$; $C \in L(Y, X)$ 
    називається \ul{правим оберненим до $A$}, якщо $AC = I_Y$. В разі якщо оператор 
    $B \in L(Y, X)$ є одночасно лівим і правим оберненим до $A$, він називається 
    (\ul{неперервно}) \ul{оберненим} до $A$ (позначення: $B = \inv{A}$), 
    а сам оператор $A$ --- \ul{неперервно оборотним}.
\end{theory}

\begin{exercise}
    Нехай оператор $A \in L(X, Y)$ має лівий обернений $B \in L(Y, X)$ та правий обернений 
    $C \in L(Y, X)$. Доведіть: 
    \begin{enumerate}
        \item $Im A = Y$; $Im B = X$;
        \item $Ker A = {0}$; $Ker C = {0}$;
        \item $B = C$ --- обернений оператор до $A$.
    \end{enumerate}
\end{exercise}

\begin{exercise}
    Нехай $A \in L(X, Y)$. Довести, що наступні дві умови еквівалентні:
    \begin{enumerate}
        \item $A$ --- неперервно оборотний;
        \item $Im A = Y$ та $\exists m > 0$ таке, що $\forall x \in X:\norm{Ax} \geq m \norm{x}$. 
    \end{enumerate}
\end{exercise}

\begin{exercise}
    Нехай $dim X < \infty$; $A$ --- лінійний оператор в $X$. Тоді $A$ --- неперервно оборотний 
    тоді й тільки тоді, коли $det A \neq 0$.
\end{exercise}

\begin{exercise}
    Нехай $X$ --- банахів простір; $A \in L(X)$; $\norm{A} < 1$. Тоді оператор $I-A$ --- 
    неперервно оборотний і при цьому $\inv{I-A} = I + \sum\limits_{n=1}^\infty A^n$.
\end{exercise}

\begin{exercise}
    Нехай оператор $A: \ell_2 \rightarrow \ell_2$ визначено формулою : 
    $A\vec{x} = (a_1x_1, a_2x_2, ...)$, де $\underset{n}{sup}|a_n| < \infty$. Довести: 
    $A$ --- неперервно оборотний тоді й тільки тоді, коли $\underset{n}{inf}|a_n| > 0$. 
    Знайти $\inv{A}$.
\end{exercise}

\begin{exercise}
    Нехай $X$ --- банахів простір; $A, B \in L(X)$. Нехай $A$ --- неперервно оборотний; 
    $\norm{B} < \inv{\norm{\inv{A}}}$. Довести: $A + B$ неперервно оборотний і знайти 
    $\inv{(A+B)}$.
\end{exercise}

\begin{exercise}
    Доведіть, що множина неперервно оборотних операторів в банаховому просторі $X$ є відкритою 
    в $L(X)$. 
\end{exercise}

\begin{exercise}
    Нехай $X$ --- нормований простір; $A \in L(X)$; $n \in \mathbb{N}$. Довести : оператори 
    $A$ та $A^n$ одночасно неперервно оборотні чи ні.
\end{exercise}

\begin{exercise}
    Нехай $A, B \in L(X)$; оператори $A$, $BA$ --- неперервно оборотні. Довести: оператор 
    $B$ також неперервно оборотний.
\end{exercise}

\begin{exercise}
    Нехай $X$ --- нормований простір; $A, B \in L(X)$; оператор $(I - AB)$ --- неперервно 
    оборотний; $\inv{(I - AB)} = C$. Довести: $(I - BA)$ --- неперервно оборотний і знайти 
    $\inv{(I - BA)}$.
\end{exercise}

\begin{exercise}
    Нехай $A: C[0, 1] \rightarrow C[0, 1]$ визначений формулою: $(Ax)(t) = \alpha(t)x(t)$, 
    де $\alpha \in C[0, 1]$ --- фіксована функція. Довести: $A$ -- неперервно оборотний 
    тоді й тільки тоді, коли $\alpha(t) \neq 0$ для всіх $t \in [0, 1]$. 
\end{exercise}

\begin{exercise}
    Нехай $A, B : C[-1, 1] \rightarrow C[-1, 1]$ визначено формулами: $(Ax)(t) = 
    x(t^2)$; $(Bx)(t) = x(t^3)$. Доведіть, що $\inv{A}$ не існує, а $B$ --- неперервно 
    оборотний і знайти $\inv{B}$.
\end{exercise}
            % !TEX root = ../main.tex

\begin{exercise}
    Оператори $A: \ell_2 \to \ell_2$ дослідити на неперервну оборотність:
    \begin{enumerate}
        \item $A\vec{x} = (0, x_1, x_2, ...)$;
        \item $A\vec{x} = (x_2, x_3, x_4, ...)$;
        \item $A\vec{x} = (x_1+x_2, x_2, x_3, ...)$;
        \item $A\vec{x} = (x_3, x_4, x_2, x_1, x_5, x_6, x_7, ...)$;
        \item $A\vec{x} = (x_1+x_2, x_1 - x_2 + x_3, x_2 + x_3 + x_4, x_4, x_5, ...)$.
    \end{enumerate}
\end{exercise}

\begin{exercise}
    $A \in L(X, Y)$. Довести, що якщо обернений оператор $\inv{A}$
    існує, то він єдиний.
\end{exercise}

\begin{exercise}
    Нехай $A, B \in L(X)$ і є неперервно оборотними.
    Довести: $AB$ --- неперервно оборотний і при цьому $\inv{(AB)} = \inv{B} \inv{A}$.
\end{exercise}

\begin{exercise}
    $A: C[0;1] \to C[0;1]$. Дослідити $A$ на неперервну оборотність та знайти $\inv{A}$ (якщо існує):
    \begin{enumerate}
        \item $(Ax)(t) = \int\limits_0^t x(s)ds$;
        \item $(Ax)(t) = x(t) - \int\limits_0^t x(s)ds$;
        \item $(Ax)(t) = x(t) - \int\limits_0^1 t s x(s) ds$.
    \end{enumerate}
\end{exercise}

\begin{exercise}
    При яких $\lambda \in \real$ оператор $(A_\lambda x)(t) = x(t) + \lambda \int\limits_0^1 (t+s) x(s) ds$,
    що діє в просторі $C[0;1]$, має неперервний обернений? Знайти $\inv{A_\lambda}$.
\end{exercise}

\begin{exercise}
    Оператор $A$ в просторі $L_2 [0;1]$ визначено формулою $(Ax)(t) = x(t) - \int\limits_0^t x(s) ds$.
    Дослідити $A$ на неперервну оборотність та знайти $\inv{A}$, якщо він існує.
\end{exercise}

\begin{exercise}
    Нехай $A, B \in L(X)$, $B$ --- неперервно оборотний.
    Довести $\norm{AB} \geq \norm{A} \cdot \norm{\inv{B}}^{-1}$.
\end{exercise}

\begin{exercise}
    $X$ --- банахів простір, $A \in L(X)$, $\left| \lambda\right| > \norm{A}$.
    Довести: $A - \lambda I$ --- неперервно оборотний оператор.
\end{exercise}

\begin{exercise}
    Нехай $H$ --- гільбертів простір, $A \in L(H)$, $\underset{\norm{x} = 1}{\inf} \norm{Ax} > 0$, $\mathrm{Ker} A^* = \{0\}$.
    Довести: $A$ --- неперервно оборотний.
\end{exercise}

\begin{exercise}\label{N:1_6_22}
    Розглянемо в просторах $C[0;1]$ та $L_2 [0;1]$ оператор взяття первісної \\
    $(Ax)(t) = \int\limits_0^t x(s) ds$. Довести, що у цього оператора немає
    ані лівого, ані правого оберненого.
\end{exercise}

\begin{exercise}
    Нехай оператор $A: C^1 [0;1] \to C[0;1]$ визначено формулою $(Ax)(t) = x^\prime(t)$.
    Довести, що він не є оборотним. Знайти правий обернений оператор.
\end{exercise}

\begin{exercise}
    Оператор $A$ в просторі $C[0;1]$ визначено формулою $(Ax)(t) = x(t) + \int\limits_0^1 e^{t+s}x(s) ds$.
    Довести його неперервну оборотність та знайти $\inv{A}$.
\end{exercise}

\begin{exercise}\label{N:1_6_25}
    Нехай $A: \vec{x} \mapsto (0, x_1, x_2, ...)$ --- оператор правого зсуву в $\ell_2$.
    Нехай $B \in L(\ell_2)$, $\norm{B} < 1$. Довести, що оператор $A+B$ не є оборотним.
\end{exercise}

\begin{theory}
    \ul{Наслідок:} множина оборотних операторів не щільна в $L(\ell_2)$.
\end{theory}
            % !TEX root = ../main.tex

\begin{theory}
    \begin{theorem*}[Банаха про обернений оператор]
        Нехай $X$, $Y$ --- банахові простори; $A \in L(X,Y)$;
        $A$ --- бієктивне  відображення. Тоді $A$ --- неперервно оборотний
        $\left(\exists A^{-1} \in L(Y,X)\right)$.
    \end{theorem*}
\end{theory}

\begin{exercise}
    Нехай $X$, $Y$ --- банахові простори; $A \in L(X,Y)$.
    Доведіть, що наступні три умови еквівалентні:
    \begin{enumerate}
        \item $A$ --- неперервно оборотний;
        \item $A$ має і при тому єдиний правий обернений оператор;
        \item[в)*] $A$ має і при тому єдиний лівий обернений оператор.
    \end{enumerate}
\end{exercise}

\begin{exercise}
    Навести приклад оператора $A \in L(X,Y)$ в банахових просторах для якого:
    \begin{enumerate}
        \item існує правий обернений $B\in L(Y,X)$, але немає лівого оберненого;
        \item існує лівий обернений $C\in L(Y,X)$, але немає правого оберненого.
    \end{enumerate}
\end{exercise}

\begin{exercise}
    Нехай $X$, $Y$, $Z$ --- банахові простори; $B \in L(Y,Z)$ --- ін'єктивний
    оператор ($\mathrm{Ker}B = \{0\}$); $A$ --- лінійний оператор з $X$ в $Y$;
    $BA \in L(X,Z)$, $\mathrm{Im}(BA) = Z$. Довести $A \in L(X,Y)$.
\end{exercise}

\begin{exercise}
    Нехай $\norm{\cdot}_1$, $\norm{\cdot}_2$  --- дві норми в лінійному просторі
    $X$, причому обидва простори $(X, \norm{\cdot}_1)$ та $(X, \norm{\cdot}_2)$ 
    повні. Відомо, що існує константа $C > 0$ така, що для $\forall x \in X$:
    $\norm{x}_1 \leq C \norm{x}_2$. Доведіть еквівалентність цих норм.
\end{exercise}

\begin{exercise}
    Нехай $X$ --- нормований простір, $A \in L(X)$ і існують такі комплексні числа
    $\lambda_1, \lambda_2, \dots, \lambda_n$, що $I + \lambda_1 A + \lambda_2 A^2
    + \dots + \lambda_n A^n = 0$. Довести, що $A$ --- неперервно оборотний.
\end{exercise}

\begin{exercise}
    $X$ --- нормований простір; $A, B \in L(X)$, $AB+A+I=BA+A+I=0$.
    Довести, що $A$ --- неперервно оборотний.
\end{exercise}

\begin{exercise}
    $X$ --- нормований простір; $A, B \in L(X)$, $A$ --- оборотний, $AB=BA$.
    Довести: $A^{-1}B = BA^{-1}$.
\end{exercise}

\begin{exercise}
    Нехай $H$ --- гільбертів простір, $A \in L(X)$, $A$ --- самоспряжений
    і існує таке число $c > 0$, для якого $A \geq cI$.
    Довести, що $A$ --- неперервно оборотний.
\end{exercise}

\begin{exercise}
    Навести приклад банахових просторів $X$, $Y$ і послідовності неперервно
    оборотних операторі $A_n \in L(X,Y)$, для яких $A_n \rightrightarrows A$,
    але $A$ не є оборотним.
\end{exercise}

\begin{exercise}
    Навести приклад банахового простору $X$ і неперервно оборотного оператора
    $A \in L(X)$, для якого існує послідовність необоротних операторів
     $A_n \in L(X,Y)$, що $A_n \overset{s}{\to} A$.
\end{exercise}

\begin{exercise}
    Нехай $A, A_n \in L(X, Y)$, $A_n \rightrightarrows A$.
    Довести: $A$ --- неперервно оборотний тоді й тільки тоді, коли
    виконуються дві наступні умови:
    \begin{enumerate}
        \item $\exists N$ $\forall n \geq N$: $A_n$ -- неперервно оборотні;
        \item $\underset{n \geq N}{\sup} \norm{A_n^{-1}} < \infty$.
    \end{enumerate}
\end{exercise}
            \newpage
        \section{Спектр та резольвента обмеженого лінійного оператора. 
                 Спектр та резольвента самоспряженого оператора.}
            % !TEX root = ../main.tex

\begin{theory}
    Нехай $X$ --- комплексний банахів простір; $A \in L(X)$; $\lambda \in \complex$.
    Для оператора $\lambda - A$ є наступні можливості:
    \begin{enumerate}[label=\arabic*)]
        \item ${\mathrm{Ker}(\lambda - A) = {0}}$, ${\mathrm{Im}(\lambda - A)} = X$. В цьому разі 
        за теоремою Банаха про обернений оператор, $\lambda - A$ є неперервно оборотним 
        оператором, а тому $\exists \; \inv{(\lambda - A)} \in L(X)$. Такі числа $\lambda$ 
        називаються \ul{регулярними значеннями}  оператора $A$, а множина 
        {$\rho(A) = \rho_A$} всіх регулярних значень називається 
        \ul{резольвентною множиною} оператора $A$.
    \end{enumerate}
    Операторнозначна функція $\rho_A : \lambda \mapsto \inv{(\lambda - A)} \in L(X)$ 
    називається \ul{резольвентою} оператора $A$ і позначається так:
    $R_A(\lambda) = R(\lambda; A) = R_\lambda(A) = \inv{(\lambda - A)}$
    \begin{enumerate}[label=\arabic*), resume]
        \item $\mathrm{Ker}(\lambda - A) \neq {0}$. В цьому разі існує ненульовий вектор $x_0$, 
        для якого $Ax_0 = \lambda x_0$. $\lambda$ називається 
        \ul{власним числом} оператора $A$; $x_0$ --- відповідним \ul{власним вектором}.
        Множина $\sigma_\text{т}(A)$ всіх власних чисел A утворює 
        \uline{<<точковий спектр>>} оператора $A$.
        \item $\mathrm{Ker}(\lambda - A) = {0}$, $\mathrm{Im}(\lambda - A) \neq X$. 
        У випадку нескінченновимірного простору X такі числа можуть існувати,
        найчастіше їх поділяють на дві частини:
        \begin{enumerate}[label = \ukr*)]
            \item $\overline{\mathrm{Im}(\lambda - A)} = X$;
            \item $\overline{\mathrm{Im}(\lambda - A)} \neq X$;
        \end{enumerate}
        У випадку 3а) такі числа утворюють \uline{<<неперервний спектр>>} оператора $A$ {$\sigma_\text{н}(A)$}; у випадку 3б) такі $\lambda$ 
        утворюють \uline{<<залишковий спектр>>} оператора $A$ {$\sigma_\text{з}(A)$}. 
    \end{enumerate}
    Об'єднання $\sigma_\text{н}(A) \bigvee \sigma_\text{з}(A) \bigvee \sigma_\text{т}(A) = {\sigma(A) = \sigma_A}$ називається 
    \ul{спектром} оператора $A$. Тож $\complex = \rho(A) \bigvee \sigma(A)$.
\end{theory}

\begin{exercise}
    \begin{enumerate}
        \item Доведіть, що у випадку $\dim X < \infty$ має місце рівність: 
        $\sigma_\text{т}(A) = \sigma(A)$ для будь-якого лінійного оператора 
        $A: X \rightarrow X$.
        \item Наведіть приклад комплексного банахового простору $X$ і 
        оператора $A \in L(X)$, для якого: $\mathrm{Ker} A = {0}$ та $\mathrm{Im} A \neq X$.
    \end{enumerate}
\end{exercise}

\begin{exercise}\label{N:1_7_2}
    Нехай $A \in L(X)$, $|\lambda| > \norm{A}$. Доведіть, що $\lambda \in 
    \rho(A)$.
\end{exercise}

\begin{theory}
    Число $r(A) = \sup\{|\lambda| : \lambda \in \sigma(A)\}$ називається 
    \ul{спектральним радіусом} оператора $A$. За результатом задачі \ref{N:1_7_2} 
    маємо нерівність: $r(A) < \norm{A}$.
\end{theory}

\begin{exercise}
    Нехай $A \in L(X)$, де $X$ --- комплексний банахів простір. Доведіть: 
    $\sigma(A)$ --- компакт в $\complex$.
\end{exercise}

\begin{exercise}
    Нехай $A \in L(X)$. Довести: $r(A) = \max\{|\lambda| : \lambda \in \sigma(A)\}$.
\end{exercise}

\begin{exercise}
    Нехай $\{a_n\}$ --- обмежена числова послідовність в $\complex$; оператор $A: \ell_2 
    \rightarrow \ell_2$ визначено формулою: $A\vec{x} = (a_1 x_1, a_2 x_2, ...)$. 
    Знайти $\sigma_\text{т}(A)$, $\sigma_\text{н}(A)$, $\sigma_\text{з}(A)$, $r(A)$ та 
    побудувати резольвенту $R_\lambda (A)$.
\end{exercise}

\begin{exercise}
    Нехай $P$ --- ортопроектор в гільбертовому просторі $H$. Знайти 
    $\sigma_\text{т}(P)$, $\sigma_\text{н}(P)$, $\sigma_\text{з}(P)$, $R_\lambda(P)$.
\end{exercise}

\begin{exercise}
    Для оператора $A \in L(C[0, 1])$ знайти 
    $\sigma_\text{т}(A)$, $\sigma_\text{н}(A)$, $\sigma_\text{з}(A)$; $r(A)$, $R_\lambda(A)$, 
    якщо:
    \begin{enumerate}
        \item $(Ax)(t) = tx(t)$;
        \item $(Ax)(t) = a(t)x(t)$, де $a \in C[0, 1]$;
        \item $(Ax)(t) = x(0) + tx(1)$;
        \item $(Ax)(t) = \int\limits_0^t x(s) ds$.
    \end{enumerate}
\end{exercise}

\begin{exercise}
    Для оператора $A \in L(\ell_2)$ знайти 
    $\sigma_\text{т}(A)$, $\sigma_\text{н}(A)$, $\sigma_\text{з}(A)$, $r(A)$, якщо:
    \begin{enumerate}
        \item $A\vec{x} = (x_1 + x_2, x_2, x_3, ...)$;
        \item $A\vec{x} = (x_3 , x_1, x_2, x_4, x_5, ...)$;
        \item $A\vec{x} = (x_1 , x_2, x_3, 0, 0, ...)$;
        \item $A\vec{x} = (-x_1, x_2, -x_3, ..., (-1)^n x_n, ...)$.
    \end{enumerate}
\end{exercise}

\begin{exercise}
    Нехай $H$ --- гільбертів простір. оператор $A \in L(H)$ визначений формулою: 
    $Ax = (x, x_0)x_1$, де $x_0$, $x_1$ --- фіксовані вектори. Знайти
    $\sigma_\text{т}(A)$, $\sigma_\text{н}(A)$, $\sigma_\text{з}(A)$, $r(A)$, $R_\lambda(A)$.
\end{exercise}

\begin{exercise}
    Довести, що для $\lambda, \mu \in \rho(A)$ виконуються рівності: 
    \begin{enumerate}
        \item $AR_\lambda(A) = R_\lambda(A)A$;
        \item $R_\lambda(A) - R_\mu(A) = (\mu - \lambda)R_\lambda R_\mu = (\mu - \lambda) 
        R_\mu R_\lambda$ (\ul{тотожність Гільберта}).
    \end{enumerate}
\end{exercise}
            % !TEX root = ../main.tex

\begin{exercise}
    Довести, що для обмеженого оператора в гільбертовому просторі виконується
    рівність $\sigma(A^{*}) = \set{\overline{\lambda} \mid \lambda \in \sigma (A)}$.
\end{exercise}

\begin{exercise}
    Нехай $X$ --- комплексний банахів простір, $A \in L(X)$. Довести:
    \begin{enumerate}
        \item $\left( \lambda \in \sigma(A)\right) \Rightarrow \left( \lambda^n \in \sigma(A^n)\right)$;
        \item $\sigma(A^n) = \set{\lambda^n \mid \lambda \in \sigma(A)}$ ($n\in \natur$);
        \item $\sigma(p(A)) = \set{p(\lambda) \mid \lambda \in \sigma(A)}$, де $p(A) = a_0 A^n + a_1 A^{n-1} + ... + a_{n-1} A + a_n I$ ($a_k \in \complex$);
        \item якщо $A$ --- неперервно оборотний, то $\left( \lambda \in \sigma(A)\right) \Rightarrow \left( \lambda^{-1} \in \sigma(\inv{A})\right)$.
    \end{enumerate}
    Які з цих тверджень мають місце і для дійсного банахового простору?
\end{exercise}

\begin{theory}
    \begin{theorem*}
        $r(A) = \underset{n \to \infty}{\lim} \sqrt[\leftroot{-3}\uproot{3}n]{\norm{A^n}}$
    \end{theorem*}
    \begin{theorem*}
        $\left( A \in L(X)\right) \Rightarrow (\sigma(A) \neq \varnothing)$
    \end{theorem*}
\end{theory}

\begin{exercise}
    Нехай $A, B \in L(\ell_2)$ є відповідно операторами лівого та правого зсуву:
    $A \vec{x} = (x_2, x_3, ...)$, $B\vec{x} = (0, x_1, x_2, x_3, ...)$.
    Знайти $\sigma_\text{т}(A)$, $\sigma_\text{н}(A)$, $\sigma_\text{з}(A)$,
    $\sigma_\text{т}(B)$, $\sigma_\text{н}(B)$, $\sigma_\text{з}(B)$.
\end{exercise}

\begin{exercise}
    Для операторів $A \in L(L_2 [0;1])$ знайти $\sigma_\text{т}(A)$, $\sigma_\text{н}(A)$, $\sigma_\text{з}(A)$,
    $r(A)$, $R_\Lambda(A)$, якщо:
    \begin{enumerate}
        \item $(Ax)(t) = t x(t)$;
        \item $(Ax)(t) = a(t) x(t)$, де $a(\cdot) \in C[0;1]$;
        \item $(Ax)(t) = \int\limits_0^t x(s) ds$.
    \end{enumerate}
\end{exercise}

\begin{exercise}
    Оператор $A \in L(C[0; \frac{1}{2}])$ визначено формулою $(Ax)(t) = t x(t^2)$. Довести: $r(A) = 0$.
\end{exercise}

\begin{exercise}
    Знайти спектральний радіус оператора $A \in L(X)$, визначеного формулою
    $(Ax)(t) = \int\limits_0^t K(t, s) x(s) ds$, де $K \in C([0;1] \times [0;1])$ в разі, якщо:
    \begin{enumerate}
        \item $X = C[0;1]$;
        \item $X = L_2 [0;1]$.
    \end{enumerate}
\end{exercise}

\begin{exercise}
    Оператор $A \in L(C[0;1])$ визначено як $(Ax)(t) = x(t) + \int\limits_0^t e^{t-s} x(s) ds$.
    Обчислити $r(A)$.
\end{exercise}

\begin{exercise}
    Нехай $X$ --- банахів простір, $A \in L(X)$. Довести, що спектральний радіус $A$ не зміниться, якщо
    в $X$ перейти до еквівалентної норми.
\end{exercise}

\begin{exercise}
    Довести, що будь-який непорожній компакт в $\complex$ є спектром деякого обмеженого оператора.
\end{exercise}

\begin{exercise}
    Нехай $X$ --- банахів простір, $A \in L(X)$, $\lambda \in \complex$. Довести,
    що якщо існує така послідовність $\{ x_n\} \subset X$, для якої $\norm{x_n} = 1$ ($\forall n \in \natur$)
    і $\underset{n \to \infty}{\lim} (A x_n - \lambda x_n) = 0$, то $\lambda \in \sigma (A)$.
\end{exercise}

\begin{exercise}
    Нехай $X$ --- банахів простір, $A, B \in L(X)$, $AB = BA$. Довести:
    \begin{enumerate}
        \item $r(A B) \leq r(A) \cdot r(B)$;
        \item[б)*] $r(A+B) \leq r(A) + r(B)$.
    \end{enumerate}
\end{exercise}

\begin{exercise}
    Нехай $A, B \in L(X)$. Довести:
    \begin{enumerate}
        \item $\left( 0 \notin \sigma(A)\right) \Rightarrow \left( \sigma(AB) = \sigma(BA)\right)$;
        \item $\sigma(AB) \setminus \{ 0 \} = \sigma(BA) \setminus \{ 0 \}$;
        \item $r(AB) = r(BA)$;
        \item $\left( AB - BA = \lambda I \right) \Rightarrow \left( \lambda = 0\right)$.
    \end{enumerate}
\end{exercise}

\begin{exercise}
    Нехай $A \in L(X)$, $A^2$ має власний вектор. Довести: $A$ має власний вектор.
\end{exercise}

\begin{exercise}
    $A \in L(X)$, $| \lambda | > r(A)$. 
    Довести: $\norm{R_\lambda (A)} \leq \left( | \lambda | - \norm{A}\right)^{-1}$.
\end{exercise}

\begin{exercise}
    $A \in L(X)$, $\lambda \in \rho (A)$, $\mu \in \complex$, $|\mu| \leq \norm{R_\lambda (A)}^{-1}$.
    Довести: $\lambda - \mu \in \rho (A)$.
\end{exercise}

\begin{exercise}
    $A \in L(X)$. Чи може $R_\lambda (A)$ бути цілком неперервним оператором?
\end{exercise}

\begin{exercise}
    Нехай $A, A_n \in L(X)$, $A_n \rightrightarrows A$. Довести:
    \begin{enumerate}
        \item $\left( \lambda \in \rho(A) \right) \Rightarrow \left( \exists \; N \; \forall \; n \geq N : \lambda \in \rho(A_n) \right)$;
        \item $\forall \; \varepsilon\text{-окола } (\sigma(A))_\varepsilon \text{ спектра } A \; \exists \; N \; \forall \; n \geq N : \sigma(A_n) \subset (\sigma(A))_\varepsilon $.
    \end{enumerate}
\end{exercise}

\begin{exercise}
    $A : C[0;1] \to C[0;1]$, $(Ax)(t) = x(t^2)$. Довести: $\sigma(A) \subset \set{\lambda \mid |\lambda| = 1}$.
\end{exercise}

\begin{exercise}
    Нехай $A$ --- самоспряжений оператор в гільбертовому просторі. Довести:
    \begin{enumerate}
        \item $r(A) = \norm(A)$;
        \item $\sigma_\text{з}(A) = \varnothing$;
        \item $\left( \mathrm{Im} (A - \lambda I) = H\right) \Rightarrow \left( \lambda \in \rho(A)\right)$;
        \item $\sigma_\text{т}(A) \subset \left[-\norm{A}; \norm{A}\right]$;
        \item $\sigma (A) \subset \left[-\norm{A}; \norm{A}\right]$.
    \end{enumerate}
\end{exercise}
            % !TEX root = ../main.tex

\begin{exercise}
    Нехай $A$ --- самоспряжений обмежений оператор в гільбертовому просторі,
    $m = \inf\set{\dotprod{Ax}{x} \mid \norm{x}=1}$,
    $M = \sup\set{\dotprod{Ax}{x} \mid \norm{x}=1}$.
    Довести:
    \begin{enumerate}
        \item $\sigma(A) \subset [m,M]$;
        \item $m,M \in \sigma(A)$.
    \end{enumerate}
\end{exercise}

\begin{exercise}
    Нехай $A$ --- самоспряжений обмежений оператор в $H$. Довести $(A \geq 0)
    \Leftrightarrow (\sigma(A) \subset [0,+\infty])$.
\end{exercise}

\begin{exercise}
    Нехай $A$ --- нормальний оператор в гільбертовому просторі $(AA^* = A^* A)$.
    Довести: $r(A) = \norm{A}$.
\end{exercise}

\begin{exercise}
    $A \in L(H)$ ($H$ --- гільбертів простір). Доведіть: $AA^*$ та $A^* A$ мають
    однакові не нульові власні числа. А нульові?
\end{exercise}

\begin{exercise}
    Нехай $A$ --- самоспряжений обмежений оператор в $H$, $\sigma(A) = \sigma_T(A)
    = \{0,1\}$. Довести: $A$ --- ортогональний проектор.
\end{exercise}

\begin{exercise}
    Нехай $A$ --- нормальний оператор в комплексному гільбертовому просторі;
    $\sigma(A) \subset \real$. Довести: $A$ --- самоспряжений.
\end{exercise}

\begin{exercise}
    Нехай $A \in L(H)$.
    \begin{enumerate}
        \item\label{N:1_7_36_a} Довести: $\norm{A} = (r(A^*A))^{\frac{1}{2}}$;
        \item Користуючись формулою пункту \ref{N:1_7_36_a} знайти операторну
        норму оператора в $\real^2$, якому відповідає матриця $\begin{pmatrix}
        1 & 2 \\ 3 & 4 \end{pmatrix}$ в канонічному базисі.
    \end{enumerate}
\end{exercise}

\begin{exercise}
    Знайти норму оператора $A: L_2[0;1] \to L_2[0;1]$, що визначений формулою
    $(Ax)(t) = \int\limits^t_0 x(s)ds$.
\end{exercise}
            \newpage
        \section{Спектр компактного оператора, теореми Фредгольма.}
            % !TEX root = ../main.tex

\begin{exercise}
    Нехай $A$ --- компактний оператор в нескінченновимірному нормованому просторі.
    Чи може власний підпростір, що відповідає ненульовому власному числу оператора $A$,
    бути нескінченновимірним? Чи може власний підпростір, що відповідає нульовому власному числу,
    бути нескінченновимірним?
\end{exercise}

\begin{exercise}
    Нехай компактний самоспряжений оператор $A$ в нескінченновимірному
    гільбертовому просторі має скінченну кількість власних чисел. Довести, що $\lambda = 0$
    --- власне число оператора $A$.
\end{exercise}

\begin{theory}
    \begin{theorem*}
        Нехай $A$ --- компактний оператор в гільбертовому просторі. Тоді всі
        ненульові точки $\sigma(A)$ є власними числами і відповідні власні підпростори є скінченновимірними,
        $\sigma(A)$ не більш, ніж зліченний, і єдиною граничною точкою спектра може бути лише $\lambda = 0$.
        Результат має місце і в банаховому просторі.
    \end{theorem*}
\end{theory}

\begin{exercise}
    Нехай $A$ --- компактний оператор в нескінченновимірному банаховому просторі.
    Доведіть, що $0 \in \sigma (A)$.
\end{exercise}

\begin{exercise}
    Чи може бути наступна множина спектром компактного оператора? В разі позитивної відповіді навести приклад.
    \begin{enumerate}
        \item $\{0; 1\}$;
        \item $[0; 1]$;
        \item $\set{\frac{1}{n} \mid n \in \natur}$;
        \item $\set{1-\frac{1}{n} \mid n \in \natur}$;
        \item $\{0\} \cup \set{\frac{1}{n} \mid n \in \natur}$;
        \item $\{0; i\}$.
    \end{enumerate}
\end{exercise}

\begin{exercise}
    Навести приклади компактних операторів $A$ в нескінченновимірному
    банаховому просторі, для яких:
    \begin{enumerate}
        \item $0 \in \sigma_{\text{т}} (A)$;
        \item $0 \in \sigma_{\text{н}} (A)$;
        \item $0 \in \sigma_{\text{з}} (A)$.
    \end{enumerate}
\end{exercise}

\begin{exercise}
    В функціональному просторі $X$ розглядається інтегральний оператор 
    $(Ax)(t) = \int\limits_0^t x(s) ds$. Довести його компактність і дослідити
    $\sigma_{\text{т}} (A)$, $\sigma_{\text{н}} (A)$, $\sigma_{\text{з}} (A)$
    у наступних випадках:
    \begin{enumerate}
        \item $X = C[0;1]$;
        \item $X = L_2 [0;1]$.
    \end{enumerate}
\end{exercise}

\begin{theory}
    Компактний оператор $A$, який діє в банаховому просторі $X$, називається
    \ul{оператором Вольтерра}, якщо $\sigma(A) = 0$.
\end{theory}

\begin{exercise}
    Нехай $K \in C([0;1] \times [0;1])$. Доведіть, що інтегральний оператор
    $(Ax)(t) = \int\limits_0^t K(t,s) x(s)ds$ є оператором Вольтерра
    в просторі $X$ у наступних випадках:
    \begin{enumerate}
        \item $X = C[0;1]$;
        \item $X = L_2 [0;1]$. 
    \end{enumerate}
\end{exercise}

\begin{exercise}
    Довести, що інтегральний оператор в $L_2[a;b]$
    $(Ax)(t) = \int\limits_a^b K(t,s) x(s)ds$ з неперервним
    <<ядром>> $K(t,s)$ є самоспряженим тоді й тільки тоді, коли
    $K(t,s) = \overline{K(s,t)}$ для всіх $(t,s) \in [a;b] \times [a;b]$.
\end{exercise}

\begin{exercise}
    Знайти спектр, власні числа та власні функції оператора $A$,
    що визначений на $L_2[a;b]$, формулою $(Ax)(t) = \int\limits_a^b K(t,s) x(s)ds$, якщо:
    \begin{enumerate}
        \item $K(t,s) = \begin{cases}
            t(1-s), & t \leq s \\
            s(1-t), & s \leq t
        \end{cases}$, $a = 0, b = 1$;
        \item $K(t,s) = \begin{cases}
            \sin(t)\sin(1-s), & t \leq s \\
            \sin(1-t)\sin(s), & s \leq t
        \end{cases}$, $a = 0, b = \pi$;
        \item $K(t,s) = \cos(t-s)$, $a = 0, b = 2\pi$;
        \item $K(t,s) = ts + t^2 s^2$, $a = 0, b = 1$.
    \end{enumerate}
\end{exercise}
            % !TEX root = ../main.tex

\begin{theory}
    Нехай $A$ --- лінійний оператор в гільбертовому просторі $H$. 
    Для оператора $A = I + T$ ($T$ --- компактний оператор) виконуються 
    3 \ul{теореми Фредгольма}:
    \begin{enumerate}
        \item[1)] Рівняння $Ax = y$ має розв'язок для тих і 
        тільки тих $y$, які ортогональні кожному розв'язку рівняння 
        $A^*u = 0$ ($\mathrm{Im} A \oplus \mathrm{Ker} A^* = H$).
        \item[2)] Виконується \ul{альтернатива}: або рівняння $Ax = y$ 
        має і при тому єдиний розв'язок при кожному $y \in H$ або рівняння 
        $Ax = 0$ має ненульовий розв'язок ($0 \in \rho(A) \cup \sigma_\text{т}(A)$).
        \item[3)] Рівняння $Ax = 0$ та $A^* u = 0$ мають і при тому 
        однакову скінченну кількість лінійно незалежних розв'язків 
        ($\dim \mathrm{Ker} A = \dim \mathrm{Ker} A^* < \infty$)
    \end{enumerate}
\end{theory}

\begin{exercise}
    Довести теореми Фредгольма.
\end{exercise}

\begin{exercise}
    При яких $\lambda \in \complex$ рівняння $x(t) = \lambda 
    \intl{a}{b}e^{t-s}x(s)ds$ має ненульовий розв'язок в просторі 
    $L_2[a, b]$?
\end{exercise}

\begin{exercise}\label{N:1_8_12}
    При яких $f \in L_2 [0, \pi]$ інтегральне рівняння $x(t) = 
    \intl{0}{\pi} sin(t-s)x(s)ds + f(t)$ має розв'язок в просторі 
    $L_2[0, \pi]$?
\end{exercise}

\begin{exercise}
    Для оператора $A \in L(\ell_2)$ перевірити, які з теорем 
    Фредгольма і за яких умов виконуються для рівняння $Ax = y$ у 
    наступних випадках:
    \begin{enumerate}
        \item $Ax = (\lambda_1 x_1, \lambda_2 x_2, ...)$, де 
        $\{\lambda_n\}_{n=1}^\infty$ --- обмежена послідовність;
        \item $A$ --- оператор правого зсуву в $\ell_2$;
        \item $A$ --- оператор лівого зсуву в $\ell_2$;
        \item[г)*] $A$ --- сума операторів правого та лівого 
        зсуву в $\ell_2$;
    \end{enumerate}
\end{exercise}

\begin{exercise}
    Нехай $\{e_n\}$ --- ортонормований базис гільбертового простору $H$,
    $\lambda_n \in \real$; $\lambda_n \rightarrow 0$,
    $A: H \rightarrow H$ --- 
    такий лінійний оператор, що для кожного $x \in H$ має місце рівність:
    $Ax = \suml{n=1}{\infty} \lambda_n \dotprod{x}{e_n}e_n$.
    Довести, що $A$ --- цілком неперервний самоспряжений оператор.
\end{exercise}

\begin{theory}
    \begin{theorem*} [Гільберта-Шмідта]
    Нехай $H$ --- гільбертів простір, 
    $A$ --- компактний самоспряжений оператор в $H$. Тоді існує ортонормована 
    система векторів $\{e_n\}$ ($1 \leq n \leq N \leq \infty$) та 
    числовий набір $\{\lambda_n\} \subset \real \backslash \{0\}$ 
    ($1 \leq n \leq N \leq \infty$) такі, що $e_n$ --- власні вектори 
    оператора $A$, що відповідають власним числам $\lambda_n$,
    $\lambda_n \rightarrow 0$ за умови $N = \infty$ і при цьому для кожного 
    $x \in H$ виконуються рівності:
    \begin{equation*}
        x = \suml{n=1}{N}\dotprod{x}{e_n}e_n + \tilde{x}, \text{де } 
        \tilde{x} \in \mathrm{Ker}A;
    \end{equation*}
    \begin{equation*}
        Ax = \suml{n=1}{N}\lambda_n \dotprod{x}{e_n}e_n 
        \text{ (при }
        N = \infty
        \text{ цей ряд називають \ul{рядом Шмідта})}.
    \end{equation*}
    \end{theorem*}
\end{theory}
            % !TEX root = ../main.tex

\begin{exercise}
    Нехай $K(t,s) = \sum\limits^\infty_{n=1} \frac{1}{n^2} \sin(n \pi t) \sin(n \pi s)$,
    оператор $A$ діє в $L_2[-1;1]$ за формулою $(Ax)(t) = \int\limits^1_{-1} K(t,s) x(s) ds$.
    \begin{enumerate}
        \item Довести: $A$ --- компактний самоспряжений оператор;
        \item знайти $\sigma(A)$, $r(A)$, $\norm{A}$;
        \item зобразити $A$ у вигляді ряду Шмідта.
    \end{enumerate}
\end{exercise}

\begin{exercise}
    Нехай $K(t) = \sum\limits_{n \in \integers} C_n e^{i n t}$. Розглянемо в $L_2[-\pi; \pi]$
    оператор $A$ за формулою $(Ax)(t) = \int\limits^\pi_{-\pi} K(t-s) x(s) ds$.
    \begin{enumerate}
        \item Довести: $A$ --- самоспряжений оператор Гільберта-Шмідта; функції \\$\varphi_n(t) = 
        \frac{1}{\sqrt{2\pi}} e^{i n t}$, $n \in \integers$ є власними функціями оператора $A$;
        \item знайти $\sigma(A)$, $r(A)$, $\norm{A}$;
        \item зобразити $A$ у вигляді ряду Шмідта.
    \end{enumerate}
\end{exercise}

\begin{exercise}
    Знайти спектри наступних операторів $A \in L(C[a;b])$:
    \begin{enumerate}
        \item $(Ax)(t) = \int\limits^1_0 (t+s)x(s)ds$, $a=0$, $b=1$;
        \item $(Ax)(t) = x(t) + \int\limits^1_0 (t+s)x(s)ds$, $a=0$, $b=1$;
        \item $(Ax)(t) = \int\limits^{2\pi}_0 \cos(t+s)x(s)ds$, $a=0$, $b=2\pi$.
    \end{enumerate}
\end{exercise}

\begin{exercise}
    Довести, що для будь-якої функції $f \in C[a;b]$ (варіант: $f \in L_2[a;b]$)
    наступне рівняння має розв'язок в $L_2[a;b]$:
    \begin{enumerate}
        \item $x(t) = -\int\limits^1_0 x(s) ds + f(t)$, $a=0$, $b=1$;
        \item $x(t) = -\int\limits^1_0 (ts+2)x(s)ds + f(t)$, $a=0$, $b=1$;
        \item $x(t) = -\int\limits^1_0 \cos(t-s)x(s)ds + f(t)$, $a=0$, $b=1$;
        \item $x(t) = -\int\limits^1_0 e^{ts}x(s)ds + f(t)$, $a=0$, $b=1$;
        \item $x(t) = \int\limits^1_0 (t-s)x(s)ds + f(t)$, $a=0$, $b=1$;
        \item $x(t) = \int\limits^\pi_0 \sin(t-s)x(s)ds + f(t)$, $a=0$, $b=\pi$;
        \item $x(t) = \int\limits^1_0 (t-s)\cos(t+s)x(s)ds + f(t)$, $a=0$, $b=1$;
    \end{enumerate}
\end{exercise}

\begin{theory}
    \ul{Підказ}: якщо для $A\in L(K)$ $\forall x \; \dotprod{Ax}{x} \leq 0$,
    то $(x=Ax) \Rightarrow (x=0)$.
\end{theory}

\begin{exercise}
    Використовуючи теорему Фредгольма, довести, що наступні рівняння мають розв'язок в $L_2[a;b]$
    при заданих $f$ та довільних $\lambda$:
    \begin{enumerate}
        \item $x(t) = \lambda \int\limits^1_{-1} x(s) ds + f(t)$;
              $f(t) = t, t^3, e^{t^2}\sin t$, $a=-1$, $b=1$;
        \item $x(t) = \lambda \int\limits^{2\pi}_0 \sin(t+s) x(s) ds + f(t)$; 
              $f(t) = 1, \sin 2t, \cos 2t$, $a=0$, $b=2\pi$.
    \end{enumerate}
\end{exercise}

\begin{exercise}
    Використовуючи теорему Фредгольма, довести, що наступні рівняння мають розв'язок в $L_2[a;b]$:
    \begin{enumerate}
        \item $x(t) = \int\limits^1_{-1} \left(
                \sum\limits^\infty_{k=1} \frac{t^k}{k!} s^{2k}
            \right) x(s) ds + t^3$, $a=-1$, $b=1$;
        \item $x(t) = \int\limits^{2\pi}_0 \left(
                \sum\limits^\infty_{k=1} e^{-kt} \frac{\cos(ks)}{k^2}
            \right) x(s) ds + \sin 2t$, $a=0$, $b=2\pi$.
    \end{enumerate}
\end{exercise}

\begin{exercise}
    Розв'язати рівняння в просторі $L_2[a;b]$ з симетричним ядром, зведенням їх диференціюванням
    до відповідної крайової задачі:
    \begin{enumerate}
        \item $x(t) = \int\limits^1_0 K(t,s) x(s) ds + t^2 - t$,
        $K(t,s) = \begin{cases}
            (t-1)s, & 0 \leq s \leq t \leq 1 \\
            t(s-1), & 0 \leq t < s \leq 1
        \end{cases}$;
        \item $x(t) = \int\limits^\pi_0 K(t,s) x(s) ds + 1$,\
        $K(t,s) = \begin{cases}
            2 \sin (t) \cos (s), & 0 \leq t \leq s \leq \pi \\
            2 \sin (s) \cos (t), & 0 \leq s < t \leq \pi .
        \end{cases}$
    \end{enumerate}
\end{exercise}
            \newpage
    \chapter{Необмежені лінійні оператори. Операторні півгрупи.}
        \section{Необмежені лінійні оператори.}
            % !TEX root = ../main.tex

\label{N:2_1_6}
            % !TEX root = ../main.tex

\begin{exercise}
    $X$ --- банахів простір, $Y$ --- замкнений підпростір, 
    $P$ --- проектор $X$ на $Y$ паралельно $Z$ ($Z = Ker P$). 
    Довести: $P$ --- обмежений тоді й тільки тоді, коли $Z$ --- 
    замкнений підпростір в $X$.
\end{exercise}

\begin{exercise}
    Нехай $X$, $Y$ --- замкнені підпростори в банаховому просторі $Z$, 
    $Z = X \dotplus Y$. Довести, що існує число $K$, для якого при всіх $z \in Z$ 
    виконуються нерівності: $\norm{x} \leq K\norm{z}$; $\norm{x} \leq K\norm{z}$ (тут 
    $x \in X$, $y \in Y$, $z = x + y$). 
\end{exercise}

\begin{exercise}
    Розглянемо оператори $i_X$: $X \ni x \rightarrow 
    \sumpair{x}{0} \in X \dotplus Y$;
    $i_Y$: $Y $\reflectbox{$\in$}$ y \rightarrow 
    \sumpair{0}{y} \in X \dotplus Y$;
    $P_X$: $X \dotplus Y $\reflectbox{$\in$}$ \sumpair{x}{y} \rightarrow 
    x \in X $;
    $P_Y$: $X \dotplus Y $\reflectbox{$\in$}$ \sumpair{x}{y} \rightarrow 
    y \in Y $. $X$ та $Y$ --- нормовані простори.
    \begin{enumerate}
        \item Перевірити, що для норми $\norm{\sumpair{x}{y}} = 
        \norm{x}_X + \norm{y}_Y$ в $X \dotplus Y$ оператори $i_X$, $i_Y$, 
        $P_X$, $P_Y$ --- обмежені і знайти їх норми;
        \item нехай $\norm{\cdot}_1$ --- інша норма на $X \dotplus Y$; 
        оператори $i_X$, $i_Y$, $P_X$, $P_Y$ --- обмежені. Чи можна 
        стверджувати, що $\norm{\cdot}_1 \sim \norm{\cdot}$?
        \item нехай $\norm{\cdot}_1$ --- інша норма на $X \dotplus Y$;
        $X$, $Y$, $X \dotplus Y$ --- банахові. Виконуються умови: 
        $\norm{\sumpair{x}{0}}_2 = \norm{x}_X$; $\norm{\sumpair{0}{y}} 
        = \norm{y}_Y$. Чи можна 
        стверджувати, що $\norm{\cdot}_2 \sim \norm{\cdot}$?
    \end{enumerate}
\end{exercise}

\begin{exercise}\label{N:2_1_18}
    Нехай $\{a_n\}$ --- числова послідовність; оператор $A: \ell_2 
    \rightarrow \ell_2$ визначено умовою: $A\vec{x} = 
    (a_1x_1, a_2x_2, ...)$. Нехай $\inf\{|\frac{a_n}{n}|\} > 0$ та 
    $D(A) = \{\vec{x} \in \ell_2 | \suml{n=1}{\infty}|a_n|^2|x_n|^2 
    < \infty\}$. Довести лінійність, необмеженість, замкненість оператора 
    $A$ та його щільну визначеність ($\overline{D(A)} = \ell_2$).
\end{exercise}

\begin{exercise}
    Вивести теорему Банаха про обернений оператор з теореми Банаха про 
    замкнений графік.
\end{exercise}

\begin{exercise}
    Довести, що функціонал $\varphi: x \rightarrow x(0)$, що діє в 
    $L_2[0, 1]$ з областю визначення $D(\varphi) = C[0, 1]$ не є 
    замкненим.
\end{exercise}

\begin{theory}
    Нехай $A, B$ --- лінійні оператори з $X$ в $Y$. Оператор $B: X 
    \rightarrow Y$ називається \ul{розширенням} $A$, якщо $\Gamma_A 
    \subset \Gamma_B$. Позначення: $A \subset B$. В разі, якщо 
    $\Gamma_B = \overline{\Gamma_A}$, оператор $B$ називається 
    \ul{замиканням} оператора $A$, а оператор $A$ є такий, що 
    \ul{допускає замикання}. Позначення: $B = \overline{A}$.
\end{theory}

\begin{exercise}
    Нехай $A, B: X \rightarrow Y$ --- лінійні оператори. Доведіть:
    \begin{enumerate}
        \item $(A \subset B) \Leftrightarrow (D(A) \subset D(B); 
        \text{для }x \in D(A): Ax = Bx)$;
        \item ($A$ допускає замикання) $\Leftrightarrow$ (існує 
        замкнений оператор $B$, такий, що $A \subset B$)
        \item Замикання оператора $A$ --- це найменше замкнене 
        розширення $A$.
    \end{enumerate}
\end{exercise}

\begin{exercise}
    Нехай $A: X \supset D(A) \ni x \rightarrow Ax \in Y$. Довести 
    еквівалентність двох тверджень:
    \begin{enumerate}
        \item $A$ допускає замикання;
        \item виконується умова: ($D(A) \ni x_n \rightarrow 0; 
        Ax_n \rightarrow y$) $\Rightarrow$ (y = 0).
    \end{enumerate}
\end{exercise}

\begin{exercise}
    Нехай $A: X \rightarrow Y$ --- обмежений оператор. Тоді $A$ допускає 
    замикання і $D(\overline{A}) = \overline{D(A)}$.
\end{exercise}

\begin{exercise}
    Нехай оператор $A: X \rightarrow Y$ допускає замикання. Довести:
    $D(\overline{A}) \subset \overline{D(A)}; Im(\overline{A}) \subset 
    \overline{ImA}$.
\end{exercise}

\begin{exercise}
    Нехай $A: \ell_2 \rightarrow \ell_2$ визначений формулою $A: e_n 
    \rightarrow ne_1$, де $e_n = (\underbrace{0, ..., 0}_{n-1}, 1, 0, ...)$. 
    Знайти $D(A)$ і довести, що $A$ не допускає замикання.
\end{exercise}
            % !TEX root = ../main.tex

\begin{exercise}
    Довести наступні твердження для необмежених операторів:
    \begin{enumerate}
        \item $A + B = B + A$;
        \item $A + (B + C) = (A + B) + C$;
        \item $0 \cdot A \subset 0$;
        \item $A(BC) = (AB)C$;
        \item $(A+B)C = AC + BC$;
        \item $A(B+C) \supset AB + AC$;
        \item $\left( \exists \; \inv{A}, \inv{B} \right)  \Rightarrow 
               \left( \inv{(AB)} = \inv{B}\inv{A} \right)$.
    \end{enumerate}
\end{exercise}

\begin{exercise}
    Побудувати лінійний оператор $A$ в гільбертовому просторі, для якого $\overline{D(A)} = H$,
    $D(A^2) = \{0\}$.
\end{exercise}

\begin{theory}
    Нехай $A$ --- замкнений оператор в комплексному банаховому просторі $X$.
    Для числа $\lambda \in \complex$ оператор $\lambda I - A$ також є замкненим
    (див. вправу \ref{N:2_1_6}). Якщо $\mathrm{Im}(\lambda I - A) = X$ та 
    $\mathrm{Ker}(\lambda I - A) = \{0\}$, то існує обернений оператор $\inv{(\lambda I - A)}$,
    який також є замкненим (див. вправу \ref{N:2_1_14}), а тому за теоремою Банаха оператор
    $R(\lambda; A) = \inv{(\lambda I - A)}$ є обмеженим. Такі числа $\lambda$ називають
    \ul{регулярними числами} оператора $A$. Множина всіх таких чисел утворює 
    \ul{резольвентну множину} $\rho(A) \subset \complex$ оператора $A$, а її доповнення
    в $\complex$ --- \ul{спектр} $\sigma(A)$ оператора $A$.
\end{theory}

\begin{exercise}
    Нехай $A$ --- замкнений оператор в комплексному банаховому просторі, $\rho(A)$ --- його
    резольвентна множина. Довести:
    \begin{enumerate}
        \item тотожність Гільберта: $R_\lambda(A) - R_\mu(A) = (\mu - \lambda) R_\lambda(A)R_\mu(A)$,
        ($\mu, \lambda \in \rho(A)$);
        \item $\rho(A)$ --- відкрита множина в $\complex$;
        \item операторнозначна функція $\rho(A) \ni \lambda \mapsto \inv{(\lambda I - A)}$ є аналітичною.
    \end{enumerate}
\end{exercise}

\begin{exercise}
    Навести приклад необмеженого замкненого оператора, для якого $\rho(A) = \complex$.
\end{exercise}

\begin{theory}
    Нехай $A$ --- замкнений оператор в гільбертовому просторі $H$, $\overline{D(A)} = H$.
    \ul{спряжений до $A$ оператор} $A^*$ визначено на множині $D(A^*) = \{
        y \in H \mid \exists z \; \forall x \in D(A): \dotprod{Ax}{y} = \dotprod{x}{z}
    \}$.
    При цьому $A^*: y \mapsto z$.
\end{theory}

\begin{exercise}
    Доведіть коректність означення спряженого оператора:
    \begin{enumerate}
        \item $\forall y \in D(A^*)$ відповідає єдиний вектор $z$;
        \item перевірте лінійність оператора $A^*$.
    \end{enumerate}
\end{exercise}

\begin{exercise}
    Доведіть замкненість оператора $A^*$.
\end{exercise}

\begin{exercise}
    Розглянемо в гільбертовому просторі $H \oplus H$ \uline{оператор <<повороту на 90°>>}
    $\tau: \pair{x}{y} \mapsto \pair{-y}{x}$.
    Доведіть наступні властивості оператора $\tau$:
    \begin{enumerate}
        \item $\tau$ --- лінійне взаємно однозначне перетворення простору $H \oplus H$;
        \item $\tau^2 = -\mathrm{id}$, $\tau^4 = \mathrm{id}$ 
              ($\mathrm{id}$ --- тотожне перетворення в $H \oplus H$);
        \item для кожного підпростору $M \subset H \oplus H$ виконуються властивості:
              $\tau(M^\perp) = (\tau(M))^\perp$, $\overline{\tau(M)} = \tau(\overline{M})$.
    \end{enumerate}
\end{exercise}

\begin{exercise}
    Нехай $A$ --- щільно визначений оператор в гільбертовому просторі $H$,
    $\Gamma_A$ --- його графік. Довести: 
    $\Gamma_{A^*} = (\tau(\Gamma_A))^\perp = \tau(\Gamma_A^\perp)$
\end{exercise}

\begin{exercise}
    Нехай оператор $A$ --- щільно визначений в $H$. Тоді $A$ допускає замикання тоді 
    й тільки тоді, коли $\overline{D(A^*)} = H$ і при цьому $(\overline{A})^* = A^*$, $A^{**} = \overline{A}$.
\end{exercise}

%73
\begin{exercise}
    Нехай $A$ --- оператор в $H$, $D_A = H$. Тоді $A^*$ --- обмежений оператор.
\end{exercise}

\begin{exercise}(\ul{Теорема Банаха про замкнений графік в гільбертовому просторі})
    Нехай $A: H \to H$ --- замкнений оператор, $D(A) = H$. Довести $A \in L(H)$.
\end{exercise}

\begin{exercise}
    Нехай $A$, $B$ --- оператори в $H$, $\overline{D(A)} = H$, $A \subset B$.
    Довести:
    \begin{enumerate}
        \item $B^* \subset A^*$;
        \item якщо $B$ допускає замикання, то $A$ також.
    \end{enumerate}
\end{exercise}

\begin{theory}
    Оператор $A: H \to H$ називається \ul{симетричним}, якщо для всіх $x,y \in D(A)$
    виконується умова $(Ax,y) = (x, Ay)$ та \ul{самоспряженим}, якщо $A = A^*$.
\end{theory}

\begin{exercise}
    Якщо $\overline{D(A)} = H$, то симетричність оператора $A$ рівносильна умові
    $A \subset A^*$. Доведіть.
\end{exercise}

\begin{exercise}
    Якщо оператор $A: H \to H$ --- самоспряжений, то $A$ --- симетричний і замкнений. Доведіть.
\end{exercise}

\begin{exercise}
    Якщо $A$ --- симетричний, $D(A) = H$, то $A$ --- самоспряжений та обмежений. Доведіть.
\end{exercise}

\begin{exercise}
    Для оператора $A$ із задачі \ref{N:2_1_18} знайти $D(A^*)$ та $A^*$.
\end{exercise}

\begin{exercise}
    Оператор $A$ в просторі $L_2[0;1]$ визначено формулою $(Ax)(t) = x(t^2)$ з областю визначення
    $D(A) = \left\{ x \in L_2[0;1] \mid \exists \intl{0}{1} x^2(t^2) dt \right\}$. Довести:
    \begin{enumerate}
        \item $\overline{D(A)} = L_2[0;1]$;
        \item $A$ --- лінійний необмежений оператор;
        \item знайти $D(A^*)$ та $A^*$.
    \end{enumerate}
\end{exercise}

\begin{exercise}
    Нехай $A$ --- щільно визначений оператор в $H$.
    \begin{enumerate}
        \item Довести: $(\mathrm{Im}A)^\perp = \mathrm{Ker}A^*$;
        \item Чи можна стверджувати що $(\mathrm{Ker}A)^\perp = \overline{\mathrm{Im}A^*}$?
    \end{enumerate}
\end{exercise}

\begin{exercise}
    Довести наступні твердження для необмежених операторів, що щільно визначені в гільбертовому
    просторі:
    \begin{enumerate}
        \item $(\lambda A)^* = \overline{\lambda} A^*$;
        \item $(A + B)^* \supset A^* + B^*$;
        \item $(A \in L(H)) \Rightarrow \left( (A+B)^* = A^* + B^* \right)$;
        \item $(AB)^* \supset B^* A^*$;
        \item $(A \in L(H)) \Rightarrow \left( (AB)^* = B^* A^* \right)$;
        \item $\left( \exists \; \inv{A}, A^*, (\inv{A})^* \right) \Rightarrow
               \left( \exists \; \inv{(A^*)} = (\inv{A})^* \right)$.
    \end{enumerate}
\end{exercise}

\begin{exercise}
    $A : L_2[0;1] \to L_2[0;1]$, $A = \frac{d}{dt}$,
    $D(A) = \left\{
        x \in C^1[0;1] \mid x(0) = x(1) = 0
    \right\}.$
    \begin{enumerate}
        \item Довести: $\overline{D(A)} = L_2[0;1]$;
        \item $A$ --- необмежений оператор. Довести;
        \item знайти $D(A^*)$ та $A^*$. 
    \end{enumerate}
\end{exercise}

\begin{exercise}
    В просторі комплекснозначних функцій $L_2[0;1]$ розглянемо оператор $A = i \frac{d}{dt}$
    з областю визначення
    $D(A) = \left\{
        x \in C^\infty[0;1] \mid x(0) = x(1) = 0
    \right\}.$
    Доведіть, що оператор $A$ --- симетричний, але не самоспряжений.
\end{exercise}

\begin{exercise}
    Оператор $A$ в $L_2[0;1]$ визначено формулою $(Ax)(t) = t \cdot x(0)$, $D(A) = C[0;1]$.
    Знайти $D(A^*)$ та $A^*$.
\end{exercise}

\begin{exercise}
    Нехай $A$ --- симетричний оператор в $H$, $\overline{\mathrm{Im}A} = H$.
    Доведіть: $\exists \; \inv{A}$, $\inv{A}$ --- симетричний.
\end{exercise}

\begin{exercise}
    Нехай $A \in L(H)$, $A$ --- самоспряжений, $\exists \; \inv{A}$ (необов'язково обмежений).
    Доведіть: $\inv{A}$ --- самоспряжений.
\end{exercise}

\begin{exercise}\label{N:2_1_50}
    Довести, що наступні три умови еквіваленті:
    \begin{enumerate}
        \item $A^*$ --- самоспряжений оператор в $H$;
        \item $\overline{A} = A^*$;
        \item $\overline{A}$ --- самоспряжений оператор в $H$.
    \end{enumerate}
\end{exercise}

\begin{theory}
    Оператор, що задовольняє будь-яку з умов задачі \ref{N:2_1_50} (а тому --- всі) називається
    \ul{суттєво самоспряженим}.

    \noindentОператор $A: H \to H$ називається \ul{дисипативним}, якщо виконується умова
    $(x \in D(A)) \Rightarrow (\mathfrak{Re}\dotprod{Ax}{x} \leq 0)$.
\end{theory}

\begin{exercise}
    Нехай $A$ --- дисипативний оператор в $H$, $\overline{D(A)} = H$.
    Тоді $A$ допускає замикання і $\overline{A}$ також дисипативний.
\end{exercise}

\begin{exercise}
    $A,B \in L(X,Y)$, де $X$, $Y$ --- банахові простори, $B \in K(X,Y)$ (компактний оператор),
    $\mathrm{Im} A \subset \mathrm{Im} B$. Доведіть: $A$ --- компактний оператор, якщо:
    \begin{enumerate}
        \item $\mathrm{Ker} B = \{0\}$;
        \item $\mathrm{Ker} B \neq \{0\} $.
    \end{enumerate}    
\end{exercise}
            \newpage
        \section{Операторні \texorpdfstring{$C_0$}{C0}-півгрупи.}
            % !TEX root = ../main.tex

\begin{theory}
    Нехай $X$ --- банахів простір (над полем $\real$ або $\complex$),
    $T(t)$ --- однопараметрична сім'я обмежених лінійних операторів в $X$, $t \in [0; +\infty)$.
    Сім'я операторів $T(t)$ називається \uline{(однопараметричною) операторною півгрупою}, якщо
    виконуються властивості:
    \begin{enumerate}[label = \arabic*)]
        \item $T(0) = I$;
        \item $T(t+s) = T(t) T(s)$ $(t, s \geq 0)$.
    \end{enumerate}
    Якщо, додатково, виконується умова \uline{сильної неперервності}:
    \begin{enumerate}[label = \arabic*)]
        \item[3)] $\forall \; x \in X$ функція $[0; +\infty) \ni t \mapsto T(t)x \in X$ є неперервною,
    \end{enumerate}
    то операторна півгрупа називається \uline{$c_0$-півгрупою}.
    Якщо замість 3) виконується умова 
    \begin{enumerate}[label = \arabic*)]
        \item[3$^\prime$)] $[0; +\infty) \ni t \mapsto T(t) \in L(X)$ --- неперервна операторна функція,
    \end{enumerate} 
    то півгрупу $T(t)$ називають \uline{неперервною за нормою} або
    \ul{рівномірно неперервною}.
    \noindent\uline{Генератор} $A$ $c_0$-півгрупи $T(t)$ визначається двома умовами:
    \begin{enumerate}
        \item $D(A) = \left\{ x \in X \mid \exists \; \underset{t\to 0+0}{\lim} \frac{1}{t} (T(t)x - x)\right\}$;
        \item для $x \in D(A)$ $Ax := \underset{t\to 0+0}{\lim} \frac{1}{t} (T(t)x - x)$.
    \end{enumerate}
    Інше позначення генератора --- $A = T^\prime (0)$.

    \noindent $c_0$-півгрупа називається \uline{$c_0$-півгрупою стиску},
    якщо для всіх $t \geq 0$ виконується умова $\norm{T(t)} \leq 1$.
\end{theory}

\begin{exercise}
    Довести наступні твердження:
    \begin{enumerate}
        \item Рівномірно неперервна оператора півгрупа $T(t)$ є $c_0$-півгрупою;
        \item Якщо $T(t)$ --- $c_0$-півгрупа в банаховому просторі $X$, то існують числа $M, \; \omega \in \real$,
        для яких при всіх $t \geq 0$ виконується нерівність $\norm{T(t)} \leq M e^{\omega t}$.
    \end{enumerate}
\end{exercise}

\begin{exercise}
    Нехай $A$ --- генератор $c_0$-півгрупи $T(t)$ в банаховому просторі $X$. Доведіть наступні твердження:
    \begin{enumerate}
        \item $A$ --- лінійний оператор;
        \item $\overline{D(A)} = X$;
        \item $A$ --- замкнений оператор в $X$.
    \end{enumerate}
\end{exercise}

\begin{exercise}
    Нехай $A$ --- генератор $c_0$-півгрупи $T(t)$ в банаховому просторі $X$. Довести:
    \begin{enumerate}
        \item $\forall \; x \in D(A)$ при всіх $t \geq 0$: $T(t)x \in D(A)$ і при цьому $\exists \; \frac{d}{dt} T(t)x = A T(t)x = T(t) Ax$;
        \item $\forall \; x \in D(A)$ при всіх $t \geq 0$ має місце формула $T(t)x - x = \int\limits_0^t T(s) Ax ds$;
        \item $\forall \; x \in D(A)$ при всіх $t \geq 0$: $\int\limits_0^t T(s)x ds \in D(A)$ і при цьому $T(t)x - x = A \int\limits_0^t T(s)x ds$.
    \end{enumerate}
\end{exercise}

\begin{exercise}
    Нехай $T(t)$, $S(t)$ --- дві операторні $c_0$-півгрупи в $X$, $T^\prime(0) = S^\prime(0)$. Доведіть:
    для кожного $t \geq 0$ виконується рівність $T(t) = S(t)$. 
\end{exercise}

\begin{exercise}
    Нехай для операторної півгрупи $T(t)$ в банаховому просторі $X$ крім умов
    $1$) та $2$) виконується умова $\tilde{3}$): $\norm{T(t) - I} \to 0$ при $t\to 0 + 0$ (неперервність за нормою в нулі).
    Доведіть: $T(t)$ --- рівномірно неперервна півгрупа.
\end{exercise}

\begin{exercise}
    Нехай для операторної півгрупи $T(t)$ в банаховому просторі $X$ крім умов
    $1$) та $2$) виконується умова $\hat{3}$): $\forall x \in X : \norm{T(t)x - x} \to 0$ при $t\to 0 + 0$ (сильна неперервність в нулі).
    Доведіть: $T(t)$ --- $c_0$-півгрупа.
\end{exercise}

\begin{theory}
    Для обмеженого, визначеного на всьому банаховому просторі $X$ оператора
    $A \in L(X)$ запроваджується оператор $e^{tA}$ за формулою
    $e^{tA} = \sum\limits_{n=0}^{\infty} \frac{1}{n!} t^n A^n$ ($A^0 = I$).
\end{theory}

\begin{exercise}
    Перевірте коректність означення $e^{tA}$, $A \in L(X)$:
    збіжність відповідного ряду і включення $e^{tA} \in L(X)$.
\end{exercise}
            % !TEX root = ../main.tex

\begin{exercise}
    Довести наступний \ul{критерій $c_0$ півгруп, неперервних за 
    нормою}:
    \begin{enumerate}
        \item ($A \in L(X)$) $\Rightarrow$ ($T(t) = e^{tA}$ 
        --- рівномірно неперервна півгрупа);
        \item ($T(t)$ --- рівномірно неперервна півгрупа) $\Rightarrow$ 
        ($A = T^{\prime}(0) \in L(X)$ : $T(t) = e^{tA}$ $\forall t \geq 0$);
    \end{enumerate}
\end{exercise}

\begin{exercise}
    Нехай $T(t)$ --- $c_0$-півгрупа в $X$; $\lambda \in K$ ($K$ --- основне поле). Доведіть, 
    що оператори $S(t) = e^{-\lambda t}T(t)$ утворюють $c_0$-півгрупу і при цьому 
    $S^{\prime}(0) = T^{\prime}(0) - \lambda I$. 
\end{exercise}

\begin{exercise}
    Нехай $T(t)$ --- $c_0$-півгрупа; $\norm{T(t)} \leq Me^{\omega t}$ $\forall t \geq 0$;
    $A = T^{\prime}(0)$. Для $\lambda \in \complex$, що задовольняє нерівність $Re\lambda > 0$,
    розглянемо \ul{перетворення Лапласа $R_\lambda$ півгрупи $T(t)$} $R_\lambda : 
    X \rightarrow X$ визначено формулою $R_\lambda x = \intl{0}{\infty}e^{-\lambda t} T(t)
    x dt$. Доведіть, що:
    \begin{enumerate}
        \item $\forall x \in X$, $Re \lambda > \omega$ оператор 
        $R_\lambda$ є коректно визначеним оператором з $L(X)$;
        \item ($Re \lambda > \omega$) $\Rightarrow$ ($\inv{(\lambda - A)}x = R_\lambda x$ 
        для $x \in X$), тобто ($Re \lambda > \omega$) $\Rightarrow$ ($\lambda \in \rho(A)$
        та $R(\lambda; A) = R_\lambda$).
    \end{enumerate}
\end{exercise}

\begin{exercise}
    Нехай $K = \complex$. Довести:
    \begin{enumerate}
        \item Якщо $T(t)$ --- $c_0$-півгрупа стиску, то ($Re\lambda > 0$) $\Rightarrow$
        ($\lambda \in \rho(A)$; $\norm{\inv{\lambda - A}} \leq \frac{1}{Re \lambda}$);
        \item Якщо $T(t)$ --- довільна $c_0$-півгрупа; $\norm{T(t)} \leq Me^{\omega t}$ 
        ($t \geq 0$), то ($Re \lambda > \omega$) $\Rightarrow$ 
        ($\norm{\inv{\lambda - A}} \leq \frac{M}{Re\lambda - \omega}$).
    \end{enumerate}
\end{exercise}

\begin{theory}
    \ul{Теорема Хіллє - Іосіди}. Лінійний оператор $A$ в банаховому просторі $X$ є 
    генератором $c_0$-півгрупи стиску тоді й тільки тоді, коли він задовольняє 
    наступні умови:
    \begin{enumerate}
        \item $\overline{D(A)} = X$; $A$ --- замкнений оператор;
        \item $(0, +\infty) \subset \rho(A)$ і при цьому для всіх $\lambda > 0$: 
        $\norm{R(\lambda; A)} \leq \frac{1}{\lambda}$.
    \end{enumerate}
    \ul{Теорема Хіллє-Іосіди-Міллдера-Феллера-Філліпса}. Лінійний оператор $A$ в банаховому 
    просторі $X$ є генератором $c_0$-півгрупи тоді й тільки тоді, коли він задовольняє 
    такі умови:
    \begin{enumerate}
        \item 
    \end{enumerate}
\end{theory}
            % !TEX root = ../main.tex

\begin{exercise}
    Нехай $T(t)$ --- $C_0$-півгрупа в банаховому просторі $X$, $A = T'(0)$.
    Довести:
    \begin{enumerate}
        \item $\overline{ D(A^2) } = X$;
        \item $\forall n \in \natur$: $\overline{ D(A^n) } = X$;
        \item[в)*] $\bigcap\limits^\infty_{n=1} D(A^n)$ --- щільна множина в $X$.
    \end{enumerate}
\end{exercise}

\begin{exercise}
    Нехай $T(t)$ --- $C_0$-півгрупа в $X$, $A = T'(0)$, $x \in X$.
    Довести: $\lambda\inv{(\lambda - A)} x \to x$, $\lambda \to +\infty$ $(\lambda \in \real)$.
\end{exercise}

\begin{theory}
    $C_0$-півгрупа називається \ul{рівномірно обмеженою}, якщо існує число $M>0$ таке,
    що при всіх $t \geq 0$ виконується нерівність $\norm{T(t)} \leq M$.
\end{theory}

\begin{exercise}
    Нехай $C_0$-півгрупа $T(t)$ рівномірно обмежена в банаховому просторі $X$,
    $A = T'(0)$. Довести:
    \begin{enumerate}
        \item $\mathrm{Im}A \cap \mathrm{Ker}A = \{0\}$;
        \item $\overline{\mathrm{Im}A} \cap \mathrm{Ker}A = \{0\}$.
    \end{enumerate}
    Наведіть приклад $C_0$-півгрупи в $\real^2$, для якої $\mathrm{Im}A \cap \mathrm{Ker}A \neq \{0\}$.
\end{exercise}

\begin{exercise}
    Нехай $T(t)$ --- $C_0$-півгрупа в $X$, для якої виконується умова $\forall x \in X$
    $\exists \underset{t \to +\infty}{\lim} T(t)x = T(\infty)x$, $A = T'(0)$.
    Довести:
    \begin{enumerate}
        \item $\overline{\mathrm{Im}A} = \mathrm{Ker}(T(\infty))$;
        \item $\mathrm{Im} (T(\infty)) = \mathrm{Ker} A$;
        \item $X = \mathrm{Ker}A \dotplus \overline{\mathrm{Im}A}$;
        \item $\mathrm{Im}A = \left\{x \mid \intl{0}{\infty}T(t)xdt \text{ збігається}\right\}$.
    \end{enumerate}
\end{exercise}

\begin{exercise}
    Позначимо через $X = C[0;+\infty]$ лінійний простір обмежених і рівномірно неперервних
    функцій на $[0;+\infty)$ (дійсно- або комплекснозначних). Оператори $T(t)$ задамо на $X$
    формулою: $(T(t)x)(s) = x(s+t)$, $(t \geq 0)$. Довести:
    \begin{enumerate}
        \item $X$ --- банахів простір з нормою $\norm{x} = \underset{t \geq 0}{\sup} |x(t)|$;
        \item $T(t)$ --- лінійні обмежені оператори і утворюють в $X$ $C_0$-півгрупу стиску;
        \item $(T'(0)x)(s) = x'(s)$ і при цьому $D(T'(0)) = \set{x\in X \mid x' \in X}$.
    \end{enumerate}
\end{exercise}

\begin{exercise}
    Нехай $u, f$ --- вектор-функції на $[0; \infty)$ зі значеннями в $X$,
    $f$ --- неперервна, $u$ --- диференційовна і при цьому при $t \geq 0$ виконується рівність
    $\frac{d}{dt}u(t) = A u(t) +f(t)$, де $A$ --- генератор $C_0$-півгрупи $T(t)$, $u(0) = x \in X$.
    Довести: при всіх $t \geq 0$: $u(t) = T(t)x +\intl{0}{t} T(t-s)f(x) ds$.
\end{exercise}

\begin{exercise}
    Нехай $T(t), S(t)$ --- дві $C_0$-півгрупи на банаховому просторі $X$, $T(t)S(t) = S(t)T(t)$
    для кожного $t \geq 0$. Довести:
    \begin{enumerate}
        \item $\forall s,t \geq 0$: $T(t)S(s) = S(s)T(t)$;
        \item оператори $V(t) = S(t)T(t)$ утворюють $C_0$-півгрупу в $X$;
        \item Знайти зв'язок $V'(0)$ з $T'(0)$ та $S'(0)$.
    \end{enumerate}
\end{exercise}

\begin{theory}
    $C_0$-півгрупа $V(t)$ називається \ul{добутком} $T(t)$ та $S(t)$.
\end{theory}

\begin{exercise}
    Через $C_0 (\real^2)$ позначимо лінійний простір неперервних на $\real^2$ функцій $u$,
    для яких $\underset{\norm{\vec{x}}\to \infty}{\lim} u(\vec{x}) = 0$,
    $\norm{u} = \underset{\real^2}{\sup} |u(x,y)|$. Оператори $T(t)$ та $S(t)$ на $C_0 (\real^2)$
    задано формулами $(T(t)u)(x,y) = u(x+t, y)$, $(S(t)u)(x,y) = u(x, y+t)$, ($t\geq 0$).
    \begin{enumerate}
        \item Перевірити, що $C_0 (\real^2)$ --- банахів простір; $T(t)$, $S(t)$ --- $C_0$-півгрупи на
        $C_0 (\real^2)$, $T(t)S(t) = S(t)T(t)$.
        \item Описати $C_0$-півгрупу $V(t) = T(t)S(t)$ і знайти $T'(0)$, $S'(0)$, $V'(0)$.
    \end{enumerate}
\end{exercise}
            % !TEX root = ../main.tex

\begin{exercise}
    Нехай $T(t)$ --- $C_0$-півгрупа в $X$; $T'(0) = A$. Довести еквівалентність тверджень:
    \begin{enumerate}
        \item $T(t)$ --- рівномірно неперервна;
        \item $\exists \; c > 0: \forall t \in [0; 1]: \norm{T(t) - I} \leq ct$;
        \item $\overline{\lim\limits_{\lambda \rightarrow \infty}} \norm{\lambda A (\lambda - A)^{-1}} < \infty$.
    \end{enumerate}
\end{exercise}

\begin{exercise}
    Нехай $K$ --- метричний компакт, $X = C(K)$, $T(t)$ --- $C_0$-півгрупа в просторі $X$; 
    $A = T'(0)$. Довести еквівалентність двох умов:
    \begin{enumerate}
        \item $\forall f, g \in X; \forall t \geq 0: T(t)(fg) = T(t)f \cdot T(t)g$
        (тобто $T(t)$ --- гомоморфізм алгебри $X$);
        \item $A$ --- диференціювання алгебри $X$, тобто $D(A)$ --- підалгебра в $X$
        та $A(f \cdot g) = Af \cdot g + f \cdot Ag$ для всіх $f, g \in D(A)$.
    \end{enumerate}
\end{exercise}

\begin{exercise}
    Побудувати приклади двох різних $C_0$-півгруп в сепарабельному гільбертовому просторі, 
    генератори яких співпадають на щільній підмножині.
\end{exercise}

\begin{exercise}
    Нехай $T(t)$ --- $C_0$-півгрупа в $X$. $A \subset T' (0)$, $\overline{D(A)} = X$, 
    $(\lambda - A)R_{\lambda}x = x$ для $x \in D(A)$, де $R_{\lambda}$ --- перетворення Лапласа півгрупи $T(t)$.
    Тоді: $\overline{A} = T'(0)$. Довести.
\end{exercise}

\begin{exercise}
    Нехай $T(t)$ --- $C_0$-півгрупа. $A \subset T'(0)$, $\overline{D(A)} = X$, 
    $D(A)$ інваріантна відносно операторів $T(t)$ при $t \geq 0$. Тоді $\overline{A} = T'(0)$. Довести.
\end{exercise}
            \newpage
    \chapter{Простори Соболєва та узагальнені функції}
        \section{Простір \texorpdfstring{$H^1[a;b] = W_2^1[a;b]$}{H1[a;b] = W21[a;b]}}
            % !TEX root = ../main.tex

\begin{theory}
    Функція $v \in L_2[a;b]$ (нагадаємо: елементи
    $L_2[a;b]$ --- це класи еквівалентних функцій, тому
    $v$ --- це представник відповідного класу) називається
    \uline{узагальненою похідною} функції $u \in L_2[a;b]$,
    якщо для кожної функції $w \in C_0^1 [a;b] = \set{w \in C^1[a;b] \mid \supp{w} \subset (a; b)}$
    має місце рівність $$\int\limits_a^b v \cdot w dx = -\int\limits_a^b u \cdot w^\prime dx$$
    Множина функцій з $L_2[a;b]$, для яких існує узагальнена похідна, утворює
    підпростір в $L_2[a;b]$, який позначається $H^1[a;b] = W_2^1[a;b]$.
    Узагальнена похідна функції $v$ позначається через $v^\prime$.
\end{theory}

\begin{exercise}
    Довести, що для функції $u \in H^1[a;b]$ узагальнена похідна єдина,
    тобто $(u' = v_1, u' = v_2) \Rightarrow (v_1 = v_2 \text{ майже скрізь})$.
\end{exercise}

\begin{exercise}
    Перевірити, що множина $H^1[a;b]$ --- лінійний простір, щільний в $L_2[a;b]$ (звичайно, за нормою $L_2[a;b]$).
\end{exercise}

\begin{exercise}
    Для функцій $u_1, u_2 \in H^1[a;b]$ покладемо
    $\dotprod{u_1}{u_2}_H = \int\limits_a^b u_1 u_2 dx +\int\limits_a^b u_1^{\prime} u_2^{\prime} dx$.
    Доведіть:
    \begin{enumerate}
        \item $\dotprod{\cdot}{\cdot}_H$ --- скалярний добуток в $H^1[a;b]$;
        \item множина неперервно диференційовних функцій $C^1[a;b]$ щільна в $H^1[a;b]$ за нормою, породженою цим скалярним добутком;
        \item $H_1[a;b]$ --- гільбертів простір за цим скалярним добутком.
    \end{enumerate}
\end{exercise}

\begin{exercise}\label{N:3_1_4}
    Довести сепарабельність гільбертового простору $H^1[a;b]$.
\end{exercise}

\begin{exercise}
    Довести, що вкладення $C^1[a;b] \to C[a;b]$ є компактним лінійним оператором.
\end{exercise}

\begin{exercise}
    Довести, що вкладення $H_1[a;b] \to L_2[a;b]$ є
    \begin{enumerate}
        \item обмеженим лінійним оператором;
        \item компактним лінійний оператором.
    \end{enumerate}
\end{exercise}

\begin{exercise}
    \begin{enumerate}
        \item Довести вкладення $A: H^1[a;b] \to C[a;b]$, тобто що кожний
        клас функцій $[u]$, що є елементом простору $H^1[a;b]$, містить і при тому
        єдину неперервну функцію.
        \item Довести, що відображення $A$, що переводить клас $[u]$
        у відповідну неперервну функцію, є обмеженим лінійним оператором:
        $A \in L(H^1[a;b], C[a;b])$.
        \item[в)*] Довести, що $A$ --- компактний лінійний оператор. 
    \end{enumerate}
\end{exercise}

\begin{exercise}
    Нехай $\{ f_\alpha\}$ --- передкомпактна множина в $C[a;b]$.
    \begin{enumerate}
        \item Довести, що для фіксованої функції $g \in L_2[a;b]$
        множина $\{ f_\alpha \cdot g\}$ передкомпактна в $L_2[a;b]$.
        \item Нехай $g \in H^1[a;b]$. Чи можна стверджувати, що множина
        $\{ f_\alpha \cdot g\}$ передкомпактна в $H^1[a;b]$?
    \end{enumerate}
\end{exercise}

\begin{exercise}
    Чи буде множина $M = \left\{ \frac{1}{n} \sin{(nt)}\right\}_{n=1}^{\infty}$
    передкомпактною в:
    \begin{enumerate}
        \item $C[0;1]$;
        \item $C^1[0;1]$;
        \item $H^1[0;1]$?
    \end{enumerate}
\end{exercise}

\begin{exercise}
    Довести, що вказаний функціонал на $H^1[-\pi; \pi]$
    є лінійним обмеженим і знайти його норму.
    \begin{enumerate}
        \item $\varphi(x) = \int\limits_{-\pi}^{\pi} (x(t) \sin(t) + x'(t) \cos(t)) dt$;
        \item $\varphi(x) = \int\limits_{-\pi}^{\pi} (x(t) \cos(t) + x'(t) \sin(t)) dt$;
        \item[в)*] $\varphi(x) = x(0)$ ($\varphi([u]) = u(0)$).
    \end{enumerate}
\end{exercise}

\begin{exercise}
    $x \in H^1[a;b]$, $y \in C^1[a;b]$. Довести: $xy \in H^1[a;b]$.
\end{exercise}

\begin{exercise}
    Довести, що $\mathring{H}^1[a;b] = \set{x \in H^1[a;b] \mid x(a) = x(b) = 0}$
    є замкненим підпростором в $H^1[a;b]$. Знайти $\left(\mathring{H}^1[a;b]\right)^{\perp}$. 
\end{exercise}

\begin{exercise}
    Довести, що множина $M = \left\{ x \in H^1[a;b] \mid \int\limits_a^b x(t) = 0\right\}$
    є замкненим підпростором в $H^1[a;b]$. Знайти $M^{\perp}$.
\end{exercise}

\begin{exercise}
    Які з функцій $\sgn t$, $|t|$ не лежать в $H^1[-\pi; \pi]$?
\end{exercise}

\begin{exercise}\label{N:3_1_15}
    Довести, що вкладення $H^1[0; \pi]$ в $C[0; \pi]$ є строгим:
    $\exists \; x \in C[0; \pi]$ такий, що $x \notin H^1[0;\pi]$.
\end{exercise}

\begin{exercise}
    Оператор $A$ в $H = H^1[0;1]$ визначено формулою $(Ax)(t) = t x(t)$.
    Знайти $\norm{A}$, $\mathrm{Ker}{A}$. Чи вірно, що $\mathrm{Im}{A} = H$,
    $\overline{\mathrm{Im}{A}} = H$?
\end{exercise}

\begin{exercise}\label{N:3_1_17}
    Оператор $H^1[0;1] \to L_2[0;1]$ визначено формулою $Ax = x'$.
    \begin{enumerate}
        \item Знайти $\mathrm{Im}{A}$, $\mathrm{Ker}{A}$, $\norm{A}$.
        \item Довести, що оператор $A$ є замкненим.
        \item Довести, що оператор $A$ є обмеженим.
        \item Чи буде оператор $A$ компактним?
    \end{enumerate}
\end{exercise}
            \newpage
        \section{Узагальнені функції}
            % !TEX root = ../main.tex

\begin{theory}
    % \ul{Основні функції}\\
    \ul{Основні функції} --- це фінітні функції $\varphi: \real \to \real$ класу $C^\infty$.
    \ul{Фінітність функції} $\varphi$ --- це обмеженість її носія, тобто $\exists\;[a,b]:
    \supp \varphi = \overline{\set{x \mid \varphi(x)\neq 0}} \subset [a,b]$. Множину основних
    функцій позначаємо через $\mathcal{D}$.\\
    \ul{Збіжність послідовності} $\varphi_n$ основних функцій до (основної функції) $\varphi$
    розуміємо в наступному сенсі --- виконуються дві умови:
    \begin{enumerate}
        \item $\exists \;[a,b]: \forall n: \supp\varphi_n \subset [a,b]$;
        \item $\forall \; m \in \{0\} \cup \natur$:
        $\varphi_n^{(m)} \underset{\real}{\rightrightarrows} \varphi^{(m)}, n\to\infty$
    \end{enumerate}
    (рівномірна збіжність на $\real$ послідовності самих функцій $\varphi_n$ та їх похідних
    до відповідних похідних функції $\varphi$). Позначення: $\varphi_n \underset{\mathcal{D}}{\to} \varphi$.
\end{theory}

\begin{exercise}
    Нехай $f(x) = \begin{cases}
        e^{-\frac{1}{1-x^2}}, & |x| < 1 \\
        0, & |x| \geq 1
    \end{cases}$.
    \begin{enumerate}
        \item Доведіть, що $f \in \mathcal{D}$;
        \item Доведіть, що послідовність функцій $\varphi_n(x)=\frac{1}{n}f(x)$ збігається
              в $\mathcal{D}$ до функції $\varphi \equiv 0$;
        \item Доведіть, що $\mathcal{D}$ --- лінійний простір.
    \end{enumerate}
\end{exercise}

\begin{exercise}
    Доведіть, що в $\mathcal{D}$ не існує такої метрики $\rho$, що збіжність
    $\varphi_n \underset{\mathcal{D}}{\to} \varphi$ рівносильна збіжності за цією метрикою:
    $\rho(\varphi_n,\varphi) \to 0$.
\end{exercise}

\begin{exercise}
    Доведіть, що $\mathcal{D}$ \ul{повний} відносно вказаної збіжності в наступному сенсі:
    якщо для послідовності $\varphi_n \in \mathcal{D}$ виконуються умови:
    \begin{enumerate}
        \item $\exists \; [a,b]: \forall n: \supp\varphi_n \subset [a,b]$;
        \item $\forall \; m \in \{0\} \cup \natur$: послідовність $\varphi_n^{(m)}$ рівномірно
        збігається при $n\to\infty$ на $\real$,
    \end{enumerate}
    то існує функція $\varphi \in \mathcal{D}$, до якої $\varphi_n$ збігається
    в сенсі $\mathcal{D}$.
\end{exercise}

\begin{exercise}
    Нехай $\varphi \in \mathcal{D}$, $\varphi \not\equiv 0$. Чи збігається в $\mathcal{D}$ послідовність:
    \begin{enumerate}
        \item $\varphi_n(x) = \frac{1}{n}\varphi\left(\frac{x}{n}\right)$;
        \item $\varphi_n(x) = \varphi(x+n)$?
    \end{enumerate}
\end{exercise}

\begin{exercise}
    $\varphi \in \mathcal{D}$. Довести, що наступні умови еквівалентні:
    \begin{enumerate}
        \item $\exists \; \psi \in \mathcal{D}:\; \varphi = \psi'$;
        \item $\int\limits_{\real} \varphi(x) dx = 0$.
    \end{enumerate}
\end{exercise}

\begin{exercise}
    Довести, що кожну функцію $\varphi \in \mathcal{D}$ можна представити у вигляді:
    $\varphi(x) = \varphi_0(x) \int\limits_{\real} \varphi(y) dy  + \psi'(x)$, де
    $\psi \in \mathcal{D}$, а $\varphi_0$ --- довільна функція в $\mathcal{D}$, для
    якої виконується рівність $\int\limits_{\real} \varphi_0(x) dx = 1$.
\end{exercise}

\begin{exercise}
    Нехай $\varphi_n, \psi_n, \varphi, \psi \in \mathcal{D}$;
    $\varphi_n \underset{\mathcal{D}}{\to} \varphi$, $\psi_n \underset{\mathcal{D}}{\to} \psi$.
    Доведіть:
    \begin{enumerate}
        \item $\varphi_n +\psi_n \underset{\mathcal{D}}{\to} \varphi +\psi$;
        \item $\varphi_n \psi_n \underset{\mathcal{D}}{\to} \varphi \psi$;
        \item $\forall \; m \in \natur$: $\varphi_n^{(m)} \underset{\mathcal{D}}{\to} \varphi^{(m)}$.
    \end{enumerate}
\end{exercise}

\begin{theory}
    Функцію $f: \real \to \real$ називають <<\ul{звичайною}>>, якщо вона є інтегровною
    відносно міри $\lambda_1$ на кожному обмеженому проміжку. Логічне позначення множини
    всіх таких функцій: $L_{1,\mathrm{loc}}(\real)$ (точніше, елементи $L_{1,\mathrm{loc}}(\real)$
    --- це класи функцій, рівних майже скрізь).
\end{theory}

\begin{exercise}
    Нехай $f \in L_{1,\mathrm{loc}}(\real)$; $\varphi, \varphi_n \in \mathcal{D}$;
    $\varphi_n \underset{\mathcal{D}}{\to} \varphi$. Доведіть:
    \begin{enumerate}
        \item $\lim\limits_{n\to\infty} \intl{\real}{} f\cdot \varphi_n \,dx
              = \intl{\real}{} f\cdot \varphi \,dx$;
        \item $L_{1,\mathrm{loc}}(\real)$ --- лінійний простір;
        \item $\ln|x| \in L_{1,\mathrm{loc}}(\real)$;
        \item $\frac{1}{x} \not\in L_{1,\mathrm{loc}}(\real)$;
    \end{enumerate}
\end{exercise}

\begin{theory}
    % \ul{Узагальнені функції}\\
    \ul{Узагальненою функцією} $f$ називається лінійний неперервний функціонал на $\mathcal{D}$.
    Неперервність означає, що виконується імплікація: $(\varphi, \varphi_n \in \mathcal{D};
    \varphi_n \underset{\mathcal{D}}{\to} \varphi)  \Rightarrow (f(\varphi_n) \to f(\varphi))$.
    Значення $f(\varphi)$ позначається також $\pair{f}{\varphi}$. Множину узагальнених
    функцій позначаємо через $\mathcal{D}'$.
\end{theory}

\begin{exercise}
    Перевірити, що $\mathcal{D}'$ --- лінійний простір відносно стандартних операцій:
    $\pair{f_1 + f_2}{\varphi} \coloneqq \pair{f_1}{\varphi} + \pair{f_2}{\varphi}$;
    $\pair{\lambda f}{\varphi} \coloneqq \lambda \pair{f}{\varphi}$.
\end{exercise}

\begin{exercise}
    Довести, що наступні функціонали на $\mathcal{D}$ є узагальненими функціями:
    \begin{enumerate}
        \item $\pair{f}{\varphi} = \varphi(0)$;
        \item $\pair{f}{\varphi} = \varphi'(0) + \varphi(1)$;
        \item $\pair{f}{\varphi} = \intl{0}{1} (\sgn x) \varphi'(x) dx$;
        \item $\pair{f}{\varphi} = \varphi''(1) + \intl{\real}{} e^x \varphi'(x) dx$;
        \item $\pair{f}{\varphi} = \intl{-1}{\infty} ln|x| \cdot \varphi''(x) dx$.
    \end{enumerate}
\end{exercise}

\begin{theory}
    Узагальнена функція $f$ називається \ul{регулярною}, якщо існує звичайна
    функція $g$, для якої при всіх $\varphi \in \mathcal{D}$ має місце
    рівність $\pair{f}{\varphi} = \intl{\real}{} g\cdot \varphi \,dx$.
    В протилежному випадку $f$ називається \ul{нерегулярною (сингулярною)}.
\end{theory}
            % !TEX root = ../main.tex
\begin{exercise}\label{N:3_2_11}
    Доведіть, що якщо $f_1, f_2 \in L_{1, loc}(\real)$ та для 
    $\forall \; \varphi \in \mathcal{D}$ має місце рівність 
    $\int\limits_{\real} f_1 \cdot \varphi dx = \int\limits_{\real} f_2 \cdot \varphi dx$,
    то $f_1 = f_2$ майже скрізь, тобто $f_1 = f_2$ в сенсі простору $L_{1, loc}(\real)$.
\end{exercise}
\begin{theory}
    Результат задач \ref{N:3_2_8}\ref{N:3_2_8a} та \ref{N:3_2_11} дає підставу казати
    про вкладення простору $L_{1, loc}(\real)$ в $\mathcal{D}'$. Таким чином, маємо трійку просторів
    $\mathcal{D} \subset L_{1, loc}(\real) \subset \mathcal{D}'$.
\end{theory}

\begin{exercise}\label{N:3_2_12}
    Перевірити, чи є наступні функціонали узагальненими функціями і в разі позитивної відповіді перевірити,
    чи є вони сингулярними.
    \begin{enumerate}
        \item $\pair{f}{\varphi} = \varphi(0)$;
        \item $\pair{f}{\varphi} = \varphi(a)$ ($a \in \real$);
        \item $\pair{f}{\varphi} = \suml{n=1}{\infty}\varphi^{(n)}(n)$;
        \item $\pair{f}{\varphi} = \mathrm{V.P.}\intl{\real}{} \frac{\varphi(x)}{x} dx$;
        \item $\pair{f}{\varphi} = \intl{\real}{} \varphi''(x) e^{-x} dx$;
        \item $\pair{f}{\varphi} = \mathrm{V.P.}\intl{\real}{} \frac{\varphi(x) - \varphi(0)}{x^2} dx$.
    \end{enumerate}
\end{exercise}
\begin{theory}
    Узагальнені функції з пунктів а), б), г), д) номеру \ref{N:3_2_12} позначаються відповідно
    $\delta$, $\delta_a$, $\mathcal{P}\frac{1}{x}$ (або $\mathrm{V.P.} \frac{1}{x}$),
    $\mathcal{P}\frac{1}{x^2}$ (або $\mathrm{V.P.} \frac{1}{x^2}$). Іноді узагальнені функції 
    $\mathcal{P}\frac{1}{x}$ та $\mathcal{P}\frac{1}{x^2}$ позначаються коротше: $\frac{1}{x}$ та
    $\frac{1}{x^2}$ відповідно.
    Узагальнені функції $\delta$ та $\delta_a$ називаються 
    відповідно $\delta$-функцією Дірака та $\delta$-функцією Дірака, зосередженою в точці $a$.
\end{theory}
\begin{theory}
    Нехай $f \in \mathcal{D}'$, $h \in C^{\infty}(\real)$. За означенням \ul{добуток} $h\cdot f$ --- це 
    функціонал, дія якого на кожну основну функцію $\varphi$ визначена за правилом $\pair{h\cdot f}{\varphi} = \pair{f}{h\cdot \varphi}$.

    \noindent Для узагальненої функції $f \in\mathcal{D}'$ запроваджуються \ul{похідна} за правилом
    $\pair{f'}{\varphi} = -\pair{f}{\varphi'}$ (тут $\varphi$ --- довільна основна функція).
    Індуктивно запроваджуються похідні вищих порядків: $f^{(n)} := \left( f^{(n-1)}\right)'$.

    \noindent Нехай $f, f_n \in \mathcal{D}'$. \ul{Збіжність} $f_n \underset{\mathcal{D}'}{\to} f$ означає,
    що для кожної $\varphi \in \mathcal{D}$ є збіжність $\pair{f_n}{\varphi} \to \pair{f}{\varphi}$, $n \to \infty$.
    
    \noindent \ul{Збіжність ряду} $\suml{n=1}{\infty} f_n$ в $\mathcal{D}'$ --- це
    збіжність в $\mathcal{D}'$ послідовності його часткових сум.
\end{theory}
\begin{exercise}
    Нехай $f \in \mathcal{D}'$, $h \in  C^{\infty}(\real)$. Довести:
    \begin{enumerate}
        \item $h \cdot f \in \mathcal{D}'$;
        \item $(f = \delta) \Rightarrow (h\cdot f = h(0) \cdot \delta)$;
        \item $(g \in C^{\infty}(\real)) \Rightarrow (g\cdot(h\cdot f) = (g \cdot h) \cdot f)$;
        \item $\forall n \in \natur: \pair{f^{(n)}}{\varphi} = (-1)^n \pair{f}{\varphi^{(n)}}$;
        \item $(h \cdot f)' = h'\cdot f + h\cdot f'$.
    \end{enumerate}
\end{exercise}
\begin{exercise}
    Довести твердження:
    \begin{enumerate}
        \item $x^n \cdot \mathcal{P}\frac{1}{x} = x^{n-1}$ ($n \in \natur$);
        \item $x \cdot \mathcal{P}\frac{1}{x^2} = \mathcal{P}\frac{1}{x}$;
        \item $x^n \cdot \mathcal{P}\frac{1}{x^2} = x^{n-2}$ ($n \geq 2$).
    \end{enumerate}
\end{exercise}
\begin{exercise}
    Нехай $f$ --- регулярна узагальнена функція ($f \in L_{1, loc}$). Довести, що якщо
    $f \in C^1(\real)$, рівність $\pair{f'}{\varphi} = -\pair{f}{\varphi'}$ еквівалента
    формулі інтегрування частинами: $\intl{\real}{} f' \cdot \varphi dx = - \intl{\real}{} f\cdot \varphi' dx$.
    Таким чином, для функцій $f$ класу $C^1(\real)$ класична похідна співпадає з похідною $f'$ в сенсі
    теорії узагальнених функцій.
\end{exercise}
\begin{exercise}
    Обчислити:
    \begin{enumerate}
        \item $\eta'$, де $\eta(x) = \begin{cases}
            1, & x \geq 0 \\
            0, & x < 0 
        \end{cases}$ --- функція Хевісайда;
        \item $\left( |x| \right)'$;
        \item $\left( |x| \right)''$;
        \item $\left( x\cdot \sgn x\right)'$;
        \item $\left( [x] \right)'$;
        \item $\left(\eta(x) \cdot \sin x \right)'$;
        \item $\left(\sgn(\cos x)\right)'$.
    \end{enumerate}
\end{exercise}
\begin{exercise}
    Довести рівності:
    \begin{enumerate}
        \item $\left( \ln |x|\right)' = \mathcal{P}\frac{1}{x}$;
        \item $\left(\mathcal{P}\frac{1}{x}\right)' = -\mathcal{P}\frac{1}{x^2}$;
        \item $\left(\mathcal{P}\frac{1}{x^2}\right)' = -2\mathcal{P}\frac{1}{x^3}$,
        де $\pair{\mathcal{P}\frac{1}{x^3}}{\varphi} = \mathrm{V.P.}\intl{\real}{} \frac{\varphi(x) - \varphi(0) - x \varphi'(0)}{x^3} dx$.
    \end{enumerate}
\end{exercise}
\begin{exercise}
    Нехай $f, f_n \in \mathcal{D}'$, $f_n \underset{\mathcal{D}'}{\to} f$, $h \in C^{\infty}(\real)$. Доведіть:
    \begin{enumerate}
        \item $f'_n \underset{\mathcal{D}'}{\to} f'$;
        \item $h\cdot f_n \underset{\mathcal{D}'}{\to} h\cdot f$.
    \end{enumerate}
\end{exercise}
\begin{exercise}
    Довести, що в $\mathcal{D}'$ виконуються збіжності:
    \begin{enumerate}
        \item $\frac{\varepsilon}{\pi (x^2 + \varepsilon^2)} \to \delta$, $\varepsilon \to 0 +$;
        \item $\arctg(nx) \to \frac{\pi}{2} \sgn x$, $n \to \infty$;
        \item $\frac{1}{\sqrt{\varepsilon}} e^{-\frac{x^2}{2}} \to \sqrt{\pi} \cdot \delta$, $\varepsilon \to 0 +$;
        \item $\frac{1}{x} \sin \frac{x}{\varepsilon} \to \pi \cdot \delta$, $\varepsilon \to 0 +$;
        \item $\frac{\varepsilon}{x^2} \sin^2 \frac{x}{\varepsilon} \to \pi \cdot \delta$, $\varepsilon \to 0 +$;
        \item $t^n e^{ixt} \to 0$, $t \to +\infty$ ($n \geq 0$);
        \item $\frac{1}{\varepsilon} f\left( \frac{x}{\varepsilon}\right) \to \delta \cdot \intl{a}{b} f(x) dx$, $\varepsilon \to 0+$ ($f \in L_1 [a; b]$). 
    \end{enumerate}
\end{exercise}
\begin{exercise}
    Довести, що узагальнені функції $\delta, \delta', \delta'', ..., \delta^{(n)}$
    лінійно незалежні в $\mathcal{D}'$.
\end{exercise}
\begin{exercise}\label{N:3_2_21}
    Довести рівності в $\mathcal{D}'$:
    \begin{enumerate}
        \item $2 \suml{k \in \mathbb{Z}}{} \delta_{k \pi} = |\sin x|'' + |\sin x|$;
        \item $2 \suml{k \in \mathbb{Z}}{} \delta_{\left(k+\frac{1}{2}\right) \pi} = |\cos x|'' + |\cos x|$.
    \end{enumerate}
\end{exercise}
\begin{exercise}
    Доведіть, що загальний розв'язок рівняння $y' = 0$ в $\mathcal{D}'$ має вид $y = C$ 
    (тобто, рівняння має лише <<класичний>> розв'язок).
\end{exercise}
\begin{exercise}
    Знайти загальні розв'язки рівнянь в $\mathcal{D}'$:
    \begin{enumerate}
        \item $y'' = 0$;
        \item $y' = \eta$ ($\eta$ --- функція Хевісайда);
        \item $y'' = \delta$;
        \item $x y' = \mathcal{P}\frac{1}{x}$;
        \item $u' = u$;
        \item $u'' = u$;
        \item $u'' = -u$; 
    \end{enumerate}
\end{exercise}
\begin{exercise}
    Довести, що $u = C_1 + C_2 \eta(x) + \ln|x|$, де $C_1$ та $C_2$ --- довільні сталі,
    є загальним розв'язком рівняння $x u' = 1$.
\end{exercise}
\begin{exercise}
    Нехай ряд $\suml{n=0}{\infty} a_n \delta^{(n)}$ ($a_n \in \real$) збігається в просторі $\mathcal{D}'$.
    Довести, що існує таке $N$, що $a_n = 0$ при $n \geq N$.
\end{exercise}
\begin{exercise}
    Побудувати послідовність функцій $g_n \in L_{1, loc}(\real)$,
    яка збігається в $\mathcal{D}'$ до функції $\delta^{(m)}$ ($m \in \natur$).
\end{exercise}
\begin{theory}
    Нехай $f \in \mathcal{D}'$, $U$ --- відкрита множина в $\real$. Позначення
    $f |_{U} = 0$ (обмеження $f$ на $U$ дорівнює нулю) означає, що
    виконується умова: $(\varphi \in \mathcal{D}, \supp \varphi \subset U) \Rightarrow (\pair{f}{\varphi} = 0)$.
    
    \noindent Замкнена множина $M \subset \real$ називається \ul{носієм} узагальненої
    функції $f$, якщо $U = \real \setminus M$ --- найбільша відкрита множина, для якої $f |_{U} = 0$.
    Позначення: $M = \supp f$.

    \noindent Нехай $f$ --- сингулярна узагальнена функція. Найменше натуральне число $n$, для якого існує регулярна функція $g$, що задовольняє умову 
    $g^{(n)} = f$, називається \ul{порядком сингулярності} $f$.
\end{theory}
\begin{exercise*}
    Нехай $f \in \mathcal{D}'$, $\{ U_\alpha\}$ --- сім'я всіх відкритих множин, для яких $f |_{U_\alpha} = 0$.
    Позначимо $U = \bigcup\limits_{\alpha} U_\alpha$. Доведіть: $f |_{U} = 0$, тобто $U$ --- найбільша
    відкрита множина, для якої виконується ця рівність. Тим самим $U \in \{ U_\alpha\}$ і означення $\supp f$ 
    є коректним.
\end{exercise*}
\begin{exercise}
    Нехай $f \in \mathcal{D}'$. Доведіть: $\supp f' \subset \supp f$.
\end{exercise}
\begin{exercise}
    Нехай $f = \suml{k=0}{m} a_k \delta^{(k)}$, $m \in \natur$, $a_k \in \real$, $a_m \neq 0$.
    \begin{enumerate}
        \item Знайти $\supp f$;
        \item Знайти порядок сингулярності функції $f$.
    \end{enumerate}
\end{exercise}
\begin{exercise}
    Доведіть, що кожна сингулярна узагальнена функція $f \in \mathcal{D}'$ з одноточковим носієм ($\supp f = \{ a\}$) має вид
    $f = \suml{k=0}{m} = c_k \delta_a^{(k)}$ ($m \in \natur$, $c_k \in \real$), а тому має скінченний порядок сингулярності.
\end{exercise}
\begin{exercise*}
    Довести, що кожна узагальнена функція $f \in \mathcal{D}'$ з компактним носієм має скінченний порядок сингулярності.
\end{exercise*}
\begin{exercise*}
    Чи можна ввести в просторі $\mathcal{D}'$ метрику, збіжність за якою співпадала б зі збіжністю в $\mathcal{D}'$?
\end{exercise*}
        \newpage
    \chapter*{Підказки}
    \addcontentsline{toc}{chapter}{Підказки}
        % !TEX root = ../main.tex

\noindent\ref{N:1_1_9}. Нехай $\bigcap\limits_{k=1}^m {Ker} \varphi_k \subset {Ker}\varphi$.
Без втрати загальності можна вважати, що $\forall k \in \set{1, ..., m}: {Ker} \varphi_k \not\supset \bigcap\limits_{j \neq k} {Ker} \varphi_j$.
Звідси зробіть висновок, що
$\forall k \in \set{1, ..., n}$ $\exists \; x_k \in L$
такий, що $\forall j \in \set{1, ..., m} : \varphi_j(x_k) = \delta_{jk}$ (<<символ Кронекера>>)
і розгляньте функціонал $\sum\limits_{k=1}^m \varphi(x_k) \cdot \varphi_k$.

\noindent\ref{N:1_1_12}.
ґ) $\norm{x}_1 = \underset{0\leq t \leq 1}{\max} |x(t)| + \underset{0\leq t \leq 1}{\max} |x'(t)|$, тож $\norm{A} \leq 1$.
Зворотну нерівність можна одержати за допомогою послідовності $x_n(t) = \frac{1}{n} t^n$.

\noindent з) $\norm{A\vec{x}} \leq 2 \cdot \sum\limits_{k=1}^{\infty} |x_k| = 2 \norm{\vec{x}}$, $A \vec{e_1} = (1, 1, 0, ...)$. $\norm{A} = 2$.

\noindent і) $\norm{Ax}^2 = \int\limits_0^1 \varphi^2 (t) x^2(t) dt \leq \left(\underset{[0;1]}{\max} |\varphi(t)|\right)^2 \cdot \norm{x}^2 $, 
тому $\norm{A} \leq \underset{[0;1]}{\max} |\varphi(t)|$.
З іншого боку, якщо $\underset{[0;1]}{\max} |\varphi(t)| = C > 0$ (випадок $C = 0$ очевидний), то
$\forall \varepsilon \in (0, C) \; \exists \; \text{проміжок } \Delta \subset [0;1]$,
на якому $|\varphi(t)| > C - \varepsilon$. Нехай $x = j_\Delta$ ($x(t) = 1$, якщо $t \in \Delta$ та $x(t) = 0$, якщо $t \notin \Delta$).
Тоді $\norm{x}^2 = \mu(\Delta)$, $\norm{Ax}^2 \geq \int\limits_\Delta (C-\varepsilon)^2 dt = (C-\varepsilon)^2 \cdot \mu(\Delta)$ (тут $\mu$ --- міра <<довжина>>).
Звідси отримуємо $\norm{A} = \underset{[0;1]}{\max} |\varphi(t)|$.

\noindent ї) $\left| (Ax)(t) \right|^2 = t^2 \cdot \left( \int\limits_0^1 s x(s) ds\right)^2 \leq t^2 \cdot \left( \int\limits_0^1 s^2 ds\right)^2 \cdot \left( \int\limits_0^1 x^2(s) ds\right)^2 = \frac{1}{3} \cdot t^2 \cdot \norm{x}^2$.
$\norm{Ax}^2 = \int\limits_0^1 \left| (Ax)(t) \right|^2 dt \leq \frac{1}{9} \cdot \norm{x}^2$, звідки $\norm{A} \leq \frac{1}{3}$.
Зворотну нерівність отримуємо підстановкою $x(t) = t$.

\end{document}