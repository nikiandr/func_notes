% !TEX root = ../main.tex

\begin{theory}
    Нехай $X$ --- банахів простір (над полем $\real$ або $\complex$),
    $T(t)$ --- однопараметрична сім'я обмежених лінійних операторів в $X$, $t \in [0; +\infty)$.
    Сім'я операторів $T(t)$ називається \uline{(однопараметричною) операторною півгрупою}, якщо
    виконуються властивості:
    \begin{enumerate}[label = \arabic*)]
        \item $T(0) = I$;
        \item $T(t+s) = T(t) T(s)$ $(t, s \geq 0)$.
    \end{enumerate}
    Якщо, додатково, виконується умова \uline{сильної неперервності}:
    \begin{enumerate}[label = \arabic*)]
        \item[3)] $\forall \; x \in X$ функція $[0; +\infty) \ni t \mapsto T(t)x \in X$ є неперервною,
    \end{enumerate}
    то операторна півгрупа називається \uline{$c_0$-півгрупою}.
    Якщо замість 3) виконується умова 
    \begin{enumerate}[label = \arabic*)]
        \item[3$^\prime$)] $[0; +\infty) \ni t \mapsto T(t) \in L(X)$ --- неперервна операторна функція,
    \end{enumerate} 
    то півгрупу $T(t)$ називають \uline{неперервною за нормою} або
    \ul{рівномірно неперервною}.
    \noindent\uline{Генератор} $A$ $c_0$-півгрупи $T(t)$ визначається двома умовами:
    \begin{enumerate}
        \item $D(A) = \left\{ x \in X \mid \exists \; \underset{t\to 0+0}{\lim} \frac{1}{t} (T(t)x - x)\right\}$;
        \item для $x \in D(A)$ $Ax := \underset{t\to 0+0}{\lim} \frac{1}{t} (T(t)x - x)$.
    \end{enumerate}
    Інше позначення генератора --- $A = T^\prime (0)$.

    \noindent $c_0$-півгрупа називається \uline{$c_0$-півгрупою стиску},
    якщо для всіх $t \geq 0$ виконується умова $\norm{T(t)} \leq 1$.
\end{theory}

\begin{exercise}
    Довести наступні твердження:
    \begin{enumerate}
        \item Рівномірно неперервна оператора півгрупа $T(t)$ є $c_0$-півгрупою;
        \item Якщо $T(t)$ --- $c_0$-півгрупа в банаховому просторі $X$, то існують числа $M, \; \omega \in \real$,
        для яких при всіх $t \geq 0$ виконується нерівність $\norm{T(t)} \leq M e^{\omega t}$.
    \end{enumerate}
\end{exercise}

\begin{exercise}
    Нехай $A$ --- генератор $c_0$-півгрупи $T(t)$ в банаховому просторі $X$. Доведіть наступні твердження:
    \begin{enumerate}
        \item $A$ --- лінійний оператор;
        \item $\overline{D(A)} = X$;
        \item $A$ --- замкнений оператор в $X$.
    \end{enumerate}
\end{exercise}

\begin{exercise}
    Нехай $A$ --- генератор $c_0$-півгрупи $T(t)$ в банаховому просторі $X$. Довести:
    \begin{enumerate}
        \item $\forall \; x \in D(A)$ при всіх $t \geq 0$: $T(t)x \in D(A)$ і при цьому $\exists \; \frac{d}{dt} T(t)x = A T(t)x = T(t) Ax$;
        \item $\forall \; x \in D(A)$ при всіх $t \geq 0$ має місце формула $T(t)x - x = \int\limits_0^t T(s) Ax ds$;
        \item $\forall \; x \in D(A)$ при всіх $t \geq 0$: $\int\limits_0^t T(s)x ds \in D(A)$ і при цьому $T(t)x - x = A \int\limits_0^t T(s)x ds$.
    \end{enumerate}
\end{exercise}

\begin{exercise}
    Нехай $T(t)$, $S(t)$ --- дві операторні $c_0$-півгрупи в $X$, $T^\prime(0) = S^\prime(0)$. Доведіть:
    для кожного $t \geq 0$ виконується рівність $T(t) = S(t)$. 
\end{exercise}

\begin{exercise}
    Нехай для операторної півгрупи $T(t)$ в банаховому просторі $X$ крім умов
    $1$) та $2$) виконується умова $\tilde{3}$): $\norm{T(t) - I} \to 0$ при $t\to 0 + 0$ (неперервність за нормою в нулі).
    Доведіть: $T(t)$ --- рівномірно неперервна півгрупа.
\end{exercise}

\begin{exercise}
    Нехай для операторної півгрупи $T(t)$ в банаховому просторі $X$ крім умов
    $1$) та $2$) виконується умова $\hat{3}$): $\forall x \in X : \norm{T(t)x - x} \to 0$ при $t\to 0 + 0$ (сильна неперервність в нулі).
    Доведіть: $T(t)$ --- $c_0$-півгрупа.
\end{exercise}

\begin{theory}
    Для обмеженого, визначеного на всьому банаховому просторі $X$ оператора
    $A \in L(X)$ запроваджується оператор $e^{tA}$ за формулою
    $e^{tA} = \sum\limits_{n=0}^{\infty} \frac{1}{n!} t^n A^n$ ($A^0 = I$).
\end{theory}

\begin{exercise}
    Перевірте коректність означення $e^{tA}$, $A \in L(X)$:
    збіжність відповідного ряду і включення $e^{tA} \in L(X)$.
\end{exercise}