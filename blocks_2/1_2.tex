% !TEX root = ../main.tex

\begin{exercise}
    $X$ --- банахів простір, $Y$ --- замкнений підпростір, 
    $P$ --- проектор $X$ на $Y$ паралельно $Z$ ($Z = \mathrm{Ker} P$). 
    Довести: $P$ --- обмежений тоді й тільки тоді, коли $Z$ --- 
    замкнений підпростір в $X$.
\end{exercise}

\begin{exercise}
    Нехай $X$, $Y$ --- замкнені підпростори в банаховому просторі $Z$, 
    $Z = X \dotplus Y$. Довести, що існує число $K$, для якого при всіх $z \in Z$ 
    виконуються нерівності: $\norm{x} \leq K\norm{z}$; $\norm{x} \leq K\norm{z}$ (тут 
    $x \in X$, $y \in Y$, $z = x + y$). 
\end{exercise}

\begin{exercise}
    Розглянемо оператори $i_X$: $X \ni x \rightarrow 
    \sumpair{x}{0} \in X \dotplus Y$;
    $i_Y$: $Y \ni y \rightarrow 
    \sumpair{0}{y} \in X \dotplus Y$;
    $P_X$: $X \dotplus Y \ni \sumpair{x}{y} \rightarrow 
    x \in X $;
    $P_Y$: $X \dotplus Y \ni \sumpair{x}{y} \rightarrow 
    y \in Y $. $X$ та $Y$ --- нормовані простори.
    \begin{enumerate}
        \item Перевірити, що для норми $\norm{\sumpair{x}{y}} = 
        \norm{x}_X + \norm{y}_Y$ в $X \dotplus Y$ оператори $i_X$, $i_Y$, 
        $P_X$, $P_Y$ --- обмежені, і знайти їх норми;
        \item Нехай $\norm{\cdot}_1$ --- інша норма в $X \dotplus Y$, 
        оператори $i_X$, $i_Y$, $P_X$, $P_Y$ --- обмежені. Чи можна 
        стверджувати, що $\norm{\cdot}_1 \sim \norm{\cdot}$?
        \item Нехай $\norm{\cdot}_2$ --- інша норма в $X \dotplus Y$,
        $X$, $Y$, $X \dotplus Y$ --- банахові. Виконуються умови: 
        $\norm{\sumpair{x}{0}}_2 = \norm{x}_X$; $\norm{\sumpair{0}{y}} 
        = \norm{y}_Y$. Чи можна 
        стверджувати, що $\norm{\cdot}_2 \sim \norm{\cdot}$?
    \end{enumerate}
\end{exercise}

\begin{exercise}\label{N:2_1_18}
    Нехай $\{a_n\}$ --- числова послідовність; оператор $A: \ell_2 
    \rightarrow \ell_2$ визначено умовою: $A\vec{x} = 
    (a_1x_1, a_2x_2, ...)$. Нехай $\inf\{|\frac{a_n}{n}|\} > 0$ та 
    $D(A) = \{\vec{x} \in \ell_2 | \suml{n=1}{\infty}|a_n|^2|x_n|^2 
    < \infty\}$. Довести лінійність, необмеженість, замкненість оператора 
    $A$ та його щільну визначеність ($\overline{D(A)} = \ell_2$).
\end{exercise}

\begin{exercise}
    Вивести теорему Банаха про обернений оператор з теореми Банаха про 
    замкнений графік.
\end{exercise}

\begin{exercise}
    Довести, що функціонал $\varphi: x \rightarrow x(0)$, що діє в 
    $L_2[0; 1]$ з областю визначення $D(\varphi) = C[0; 1]$, не є 
    замкненим.
\end{exercise}

\begin{theory}
    Нехай $A, B$ --- лінійні оператори з $X$ в $Y$. Оператор $B: X 
    \rightarrow Y$ називається \ul{розширенням} $A$, якщо $\Gamma_A 
    \subset \Gamma_B$. Позначення: $A \subset B$. В разі, якщо 
    $\Gamma_B = \overline{\Gamma_A}$, оператор $B$ називається 
    \ul{замиканням} оператора $A$, а оператор $A$ є таким, що 
    \ul{допускає замикання}. Позначення: $B = \overline{A}$.
\end{theory}

\begin{exercise}
    Нехай $A, B: X \rightarrow Y$ --- лінійні оператори. Доведіть:
    \begin{enumerate}
        \item $(A \subset B) \Leftrightarrow (D(A) \subset D(B); 
        \text{для }x \in D(A): Ax = Bx)$;
        \item ($A$ допускає замикання) $\Leftrightarrow$ ($\exists$ 
        замкнений оператор $B$ такий, що $A \subset B$);
        \item Замикання оператора $A$ --- це найменше замкнене 
        розширення $A$.
    \end{enumerate}
\end{exercise}

\begin{exercise}
    Нехай $A: X \supset D(A) \ni x \rightarrow Ax \in Y$. Довести 
    еквівалентність двох тверджень:
    \begin{enumerate}
        \item $A$ допускає замикання;
        \item виконується умова: $(D(A) \ni x_n \rightarrow 0; 
        Ax_n \rightarrow y) \Rightarrow (y = 0)$.
    \end{enumerate}
\end{exercise}

\begin{exercise}
    Нехай $A: X \rightarrow Y$ --- обмежений оператор. Тоді $A$ допускає 
    замикання і $D(\overline{A}) = \overline{D(A)}$.
\end{exercise}

\begin{exercise}
    Нехай оператор $A: X \rightarrow Y$ допускає замикання. Довести:
    $D(\overline{A}) \subset \overline{D(A)}$, $\mathrm{Im}(\overline{A}) \subset 
    \overline{\mathrm{Im}A}$.
\end{exercise}

\begin{exercise}
    Нехай $A: \ell_2 \rightarrow \ell_2$ визначений формулою $A: e_n 
    \rightarrow ne_1$, де $e_n = (\underbrace{0, ..., 0}_{n-1}, 1, 0, ...)$. 
    Знайти $D(A)$ і довести, що $A$ не допускає замикання.
\end{exercise}