% !TEX root = ../main.tex

\begin{exercise}
    Довести наступний \ul{критерій $c_0$-півгруп, неперервних за 
    нормою}:
    \begin{enumerate}
        \item ($A \in L(X)$) $\Rightarrow$ ($T(t) = e^{tA}$ 
        --- рівномірно неперервна півгрупа);
        \item ($T(t)$ --- рівномірно неперервна півгрупа) $\Rightarrow$ 
        ($A = T^{\prime}(0) \in L(X)$ : $T(t) = e^{tA}$ $\forall t \geq 0$).
    \end{enumerate}
\end{exercise}

\begin{exercise}
    Нехай $T(t)$ --- $c_0$-півгрупа в $X$, $\lambda \in K$ ($K$ --- основне поле). Доведіть, 
    що оператори $S(t) = e^{-\lambda t}T(t)$ утворюють $c_0$-півгрупу і при цьому 
    $S^{\prime}(0) = T^{\prime}(0) - \lambda I$. 
\end{exercise}

\begin{exercise}
    Нехай $T(t)$ --- $c_0$-півгрупа, $\forall t \geq 0 : \norm{T(t)} \leq Me^{\omega t}$;
    $A = T^{\prime}(0)$. Для $\lambda \in \complex$, що задовольняє нерівність $\mathfrak{Re}\lambda > 0$,
    розглянемо \ul{перетворення Лапласа $R_\lambda$ півгрупи $T(t)$} $R_\lambda : 
    X \rightarrow X$, визначене формулою $R_\lambda x = \intl{0}{\infty}e^{-\lambda t} T(t)
    x dt$. Доведіть, що:
    \begin{enumerate}
        \item $\forall x \in X$, $\mathfrak{Re} \lambda > \omega$ оператор 
        $R_\lambda$ є коректно визначеним оператором з $L(X)$;
        \item ($\mathfrak{Re} \lambda > \omega$) $\Rightarrow$ ($\inv{(\lambda - A)}x = R_\lambda x$ 
        для $x \in X$), тобто ($\mathfrak{Re} \lambda > \omega$) $\Rightarrow$ ($\lambda \in \rho(A)$
        та $R(\lambda; A) = R_\lambda$).
    \end{enumerate}
\end{exercise}

\begin{exercise}
    Нехай $K = \complex$. Довести:
    \begin{enumerate}
        \item Якщо $T(t)$ --- $c_0$-півгрупа стиску, то ($\mathfrak{Re}\lambda > 0$) $\Rightarrow$
        ($\lambda \in \rho(A)$; $\norm{\inv{\lambda - A}} \leq \frac{1}{\mathfrak{Re} \lambda}$);
        \item Якщо $T(t)$ --- довільна $c_0$-півгрупа, $\norm{T(t)} \leq Me^{\omega t}$ 
        ($t \geq 0$), то ($\mathfrak{Re} \lambda > \omega$) $\Rightarrow$ 
        ($\norm{\inv{\lambda - A}} \leq \frac{M}{\mathfrak{Re}\lambda - \omega}$).
    \end{enumerate}
\end{exercise}

\begin{theory}
    \begin{theorem*}[Хіллє, Іосіда]
        Лінійний оператор $A$ в банаховому просторі $X$ є 
    генератором $c_0$-півгрупи стиску тоді й тільки тоді, коли він задовольняє 
    наступні умови:
    \begin{enumerate}
        \item $\overline{D(A)} = X$, $A$ --- замкнений оператор;
        \item $(0, +\infty) \subset \rho(A)$ і при цьому для всіх $\lambda > 0 : \norm{R(\lambda; A)} \leq \frac{1}{\lambda}$.
    \end{enumerate}
    \end{theorem*}
    \begin{theorem*}[Хіллє, Іосіда, Міллдер, Феллер, Філліпс]
        Лінійний оператор $A$ в банаховому 
    просторі $X$ є генератором $c_0$-півгрупи тоді й тільки тоді, коли він задовольняє 
    такі умови:
    \begin{enumerate}
        \item $\overline{D(A)} = X$, $A$ --- замкнений оператор;
        \item $\exists \; M \geq 1, \omega \in \real : \forall \lambda > \omega, \forall \; n \in \natur : \lambda \in \rho(A)$ та
        $\norm{\lambda - A}^{-n} \leq \frac{M}{(\lambda - \omega)^n}$.
    \end{enumerate}
    \end{theorem*}
\end{theory}