% !TEX root = ../main.tex

\begin{theory}
    Нехай $X$, $Y$ --- нормовані простори з нормами $\norm{\cdot}_X$ та 
    $\norm{\cdot}_Y$ відповідно; $I_X$ та $I_Y$ --- відповідні тотожні 
    оператори; $A \in L(X, Y)$. Оператор $B \in L(Y, X)$ називається 
    \ul{лівим оберненим до $A$}, якщо $BA = I_X$; $C \in L(Y, X)$ 
    називається \ul{правим оберненим до $A$}, якщо $AC = I_Y$. В разі якщо оператор 
    $B \in L(Y, X)$ є одночасно лівим і правим оберненим до $A$, він називається 
    (\ul{неперервно}) \ul{оберненим} до $A$ (позначення: $B = \inv{A}$), 
    а сам оператор $A$ --- \ul{неперервно оборотним}.
\end{theory}

\begin{exercise}
    Нехай оператор $A \in L(X, Y)$ має лівий обернений $B \in L(Y, X)$ та правий обернений 
    $C \in L(Y, X)$. Доведіть: 
    \begin{enumerate}
        \item $\mathrm{Im} A = Y$, $\mathrm{Im} B = X$;
        \item $\mathrm{Ker} A = {0}$, $\mathrm{Ker} C = {0}$;
        \item $B = C$ --- обернений оператор до $A$.
    \end{enumerate}
\end{exercise}

\begin{exercise}
    Нехай $A \in L(X, Y)$. Довести, що наступні дві умови еквівалентні:
    \begin{enumerate}
        \item $A$ --- неперервно оборотний;
        \item $\mathrm{Im} A = Y$ та $\exists \; m > 0$ таке, що $\forall x \in X:\norm{Ax} \geq m \norm{x}$. 
    \end{enumerate}
\end{exercise}

\begin{exercise}
    Нехай $\mathrm{dim} X < \infty$, $A$ --- лінійний оператор в $X$. Тоді $A$ --- неперервно оборотний 
    тоді й тільки тоді, коли $\mathrm{det} A \neq 0$.
\end{exercise}

\begin{exercise}
    Нехай $X$ --- банахів простір; $A \in L(X)$, $\norm{A} < 1$. Тоді оператор $I-A$ --- 
    неперервно оборотний і при цьому $\inv{I-A} = I + \sum\limits_{n=1}^\infty A^n$.
\end{exercise}

\begin{exercise}
    Нехай оператор $A: \ell_2 \rightarrow \ell_2$ визначено формулою: 
    $A\vec{x} = (a_1x_1, a_2x_2, ...)$, де $\underset{n}{\sup}|a_n| < \infty$. Довести: 
    $A$ --- неперервно оборотний тоді й тільки тоді, коли $\underset{n}{\inf}|a_n| > 0$. 
    Знайти $\inv{A}$.
\end{exercise}

\begin{exercise}
    Нехай $X$ --- банахів простір, $A, B \in L(X)$. Нехай $A$ --- неперервно оборотний, 
    $\norm{B} < \inv{\norm{\inv{A}}}$. Довести: $A + B$ неперервно оборотний і знайти 
    $\inv{(A+B)}$.
\end{exercise}

\begin{exercise}
    Доведіть, що множина неперервно оборотних операторів в банаховому просторі $X$ є відкритою 
    в $L(X)$. 
\end{exercise}

\begin{exercise}
    Нехай $X$ --- нормований простір, $A \in L(X)$, $n \in \mathbb{N}$. Довести: оператори 
    $A$ та $A^n$ одночасно неперервно оборотні чи ні.
\end{exercise}

\begin{exercise}
    Нехай $A, B \in L(X)$, оператори $A$, $BA$ --- неперервно оборотні. Довести: оператор 
    $B$ також неперервно оборотний.
\end{exercise}

\begin{exercise}
    Нехай $X$ --- нормований простір; $A, B \in L(X)$; оператор $(I - AB)$ --- неперервно 
    оборотний; $\inv{(I - AB)} = C$. Довести: $(I - BA)$ --- неперервно оборотний, і знайти 
    $\inv{(I - BA)}$.
\end{exercise}

\begin{exercise}
    Нехай $A: C[0, 1] \rightarrow C[0, 1]$ визначений формулою: $(Ax)(t) = \alpha(t)x(t)$, 
    де $\alpha \in C[0, 1]$ --- фіксована функція. Довести: $A$ -- неперервно оборотний 
    тоді й тільки тоді, коли $\alpha(t) \neq 0$ для всіх $t \in [0, 1]$. 
\end{exercise}

\begin{exercise}
    Нехай $A, B : C[-1, 1] \rightarrow C[-1, 1]$ визначено формулами: $(Ax)(t) = 
    x(t^2)$; $(Bx)(t) = x(t^3)$. Доведіть, що $\inv{A}$ не існує, а $B$ --- неперервно 
    оборотний і знайти $\inv{B}$.
\end{exercise}