% !TEX root = ../main.tex

\begin{theory}
    Нехай $X$ --- нормований простір; $X^*$ --- його спряжений. Послідовність $\{x_n\}$ 
    векторів простору $X$ називається \ul{сильно збіжною} до вектора $x \in X$, якщо 
    $\norm{x_n - x} \underset{n \rightarrow \infty}{\rightarrow} 0$. Це є звичайна збіжність 
    за нормою. Позначення: $x_n \rightarrow x$.

    Послідовність $x_n \in X$ називається \ul{слабо} (або \ul{слабко}) збіжною до $x \in X$, 
    якщо $\forall \varphi \in X^* : \varphi(x_n) \underset{n \rightarrow \infty}{\rightarrow} 
    \varphi(x)$. Позначення: $x_n \underset{\text{сл.}}{\rightarrow} x$ 
    (або $x_n \rightharpoonup x$).

    Послідовність функціоналів $\varphi_n \in X$ \ul{сильно збігається} до $\varphi \in X^*$, 
    якщо $\norm{\varphi_n - \varphi} \underset{n \rightarrow \infty}{\rightarrow} 0$ 
    (збіжність за нормою в $X^*$). \ul{Слабка збіжність} $\varphi_n$ до $\varphi$ --- 
    це просто поточкова збіжність  $\forall x \in X : \varphi_n(x) \underset{n \rightarrow \infty}{\rightarrow} \varphi(x)$.
    Позначаємо її так: $\varphi_n \overset{*}{\rightarrow} \varphi$. 
    Її також називають $*$-слабкою, бо в $X^*$ є і інша слабка збіжність: $\forall \alpha 
    \in X^{**} : \alpha(\varphi_n) \rightarrow \alpha(\varphi)$ (як в будь-якому 
    нормованому просторі).

    Для послідовності операторів $A_n \in L(X, Y)$ будемо розглядати \ul{збіжність за нормою} 
    (або \ul{рівномірну}): $\norm{A_n - A} \underset{n \rightarrow \infty}{\rightarrow} 0$. 
    Позначення: $A_n \rightrightarrows A$.

    Також нам важлива \ul{сильна збіжність}, яка визначена умовою: 
    $A_n x \rightarrow Ax$ для $\forall x \in X$. Позначення: $A_n \overset{s}{\rightarrow} A$ 
    (або інакше: <<$A_n \rightarrow A$ сильно>>).

    Є ще <<слабка операторна збіжність>> $A_n \underset{\text{сл.}}{\rightarrow} A$ 
    (або $A_n \rightharpoonup A$). Слабка збіжність за означенням --- це умова: 
    $\forall x \in X$, $\forall \varphi \in X^*$: $\varphi(A_nx) \rightarrow \varphi(Ax)$.
\end{theory}

\begin{exercise}
    Доведіть, що у випадку гільбертового простору $H$ слабка збіжність $x_n 
    \underset{\text{сл.}}{\rightarrow} x$ ($x, x_n \in H$) рівносильна умові: 
    $\forall y \in H : \dotprod{x_n}{y} \rightarrow \dotprod{x}{y}$.
\end{exercise}

\begin{exercise}
    Доведіть, що сильна збіжність послідовності $x_n$ нормованого простору $X$ гарантує 
    її слабку збіжність. Доведіть, що у випадку скінченновимірного $X$ має місце і зворотній 
    факт. Наведіть приклад нескінченновимірного простору $X$ і слабко збіжної послідовності 
    $x_n$, яка не має сильної границі.
\end{exercise}

\begin{exercise}
    Нехай $\{x_n\}$ --- слабо збіжна послідовність векторів гільбертового простору $H$. 
    Доведіть : 
    \begin{enumerate}[label=\ukr*)]
        \item послідовність $\{x_n\}$ --- обмежена;
        \item послідовність $\{x_n\}$ не може мати двох різних слабких 
        границь.
    \end{enumerate}
\end{exercise}

\begin{exercise}
    Нехай $\{x_n\}$ --- слабо збіжна послідовність векторів в нормованому 
    просторі $X$. Доведіть :
    \begin{enumerate}[label=\ukr*)]
        \item послідовність $\{x_n\}$ --- обмежена;
        \item послідовність $\{x_n\}$ не може мати двох різних 
        слабких границь.
    \end{enumerate}
\end{exercise}

\begin{exercise}
    Нехай $X$ - банахів простір; $\varphi_n \in X^*$, $\varphi_n$ --- 
    слабко збіжна. Доведіть:
    \begin{enumerate}[label=\ukr*)]
        \item послідовність $\{\varphi_n\}$ обмежена (за нормою в $X^*$);
        \item послідовність $\{\varphi_n\}$ не може мати двох різних слабких 
        границь.
    \end{enumerate}
\end{exercise}

\begin{exercise}
    Нехай $A_n$, $A \in L(X, Y)$. Доведіть: 
    \begin{enumerate}
        \item $(A_n \rightrightarrows A) \Rightarrow (A_n 
        \overset{s}{\rightarrow} A)$;
        \item взагалі кажучи: $(A_n 
        \overset{s}{\rightarrow} A) \nRightarrow (A_n \rightrightarrows A)$;

        Розгляньте приклад: $H = \ell_2$, $P_nx = (x_1, x_2, ..., x_n, 0, 0, ...)$;

        \item $(A_n \overset{s}{\rightarrow} A) \Rightarrow (\underset{n}{\sup}\norm{A_n} 
        < \infty$), $X$ --- повний простір;

        \item $A_n$ не може мати двох різних сильних границь.
    \end{enumerate}
\end{exercise}

\begin{exercise}
    Наступні послідовності $x_n \in X$ дослідити на сильну і слабку збіжність: 
    \begin{enumerate}
        \item $X = \ell_2$; $\vec{x_n} = (1, \frac{1}{2}, \frac{1}{3}, ..., \frac{1}{n}, 0, 0
        , ...)$;
        \item $X = \ell_2$; $\vec{x_n} = ( \underbrace{0, ..., 0}_{n-1} ,
        1, \frac{1}{2}, \frac{1}{3}, ...)$;
        \item $X = \ell_2$; $\vec{x_n} = ( \underbrace{1, ..., 1}_{n-1} ,
        \frac{1}{n}, \frac{1}{n+1}, ...)$;
        \item $X = L_2[0, 1]$; $x_n(t) = t^n$;
        \item $X = L_2[0, 1]$; $x_n(t) =  \begin{cases}
            \sqrt{n} & t \in [0; \frac{1}{n}] \\
            0 & t \in (\frac{1}{n}; 1]
        \end{cases}$;
        \item $X = L_2[0, 1]$; $x_n(t) = e^{int}$;
        \item $X = L_2[0, 1]$; $x_n(t) = sin(2^nt)$;
        \item $X = \ell_p \; (1 < p < \infty)$; $x_n = ( \underbrace{0, ..., 0}_{n-1} ,
        1, \frac{1}{2}, \frac{1}{3}, ...)$;
        \item $X = c_0$; $x_n = ( \underbrace{0, ..., 0}_{n-1} ,
        1, \frac{1}{2}, \frac{1}{3}, ...)$.
    \end{enumerate}
\end{exercise}