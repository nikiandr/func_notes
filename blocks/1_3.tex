% !TEX root = ../main.tex

\begin{exercise}
    Лінійний функціонал $\varphi: C[a; b] \rightarrow \mathbb{R}$ називається \underline{невід'ємним},
    якщо для кожної невід'ємної на $[a; b]$ функції $x \in C[a, b]$ має місце нерівність 
    $\varphi(x) \geq 0$. Довести, що невід'ємний функціонал $\varphi$ є неперервним і при цьому
    $\norm{\varphi} = \varphi(\mathds{1})$, де $\mathds{1}$ --- тотожно одинична функція.
\end{exercise}

\begin{exercise}
    Довести, що лінійний оператор $A$, який діє в нормованому просторі $X$, є неперервним тоді
    й тільки тоді, коли множина $\set{x \in X \mid \norm{Ax} < 1}$ має внутрішні точки.
\end{exercise}

\begin{exercise}
    Нехай $X$ --- банахів простір; $A \in L(X)$ та існує $c > 0$ таке, що для кожного $x \in X$
    виконується нерівність $\norm{Ax} \geq c \norm{x}$. Довести, що $\mathrm{Im} A$ є замкненим підпростором.
\end{exercise}

\begin{exercise}
    Довести, що після заміни норм в нормованих просторах $X$ і $Y$ на еквівалентні,
    нова норма в $L\left( X, Y\right)$ буде еквівалентна старій.
\end{exercise}

\begin{exercise}
    Нехай $X$ --- комплексний нормований простір, $\varphi \in X^*$. Довести $\norm{\varphi} = \underset{\norm{x}=1}{\sup}\mathfrak{Re}\varphi(x)$.
\end{exercise}

\begin{exercise}
    Нехай лінійний функціонал, визначений на нормованому просторі  $X$ --- необмежений.
    Довести, що в будь-якому околі нуля він приймає всі дійсні значення.
\end{exercise}

\begin{exercise}
    Нехай $X,Y$ --- нормовані простори, $Z$ --- замкнений підпростір в $X$. Покладемо 
    $M = \set{A \in L(X,Y) \mid \mathrm{Ker} A = Z}$. Чи буде $M$ замкненим підпростором в $L\left( X, Y\right)$?
\end{exercise}

\begin{exercise}
    Нехай $X,Y$ --- нормовані простори; $Z$ --- замкнений підпростір в $X$. Покладемо 
    $M = \set{A \in L(X,Y) \mid \mathrm{Ker} A \supset Z}$. Чи буде $M$ замкненим підпростором в $L\left( X, Y\right)$?
\end{exercise}

\begin{exercise}
    Нехай $X$ --- нормований простір; $A \in L(X)$. Покладемо $M = \{B \in L(X) \;|\; AB=0\}$;
    $N = \set{B \in L(X) \mid AB=BA}$. Довести, що $M$ та $N$ --- замкнені підпростори в $L(X)$.
\end{exercise}

\begin{theory}
    \begin{theorem*}[Ган, Банах]
        Для кожного ненульового вектора $x$ нормованого простору $X$ існує $\varphi \in X^*$,
        для якого $\norm{\varphi} = 1$ та $\varphi(x) = \norm{x}$. 
    \end{theorem*}
\end{theory}

\begin{exercise} 
    \begin{enumerate}[label=\ukr*)]
        \item Нехай $X$ --- нормований простір; $x,y \in X$. Довести $(x=y) \Leftrightarrow 
        \big(\forall\varphi \in X^*: \varphi(x) = \varphi(y)\big)$;
        \item Нехай $X$ --- сепарабельний нормований простір. Довести існування зліченної сім'ї
        функціоналів $\varphi_n \in X^*$, що розділяють точки $X$, тобто 
        $(x\neq y) \Leftrightarrow \big(\exists n \in \mathbb{N}: \varphi_n(x) \neq \varphi_n(y)\big)$.
    \end{enumerate}
\end{exercise}

\begin{theory}
    Кожен вектор $x$ нормованого простору $X$ можна розглядати як функцію $\tilde{x}$, яка 
    діє на $X^*$ за правилом $\tilde{x}(\varphi)=\varphi(x)$.
\end{theory}

\begin{exercise}\label{N:1_1_42}
    Доведіть:~
    \begin{enumerate}[label=\ukr*)]
        \item Для будь-якого $x \in X$ функція $\tilde{x}$ є лінійним обмеженим функціоналом на просторі $X^*$
        і його норма, як елемента $X^{**}$, співпадає з нормою $x \in X$;
        \item відображення $\alpha: X \ni x \mapsto \tilde{x} \in X^{**}$ є ізометричним вкладенням
        простору $X$ в $X^{**}$, тобто $\alpha$ --- лінійний оператор; $\mathrm{Ker}\alpha = \{0\}$; 
        $\norm{\alpha(x)} = \norm{\tilde{x}} = \norm{x}$.
    \end{enumerate}
\end{exercise}

\begin{theory}
    В разі, якщо $\mathrm{Im}\alpha = X^{**}$, тобто $\alpha$ --- ізоморфізм просторів $X$ та $X^{**}$,
    вихідний простір $X$ називається \underline{рефлексивним}.
\end{theory}

\begin{exercise}
    Доведіть, що простори  $\ell_p$, де $1<p<\infty$, є рефлексивними, а простори $\ell_1$ та $\ell_\infty$ --- ні.
\end{exercise}

\begin{exercise}\label{N:1_1_44}
    Чи буде простір $L(X)$, де $X=L_2[0;1]$, сепарабельним?
\end{exercise}

\begin{exercise}\label{N:1_1_45}
    Нехай $X$, $Y$ --- нормовані простори; $A \in L\left( X, Y\right)$; $B, C \in L\left( Y, X\right)$;
    $AB = I_Y$; $CA = I_X$ ($I_X$ --- тотожній оператор в $X$).
    Довести $B = C = A^{-1} \in L\left( Y, X\right)$ --- обернений оператор до $A$.
\end{exercise}

\begin{theory}
    \begin{theorem*}[Банах, Штейнгауз; принцип рівномірної обмеженості]
        Нехай $X, Y$ --- нормовані простори; $X$ --- повний; $A_n \in L\left( X, Y\right),\; n\in \mathbb{N}$;
        $\forall x \in X, \; \exists C(x) > 0$ таке, що $\forall n \in \mathbb{N}: \norm{A_n x} \leq C(x)$.
        Тоді $\exists \; C>0: \forall n \in \mathbb{N}: \norm{A_n} \leq C$.
    \end{theorem*}
\end{theory}

\begin{exercise}\label{N:1_1_46}
    Нехай $\alpha_n$ --- числова послідовність в $\mathbb{R}$, для якої 
    $\sum\limits^\infty_{k=1} \alpha^2_k =\infty$. Довести, що існує така послідовність 
    $\beta_n$ в $\mathbb{R}$, для якої $\sum\limits^\infty_{k=1} \beta^2_k < \infty$, 
    але $\sum\limits^\infty_{k=1} \alpha_k\beta_k$ --- розбіжний ряд.
\end{exercise}

\begin{exercise}
    Нехай $X$ --- нормований простір, $L$ --- замкнений підпростір $X$; $\mathrm{codim}L \geq 1$.
    Тоді $\exists \; \varphi \in X^*$; $\varphi \neq 0$, для деякого $L \subset \mathrm{Ker}\varphi$.
\end{exercise}

\begin{exercise}\label{N:1_1_48}
    Нехай $X, Y$ --- банахові простори над полем $K$ ($K$ = $\mathbb{R}$ або $\mathbb{C}$).
    $\varphi: X \times Y \rightarrow K$ --- білінійний функціонал, що має наступні властивості:
    \begin{enumerate}[label=\ukr*)]
        \item $\forall y \in Y$: $\varphi(\,\cdot\,, y)$ --- неперервний по $x$;
        \item $\forall x \in X$: $\varphi(x, \,\cdot\,)$ --- неперервний по $y$.
    \end{enumerate}
    Довести існування константи $k$ такої, що для кожного $(x,y) \in X \times Y$ має місце нерівність
    $|\varphi(x,y)| \leq k \norm{x}\cdot\norm{y}$.
\end{exercise}