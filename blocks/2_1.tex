% !TEX root = ../main.tex

\begin{theory}
    Нехай $H$ --- нескінченновимірний гільбертів простір (дійсний або комплексний); 
    $\{e_1, e_2, ...\}$ --- зліченна ортонормована система векторів в $H$ (тобто $\dotprod{e_i}{e_j} 
    = \delta_{ij}$). Для $x \in H$ покладемо $c_n = \dotprod{x}{e_n}$. Ряд $\sum\limits_{n=1}^\infty 
    c_n e_n$ називається \underline{рядом Фур'є} вектора $x$ за ортонормованою системою 
    $\{e_n\}$, $c_n$ --- \underline{коефіцієнтами Фур'є}.
\end{theory}

\begin{exercise}
    Довести \underline{нерівність Бесселя}: $\sum\limits_{n=1}^\infty |\dotprod{x}{e_n}|^2 \leq 
    \norm{x}^2$ ($\norm{x} = \sqrt{\dotprod{x}{x}}$).
\end{exercise}

\begin{theory}
    
    Ортонормована система $\{e_n\}$ в $H$ називається \underline{повною}, якщо її лінійна 
    оболонка (л.о.) щільна в $H$:
     $\forall x \in H, \;\forall \varepsilon > 0 \; \exists \; \{\alpha_1, ..., 
    \alpha_m\}$: $\norm{\sum\limits_{k=1}^m \alpha_k e_k - x} < \varepsilon$.

    \noindent Ортонормована система $\{e_n\}$ називається \underline{замкненою}, якщо для кожного 
    $x \in H$ має місце рівність: $\norm{x}^2 = \sum\limits_{n=1}^\infty |\dotprod{x}{e_n}|^2$ 
    (\underline{рівність Парсеваля}).

    \noindent Повна ортонормована система $\{e_n\}$ називається \ul{ортонормованим 
    базисом}.
\end{theory}

\begin{exercise}
    Нехай $\{e_n\}$ --- ортонормована система векторів в гільбертовому просторі $H$. 
    Довести еквівалентність трьох умов:
    \begin{enumerate}[label=\ukr*)]
        \item система векторів $\{e_n\}$ повна в $H$;
        \item система векторів $\{e_n\}$ замкнена в $H$;
        \item $\forall x \in H$ ряд $\sum\limits_{n=1}^\infty \dotprod{x}{e_n}e_n$ збігається до $x$
        (тобто $\norm{x - \sum\limits_{n=1}^m \dotprod{x}{e_n}e_n}
        \underset{m\rightarrow\infty}{\rightarrow}0$)
    \end{enumerate}
\end{exercise}

\begin{exercise}
    Довести еквівалентність двох умов:
    \begin{enumerate}[label=\ukr*)]
        \item $H$ --- сепарабельний гільбертів простір;
        \item в просторі $H$ існує ортонормований базис.
    \end{enumerate}
\end{exercise}
\begin{theory}
    Ортонормована система векторів $\{e_n\}$ в $H$ називається \underline{тотальною}, 
    якщо умова <<$\dotprod{x}{e_n} = 0$ $\forall n \in N$>> виконується лише для вектора $x=0$.
\end{theory}
\begin{exercise}[лема Ріса-Фішера]
    Нехай $\{e_n\}$ --- ортонормована система векторів в $H$, $\{c_n\}$ --- числова послідовність,
    для якої ряд $\sum\limits_{n=1}^\infty |c_n|^2$  збіжний. Тоді існує $x \in H$, 
    для якого при всіх $n \in \mathbb{N}$ $c_n = \dotprod{x}{e_n}$, і при цьому 
    $\norm{x}^2 = \sum\limits_{n=1}^\infty |c_n|^2$.
\end{exercise}

\begin{exercise}
    Нехай $\{e_n\}$ --- ортонормована система в $H$. Тоді дві умови еквівалентні:
    \begin{enumerate}[label=\ukr*)]
        \item $\{e_n\}$ --- ортонормований базис в $H$;
        \item $\{e_n\}$ --- тотальна система векторів в $H$.
    \end{enumerate}
\end{exercise}

\begin{theory}
    Гільбертові простори $H_1$ і $H_2$ називають \underline{ізоморфними}, якщо існує 
    лінійний оператор $A: H_1 \rightarrow H_2$, для якого виконуються рівності
    $\mathrm{Ker} A = \{0\}$, $\mathrm{Im} A = H_2$, $\dotprod{Ax}{Ay}_2 = \dotprod{x}{y}_1$ (тут $x$, $y$ --- довільні 
    вектори в $H_1$, $\dotprod{\cdot}{\cdot}_k$ --- скалярний добуток в $H_k$). 
    Такий оператор $A$ називається \underline{ізоморфізмом}. Позначення: $H_1 \cong H_2 $.
\end{theory}

\begin{exercise}
    Нехай $A: H_1 \rightarrow H_2$ --- ізоморфізм. Доведіть, що в означенні ізоморфізму:
    \begin{enumerate}[label=\ukr*)]
        \item умова <<$\mathrm{Ker} A = \{0\}$>> є зайвою;
        \item умова <<$\dotprod{Ax}{Ay} = \dotprod{x}{y}$ для всіх $x, y \in H_1$>> може бути 
        замінена на таку: <<$\norm{Ax} = \norm{x}$ для всіх $x \in H_1$>>.
    \end{enumerate}
\end{exercise}