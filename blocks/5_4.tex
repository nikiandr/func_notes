% !TEX root = ../main.tex

\begin{exercise}
    Нехай $\{e_n\}$ та $\{f_n\}$ --- два ортонормованих базиси в гільбертовому просторі $H$, 
    $A \in L(H)$. Довести рівність: 
    $\sum\limits_{n = 1}^\infty \norm{A e_n}^2 = \sum\limits_{n = 1}^\infty \norm{A f_n}^2$ (точніше: якщо
    один ряд збігається, то збігається інший та вони мають однакові суми).
\end{exercise}

\begin{theory}
    $\norm{A}_2 = \left(\sum\limits_{n = 1}^\infty \norm{A e_n}^2\right)^{\frac{1}{2}}$ називається
    \underline{абсолютною нормою} оператора $A$.
\end{theory}

\begin{exercise}
    Нехай $A \in S_2 (H)$; $B \in L(H)$. Довести наступні твердження:
    \begin{enumerate}
        \item $A^* \in S_2 (H)$; $\norm{A^*}_2 = \norm{A}_2$;
        \item $\norm{A} \leq \norm{A}_2$;
        \item $AB \in S_2 (H)$; $BA \in S_2 (H)$; $\norm{AB}_2 \leq \norm{A}_2 \cdot \norm{B}$; 
        $\norm{BA}_2 \leq \norm{A}_2 \cdot \norm{B}$;
        \item $A \in K(H)$.
    \end{enumerate}
\end{exercise}

\begin{exercise}
    Довести наступні твердження:
    \begin{enumerate}
        \item $S_2(H)$ є лінійним простором за стандартними операціями над лінійними операторами; 
        \item $S_2(H)$ є гільбертовим простором із скалярним добутком:
         
        $(A, B) = \sum\limits_{n = 1}^\infty \left(A e_n, B e_n\right)$, причому сума цього ряду від вибору
        ортонормованого базису не залежить. Зокрема, $S_2(H)$ --- банахів за нормою $\norm{\cdot}_2$;
        \item $S_2(H)$ є незамкненою множиною в $L(H)$ (за нормою в $L(H)$).
    \end{enumerate}
\end{exercise}

\begin{exercise}\label{N:1_5_42}
    Нехай $\{a_n\}$ --- числова послідовність; оператор $A:\ell_2 \rightarrow \ell_2$ визначений формулою:
    $A \vec{x} = \left(a_1 x_1, a_2 x_2, \dots \right)$. З'ясувати:
    \begin{enumerate}
        \item за якою умовою на послідовність $\{a_n\}$ оператор $A$ буде оператором Гільберта-Шмідта;
        \item за якою умовою на послідовність $\{a_n\}$ оператор $A$ буде компактним, але не оператором Гільберта-Шмідта?
    \end{enumerate}
\end{exercise}

\begin{theory}
    Лінійний оператор $A$ на сепарабельному гільбертовому просторі називається \underline{ядерним}, якщо він
    може бути представлений як добуток: $A = BC$, де $B, C \in S_2(H)$. Множину всіх ядерних операторів позначимо
    через $S_1(H)$.
\end{theory}

\begin{exercise}
    Довести: $S_1(H) \subset S_2(H)$.
\end{exercise}

\begin{exercise}\label{N:1_5_44}
    За якої умови на $\{a_n\}$ оператор в $\ell_2$ із задачі \ref{N:1_5_42} буде:
    \begin{enumerate}
        \item ядерним;
        \item оператором Гільберта-Шмідта, але не ядерним.
    \end{enumerate}
\end{exercise}

\begin{exercise}
    Довести:
    \begin{enumerate}
        \item $(A \in S_1(H)) \Rightarrow (A^* \in S_1(H))$;
        \item $(A \in S_1(H)) \Rightarrow$ (число $Tr A = \sum\limits_{n = 1}^\infty (A e_n, e_n)$ є скінченним
        і не залежить від вибору ортонормованого базису $\{e_n\}$);
        \item $(A \in S_1(H); B \in L(H)) \Rightarrow (AB \in S_1(H); BA \in S_1(H))$; 
        \item[г)*] $(A \in S_1(H)) \Rightarrow$ (ряд $\sum\limits_{n = 1}^\infty |(A e_n, e_n)|$ збіжний для будь-якого ортонормованого базису).
    \end{enumerate}
\end{exercise}

\begin{exercise}
    Нехай $A, B \in S_2(H)$. Довести: $Tr (AB) = Tr (BA)$.
\end{exercise}
