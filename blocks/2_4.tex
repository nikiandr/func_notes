% !TEX root = ../main.tex

\begin{exercise}
    Довести, що множина $M = \{ \vec{x} \in l_2 | \sum\limits_{n = 1}^{\infty}x_n = 0\}$
    щільна в $l_2$.
\end{exercise}

\begin{exercise}
    Нехай $M$ --- замкнена опукла множина в гільбертовому просторі $H$. Довести, що в $M$
    існує і притому єдиний вектор із найменшою нормою.
\end{exercise}

\begin{exercise}
    На просторі $C[0;1]$ розглянемо функціонал $\varphi$, який визначимо формулою: 
    $\varphi(x) = \int\limits_{0}^{\frac{1}{2}}x(t)dt - \int\limits_{\frac{1}{2}}^{1}x(t)dt$. Тоді
    $M = \{x \in C[0; 1] | \varphi(x) = 1\}$ опукла замкнена множина, яка не містить найближчого до 
    нуля елемента. Довести. Порівняйте із задачею 1.2.30.
\end{exercise}

\begin{exercise}
    Нехай послідовність неперервно диференційованих функцій утворює ортонормовану систему
    в $L_2[0; 2\pi]$. Доведіть, що похідні цих функцій не можуть бути обмеженими у сукупності.
\end{exercise}

\begin{exercise}
    Побудувати конкретний ізоморфізм просторів $L_2[0;1]$ та $l_2$.
\end{exercise}

\begin{theory}
    Нехай $L$ --- замкнений підпростір гільбертова простору $H$; $x \in H$. Вектор
    $y \in L$ називається \underline{(ортогональною) проекцією} вектора $x$ на $L$, якщо
    $x - y \bot L$. Вектор $x - y$ називається \underline{ортогональною складовою} при
    проектуванні $x$ на $L$. Позначення: $y = pr_L x$; $x - y = ort_L x$.
\end{theory}

\begin{exercise}
    Нехай $L$ --- замкнений підпростір $H$; $x \in H$; $y = pr_L x$; $z \in L, z \neq y$.
    Доведіть: $\norm{x - y} < \norm{x - z}$ (екстремальна властивість ортогональної проекції).
\end{exercise}

\begin{exercise}
    Нехай $L$ --- замкнений підпростір в $H$; $x \in H$. Довести існування та єдиність $pr_L x$:
    \begin{enumerate}[\label = \ukr*)]
        \item в разі, якщо $L$ --- сепарабельний підпростір;
        \item для загального випадку.
    \end{enumerate}
\end{exercise}

\begin{exercise}
    Нехай $L$ --- замкнений підпростір гільбертова простору $H$; $M$ --- замкнений підпростір
    в $L$; $x \in H$. ДоведітьЖ
    \begin{enumerate}[\label = \ukr*)]
        \item $L$ --- гільбертів простір, що успадковує скалярний добуток простору $H$;
        \item $M$ є замкненим підпростором в $H$;
        \item $pr_M (pr_L x) = pr_M x$
    \end{enumerate}
\end{exercise}
