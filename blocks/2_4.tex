% !TEX root = ../main.tex

\begin{exercise}
    Довести, що множина $M = \{ \vec{x} \in \ell_2 | \sum\limits_{n = 1}^{\infty}x_n = 0\}$
    щільна в $\ell_2$.
\end{exercise}

\begin{exercise}\label{N:1_2_30}
    Нехай $M$ --- замкнена опукла множина в гільбертовому просторі $H$. Довести, що в $M$
    існує і притому єдиний вектор із найменшою нормою.
\end{exercise}

\begin{exercise}
    На просторі $C[0;1]$ розглянемо функціонал $\varphi$, який визначимо формулою: 
    $\varphi(x) = \int\limits_{0}^{\frac{1}{2}}x(t)dt - \int\limits_{\frac{1}{2}}^{1}x(t)dt$. Тоді
    $M = \set{x \in C[0; 1] \mid \varphi(x) = 1}$ опукла замкнена множина, яка не містить найближчого до 
    нуля елемента. Довести. Порівняйте із задачею \ref{N:1_2_30}.
\end{exercise}

\begin{exercise}
    Нехай послідовність неперервно диференційовних функцій утворює ортонормовану систему
    в $L_2[0; 2\pi]$. Доведіть, що похідні цих функцій не можуть бути обмеженими у сукупності.
\end{exercise}

\begin{exercise}
    Побудувати конкретний ізоморфізм просторів $L_2[0;1]$ та $\ell_2$.
\end{exercise}

\begin{theory}
    Нехай $L$ --- замкнений підпростір гільбертового простору $H$; $x \in H$. Вектор
    $y \in L$ називається \uline{(ортогональною) проекцією} вектора $x$ на $L$, якщо
    $x - y \bot L$. Вектор $x - y$ називається \uline{ортогональною складовою} при
    проектуванні $x$ на $L$. Позначення: $y = pr_L x$; $x - y = ort_L x$.
\end{theory}

\begin{exercise}
    Нехай $L$ --- замкнений підпростір $H$, $x \in H$, $y = pr_L x$, $z \in L, z \neq y$.
    Доведіть: $\norm{x - y} < \norm{x - z}$ (екстремальна властивість ортогональної проекції).
\end{exercise}

\begin{exercise}
    Нехай $L$ --- замкнений підпростір $H$; $x \in H$. Довести існування та єдиність ${pr}_L x$:
    \begin{enumerate}[label=\ukr*)]
        \item в разі, якщо $L$ --- сепарабельний підпростір;
        \item для загального випадку.
    \end{enumerate}
\end{exercise}

\begin{exercise}
    Нехай $L$ --- замкнений підпростір гільбертового простору $H$, $M$ --- замкнений підпростір $L$, $x \in H$.
    Доведіть:
    \begin{enumerate}[label=\ukr*)]
        \item $L$ --- гільбертів простір, що успадковує скалярний добуток простору $H$;
        \item $M$ є замкнений підпростором $H$;
        \item ${pr}_M({pr}_L x) = {pr}_M x$.
    \end{enumerate}
\end{exercise}