% !TEX root = ../main.tex

\begin{exercise}
    Нехай $L$ --- замкнений підпростір в $H$, $L \neq \{0\}$. 
    Відображення $P: H \rightarrow H$ визначене формулою 
    $P: x \rightarrow {pr}_L x$. Довести:
    \begin{enumerate}[label=\ukr*)]
        \item $P$ --- лінійний обмежений оператор в $H$;
        \item $\norm{P} = 1$, $\mathrm{Im}P = L$, $\mathrm{Ker} P = L^\bot$;
        \item $P^2 = P$, $L = \set{x \mid x=Px}$, $\dotprod{Px}{y}=\dotprod{x}{Py}$,
        $\dotprod{Px}{x} = \norm{Px}^2$ ($\forall x,y \in H$);
        \item оператор $Q: x \rightarrow {pr}_{L^\bot}x$ пов'язаний з $P$ 
        співвідношенням: $Q = I-P$ (тут $I$ --- тотожній оператор в $H$);
        \item $PQ = QP = 0$, $\mathrm{KerP} = \mathrm{Im}Q$, $\mathrm{Im}P = \mathrm{Ker}Q$;
        \item якщо $M$ --- замкнений підпростір в $L$ і для $x \in H$:
        $P_1 x = {pr}_M x$, то $P_1 P = P P_1 = P_1$,
        $\dotprod{P_1x}{x} = \dotprod{Px}{x}$ ($\forall x \in H$).
    \end{enumerate}
\end{exercise}

\begin{theory}
    Лінійний оператор $P$ із задачі 1.3.1 називається \uline{ортопроектором}.
\end{theory}

\begin{exercise}
    Нехай $X$ --- нормований простір, $P \in L(X)$.
    $P$ називається \uline{проектором}, якщо $P^2=P$. Доведіть наступні 
    властивості (обмеженого) проектора:
    \begin{enumerate}[label=\ukr*)]
        \item $M = KerP$ --- замкнений підпростір;
        \item оператор $Q = I - P$ також є (обмеженим) проектором;
        \item $L=ImP=KerQ=\set{x \mid Px=x}$ --- замкнений підпростір в $X$;
        \item $X = L \dotplus M$ (тобто $\forall x \in X$ має, і притому 
        єдиний розклад: $x=x_1+x_2$, де $x_1 \in L$, $x_2 \in M$).
    \end{enumerate}
\end{exercise}

\begin{theory}
    Оператор $P$ називається \uline{проектором на $L$ паралельно $M$}.
\end{theory}

\begin{exercise}[теорема Фреше-Ріса]
    Нехай $H$ --- гільбертів простір над полем $K$ ($K=\mathbb{R}$ 
    або $\mathbb{C}$), $y \in H$, $\varphi_y : H \rightarrow K$ визначено 
    формулою $\phi_y(x) = \dotprod{x}{y}$. Довести:
    \begin{enumerate}[label=\ukr*)]
        \item $\varphi_y \in H^*$, $\norm{\phi_y} = \norm{y}$;
        \item $\forall \varphi \in H^*$ $\exists! \; y\in H$, для якого 
        $\varphi = \varphi_y$ (при цьому $y\in(Ker\varphi)^\bot$).
    \end{enumerate}
\end{exercise}

\begin{exercise}
    Застосувати теорему Фреше-Ріса для розв'язання задачі \ref{N:1_1_5}
    (в, г, ґ, е, ж).
\end{exercise}

\begin{exercise}
    Нехай $H$ --- гільбертів простір; $A$ --- лінійний оператор. 
    Довести, що $A \in L(H)$ тоді й тільки тоді, коли існує 
    $C > 0$ таке, що для кожного $x, y \in H$ має місце нерівність:
    $|\dotprod{Ax}{y}| \leq C\norm{x}\norm{y}$. При цьому $\norm{A} = 
    \underset{x,y \neq 0}{\sup} \frac{|\dotprod{Ax}{y}|}{\norm{x}\norm{y}}
    =\underset{\norm{x},\norm{y} \leq 1}{\sup} |\dotprod{Ax}{y}| = 
    \underset{\norm{x} = \norm{y} = 1}{\sup} |\dotprod{Ax}{y}|$.
\end{exercise}

\begin{theory}
    Нехай $A \in L(H)$. \uline{Спряженим оператором} до $A$ називається 
    такий оператор $B \in L(H)$, для якого при всіх $x, y \in H$ 
    виконується рівність $\dotprod{Ax}{y} = \dotprod{x}{By}$. 
    Позначення: $B = A^*$. 
\end{theory}

\begin{exercise}
    Довести коректність означення $A^*$ (тобто для кожного $A\in L(H)$ 
    $\exists! \; A^* \in L(H)$).
\end{exercise}

\begin{exercise}
    Нехай $A, B \in L(H)$, $\alpha \in K$. Доведіть наступні властивості:
    \begin{enumerate}[label=\ukr*)]
        \item $(A+B)^* = A^* + B^*$;
        \item $0^*=0$; $I^* = I$;
        \item $(\alpha A)^* = \overline{\alpha}A^*$
        \item $(A^*)^* = A^{**} = A$;
        \item якщо існує $A^{-1} \in L(H)$, то існує $(A^*)^{-1} \in 
        L(H)$ і при цьому $(A^*)^{-1} = (A^{-1})^*$;
        \item $\norm{A^*} = \norm{A}$.
    \end{enumerate}
\end{exercise}

\begin{theory}
    Оператор $A \in L(H)$ називається \uline{самоспряженим}, якщо 
    $A = A^*$.
\end{theory}

\begin{exercise}
    Нехай $A, B \in L(H)$; $A, B$ --- самоспряжені. Довести:
    \begin{enumerate}[label=\ukr*)]
        \item $A+B$ --- самоспряжений;
        \item $\alpha \in \mathbb{R}, \alpha A $ 
        --- самоспряжений;
        \item $0, I$ --- самоспряжені;
        \item $AB$ -- самоспряжений $\Rightarrow AB = BA$;
        \item якщо існує $A^{-1} \in L(H)$, то $A^{-1}$ --- 
        самоспряжений.
    \end{enumerate}
\end{exercise}

\begin{exercise}
    Знайти спряжений оператор до $A: H\rightarrow H$ в наступних 
    прикладах:
    \begin{enumerate}[label=\ukr*)]
        \item $H = \ell_2$; $A\vec{x} = (\alpha_1x_1, \alpha_2x_2, ...)$
        ($\alpha_k \in \mathbb{C}$, $\underset{k \in \mathbb{N}}{\sup}|\alpha_k| < \infty$);
        \item $H = \ell_2$; $A\vec{x} = (0, x_1, x_1, ...)$;
        \item $H = \ell_2$; $A\vec{x} = (x_5, x_6, x_7, ...)$;
        \item $H = \ell_2$; $A\vec{x} = (x_1, x_2, x_3,0,0,...)$;
        \item $H = \ell_2$; $A\vec{x} = (\underbrace{0,...,0}_m, 
        x_1, 0, 0, ...)$;
        \item $H = \ell_2$; $A\vec{x} = (2x_1 + 5x_2, x_2, x_3, ...)$;
        \item $H = \ell_2$; $A\vec{x} = (0, 0, x_1 + x_2, x_1 - x_2, 0,  
        0, ...)$;
        \item $H = L_2[0,1]$; $(Ax)(t) = \int\limits_0^t x(\tau) d\tau$;
        \item $H = L_2[0,1]$; $(Ax)(t) = x(t^\alpha)$, $\alpha \in (0,1)$;
        \item $H = L_2[0,1]$; $(Ax)(t) = \int\limits_0^1 e^{t+\tau}x(\tau) 
        d\tau$;
        \item $H = L_2[0,1]$; $(Ax)(t) = \int\limits_0^1 
        sin(t+2\tau) x(\tau) d\tau$;
    \end{enumerate}
\end{exercise}