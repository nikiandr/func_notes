% !TEX root = ../main.tex

\begin{exercise}
    Нехай $\{x_n\}$ --- ортогональна система векторів в гільбертовому просторі $H$.
    Довести еквівалентність наступних тверджень:
    \begin{enumerate}
        \item ряд $\sum\limits^\infty_{n=1} x_n$ збігається сильно;
        \item ряд $\sum\limits^\infty_{n=1} x_n$ збігається слабо (тобто слабо збігається
        послідовність його часткових сум);
        \item числовий ряд $\sum\limits^\infty_{n=1} \norm{x_n}^2$ --- збіжний.
    \end{enumerate}
\end{exercise}

\begin{theory}
    Послідовність $\{x_n\}$ в нормованому просторі $X$ називається \ul{слабо фундаментальною},
    якщо для кожного $\varphi \in X^*$ числова послідовність $\{\varphi(x_n)\}$ є фундаментальною.
    
    Послідовність операторів $A_n \in L(X,Y)$ називається \ul{сильно фундаментальною},
    якщо для кожного $x \in X$ послідовність векторів $\{A_n x\} \subset Y$ є фундаментальною
    за нормою.
\end{theory}

\begin{exercise}
    \begin{enumerate}
        \item Довести, що слабо збіжна послідовність векторів в банаховому просторі 
        є слабо фундаментальною;
        \item Довести, що послідовність векторів $\vec{x}_n = 
        (\underbrace{1,\dots,1}_{n},0,0,\dots)$ є слабо фундаментальною, але не слабо 
        збіжною в банаховому просторі $c_0$;
        \item Довести, що слабо фундаментальна послідовність в гільбертовому просторі 
        є слабо збіжною.
    \end{enumerate}
\end{exercise}

\begin{exercise}
    Довести, що послідовність $\vec{x}_n$ векторів простору $\ell_2$ слабо збігається тоді
    й тільки тоді, коли виконуються дві умови:
    \begin{enumerate}
        \item послідовність $\vec{x}_n = (x_n^1,x_n^2,\dots)$ рівномірно обмежена, тобто
        $\exists C > 0 \; \forall n$: $\norm{\vec{x}_n} \leq C$;
        \item $\forall k \in \mathbb{N}$ числова послідовність $\{x_n^k\}$ збігається при
        $n \to \infty$ (<<\ul{покоординатна збіжність}>>).
    \end{enumerate}
\end{exercise}

\begin{exercise}
    Нехай $X$, $Y$ --- банахові простори, послідовність операторів $A_n \in L(X,Y)$ сильно
    фундаментальна. Довести, що існує оператор $A \in L(X,Y)$, до якого послідовність $A_n$
    збігається сильно (<<\ul{повнота $L(X,Y)$ відносно сильної операторної збіжності}>>).
\end{exercise}

\begin{exercise}
    Нехай $X$, $Y$ --- банахові простори; $A, A_n \in L(X,Y)$; $x, x_n \in X$;
    $A_n \overset{s}{\to} A$; $x_n \to x$ (за нормою). Доведіть $A_n x_n \to Ax$ (за нормою).
\end{exercise}

\begin{exercise}
    Нехай $X$, $Y$, $Z$ --- нормовані простори; $A, A_n \in L(X,Y)$; $B, B_n \in L(Y,Z)$;
    $A_n \rightrightarrows A$; $B_n \rightrightarrows B$.
    Довести $B_n A_n \rightrightarrows BA$.
\end{exercise}

\begin{exercise}
    Нехай $X$, $Y$, $Z$ --- банахові простори; $A, A_n \in L(X,Y)$; $B, B_n \in L(Y,Z)$;
    $A_n \overset{s}{\to} A$; $B_n \overset{s}{\to} B$.
    Довести $B_n A_n \overset{s}{\to} BA$.
\end{exercise}

\begin{exercise}\label{N:1_4_26}
    Нехай $H$ --- сепарабельний гільбертів простір, $Z$ --- підмножина в $H$.
    Довести еквівалентність двох умов:
    \begin{enumerate}
        \item $Z$ --- обмежена множина в $H$;
        \item кожна послідовність точок $x_n \in Z$ містить слабо збіжну (в $H$) 
        підпослідовність.
    \end{enumerate}
\end{exercise}

\begin{theory}
    Умова б) називається умовою <<\uline{слабкої компактності}>> множини $Z$,
    а твердження задачі \ref{N:1_4_26} є спрощеним варіантом теореми Банаха-Алаоглу.
\end{theory}

\begin{exercise}
    $A_n$ --- самоспряжені обмежені оператори в гільбертовому просторі $H$,
    $A_n \overset{s}{\to} A$. Довести $A$ --- самоспряжений оператор.
    Якщо, на додачу, $A_n \geq 0$ при всіх $n$, то $A \geq 0$.
\end{exercise}

\begin{exercise}
    $A_n$ --- самоспряжені обмежені оператори в гільбертовому просторі $H$,
    $A_1 \leq A_2 \leq \dots$; $\exists C > 0 \; \forall n \in \mathbb{N}$: $\norm{A_n}\leq C$.
    Тоді існує самоспряжений оператор $A$ такий, що $A_n \overset{s}{\to} A$
    (аналог теореми Вейєрштрасса).
\end{exercise}

\begin{exercise}
    $A_n \in L(L_2[0;1])$. Оператори $A_n$ визначено формулою $(A_n x)(t) = a_n(t) x(t)$.
    Дослідити послідовність $\{A_n\}$ на сильну та рівномірну збіжність якщо:
    \begin{enumerate}
        \item $a_n(t) = t^n$;
        \item $a_n(t) = t^n (1-t)$.
    \end{enumerate}
\end{exercise}

\begin{exercise*}
    Довести, що в просторі $\ell_1$ сильна збіжність послідовності векторів співпадає зі слабкою.
\end{exercise*}