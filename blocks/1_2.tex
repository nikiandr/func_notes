% !TEX root = ../main.tex
\begin{exercise}
    Нехай $\left\{c_n\right\}$ --- числова послідовність. Довести, що оператор
    $A : \ell_2 \ni (x_1, x_2, ...) = \vec{x} \mapsto \vec{y} = (c_1x_1, c_2x_2, ...) \in \ell_2$
    є обмеженим (та визначеним на всьому $\ell_2$) тоді й тільки тоді, коли $c = \underset{n\in\mathbb{N}}{\sup} |c_n| < +\infty$, 
    і при цьому $\norm{A} = c$.
\end{exercise}

\begin{exercise}
    Нехай $K \in C\left( [a;b] \times [a;b]\right)$ і оператор $A: C\left( [a;b]\right) \rightarrow C\left( [a;b]\right)$
    визначено формулою $(Ax)(t) = \int\limits_a^b K(t, s) x(s) ds$. Довести лінійність і обмеженість оператора $A$.
\end{exercise}

\begin{exercise}
    Довести, що ядро лінійного неперервного оператора замкнене. Чи завжди замкнена його область значень?
\end{exercise}

\begin{exercise}
    Нехай $A: X \rightarrow Y$ --- обмежений лінійний оператор, $Z$ --- передкомпакт в $X$.
    Довести $A(Z)$ --- передкомпакт в $Y$. Чи завжди образ компакта під дією обмеженого оператора буде компактом?
\end{exercise}

\begin{exercise}\label{N:1_1_17}
    Нехай $\varphi$ --- лінійний функціонал на нормованому просторі $X$. 
    Довести, що $\varphi$ --- обмежений тоді й тільки тоді, коли його ядро замкнене. 
\end{exercise}

\begin{exercise*}
    Нехай $A$ --- лінійний оператор з $X$ в $Y$, $\mathrm{dim} Y < \infty$.
    Доведіть, що $A$ --- обмежений тоді й тільки тоді, коли його ядро замкнене.
\end{exercise*}

\begin{exercise}\label{N:1_1_19}
    Нехай $A$ --- лінійний оператор з $X$ в $Y$. Чи завжди з умови замкненості ядра $A$ випливає його обмеженість?
\end{exercise}

\begin{exercise}\label{N:1_1_20}
    Нехай $\varphi$ --- лінійний функціонал на нормованому просторі. Довести $\varphi$ --- неперервний тоді й тільки тоді, коли
    для будь-якої послідовності $x_n \rightarrow 0$ числова послідовність $\left\{\varphi(x_n)\right\}$ --- обмежена.
\end{exercise}

\begin{exercise}\label{N:1_1_21}
    Нехай $\varphi$ --- ненульовий лінійний функціонал на нормованому просторі $X$. Довести, що для будь-якого $x \in X$ виконується рівність
    $\rho \left( x, \mathrm{Ker}\varphi\right) = \inf \left\{\norm{x - y} \; | \; y \in \mathrm{Ker} \varphi\right\} = \frac{|\varphi(x)|}{\norm{\varphi}}$.
\end{exercise}

\begin{exercise}\label{N:1_1_22}
    Нехай $\varphi$ --- ненульовий лінійний функціонал на нормованому просторі $X$, $a \in K$ ($\mathbb{R}$ або $\mathbb{C}$).
    Позначимо $L = \left\{x\in X \; | \; \varphi(x) = a\right\}$. Довести $\rho(0, L) = \frac{|a|}{\norm{\varphi}}$.
\end{exercise}

\begin{theory}
    Обмежені лінійні оператори з нормованого простору $X$ в нормований простір $Y$ утворюють лінійний простір за поточковими операціями:
    $(A+B)x = Ax + Bx$, $(\lambda\cdot A)x = \lambda \cdot Ax$. Цей простір є нормований із стандартною операторною нормою.
    Найчастіше він позначається так: $L\left( X, Y\right)$ або $\left\{X \rightarrow Y\right\}$. В разі, якщо $X = Y$, застосовується
    скорочене позначення $L(X)$. Якщо $Y = K$ (основне поле), то $L\left( X, K\right)$ позначається $X^*$ і називається 
    \uline{простором, спряженим} до $X$.
\end{theory}

\begin{exercise}
    Довести, що в разі, якщо простір $Y$ є повним, простір $L\left( X, Y\right)$ також є повним.
    Зокрема, для кожного нормованого простору $X$ спряжений простір $X^*$ є банаховим.
\end{exercise}

\begin{exercise}
    Нехай $X$, $Y$, $Z$ --- нормовані простори. Довести:
    \begin{enumerate}[label=\ukr*)]
        \item $\left( A \in L\left( X, Y\right); B \in L\left( Y, Z\right)\right) \Rightarrow \left( B \circ A \in L\left( X, Z\right); \norm{B\circ A} \leq \norm{B} \cdot \norm{A}\right)$;
        \item $\left( A \in L(X), n\in\mathbb{N}\right) \Rightarrow \left( A^n \in L(X) ; \norm{A^n} \leq \norm{A}^n\right)$;
        \item $\left( X \text{ --- банахів}; A_n \in L(X), n \in \mathbb{N} ; \text{ ряд } \sum\limits_{n=1}^{\infty} \norm{A_n} \text{ --- збіжний}\right) \Rightarrow$ 

        $\Rightarrow \left( \exists A \in L(X): \norm{A - \sum\limits_{k=1}^n A_k} \rightarrow 0, n\rightarrow \infty \right)$.
    \end{enumerate}
\end{exercise}

\begin{exercise}
    Нехай $X$ --- банахів простір, $A \in L(X)$. $e^A = \exp{A} := I + \sum\limits_{n=1}^{\infty} \frac{1}{n!}A^n$, $I$ --- тотожний оператор.
    Довести: існування, $\exp{A} \in L(X)$, $\norm{e^A} \leq e^{\norm{A}}$.
\end{exercise}

\begin{exercise}\label{N:1_1_26}
    Нехай $X$ --- банахів простір, $A\in L(X)$. Довести, що ряд $\sum\limits_{k=0}^{\infty} A^k$ ($A^0 := I$)
    збігається тоді й тільки тоді, коли існує натуральне число $n$, для якого виконується нерівність $\norm{A^n} < 1$.
\end{exercise}

\begin{exercise}\label{N:1_1_27}
    Знайти образ та ядро оператора $A \in L\left( X, Y\right)$, визначеного формулою $Ax = \varphi(x) y$,
    де $\varphi \in X^*$ --- фіксований обмежений лінійний функціонал на $X$, $y$ --- фіксований вектор з $Y$.
    Знайти норму $\norm{A}$.
\end{exercise}

\begin{theory}
    \underline{Рангом} оператора $A$ називається число $\mathrm{rank}A := \mathrm{dim} \mathrm{Im}A$.
    В разі, якщо $\mathrm{rank} A < \infty$, оператор $A$ називається
    \uline{оператором скінченного рангу} або \uline{скінченновимірним оператором}.
\end{theory}

\begin{exercise}\label{N:1_1_28}
    Нехай $A \in L\left( X, Y\right)$. Довести: $A$ --- оператор скінченного рангу тоді й тільки тоді, коли він допускає представлення
    $Ax = \sum\limits_{k=1}^n \varphi_k(x) y_k$, де $\varphi_k \in X^*$, $y_k \in Y$.
\end{exercise}

\begin{theory}
    Оператори $A, B: \ell_2 \rightarrow \ell_2$, $A : \vec{x} \mapsto (0, x_1, x_2, ...)$, $B : \vec{x} \mapsto (x_2, x_3, x_4, ...)$
    називаються відповідно операторами \underline{правого} та \underline{лівого зсуву}.
\end{theory}

\begin{exercise}
    Знайти норми $\norm{A}$ та $\norm{B}$ операторів зсуву.
\end{exercise}

\begin{theory}
    Нормовані простори $(X, \norm{\cdot}_X)$, $(Y, \norm{\cdot}_Y)$ називаються 
    \underline{ізоморфними}, якщо існує лінійний оператор $A: X \rightarrow Y$, для якого $\mathrm{Ker}A = \{0\}$,
    $\mathrm{Im}A = Y$ і для кожного $x \in X$ має місце $\norm{Ax}_Y = \norm{x}_X$. Такий оператор
    називається \underline{ізоморфізмом}. Позначення: $X \cong Y$.
\end{theory}

\begin{exercise}\label{N:1_1_30}
    Довести наступні ізоморфізми:
    \begin{enumerate}[label=\ukr*)]
        \item $\ell_1^* \cong \ell_\infty$;
        \item $c_0^* \cong \ell_1$;
        \item $\left( 1<p<\infty\right) \Rightarrow \left( \ell_p^* \cong \ell_q, \text{ де } \frac{1}{p} + \frac{1}{q} = 1\right)$.
    \end{enumerate}
    Тут $c_0 = \left\{ \vec{x} = (x_1, x_2, ...) \; | \; \underset{n\rightarrow\infty}{\lim} x_n = 0\right\}$ 
    з нормою $\norm{\vec{x}} = \sup\left\{|x_n| \; | \; n \in \mathbb{N}\right\}$.
\end{exercise}

\begin{exercise*}\label{N:1_1_31}
    Нехай $X$ --- нормований простір, $X^*$ --- сепарабельний простір. Довести сепарабельність простору $X$.
    Чи можна стверджувати, що спряжений простір до сепарабельного простору $X$ завжди є сепарабельним?
\end{exercise*}