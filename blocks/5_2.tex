% !TEX root = ../main.tex

\begin{exercise}\label{N:1_5_14}
    Довести, що оператор $A: C[a, b] \rightarrow C[a, b]$, визначений формулою : $(Ax)(t) = 
    f(t)x(t)$ ($f \in C[a, b]$; $\exists t \in [a, b] : f(t) \neq 0$) не є компактним.
\end{exercise}

\begin{exercise}\label{N:1_5_15}
    З'ясувати, які з наведених нижче операторів $A : C[0, 1] \rightarrow C[0, 1]$ 
    є компактними:
    \begin{enumerate}
        \item $(Ax)(t) = x(0) + tx(\frac{1}{2}) + t^2 x(1)$;
        \item $(Ax)(t) = \int\limits_0^t sx(s) ds$
        \item $(Ax)(t) = x(t^2)$
        \item $(Ax)(t) = \int\limits_0^1 x(ts) ds$
    \end{enumerate}
\end{exercise}

\begin{exercise}
    З'ясувати, які з наведених нижче операторів $A : \ell_2 \rightarrow \ell_2$ 
    є компактними:
    \begin{enumerate}
        \item $A\vec{x} = (0, x_1, x_2, x_3, ...)$;
        \item $A\vec{x} = (x_2, x_3, ..., x_{10}, 0, 0, ...)$;
        \item $A\vec{x} = (x_{100}, \frac{1}{2}x_{101} ,\frac{1}{3}x_{102}, ...)$.
    \end{enumerate}
\end{exercise}

\begin{exercise}\label{N:1_5_17}
    З'ясувати, які з наведених нижче операторів $A : L_2[0, 1] \rightarrow L_2[0, 1]$ 
    є компактними:
    \begin{enumerate}
        \item $(Ax)(t) = \int\limits_0^1 tsx(s) ds$;
        \item $(Ax)(t) = \int\limits_0^t x(s) ds$;
        \item $(Ax)(t) = t x(t)$;
        \item $(Ax)(t) = x(\sqrt{t})$;
        \item $(Ax)(t) = \int\limits_0^t tsx(s) ds$;
    \end{enumerate}
\end{exercise}

\begin{exercise}
    Нехай $F \in C([a, b] \times [a, b])$; оператори $A$ та $B$ задаються в $X$ формулами:
    $(Ax)(t)=\int\limits_a^b F(t, s)x(s)ds$; $(Bx)(t) = \int\limits_a^t F(t, s)x(s)ds$.
    Довести їх компактність у випадках:
    \begin{enumerate}
        \item $X = C[a, b]$;
        \item $X = L_2[a, b]$.
    \end{enumerate}
\end{exercise}

\begin{exercise}
    Нехай $H$ --- сепарабельний гільбертів простір; $A \in K(H)$. Довести: $A$ 
    досягає своєї норми на замкненій одиничній сфері (тобто $\norm{A} = 
    \underset{\norm{x} = 1}{\max}\norm{Ax}$).
\end{exercise}

\begin{exercise}\label{N:1_5_20}
    Нехай $Z$ --- замкнений підпростір $C[a, b]$, який є підмножиною в $C^1[a, b]$. 
    Довести: $\mathrm{dim} Z < \infty$.
\end{exercise}

\begin{exercise}
    Нехай $X$ --- сепарабельний гільбертів простір; $A \in K(X)$. Довести: образ замкненої 
    одиничної кулі $B[0; 1]$ є компактом.
\end{exercise}

\begin{exercise}
    Довести, що будь-який компактний оператор в гільбертовому просторі є рівномірною 
    границею послідовності скінченновимірних операторів.
\end{exercise}

\begin{exercise}\label{N:1_5_23}
    Довести, що будь-який обмежений оператор в сепарабельному гільбертовому просторі $H$ є 
    сильною границею послідовності:
    \begin{enumerate}
        \item компактних операторів;
        \item скінченновимірних операторів.
    \end{enumerate}
    А в несепарабельному просторі $H$ це вірно?
\end{exercise}

\begin{exercise}
    Довести, що оператор диференціювання $Ax = x^\prime$, що діє з $C^1[a, b]$ в $C[a, b]$ 
    не є компактним.
\end{exercise}