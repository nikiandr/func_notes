% !TEX root = ../main.tex

\begin{theory}
    Нехай $A$ --- лінійний оператор в гільбертовому просторі $H$. 
    Для оператора $A = I + T$ ($T$ --- компактний оператор) виконуються 
    3 \ul{теореми Фредгольма}:
    \begin{enumerate}
        \item[1)] Рівняння $Ax = y$ має розв'язок для тих і 
        тільки тих $y$, які ортогональні кожному розв'язку рівняння 
        $A^*u = 0$ ($\mathrm{Im} A \oplus \mathrm{Ker} A^* = H$).
        \item[2)] Виконується \ul{альтернатива}: або рівняння $Ax = y$ 
        має і при тому єдиний розв'язок при кожному $y \in H$ або рівняння 
        $Ax = 0$ має ненульовий розв'язок ($0 \in \rho(A) \cup \sigma_\text{т}(A)$).
        \item[3)] Рівняння $Ax = 0$ та $A^* u = 0$ мають і при тому 
        однакову скінченну кількість лінійно незалежних розв'язків 
        ($\dim \mathrm{Ker} A = \dim \mathrm{Ker} A^* < \infty$)
    \end{enumerate}
\end{theory}

\begin{exercise}
    Довести теореми Фредгольма.
\end{exercise}

\begin{exercise}
    При яких $\lambda \in \complex$ рівняння $x(t) = \lambda 
    \intl{a}{b}e^{t-s}x(s)ds$ має ненульовий розв'язок в просторі 
    $L_2[a, b]$?
\end{exercise}

\begin{exercise}\label{N:1_8_12}
    При яких $f \in L_2 [0, \pi]$ інтегральне рівняння $x(t) = 
    \intl{0}{\pi} sin(t-s)x(s)ds + f(t)$ має розв'язок в просторі 
    $L_2[0, \pi]$?
\end{exercise}

\begin{exercise}
    Для оператора $A \in L(\ell_2)$ перевірити, які з теорем 
    Фредгольма і за яких умов виконуються для рівняння $Ax = y$ у 
    наступних випадках:
    \begin{enumerate}
        \item $Ax = (\lambda_1 x_1, \lambda_2 x_2, ...)$, де 
        $\{\lambda_n\}_{n=1}^\infty$ --- обмежена послідовність;
        \item $A$ --- оператор правого зсуву в $\ell_2$;
        \item $A$ --- оператор лівого зсуву в $\ell_2$;
        \item[г)*] $A$ --- сума операторів правого та лівого 
        зсуву в $\ell_2$;
    \end{enumerate}
\end{exercise}

\begin{exercise}
    Нехай $\{e_n\}$ --- ортонормований базис гільбертового простору $H$,
    $\lambda_n \in \real$; $\lambda_n \rightarrow 0$,
    $A: H \rightarrow H$ --- 
    такий лінійний оператор, що для кожного $x \in H$ має місце рівність:
    $Ax = \suml{n=1}{\infty} \lambda_n \dotprod{x}{e_n}e_n$.
    Довести, що $A$ --- цілком неперервний самоспряжений оператор.
\end{exercise}

\begin{theory}
    \begin{theorem*} [Гільберта-Шмідта]
    Нехай $H$ --- гільбертів простір, 
    $A$ --- компактний самоспряжений оператор в $H$. Тоді існує ортонормована 
    система векторів $\{e_n\}$ ($1 \leq n \leq N \leq \infty$) та 
    числовий набір $\{\lambda_n\} \subset \real \backslash \{0\}$ 
    ($1 \leq n \leq N \leq \infty$) такі, що $e_n$ --- власні вектори 
    оператора $A$, що відповідають власним числам $\lambda_n$,
    $\lambda_n \rightarrow 0$ за умови $N = \infty$ і при цьому для кожного 
    $x \in H$ виконуються рівності:
    \begin{equation*}
        x = \suml{n=1}{N}\dotprod{x}{e_n}e_n + \tilde{x}, \text{де } 
        \tilde{x} \in \mathrm{Ker}A;
    \end{equation*}
    \begin{equation*}
        Ax = \suml{n=1}{N}\lambda_n \dotprod{x}{e_n}e_n 
        \text{ (при }
        N = \infty
        \text{ цей ряд називають \ul{рядом Шмідта})}.
    \end{equation*}
    \end{theorem*}
\end{theory}