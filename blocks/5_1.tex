% !TEX root = ../main.tex

\begin{theory}
    Нехай $X$, $Y$ --- нормовані простори, $A: X \to Y$ --- лінійний оператор.
    Оператор $A$ називається \ul{цілком непереревним} (або \ul{компактним}),
    якщо для кожної обмеженої множини $Z \subset X$ її образ $A(Z)$ є передкомпактною
    множиною в $Y$. 
    Множина всіх компактних операторів з $X$ в $Y$ позначається через 
    \ul{$K(X,Y)$}, а якщо $X = Y$, то через \ul{$K(X)$}.
\end{theory}

\begin{exercise}
    Нехай $X$, $Y$ --- нормовані простори, $A: X \to Y$ --- лінійний оператор. Доведіть:
    \begin{enumerate}
        \item якщо образ кулі $B(0;1) = \set{x \in X \mid \norm{x} < 1}$ є передкомпактом в $Y$, то $A$ --- компактний оператор;
        \item якщо $A$ --- компактний оператор, то він обмежений, тобто $K(X,Y) \subset L(X,Y)$.
    \end{enumerate}
\end{exercise}

\begin{exercise}
    Нехай $X$, $Y$ --- нормовані простори, $A, B, A_n \in K(X,Y)$. Довести:
    \begin{enumerate}
        \item $A+B \in K(X, Y)$;
        \item $\lambda A \in K(X, Y)$ (тут $\lambda$ --- число);
        \item $\left( A_n \rightrightarrows C, C\in L(X, Y)\right) \Rightarrow \left( C \in K(X, Y)\right)$.
    \end{enumerate}
\end{exercise}

\begin{exercise}
    Нехай $X$, $Y$, $Z$ --- нормовані простори, $A \in L(X, Y)$, $B \in L(Y, Z)$.
    Довести, що якщо принаймні один з операторів $A$ або $B$ є компактним, то й оператор $BA$ --- компактний.
\end{exercise}

\begin{theory}
    Результати трьох задач вище приводять до висновку: для нормованого простору $X$ $K(X)$ 
    є замкненим двобічним ідеалом в операторній алгебрі $L(X)$.
\end{theory}

\begin{exercise}
    Нехай $X$, $Y$ --- нормовані простори, $A: X \to Y$ --- обмежений лінійний оператор
    скінченного рангу. Довести, що тоді $A \in K(X, Y)$.
\end{exercise}

\begin{exercise}
    Довести, що тотожній оператор в нормованому просторі $X$ є компактним
    тоді й тільки тоді, коли $\mathrm{dim} X < \infty$.
\end{exercise}

\begin{exercise}
    Нехай $A \in K(X)$, $\mathrm{dim} X = \infty$. Довести, що не існує оператора
    $B \in L(X)$, для якого виконується або рівність $AB = I$, або рівність $BA = I$ 
    ($I$ --- тотожний оператор в $X$).
\end{exercise}

\begin{exercise}
    Нехай $X$, $Y$ --- нормовані простори, $A \in K(X, Y)$. Довести, що простір $\left( \mathrm{Im} A, \norm{\cdot}_Y\right)$ --- сепарабельний.
\end{exercise}

\begin{exercise}
    Нехай $X$ та $Y$ --- банахові простори, $A \in K(X, Y)$. $Z \subset \mathrm{Im} A$, 
    $Z$ --- замкнений підпростір в $Y$. Довести: $\mathrm{dim} Z < \infty$.
\end{exercise}

\begin{exercise}
    Довести, що обмежений оператор проектування в банаховому просторі $X$
    є компактним тоді й тільки тоді, коли він скінченновимірний.
\end{exercise}

\begin{exercise}
    Чи вірно, що якщо в нескінченновимірному нормованому просторі $X$ для оператора $A \in L(X)$
    виконується рівність $A^2 = 0$, то $A$ --- компактний оператор?
\end{exercise}

\begin{exercise}
    Нехай $\{e_n\}$ --- ортонормована система в гільбертовому просторі $H$, $A \in K(H)$.
    Довести: $\norm{A e_n} \to 0$, $n \to \infty$.
\end{exercise}

\begin{exercise}
    Нехай $\{a_n\}$ --- числова послідовність. Довести, що оператор $A: \ell_2 \to \ell_2$,
    визначений формулою $A\vec{x} = (a_1 x_1, a_2 x_2, ...)$ є компактним тоді й тільки тоді,
    коли $a_n \to 0$.
\end{exercise}

\begin{exercise}
    Довести, що оператор вкладення $A: C^1 [a;b] \to C[a;b]$, $(Ax)(t) = x(t)$ є компактним.
\end{exercise}