% !TEX root = ../main.tex

\begin{exercise}
    Нехай $A, B$ --- самоспряжені оператори в $H$; $A \geq B$; $B \geq A$.
    Довести: $A = B$.
\end{exercise}

\begin{exercise}
    Нехай $A$ --- самоспряжений оператор в $H$; $\lambda \geq 0$; $0 \leq A \leq \lambda I$.
    Довести: $\norm{A} \leq \lambda$.
\end{exercise}

\begin{exercise}
    Нехай $A$ --- самоспряжений оператор в $H$. Довести:
    $ (A \geq 0; \exists A^{-1} \in L(H)) \Rightarrow (A^{-1} \geq 0)$.
\end{exercise}

\begin{exercise}
    Нехай $A$ --- самоспряжений оператор в $H$; $A \geq 0$. Довести еквівалентність наступних умов:
    \begin{enumerate}[label=\ukr*)]
        \item $\overline{\rm \mathrm{Im}A} = H$;
        \item $\mathrm{Ker}A = \{0\}$;
        \item $(Ax, x) > 0$ для $\forall x \in H \setminus \{0\}$.
    \end{enumerate}
\end{exercise}

\begin{exercise}\label{N:1_3_39}
    Нехай $A$ --- лінійний оператор в гільбертовому просторі $H$ і для $\forall x, y \in H$ виконується
    рівність: $(Ax, y) = (x, Ay)$. Довести: $A \in L(H)$ (а тому $A$ --- самоспряжений).
\end{exercise}

\begin{exercise}
    Нехай числова послідовність $\vec{a} = (a_1, a_2, \dots)$ така, що для кожного $\vec{x} \in \ell_2$
    ряд $\sum\limits_{n = 1}^\infty a_n x_n$ збігається. Тоді $\vec{a} \in \ell_2$ і формула 
    $\varphi(\vec{x}) = \sum\limits_{n = 1}^\infty a_n x_n$ задає неперервний лінійний функціонал на $\ell_2$.
    Довести.
\end{exercise}

\begin{exercise}\label{N:1_3_41}
    Нехай $A$ --- самоспряжений оператор в $H$; $\exists m > 0: (Ax, x) \geq m \norm{x}^2$ для $\forall x \in H$.
    Довести: $\forall f \in H$ рівняння $Ax = f$ має і при тому єдиний розв'язок.
\end{exercise}

\begin{exercise}
    Використовуючи результат задачі \ref{N:1_3_41}, довести розв'язність в дійсному просторі $L_2[0; 1]$ рівняння
    $x(t) = \int\limits_{0}^1 K(t, s)x(s)ds + f(t)$ для таких функцій $K$:
    \begin{enumerate}[label=\ukr*)]
        \item $K(t, s) = -e^{ts}$;
        \item $K(t, s) = \sin(ts)$;
        \item $K(t, s) = -\sum\limits_{n = 1}^\infty \frac{\sin(nt)\sin(ns)}{n^2}$;
        \item $K(t, s) = -\sum\limits_{n = 1}^\infty t^n(1-t)s^n(1-s)$.
    \end{enumerate}
\end{exercise}

\begin{exercise}
    Для функціоналів на $L_2[0; 1]$ вказати такий вектор $h \in L_2[0; 1]$, що $\varphi(x) = (x, h)$ для $\forall x$:
    \begin{enumerate}[label=\ukr*)]
        \item $\varphi(x) = \int\limits_{0}^{\frac{1}{2}}x(s)ds$;
        \item $\varphi(x) = \int\limits_{0}^{\frac{1}{3}}x(s)ds - \int\limits_{\frac{1}{2}}^{1}x(s)ds$;
        \item $\varphi(x) = \int_A x(s)ds$, де $A$ --- вимірна множина на $[0; 1]$.
    \end{enumerate}
\end{exercise}

\begin{exercise}
    Нехай $H$ --- нескінченновимірний гільбертів простір. Довести, що простір $L(H)$ не є сепарабельним.
\end{exercise}

\begin{exercise}
    Нехай $A \geq B \geq 0$. Довести: $\norm{A} \geq \norm{B}$.
\end{exercise}