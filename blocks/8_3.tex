% !TEX root = ../main.tex

\begin{exercise}
    Нехай $K(t,s) = \sum\limits^\infty_{n=1} \frac{1}{n^2} \sin(n \pi t) \sin(n \pi s)$,
    оператор $A$ діє в $L_2[-1;1]$ за формулою $(Ax)(t) = \int\limits^1_{-1} K(t,s) x(s) ds$.
    \begin{enumerate}
        \item Довести: $A$ --- компактний самоспряжений оператор;
        \item знайти $\sigma(A)$, $r(A)$, $\norm{A}$;
        \item зобразити $A$ у вигляді ряду Шмідта.
    \end{enumerate}
\end{exercise}

\begin{exercise}
    Нехай $K(t) = \sum\limits_{n \in \integers} c_n e^{i n t}$, $c_n \in \real$ та ряд
    $\sum\limits_{n \in \integers} |c_n|$ збігається.
    Розглянемо в $L_2[-\pi; \pi]$
    оператор $A$, заданий формулою $(Ax)(t) = \int\limits^\pi_{-\pi} K(t-s) x(s) ds$.
    \begin{enumerate}
        \item Довести: $A$ --- самоспряжений оператор Гільберта-Шмідта; функції \\$\varphi_n(t) = 
        \frac{1}{\sqrt{2\pi}} e^{i n t}$, $n \in \integers$ є власними функціями оператора $A$;
        \item знайти $\sigma(A)$, $r(A)$, $\norm{A}$;
        \item зобразити $A$ у вигляді ряду Шмідта.
    \end{enumerate}
\end{exercise}

\begin{exercise}
    Знайти спектри наступних операторів $A \in L(C[a;b])$:
    \begin{enumerate}
        \item $(Ax)(t) = \int\limits^1_0 (t+s)x(s)ds$, $a=0$, $b=1$;
        \item $(Ax)(t) = x(t) + \int\limits^1_0 (t+s)x(s)ds$, $a=0$, $b=1$;
        \item $(Ax)(t) = \int\limits^{2\pi}_0 \cos(t+s)x(s)ds$, $a=0$, $b=2\pi$.
    \end{enumerate}
\end{exercise}

\begin{exercise}
    Довести, що для будь-якої функції $f \in C[a;b]$ (варіант: $f \in L_2[a;b]$)
    наступне рівняння має розв'язок в $L_2[a;b]$:
    \begin{enumerate}
        \item $x(t) = -\int\limits^1_0 x(s) ds + f(t)$, $a=0$, $b=1$;
        \item $x(t) = -\int\limits^1_0 (ts+2)x(s)ds + f(t)$, $a=0$, $b=1$;
        \item $x(t) = -\int\limits^1_0 \cos(t-s)x(s)ds + f(t)$, $a=0$, $b=1$;
        \item $x(t) = -\int\limits^1_0 e^{ts}x(s)ds + f(t)$, $a=0$, $b=1$;
        \item $x(t) = \int\limits^1_0 (t-s)x(s)ds + f(t)$, $a=0$, $b=1$;
        \item $x(t) = \int\limits^\pi_0 \sin(t-s)x(s)ds + f(t)$, $a=0$, $b=\pi$;
        \item $x(t) = \int\limits^1_0 (t-s)\cos(t+s)x(s)ds + f(t)$, $a=0$, $b=1$;
    \end{enumerate}
\end{exercise}

\begin{theory}
    \ul{Підказ}: якщо для $A\in L(H)$ $\forall \; x \; \dotprod{Ax}{x} \leq 0$,
    то $(x=Ax) \Rightarrow (x=0)$.
\end{theory}

\begin{exercise}
    Використовуючи теорему Фредгольма, довести, що наступні рівняння мають розв'язок в $L_2[a;b]$
    при заданих $f$ та довільних $\lambda$:
    \begin{enumerate}
        \item $x(t) = \lambda \int\limits^1_{-1} x(s) ds + f(t)$;
              $f(t) = t, t^3, e^{t^2}\sin t$, $a=-1$, $b=1$;
        \item $x(t) = \lambda \int\limits^{2\pi}_0 \sin(t+s) x(s) ds + f(t)$; 
              $f(t) = 1, \sin 2t, \cos 2t$, $a=0$, $b=2\pi$.
    \end{enumerate}
\end{exercise}

\begin{exercise}
    Використовуючи теорему Фредгольма, довести, що наступні рівняння мають розв'язок в $L_2[a;b]$:
    \begin{enumerate}
        \item $x(t) = \int\limits^1_{-1} \left(
                \sum\limits^\infty_{k=1} \frac{t^k}{k!} s^{2k}
            \right) x(s) ds + t^3$, $a=-1$, $b=1$;
        \item $x(t) = \int\limits^{2\pi}_0 \left(
                \sum\limits^\infty_{k=1} e^{-kt} \frac{\cos(ks)}{k^2}
            \right) x(s) ds + \sin 2t$, $a=0$, $b=2\pi$.
    \end{enumerate}
\end{exercise}

\begin{exercise}
    Розв'язати рівняння в просторі $L_2[a;b]$ з симетричним ядром, зведенням їх диференціюванням
    до відповідної крайової задачі:
    \begin{enumerate}
        \item $x(t) = \int\limits^1_0 K(t,s) x(s) ds + t^2 - t$,
        $K(t,s) = \begin{cases}
            (t-1)s, & 0 \leq s \leq t \leq 1 \\
            t(s-1), & 0 \leq t < s \leq 1
        \end{cases}$;
        \item $x(t) = \int\limits^\pi_0 K(t,s) x(s) ds + 1$,\
        $K(t,s) = \begin{cases}
            2 \sin (t) \cos (s), & 0 \leq t \leq s \leq \pi \\
            2 \sin (s) \cos (t), & 0 \leq s < t \leq \pi .
        \end{cases}$
    \end{enumerate}
\end{exercise}