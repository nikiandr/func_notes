% !TEX root = ../main.tex

\begin{theory}
    \begin{theorem*}[Банаха про обернений оператор]
        Нехай $X$, $Y$ --- банахові простори; $A \in L(X,Y)$,
        $A$ --- бієктивне відображення. Тоді $A$ --- неперервно оборотний
        $\left(\exists \; A^{-1} \in L(Y, X)\right)$.
    \end{theorem*}
\end{theory}

\begin{exercise}
    Нехай $X$, $Y$ --- банахові простори, $A \in L(X,Y)$.
    Доведіть, що наступні три умови еквівалентні:
    \begin{enumerate}
        \item $A$ --- неперервно оборотний;
        \item $A$ має і при тому єдиний правий обернений оператор;
        \item[в)*] $A$ має і при тому єдиний лівий обернений оператор.
    \end{enumerate}
\end{exercise}

\begin{exercise}
    Навести приклад оператора $A \in L(X,Y)$ в банахових просторах, для якого:
    \begin{enumerate}
        \item існує правий обернений $B\in L(Y,X)$, але немає лівого оберненого;
        \item існує лівий обернений $C\in L(Y,X)$, але немає правого оберненого.
    \end{enumerate}
\end{exercise}

\begin{exercise}\label{N:1_6_28}
    Нехай $X$, $Y$, $Z$ --- банахові простори; $B \in L(Y,Z)$ --- ін'єктивний
    оператор ($\mathrm{Ker}B = \{0\}$); $A$ --- лінійний оператор з $X$ в $Y$;
    $BA \in L(X,Z)$, $\mathrm{Im}(BA) = Z$. Довести $A \in L(X,Y)$.
\end{exercise}

\begin{exercise}
    Нехай $\norm{\cdot}_1$, $\norm{\cdot}_2$  --- дві норми в лінійному просторі
    $X$, причому обидва простори $(X, \norm{\cdot}_1)$ та $(X, \norm{\cdot}_2)$ 
    повні. Відомо, що існує константа $C > 0$ така, що для $\forall \; x \in X$:
    $\norm{x}_1 \leq C \norm{x}_2$. Доведіть еквівалентність цих норм.
\end{exercise}

\begin{exercise}
    Нехай $X$ --- нормований простір, $A \in L(X)$ і існують такі комплексні числа
    $\lambda_1, \lambda_2, ..., \lambda_n$, що $I + \lambda_1 A + \lambda_2 A^2
    + ... + \lambda_n A^n = 0$. Довести, що $A$ --- неперервно оборотний.
\end{exercise}

\begin{exercise}
    $X$ --- нормований простір; $A, B \in L(X)$, $AB+A+I=BA+A+I=0$.
    Довести, що $A$ --- неперервно оборотний.
\end{exercise}

\begin{exercise}
    $X$ --- нормований простір; $A, B \in L(X)$, $A$ --- оборотний, $AB=BA$.
    Довести: $A^{-1}B = BA^{-1}$.
\end{exercise}

\begin{exercise}
    Нехай $H$ --- гільбертів простір, $A \in L(X)$, $A$ --- самоспряжений
    і існує таке число $c > 0$, для якого $A \geq cI$.
    Довести, що $A$ --- неперервно оборотний.
\end{exercise}

\begin{exercise}
    Навести приклад банахових просторів $X$, $Y$ і послідовності неперервно
    оборотних операторі $A_n \in L(X,Y)$, для яких $A_n \rightrightarrows A$,
    але $A$ не є оборотним.
\end{exercise}

\begin{exercise}
    Навести приклад банахового простору $X$ і неперервно оборотного оператора
    $A \in L(X)$, для якого існує послідовність необоротних операторів
     $A_n \in L(X)$, що $A_n \overset{s}{\to} A$.
\end{exercise}

\begin{exercise}\label{N:1_6_36}
    Нехай $A, A_n \in L(X, Y)$, $A_n \rightrightarrows A$.
    Довести: $A$ --- неперервно оборотний тоді й тільки тоді, коли
    виконуються дві наступні умови:
    \begin{enumerate}
        \item $\exists \; N$ $\forall n \geq N$: $A_n$ -- неперервно оборотні;
        \item $\underset{n \geq N}{\sup} \norm{A_n^{-1}} < \infty$.
    \end{enumerate}
\end{exercise}