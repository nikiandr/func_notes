% !TEX root = ../main.tex

\noindent\ref{N:1_1_6}. Припустимо, що $|x(t)|\leq 1$, але $|\varphi(x)| = 2$.
Оскільки $|\int\limits^1_0 x(t) dt| \leq 1$, $|x(0)| \leq 1$, то це можливо лише тоді,
якщо $\int\limits^1_0 x(t) dt = 1$ та $x(0) = -1$, або, навпаки: $\int\limits^1_0 x(t) dt = -1$,
$x(0) = 1$. Розглянемо перший варіант. З неперервності функції $x$ виходить існування
$\delta > 0$ такого, що $x(t) < 0$ при $t \in [0,\delta]$. Але тоді 
$\int\limits^1_0 x(t) dt < \int\limits^1_\delta x(t) dt < 1$.

\noindent\ref{N:1_1_9}. Нехай $\bigcap\limits_{k=1}^m {Ker} \varphi_k \subset {Ker}\varphi$.
Без втрати загальності можна вважати, що $\forall k \in \set{1, ..., m}: {Ker} \varphi_k \not\supset \bigcap\limits_{j \neq k} {Ker} \varphi_j$.
Звідси зробіть висновок, що
$\forall k \in \set{1, ..., n}$ $\exists \; x_k \in L$
такий, що $\forall j \in \set{1, ..., m} : \varphi_j(x_k) = \delta_{jk}$ (<<символ Кронекера>>)
і розгляньте функціонал $\sum\limits_{k=1}^m \varphi(x_k) \cdot \varphi_k$.

\noindent\ref{N:1_1_12}.
ґ) $\norm{x}_1 = \underset{0\leq t \leq 1}{\max} |x(t)| + \underset{0\leq t \leq 1}{\max} |x'(t)|$, тож $\norm{A} \leq 1$.
Зворотну нерівність можна одержати за допомогою послідовності $x_n(t) = \frac{1}{n} t^n$.

\noindent з) $\norm{A\vec{x}} \leq 2 \cdot \sum\limits_{k=1}^{\infty} |x_k| = 2 \norm{\vec{x}}$, $A \vec{e_1} = (1, 1, 0, ...)$. $\norm{A} = 2$.

\noindent і) $\norm{Ax}^2 = \int\limits_0^1 \varphi^2 (t) x^2(t) dt \leq \left(\underset{[0;1]}{\max} |\varphi(t)|\right)^2 \cdot \norm{x}^2 $, 
тому $\norm{A} \leq \underset{[0;1]}{\max} |\varphi(t)|$.
З іншого боку, якщо $\underset{[0;1]}{\max} |\varphi(t)| = C > 0$ (випадок $C = 0$ очевидний), то
$\forall \varepsilon \in (0, C) \; \exists \; \text{проміжок } \Delta \subset [0;1]$,
на якому $|\varphi(t)| > C - \varepsilon$. Нехай $x = j_\Delta$ ($x(t) = 1$, якщо $t \in \Delta$ та $x(t) = 0$, якщо $t \notin \Delta$).
Тоді $\norm{x}^2 = \mu(\Delta)$, $\norm{Ax}^2 \geq \int\limits_\Delta (C-\varepsilon)^2 dt = (C-\varepsilon)^2 \cdot \mu(\Delta)$ (тут $\mu$ --- міра <<довжина>>).
Звідси отримуємо $\norm{A} = \underset{[0;1]}{\max} |\varphi(t)|$.

\noindent ї) $\left| (Ax)(t) \right|^2 = t^2 \cdot \left( \int\limits_0^1 s x(s) ds\right)^2 \leq t^2 \cdot \left( \int\limits_0^1 s^2 ds\right)^2 \cdot \left( \int\limits_0^1 x^2(s) ds\right)^2 = \frac{1}{3} \cdot t^2 \cdot \norm{x}^2$.
$\norm{Ax}^2 = \int\limits_0^1 \left| (Ax)(t) \right|^2 dt \leq \frac{1}{9} \cdot \norm{x}^2$, звідки $\norm{A} \leq \frac{1}{3}$.
Зворотну нерівність отримуємо підстановкою $x(t) = t$.

\noindent\ref{N:1_1_17}. Нехай $\varphi \neq 0$ та $\mathrm{Ker}\varphi$ - замкнене,
$x_n \to 0$. Доведемо: $\varphi(x_n) \to 0$. Нехай $\varphi(x_n) \not\to 0$.
\uline{Випадок 1}: послідовність $\{\varphi(x_n)\}$ - обмежена. Тоді існує підпослідовність
$x_{n_k}$, для якої $\varphi(x_{n_k}) \to C \neq 0$. Візьмемо вектор $z$, для якого
$\varphi(z) = 1$. Тоді $x_{n_k} - \varphi(x_{n_k})\cdot z \in \mathrm{Ker}\varphi$;
$x_{n_k} - \varphi(x_{n_k})\cdot z \to - Cz \notin \mathrm{Ker}\varphi$, суперечність.
\uline{Випадок 2}: послідовність $\{\varphi(x_n)\}$ - необмежена. Тоді існує
підпослідовність $x_{n_k}$ така, що $\varphi(x_{n_k}) \to \infty$. Тоді 
$y_k = \frac{1}{\varphi(x_{n_k})}x_{n_k} \to 0$; $\varphi(y_k)=1, \; \forall k$,
а це суперечить випадку 1.

\noindent\ref{N:1_1_19}. Нехай $X = C^1[a,b]$, але з нормою простору $C[a,b]=Y$.
Оператор $A = \frac{d}{dt}: X \to Y$ необмежений, але його ядро $\mathrm{Ker}A$
замкнене в $X$.

\noindent\ref{N:1_1_20}. Нехай $x_n \to 0$, але $\varphi(x_n) \not\to 0$.
Тоді існує підпослідовність $x_{n_k}$ така, що $|\varphi(x_{n_k}| \geq \delta > 0$.
Тоді $y_k = \frac{1}{\sqrt{\norm{x_{n_k}}}} x_{n_k} \to 0$, але $|\varphi(y_k)|\to\infty$.

\noindent\ref{N:1_1_21}. $\rho(x,\mathrm{Ker}\varphi) \geq \inf
\{\; \frac{ |\varphi(x-y)| }{ \norm{\varphi} } \mid y \in \mathrm{Ker}\varphi \;\} = 
\frac{|\varphi(x)|}{\norm{\varphi}}$.
\uline{Зворотня нерівність}: $\forall\varepsilon > 0$ $\exists z \in X$:
$|\varphi(z)| > (\norm{\varphi} - \varepsilon)\norm{z}$;
$z = \alpha x + y$, $y \in \mathrm{Ker}\varphi$, $\alpha \neq 0$.
$|\alpha| \cdot |\varphi(x)|
> (\norm{\varphi} - \varepsilon)\cdot|\alpha|\cdot\norm{x = \frac{1}{\alpha}y}
\geq (\norm{\varphi} - \varepsilon)\cdot|\alpha|\cdot\rho(x,\mathrm{Ker}\varphi)$.
Звідси: $\frac{\varphi(x)}{\norm{\varphi} - \varepsilon} > \rho(x,\mathrm{Ker}\varphi)$
і врахувати довільність $\varepsilon$.

\noindent\ref{N:1_1_22}. Задача рівносильна \ref{N:1_1_21}, тому що якщо
$\varphi(x) = a$, то $\rho(x,\mathrm{Ker}\varphi) = \rho(0,L)$, де
$L = \set{y \mid \varphi(y) = a}$.

\noindent\ref{N:1_1_26}. Якщо ряд $\sum\limits^\infty_{k=0} A^k$ збігається,
то $\norm{A^k} \to 0$, $k \to \infty$. Якщо існує $n \in \mathbb{N}$, для
якого $\norm{A^k} < 1$, то для $m \in \set{0,1,\dots,n-1}$ ряд
$\sum\limits^\infty_{k=0} \norm{A^{m+kn}}$ збіжний, тому що $\norm{A^{m+kn}}
\leq \norm{A^m} \cdot \norm{A^n}^k$. Абсолютна збіжність вихідного ряду є
наслідком рівності $\sum\limits^\infty_{k=0} \norm{A^k} = 
\sum\limits^{n-1}_{m=0} \sum\limits^\infty_{k=0} \norm{A^{m+kn}}$.

\noindent\ref{N:1_1_27}. $\mathrm{Im}A = \text{л.о.}\{y\}$, якщо $\varphi \neq 0$;
$\mathrm{Im}A = \{0\}$, якщо $\varphi = 0$. $\mathrm{Ker}A = X$, якщо $y = 0$;
$\mathrm{Ker}A = \mathrm{Ker}\varphi$, якщо $y \neq 0$.
$\norm{Ax} = |\varphi(x)|\cdot \norm{y} \leq \norm{\varphi}\cdot\norm{y}\cdot\norm{x}$.
Тому $\norm{A} \leq \norm{\varphi}\cdot\norm{y}$. Нехай $\varphi \neq 0$, $y \neq 0$.
Беремо $\varepsilon > 0$ $\exists x \in X$: $|\varphi(x)| > (\norm{\varphi}-\varepsilon)
\norm{x}$. Тому $\norm{Ax} > (\norm{\varphi}-\varepsilon)\cdot\norm{y}\cdot\norm{x}$.
Тому $\norm{A} \geq (\norm{\varphi}-\varepsilon)\cdot\norm{y}$ і врахуйте довільність 
$\varepsilon>0$.

\noindent\ref{N:1_1_30}. а) Відображення $A: \ell_\infty \to \ell_1^*$
будуємо наступним чином: $\vec{c} = (c_1,c_2,\dots) \in \ell_\infty$,
$A\vec{c}: \ell_1 \ni \vec{x} \mapsto \sum\limits^\infty_{k=1} c_k x_k$.
При цьому ряд $\sum\limits^\infty_{k=1} |c_k x_k|$ --- збіжний, тому що послідовність
$\{c_k\}$ обмежена, а ряд $\sum\limits^\infty_{k=1} |x_k|$ --- збіжний.
$A\vec{c}$ --- лінійний функціонал на $\ell_1$. Його обмеженість --- наслідок нерівності
$|\sum\limits^\infty_{k=1} c_k x_k| \leq \sup\{|c_k|\} \cdot \sum\limits^\infty_{k=1} |x_k|$.
При цьому $\norm{A \vec{c}} \leq \norm{\vec{c}} = \sup\{|c_k|\}$. Насправді 
$\norm{A \vec{c}} = \norm{\vec{c}}$ тому, що для $\forall k: |(A \vec{c})(\vec{e}_k)| = |c_k|$,
де $\vec{e}_k = (\underbrace{0,0,\dots}_{k-1},1,0,\dots) \in \ell_1$.
Наступний крок --- це лінійність відображення $A$ (перевірте!). Залишилось довести
$\mathrm{Im}A = \ell_1^*$. Беремо $\varphi\in \ell_1^*$. Будуємо послідовність $\{c_k\}$
за правилом $c_k = \varphi(\vec{e}_k)$. Тоді $\{c_k\}$ - обмежена послідовність, 
$\vec{c} \in \ell_\infty$. При цьому $\varphi(\vec{x}) = \sum\limits^\infty_{k=1} c_k x_k$.
Остання формула --- наслідок граничного переходу $\varphi(\vec{x}) = 
\underset{n\to \infty}{\lim} \varphi(\sum\limits^\infty_{k=1} x_k \vec{e}_k)$.
Тож $\varphi = A \vec{c}$.

\noindent\ref{N:1_1_44}. Для вимірної підмножини $Z \subset [0; 1]$ розглянути оператор
$A_Z: f \mapsto f \cdot j_Z$ ($j_A$ --- індикатор множини $A$). У разі, якщо $\mu(Z) \neq 0$, 
$\norm{A_Z} = 1$ і, якщо $\mu (Z_1 \bigtriangleup Z_2) > 0$, $\norm{A_{Z_1} - A_{Z_2}} = 1$.
Розглянути $\{A_{Z_t}\}$, де $Z_t = [0, t]$.

\noindent\ref{N:1_1_45}. Умова $AB = I_Y$ гарантує, що $\mathrm{Im} A = Y$; умова $6A = I_X$
гарантує, що $\rm KerA = \{0\}$. Тому існує обернений оператор $A^{-1}$. 
Тому $B = A^{-1}AB = A^{-1}$; $C = CAA^{-1} = A^{-1}$.

\noindent\ref{N:1_1_46}. Розглянути на $\ell_2$ функціонали $\varphi_n (\vec{\beta}) = \sum\limits_{k = 1}^{n} \alpha_k \beta_k$
і застосувати теорему Банаха-Штейнгауза.

\noindent\ref{N:1_2_8}. Нехай $A: H_1 \to H_2$ --- ізоморфізм і $H_1$ --- сепарабельний
гільбертів простір. Доведемо сепарабельність простору $H_2$. Нехай система $\{e_n\}$ ---
ортонормована в $H_1$. Тоді система $\{A e_n\}$ --- ортонормована в $H_2$. Оскільки
лінійна оболонка $\{e_n\}$ щільна в $H_1$, а оператор $A$ --- обмежений і $\mathrm{Im}A = H_2$,
то вектори $f_n = A e_n$ утворюють ортонормований базис в $H_2$. Тому в $H_2$ щільна
множина лінійних комбінацій векторів $f_n$ з раціональними (у випадку поля $\mathbb{R}$)
коефіцієнтами. У випадку поля $\mathbb{C}$ слід брати комплексні коефіцієнти
$z_k = u_k + i v_k$, $u_k, v_k \in \mathbb{Q}$.

\noindent\ref{N:1_2_14}. в) Відповідь: $M^{\perp} = \overline{\{1; \cos nt, n \in \mathbb{N} \}}$.

\noindent\ref{N:1_2_16}. а) $\Rightarrow$ б) 
$(x \in H_2^{\perp}) \iff (\norm{x_2}^2 = (x, x_2) = 0) \iff (x \in H_1)$;

\noindent б) $\Rightarrow$ a) $x - pr_{H_2} x \in H_2^{\perp} = H_1$.

\noindent\ref{N:1_2_18}. ґ) $(x \in M^{\perp}+N^{\perp}) \Rightarrow (x = x_1+x_2; x_1 \perp M; x_2 \perp N)$
$\Rightarrow (x \in (M \cap N)^{\perp})$. Тому $\overline{M^{\perp} + N^{\perp}} \subset (M \cap N)^{\perp}$.
$L = M \cap N$. $M = L \oplus  M_1; N = L \oplus N_1 (M_1 = L^{\perp} \cap M; N_1 = L^{\perp} \cap N); M_1 \cap N_1 = \{0\}$.
Тому $L^{\perp} = M_1 \oplus M^{\perp} = N_1 \oplus N^{\perp}$. 
$M_1 = M_1 \cap L^{\perp} = M_1 \cap (N_1 \oplus N^{\perp}) = M_1 \cap N^{\perp}$. 
Тому $L^{\perp} = (M_1 \cap N^{\perp}) \oplus M^{\perp} \subset \overline{N^{\perp} + M^{\perp}}$.

\noindent\ref{N:1_2_21}. $M = \{(1+\frac{1}{n})\vec{e_n}\}$ 
(тут $\vec{e_n} = (\underbrace{0, \dots 0}_{n-1}, 1, 0, \dots)$).

\noindent\ref{N:1_2_24}. Нехай $d = \rho(x, M)$; $\rho(x, y_1) = d + \varepsilon_1$; $\rho(x, y_2) = d + \varepsilon_2$;
$\varepsilon_2 \in (0, \varepsilon_1)$. Оскільки $\rho(x, \frac{y_1+y_2}{2}) \geq d$, то з
рівності паралелограма для векторів $y_1 - x$ та $y_2 - x$ одержимо нерівність: 
$\rho(y_1, y_2) \leq 2 \sqrt{2d\varepsilon_1 + \varepsilon_1^2}$. Тому існує послідовність додатних чисел
$\varepsilon_n \searrow 0$, для якої $\rho(y_n, y_{n+1}) \leq \frac{1}{2} \rho(y_{n-1}, y_n)$.
Послідовність $\{y_n\}$ є фундаментальною.

\noindent\ref{N:1_2_26}. б) $\Rightarrow$ в) Застосувати теорему Банаха-Штейнгауза до функціоналів
$\varphi_n (y) = (y, \sum_{k = 1}^n x_k)$.

\noindent\ref{N:1_2_27}. Якщо $\inf \lambda_k > 0$, то нова норма еквівалентна стандартній в $\ell_2$.
Якщо $\inf \lambda_k = 0$, то в разі повноти простору для функціоналів 
$\varphi_k (y) = (\frac{1}{\lambda_k} \vec{e_k}, \vec{y})$ мали б суперечність із теоремою Банаха-Штейнгауза.

\noindent\ref{N:1_2_32}. Якщо послідовність функцій $\{f_n\}$ має обмежені у сукупності похідні, то ця
система функцій одностайно неперервна. Крім того, умови 
$\sup_n \{|f_n (x) - f_n (y)|\mid n \in \mathbb{N}; x, y \in [0, 2\pi]\} < \infty$ та 
$\int\limits_0^{2\pi} f_n^2 (x) dx = 1$, $\forall n$ разом гарантують рівномірну обмеженість системи функцій.
За теоремою Арцела-Асколі: $\{f_n\}$ --- передкомпакт в $C[0; 2\pi]$, а тому й в $L_2 [0; 2\pi]$, що
суперечить її ортонормованості.
