% !TEX root = ../main.tex

\noindent\ref{N:1_1_6}. Припустимо, що $|x(t)|\leq 1$, але $|\varphi(x)| = 2$.
Оскільки $|\int\limits^1_0 x(t) dt| \leq 1$, $|x(0)| \leq 1$, то це можливо лише тоді,
якщо $\int\limits^1_0 x(t) dt = 1$ та $x(0) = -1$, або, навпаки: $\int\limits^1_0 x(t) dt = -1$,
$x(0) = 1$. Розглянемо перший варіант. З неперервності функції $x$ виходить існування
$\delta > 0$ такого, що $x(t) < 0$ при $t \in [0,\delta]$. Але тоді 
$\int\limits^1_0 x(t) dt < \int\limits^1_\delta x(t) dt < 1$.

\noindent\ref{N:1_1_9}. Нехай $\bigcap\limits_{k=1}^m \mathrm{Ker} \varphi_k \subset \mathrm{Ker}\varphi$.
Без втрати загальності можна вважати, що $\forall k \in \set{1, ..., m}: \mathrm{Ker} \varphi_k \not\supset \bigcap\limits_{j \neq k} \mathrm{Ker} \varphi_j$.
Звідси зробіть висновок, що
$\forall k \in \set{1, ..., m}$ $\exists \; x_k \in L$
такий, що $\forall j \in \set{1, ..., m} : \varphi_j(x_k) = \delta_{jk}$ (<<символ Кронекера>>)
і розгляньте функціонал $\sum\limits_{k=1}^m \varphi(x_k) \cdot \varphi_k$.

\noindent\ref{N:1_1_12}.
ґ) $\norm{x}_1 = \underset{0\leq t \leq 1}{\max} |x(t)| + \underset{0\leq t \leq 1}{\max} |x'(t)|$, тож $\norm{A} \leq 1$.
Зворотну нерівність можна одержати за допомогою послідовності $x_n(t) = \frac{1}{n} t^n$.

\noindent з) $\norm{A\vec{x}} \leq 2 \cdot \sum\limits_{k=1}^{\infty} |x_k| = 2 \norm{\vec{x}}$, $A \vec{e_1} = (1, 1, 0, ...)$. $\norm{A} = 2$.

\noindent і) $\norm{Ax}^2 = \int\limits_0^1 \varphi^2 (t) x^2(t) dt \leq \left(\underset{[0;1]}{\max} |\varphi(t)|\right)^2 \cdot \norm{x}^2 $, 
тому $\norm{A} \leq \underset{[0;1]}{\max} |\varphi(t)|$.
З іншого боку, якщо $\underset{[0;1]}{\max} |\varphi(t)| = C > 0$ (випадок $C = 0$ очевидний), то
$\forall \varepsilon \in (0, C) \; \exists \; \text{проміжок } \Delta \subset [0;1]$,
на якому $|\varphi(t)| > C - \varepsilon$. Нехай $x = j_\Delta$ ($x(t) = 1$, якщо $t \in \Delta$ та $x(t) = 0$, якщо $t \notin \Delta$).
Тоді $\norm{x}^2 = \mu(\Delta)$, $\norm{Ax}^2 \geq \int\limits_\Delta (C-\varepsilon)^2 dt = (C-\varepsilon)^2 \cdot \mu(\Delta)$ (тут $\mu$ --- міра <<довжина>>).
Звідси отримуємо $\norm{A} = \underset{[0;1]}{\max} |\varphi(t)|$.

\noindent ї) $\left| (Ax)(t) \right|^2 = t^2 \cdot \left( \int\limits_0^1 s x(s) ds\right)^2 \leq t^2 \cdot \left( \int\limits_0^1 s^2 ds\right) \cdot \left( \int\limits_0^1 x^2(s) ds\right) = \frac{1}{3} \cdot t^2 \cdot \norm{x}^2$.
$\norm{Ax}^2 = \int\limits_0^1 \left| (Ax)(t) \right|^2 dt \leq \frac{1}{9} \cdot \norm{x}^2$, звідки $\norm{A} \leq \frac{1}{3}$.
Зворотну нерівність отримуємо підстановкою $x(t) = t$.

\noindent\ref{N:1_1_17}. Нехай $\varphi \neq 0$ та $\mathrm{Ker}\varphi$ --- замкнене,
$x_n \to 0$. Доведемо: $\varphi(x_n) \to 0$. Нехай $\varphi(x_n) \not\to 0$.
\textit{Випадок 1:} послідовність $\{\varphi(x_n)\}$ --- обмежена. Тоді існує підпослідовність
$x_{n_k}$, для якої $\varphi(x_{n_k}) \to C \neq 0$. Візьмемо вектор $z$, для якого
$\varphi(z) = 1$. Тоді $x_{n_k} - \varphi(x_{n_k})\cdot z \in \mathrm{Ker}\varphi$;
$x_{n_k} - \varphi(x_{n_k})\cdot z \to - Cz \notin \mathrm{Ker}\varphi$, суперечність.
\textit{Випадок 2:} послідовність $\{\varphi(x_n)\}$ --- необмежена. Тоді існує
підпослідовність $x_{n_k}$ така, що $\varphi(x_{n_k}) \to \infty$. Тоді 
$y_k = \frac{1}{\varphi(x_{n_k})}x_{n_k} \to 0$, $\varphi(y_k)=1 \; \forall \;k$,
а це суперечить випадку 1.

\noindent\ref{N:1_1_19}. Нехай $X = C^1[a,b]$, але з нормою простору $C[a,b]=Y$.
Оператор $A = \frac{d}{dt}: X \to Y$ необмежений, але його ядро $\mathrm{Ker}A$
замкнене в $X$.

\noindent\ref{N:1_1_20}. Нехай $x_n \to 0$, але $\varphi(x_n) \not\to 0$.
Тоді існує підпослідовність $x_{n_k}$ така, що $|\varphi(x_{n_k})| \geq \delta > 0$.
Тоді $y_k = \frac{1}{\sqrt{\norm{x_{n_k}}}} x_{n_k} \to 0$, але $|\varphi(y_k)|\to\infty$.

\noindent\ref{N:1_1_21}. $\rho(x,\mathrm{Ker}\varphi) \geq \inf
\left\{ \frac{ |\varphi(x-y)| }{ \norm{\varphi} } \mid y \in \mathrm{Ker}\varphi \right\} = 
\frac{|\varphi(x)|}{\norm{\varphi}}$.
\textit{Зворотня нерівність}: $\forall\;\varepsilon > 0$ $\exists\; z \in X$:
$|\varphi(z)| > (\norm{\varphi} - \varepsilon)\norm{z}$;
$z = \alpha x + y$, $y \in \mathrm{Ker}\varphi$, $\alpha \neq 0$.
$|\alpha| \cdot |\varphi(x)|
> (\norm{\varphi} - \varepsilon)\cdot|\alpha|\cdot\norm{x + \frac{1}{\alpha}y}
\geq (\norm{\varphi} - \varepsilon)\cdot|\alpha|\cdot\rho(x,\mathrm{Ker}\varphi)$.
Звідси: $\frac{\varphi(x)}{\norm{\varphi} - \varepsilon} > \rho(x,\mathrm{Ker}\varphi)$
і врахувати довільність $\varepsilon > 0$.

\noindent\ref{N:1_1_22}. Задача рівносильна \ref{N:1_1_21}: якщо
$\varphi(x) = a$, то $\rho(x,\mathrm{Ker}\varphi) = \rho(0,L)$, де
$L = \set{y \mid \varphi(y) = a}$.

\noindent\ref{N:1_1_26}. Якщо ряд $\sum\limits^\infty_{k=0} A^k$ збігається,
то $\norm{A^k} \to 0$, $k \to \infty$. Якщо існує $n \in \mathbb{N}$, для
якого $\norm{A^n} < 1$, то для $m \in \set{0,1,\dots,n-1}$ ряд
$\sum\limits^\infty_{k=0} \norm{A^{m+kn}}$ збіжний, тому що $\norm{A^{m+kn}}
\leq \norm{A^m} \cdot \norm{A^n}^k$. Абсолютна збіжність вихідного ряду є
наслідком рівності $\sum\limits^\infty_{k=0} \norm{A^k} = 
\sum\limits^{n-1}_{m=0} \sum\limits^\infty_{k=0} \norm{A^{m+kn}}$.

\noindent\ref{N:1_1_27}. $\mathrm{Im}A = \text{л.о.}\{y\}$, якщо $\varphi \neq 0$;
$\mathrm{Im}A = \{0\}$, якщо $\varphi = 0$. $\mathrm{Ker}A = X$, якщо $y = 0$;
$\mathrm{Ker}A = \mathrm{Ker}\varphi$, якщо $y \neq 0$.
$\norm{Ax} = |\varphi(x)|\cdot \norm{y} \leq \norm{\varphi}\cdot\norm{y}\cdot\norm{x}$.
Тому $\norm{A} \leq \norm{\varphi}\cdot\norm{y}$. Нехай $\varphi \neq 0$, $y \neq 0$.
Беремо $\varepsilon > 0$, $\exists\; x \in X$: $|\varphi(x)| > (\norm{\varphi}-\varepsilon)
\norm{x}$. Тому $\norm{Ax} > (\norm{\varphi}-\varepsilon)\cdot\norm{y}\cdot\norm{x}$.
Тому $\norm{A} \geq (\norm{\varphi}-\varepsilon)\cdot\norm{y}$ і врахуйте довільність 
$\varepsilon>0$.

\noindent\ref{N:1_1_30}. а) Відображення $A: \ell_\infty \to \ell_1^*$
будуємо наступним чином: $\vec{c} = (c_1,c_2,\dots) \in \ell_\infty$,
$A\vec{c}: \ell_1 \ni \vec{x} \mapsto \sum\limits^\infty_{k=1} c_k x_k$.
При цьому ряд $\sum\limits^\infty_{k=1} |c_k x_k|$ --- збіжний, тому що послідовність
$\{c_k\}$ обмежена, а ряд $\sum\limits^\infty_{k=1} |x_k|$ --- збіжний.
$A\vec{c}$ --- лінійний функціонал на $\ell_1$. Його обмеженість --- наслідок нерівності
$|\sum\limits^\infty_{k=1} c_k x_k| \leq \sup\{|c_k|\} \cdot \sum\limits^\infty_{k=1} |x_k|$.
При цьому $\norm{A \vec{c}} \leq \norm{\vec{c}} = \sup\{|c_k|\}$. Насправді 
$\norm{A \vec{c}} = \norm{\vec{c}}$, тому що для $\forall k: |(A \vec{c})(\vec{e}_k)| = |c_k|$,
де $\vec{e}_k = (\underbrace{0,0,\dots}_{k-1},1,0,\dots) \in \ell_1$.
Наступний крок --- це лінійність відображення $A$ (перевірте!). Залишилось довести
$\mathrm{Im}A = \ell_1^*$. Беремо $\varphi\in \ell_1^*$. Будуємо послідовність $\{c_k\}$
за правилом $c_k = \varphi(\vec{e}_k)$. Тоді $\{c_k\}$ --- обмежена послідовність, 
$\vec{c} \in \ell_\infty$. При цьому $\varphi(\vec{x}) = \sum\limits^\infty_{k=1} c_k x_k$.
Остання формула --- наслідок граничного переходу $\varphi(\vec{x}) = 
\underset{n\to \infty}{\lim} \varphi\left(\sum\limits^n_{k=1} x_k \vec{e}_k\right)$.
Тож $\varphi = A \vec{c}$.

\noindent б) Позначимо $e_n = (\underbrace{0,\dots, 0}_{n-1},1,0,\dots) \in c_0$.
Нехай $\varphi \in c_0^*$; $a_n \coloneqq \varphi(e_n)$. Доведіть збіжність ряду $\suml{n=1}{\infty}|a_n|$.
Одержимо відображення $A\!: c_0^* \ni \varphi \mapsto (\varphi(e_1),\varphi(e_2),\dots) \!\in \ell_1$. % тут пара мануальных фиксов для предотвражения оверфулла
При цьому $\suml{n=1}{m}|a_n| = \varphi\left(\suml{n=1}{m}(\pm e_n)\right) \leq \norm{\varphi}$;
$\norm{A(\varphi)}_{\ell_1} \leq \norm{\varphi}$. Очевидно: $A$ --- лінійний оператор,
$(x\in c_0; \norm{x}\leq 1, \varepsilon >0) \Rightarrow (\exists y \in c_0; y= \suml{k=1}{m}\alpha_k e_k,
|\alpha_k| \leq 1, \norm{x-y}\leq \varepsilon)$.
$|\varphi(x)| \leq \varepsilon\norm{\varphi} + \suml{k=1}{m}|\alpha_k| |\varphi(e_k)| \leq
\varepsilon \norm{\varphi} + \norm{A\varphi}_{\ell_1}$.
Тож $\norm{A\varphi}_{\ell_1} = \norm{\varphi}$. Перевіримо: $\mathrm{Im}A = \ell_1$.
Нехай $(a_1, a_2, \dots) \in \ell_1$. Перевірте, що $c_0 \ni x = (x_1, x_2, \dots) \mapsto
\suml{n=1}{\infty}a_n x_n \in \real$ --- коректно визначений обмежений функціонал на $c_0$.
Тоді $A: c_0^* \to \ell_1$ ізоморфізм лінійних просторів із збереженням норми.

\noindent\ref{N:1_1_31}. Нехай $A = \{\varphi_n\}$ --- щільна злічена множина в $X^*$.
Вектори $x_n$ в $X$ вибрано з умови: $\norm{x_n} = 1$;
$|\varphi_n(x_n)| \geq \frac{1}{2}\norm{\varphi_n}$.
Доведемо, що множина $D = \left\lbrace \suml{k=1}{m} \alpha_k x_k \mid 
\alpha_k \in \rational, m \in \natur\right\rbrace$ щільна в $X$.
$L = \overline{D}$ --- замкнений підпростір в $X$. Якщо $L \neq X$, то $\exists \varphi \in X^*$,
$\varphi \neq 0$, для якого $L \subset \mathrm{Ker}\varphi$ (див \ref{N:1_1_47}).
Беремо $\varepsilon > 0$ і нехай $\varphi_n \in A$, такий, що $\norm{\varphi_n - \varphi} < \varepsilon$.
Тоді $|(\varphi-\varphi_n)(x_n)| = |\varphi_n(x_n)| \geq \frac{1}{2} \norm{\varphi_n}$.
Тому $\norm{\varphi-\varphi_n} \geq |(\varphi-\varphi_n)(x_n)| = |\varphi_n(x_n)| \geq
\frac{1}{2}\norm{\varphi_n}$;\; $\norm{\varphi_n} <2\varepsilon$;
$\norm{\varphi} \leq \norm{\varphi_n} + \norm{\varphi - \varphi_n} < 3 \varepsilon$ і,
враховуючи довільність $\varepsilon > 0$, $\varphi = 0$, що суперечить вибору $\varphi$.

\noindent\ref{N:1_1_36}. $\sup\{\mathfrak{Re}\varphi(x) \mid \norm{x} = 1\} \leq
\sup\{ |\varphi(x)| \mid \norm{x}=1 \} = \norm{\varphi}$.
З іншого боку, $\left(\varphi(x) = |\varphi(x)| e^{i \alpha}\right) \Rightarrow
\left( \norm{e^{-i \alpha}x} = \norm{x}; \varphi(e^{-i \alpha} x) = |\varphi(x| \right)$.
Звідси зробіть висновок про протилежну нерівність.

\noindent\ref{N:1_1_42}. а) $\norm{\tilde{x}} = \sup\left\{|\varphi(x)| \mid
\varphi\in X^*; \norm{\varphi}=1\right\} \leq \norm{x}$. За теоремою Гана-Банаха 
$\exists\varphi\in X^*$, $\norm{\varphi} = 1$, для якого досягається рівність
$|\varphi(x)| = \norm{x}$.

\noindent\ref{N:1_1_44}. Для вимірної підмножини $Z \subset [0; 1]$ розглянути оператор
$A_Z: f \mapsto f \cdot j_Z$ ($j_A$ --- індикатор множини $A$). У разі, якщо $\mu(Z) \neq 0$, 
$\norm{A_Z} = 1$ і, якщо $\mu (Z_1 \bigtriangleup Z_2) > 0$, $\norm{A_{Z_1} - A_{Z_2}} = 1$.
Розглянути $\{A_{Z_t}\}$, де $Z_t = [0, t]$.

\noindent\ref{N:1_1_45}. Умова $AB = I_Y$ гарантує, що $\mathrm{Im} A = Y$; умова $BA = I_X$
гарантує, що $\rm KerA = \{0\}$. Тому існує обернений оператор $A^{-1}$, $B = A^{-1}AB = A^{-1}$; $C = CAA^{-1} = A^{-1}$.

\noindent\ref{N:1_1_46}. Розглянути на $\ell_2$ функціонали $\varphi_n (\vec{\beta}) = \sum\limits_{k = 1}^{n} \alpha_k \beta_k$
і застосувати теорему Банаха-Штейнгауза.

\noindent\ref{N:1_1_47}. Відношення еквівалентності: $(x \sim y) \Leftrightarrow (x - y \in L)$
приводить до фактор-множини $\faktor{X}{_L} \coloneqq \faktor{X}{_\sim}$, на якій коректно запроваджувати
лінійні операції: $[x] + [y] \coloneqq [x+y]$; $\lambda [x] = [\lambda x]$ ( тут $[x]$ --- клас еквівалентності $x$)
В фактор-просторі $\faktor{X}{_L}$ коректно запроваджується норма $\norm{[x]} = \inf\set{\norm{x} \mid x \in [x]}$
(перевірте аксіоми норми!). Розмірність $\faktor{X}{_L} = \mathrm{codim} L \geq 1$. Тому за теоремою
Гана-Банаха на $\faktor{X}{_L}$ існує обмежений ненульовий лінійний функціонал $\alpha$.
На $X$ функціонал будуємо за правилом $\varphi(x) \coloneqq \lambda([x])$.
$(x \in L) \Rightarrow ([x] = [0]) \Rightarrow (\varphi(x) = \alpha([0]) = 0)$.
Тому $\varphi$ --- коректно визначено $(x-y \in L) \Rightarrow (\varphi(x) = \varphi(y))$,
$\varphi \in X^*$, $\norm{\varphi} = \alpha$.

\noindent\ref{N:1_1_48}. При кожному фіксованому $x: \varphi_x(y) = \varphi(x, y)$ --- лінійний обмежений функціонал на $Y$. Тому
$|\varphi(x, y)| \leq \norm{\varphi_x} \cdot \norm{y}$. Далі для всіх $y$, що задовольняють нерівність $\norm{y} \leq 1$
розглянемо (обмежені) функціонали $\psi_y (x) = \varphi(x, y)$ на просторі $X$. $\forall x \in X$ значення цієї
сім'ї функціоналів: $\{\psi_y (x) | \norm{y} \leq 1\}$ є обмеженою множиною $(|\psi_y(x)| \leq \norm{\varphi_x})$ і тому за теоремою
Банаха-Штейнгауза: $|\varphi(x, y)| = |\psi_y(x)| \leq K\norm{x}$. Якщо тепер відмовитись від обмеженості $\{\norm{y}\}$, то приходимо до
шуканої нерівності.

\noindent\ref{N:1_2_8}. Нехай $A: H_1 \to H_2$ --- ізоморфізм і $H_1$ --- сепарабельний
гільбертів простір. Доведемо сепарабельність простору $H_2$. Нехай система $\{e_n\}$ ---
ортонормована в $H_1$. Тоді система $\{A e_n\}$ --- ортонормована в $H_2$. Оскільки
лінійна оболонка $\{e_n\}$ щільна в $H_1$, а оператор $A$ --- обмежений і $\mathrm{Im}A = H_2$,
то вектори $f_n = A e_n$ утворюють ортонормований базис в $H_2$. Тому в $H_2$ щільна
множина лінійних комбінацій векторів $f_n$ з раціональними (у випадку поля $\mathbb{R}$)
коефіцієнтами. У випадку поля $\mathbb{C}$ слід брати комплексні коефіцієнти
$z_k = u_k + i v_k$, $u_k, v_k \in \mathbb{Q}$.

\noindent\ref{N:1_2_14}. в) Відповідь: $M^{\perp} = \overline{\{1; \cos nt, n \in \mathbb{N} \}}$.

\noindent\ref{N:1_2_16}. а) $\Rightarrow$ б): 
$(x \in H_2^{\perp}) \iff (\norm{x_2}^2 = (x, x_2) = 0) \iff (x \in H_1)$;

\noindent б) $\Rightarrow$ a): $x - pr_{H_2} x \in H_2^{\perp} = H_1$.

\noindent\ref{N:1_2_18}. ґ) $(x \in M^{\perp}+N^{\perp}) \Rightarrow (x = x_1+x_2; x_1 \perp M; x_2 \perp N)$
$\Rightarrow (x \in (M \cap N)^{\perp})$. Тому $\overline{M^{\perp} + N^{\perp}} \subset (M \cap N)^{\perp}$.
$L = M \cap N$. $M = L \oplus  M_1; N = L \oplus N_1 (M_1 = L^{\perp} \cap M; N_1 = L^{\perp} \cap N); M_1 \cap N_1 = \{0\}$.
Тому $L^{\perp} = M_1 \oplus M^{\perp} = N_1 \oplus N^{\perp}$. 
$M_1 = M_1 \cap L^{\perp} = M_1 \cap (N_1 \oplus N^{\perp}) = M_1 \cap N^{\perp}$. 
Тому $L^{\perp} = (M_1 \cap N^{\perp}) \oplus M^{\perp} \subset \overline{N^{\perp} + M^{\perp}}$.

\noindent\ref{N:1_2_21}. $M = \left\{(1+\frac{1}{n})\vec{e_n}\right\}$ 
(тут $\vec{e_n} = (\underbrace{0, \dots 0}_{n-1}, 1, 0, \dots)$).

\noindent\ref{N:1_2_24}. Нехай $d = \rho(x, M)$; $\rho(x, y_1) = d + \varepsilon_1$; $\rho(x, y_2) = d + \varepsilon_2$;
$\varepsilon_2 \in (0, \varepsilon_1)$. Оскільки $\rho(x, \frac{y_1+y_2}{2}) \geq d$, то з
рівності паралелограма для векторів $y_1 - x$ та $y_2 - x$ одержимо нерівність: 
$\rho(y_1, y_2) \leq 2 \sqrt{2d\varepsilon_1 + \varepsilon_1^2}$. Тому існує послідовність додатних чисел
$\varepsilon_n \searrow 0$, для якої $\rho(y_n, y_{n+1}) \leq \frac{1}{2} \rho(y_{n-1}, y_n)$.
Послідовність $\{y_n\}$ є фундаментальною.

\noindent\ref{N:1_2_26}. б) $\Rightarrow$ в): застосувати теорему Банаха-Штейнгауза до послідовності функціоналів
$\varphi_n (y) = \left(y, \sum\limits_{k = 1}^n x_k\right)$.

\noindent\ref{N:1_2_27}. Якщо $\inf \lambda_k > 0$, то нова норма еквівалентна стандартній в $\ell_2$.
Якщо $\inf \lambda_k = 0$, то в разі повноти простору для функціоналів 
$\varphi_k (y) = (\frac{1}{\lambda_k} \vec{e_k}, \vec{y})$ була б суперечність із теоремою Банаха-Штейнгауза.

\noindent\ref{N:1_2_32}. Якщо послідовність функцій $\{f_n\}$ має обмежені у сукупності похідні, то ця
система функцій одностайно неперервна. Крім того, умови 
$\underset{n}{\sup} \{|f_n (x) - f_n (y)|\mid n \in \mathbb{N}; x, y \in [0, 2\pi]\} < \infty$ та 
$\int\limits_0^{2\pi} f_n^2 (x) dx = 1 \; \forall \;n \in \mathbb{N}$ разом гарантують рівномірну обмеженість системи функцій.
За теоремою Арцела-Асколі: $\{f_n\}$ --- передкомпакт в $C[0; 2\pi]$, а тому й в $L_2 [0; 2\pi]$, що
суперечить її ортонормованості.

\noindent\ref{N:1_3_14}. б) Скористаємось тим, що $A^* A$ --- самоспряжений оператор та
результатом задачі \hyperref[N:1_3_12_h]{\ref*{N:1_3_12} \ref*{N:1_3_12_h}}. $\norm{A^2}^2 = \norm{(A^2)^*A^2} = 
\norm{(A^* A)^2} = \norm{A^* A}^2 = \norm{A}^4$.

\noindent\ref{N:1_3_18}. в) $(P_1 - P_2 \text{ --- ортопроектор}) \Leftrightarrow
\left( \dotprod{(P_1 - P_2)x}{x} = \norm{(P_1 - P_2)x}^2 \geq 0 \right)$;\\
$(P_1 \geq P_2) \Rightarrow \Big( 
(x \in H_2) \Rightarrow \left( \norm{x}^2 = \dotprod{P_2 x}{x} \leq \dotprod{P_1 x}{x} = \norm{P_1 x}^2\right)
\Rightarrow (x \in H_1)\Big) \Rightarrow (H_2 \subset H_1)$;
$(H_2 \subset H_1) \Rightarrow (H_2 \perp (H_1 \ominus H_2)).$
Нехай $H_3 \coloneqq H_1 \ominus H_2$, $P_3$ --- ортопроектор на $H_3$.
Скористаємось \hyperref[N:1_3_18_a]{\ref*{N:1_3_18} \ref*{N:1_3_18_a}}: $P_2 + P_3 = P_1$,
а тому $P_1 - P_2 = P_3$;

\noindent г) Якщо $P_1 P_2 = P_2$, то $P_2 P_1 = (P_1 P_2)^* = P_2$. Тому
$(P_1 - P_2)^2 = P_1 - P_2$ та $P_1 \geq P_2$. Якщо $P_1 \geq P_2$, то $H_2 \subset H_1$,
а тому $\forall x \in H$: $P_1 P_2 x = P_2 x$.

\noindent\ref{N:1_3_20}. a) Нехай  $\dim H_1 > \dim H_2$, $P_2$ --- ортопроектор на $H_2$.
$P_2 \big|_{H_1} : H_1 \to H_2$, $\mathrm{Ker}\left(P_2 \big|_{H_1}\right) \neq \{0\}$.
Тож існує $x \neq 0$: $x \in H_1$, $P_2 x = 0$. Тому $(P_1 - P_2)x = x$ і $\norm{P_1 - P_2} \geq 1$.

\noindent б) $H \in \real^3$; $H_1 = \text{л.о.}\{\vec\imath,\vec\jmath\}$; $H_2 = \{\vec k\}$.
$P_1 - P_2: a\vec\imath + b\vec\jmath + c\vec k \mapsto a\vec\imath + b\vec\jmath - c\vec k$.
$\norm{(P_1 - P_2)\vec x} = \norm{\vec x}$.

\noindent\ref{N:1_3_22}. б) $((I + A^* A)x = 0) \Rightarrow \left(0 = \dotprod{(I + A^* A)x}{x}
= \norm{x}^2 + \norm{Ax}^2\right) \Rightarrow (x=0) \Rightarrow (\mathrm{Ker}(I + A^* A) = \{0\})$.
Оскільки $B = I + A^* A$ --- самоспряжений, то $\overline{\mathrm{Im}B} =
(\mathrm{Ker}B)^\perp = H$. Замкненість $\mathrm{Im}B$ та обмеженість $B^{-1}$ є наслідком
нерівності $\norm{Bx} \geq \norm{x}$ (обміркуйте!). 

\noindent\ref{N:1_3_23}. а) $\Rightarrow$ б): $\left(\exists \;A^{-1} \in L(H)\right) \Rightarrow
\left(\exists\;(A^*)^{-1} = (A^{-1})^* \in L(H)\right)$;
$\exists \;m>0$: $\norm{Ax} \geq m\norm{x}$. Тому $\dotprod{A^* A x}{x} = \norm{Ax}^2 \geq
m^2 \norm{x}^2$, $\beta = m^2$;

\noindent б) $\Rightarrow$ а): $(A^* A \geq \beta I) \Rightarrow (\norm{Ax} \geq 
\sqrt{\beta}\norm{x})$. Аналогічно $\norm{A^* x} \geq \sqrt{\alpha}\norm{x}$ $(\forall x \in H)$.
Тому $\mathrm{Ker}A = \mathrm{Ker}A^* = \{0\}$, $\mathrm{Im}A = \mathrm{Im}A^* = H$ та
$A^{-1}, (A^*)^{-1} \in L(H)$.

\noindent\ref{N:1_3_25}. Нехай $P_x$ --- ортопроектор на л.о.$\{x\}$. Тоді $(P_x Ax = 
AP_x x = Ax) \Rightarrow (Ax = \lambda(x)x)$. Нехай $L = \text{л.о.}\{x,y\}$;
$Bx = y$, $By = x$ та $Bz = z$ для $z \in L^\perp$. $BAx = \lambda(x)y; ABx = Ay = \lambda(y)y$.
Тож $\lambda(x) = \lambda(y)$.

\noindent\ref{N:1_3_33}. $A^* x = \dotprod{x}{z}y$. Тепер скористайтесь задачею \ref{N:1_1_28}
та теоремою Фреше-Рісса.

\noindent\ref{N:1_3_39}. Нехай $\varphi_y: x \mapsto \dotprod{Ax}{y} = \dotprod{x}{Ay}$.
$\varphi_y \in H^*$ і при цьому $\norm{\varphi_y} = \norm{Ay}$. Якщо 
$\underset{\norm{y}=1}{\sup} \norm{\varphi_y} = \infty$, то існують $y_n$ такі,
що $\norm{y_n} = 1$, $\norm{\varphi_{y_n}} \to \infty$.
Але $\forall x \in H$: $|\varphi_{y_n}(x)| \leq \norm{Ax}$ і треба скористатися теоремою Банаха-Штейнгауза.

\noindent\ref{N:1_4_13}. в) $\Rightarrow$ а). $x_n \rightarrow 0$. Нехай $A x_n \not\rightarrow 0$. Тоді існує
$x_{n_k} \rightarrow 0$ така, що $\norm{A x_{n_k}} \geq \delta > 0$. Тоді 
$y_k = \frac{1}{\sqrt{\norm{x_{n_k}}}} x_{n_k} \rightarrow 0$, але $\{A y_k\}$ --- необмежена.

\noindent\ref{N:1_4_14}. Використовуємо \ref{N:1_4_13}. $(x_n \xrightarrow[\text{сл}]{} x) \Rightarrow (Ax_n \xrightarrow[\text{сл}]{} Ax)$.

\noindent\ref{N:1_4_15}. а) \ul{в)$\Rightarrow$а)} Скористаємось ізометричним вкладенням $Y \subset 
Y^{**}$. Якщо $x_n \rightarrow 0$, але $Ax_n \nrightarrow 0$, то $\exists \delta > 0$ та 
$\{x_{n_k}\}$ такі, що $\norm{Ax_{n_k}} \geq \delta$. Тоді $z_k = 
\frac{x_{n_k}}{\norm{x_{n_k}}} \rightarrow 0$, але послідовність $y_k = Az_k$ --- необмежена 
в $Y^{**}$, при тому, що за умовою в) $\{y_k\}$ --- слабко збіжна послідовность функціоналів на 
банаховому просторі $Y^*$. Повнотою просторів $X$ та $Y$ не користуємось.

\noindent\ref{N:1_4_15}. б) Оскільки для $\forall \varphi \in Y^*$ для $x_1, x_2 \in X$ 
виконується рівність: $\varphi(A(x_1 + x_2)) = \varphi(Ax_1 + Ax_2)$, то, застосовуючи Гана-Банаха, 
$A(x_1 + x_2) = Ax_1 + Ax_2$. Аналогічно $A(\lambda x) = \lambda Ax$. Обмеженість оператора $A$ 
довести з використанням імплікації \ul{в) $\Rightarrow$ а)} пункту а).

\noindent\ref{N:1_4_16}. б) 
$\left(x_n \xrightarrow[\text{сл}]{} x; \norm{x_n} \rightarrow \norm{x}\right) \Rightarrow (\norm{x_n - x}^2 = \norm{x_n}^2 + \norm{x}^2 - (x_n, x) - (x, x_n) \\ \rightarrow 0)$.
$\left(x_n \xrightarrow[\text{сл}]{} x; \overline{\lim\limits_{n \rightarrow \infty}} \norm{x_n} \leq \norm{x}\right)$
$\Rightarrow \left(\overline{\lim\limits_{n \rightarrow 0}} \norm{x_n - x}^2 = \overline{\lim\limits_{n \rightarrow 0}} (\norm{x_n}^2 - \norm{x}^2) \leq 0\right)$
$\Rightarrow (\norm{x_n - x} \rightarrow 0)$.

\noindent\ref{N:1_4_20}. в) Нехай $\varphi(y) = \lim\limits_{n \rightarrow \infty} \dotprod{y}{x_n}$. 
$\varphi \in H^*$(теорема Банаха-Штейнгауза). Далі застосувати теорему Рісса.

\noindent\ref{N:1_4_22}. Позначимо через $A: X \rightarrow Y$ відображення $Ax = \lim\limits_{n \rightarrow \infty} A_n x$.
$A(\alpha x + \beta y) = \lim\limits_{n \rightarrow \infty} A_n(\alpha x + \beta y) = \alpha Ax + \beta Ay$.
$\exists C > 0: \forall n: \norm{A_n} \leq C$ (за теоремою Банаха-Штейнгауза). Тому 
$\norm{Ax} = \lim\limits_{n \rightarrow \infty} \norm{A_n x} \leq C\norm{x}$. Тож $A \in L(X, Y)$.

\noindent\ref{N:1_4_28}. Достатньо довести сильну фундаментальність послідовності $\{A_n\}$. Використовуємо задачу \ref{N:1_3_21} б):

$(m \geq n) \Rightarrow (B = A_m - A_n \geq 0) \Rightarrow \left(\norm{A_m x - A_n x}^2 \leq 2C((A_m x, x) - (A_n x, x))\right)$ та
збіжність $\{(A_n x, x)\}$ за класичною теоремою Вейєрштрасса. 

\noindent\ref{N:1_5_14}. $\exists \Delta = [\alpha,\beta] \subset [a,b]$ такий, що
$(t \in [\alpha,\beta]) \Rightarrow (\inf|f(t)| > 0)$. Нехай $|f| \Big|_\Delta
\geq \delta > 0$. Тоді оператор $\tilde{A}: C(\Delta) \to C(\Delta)$, що визначений
формулою: $C(\Delta) \ni x(\cdot) \mapsto f \Big|_\Delta \cdot x(\cdot) \in C(\Delta)$
не є компактним (обміркуйте) (образ кулі містить кулю). Розглянемо оператори 
$B: C[a,b] \ni x \mapsto x \Big|_\Delta \in C(\Delta)$ та $\tilde{B}: C(\Delta) \to C[a,b]$,
що будується так: $(\tilde{B}x)(t) = x(t)$, якщо $t\in \Delta$; $(\tilde{B}x)(t) = x(\beta)$,
для $t\geq \beta$; $(\tilde{B}x)(t) = x(\alpha)$, для $t\leq \alpha$. Тоді
$B \in L(C[a,b], C(\Delta))$; $\tilde{B} \in L(C(\Delta), C[a,b])$; $\tilde{A} = B \cdot A
\cdot \tilde{B}$ і припущення про компактність $A$ приводить до суперечності.

\noindent\ref{N:1_5_15}. г) $x_n(t) = t^{\frac{1}{n}}$ --- обмежена послідовність;
$(Ax_n)(t)$ не має збіжної підпослідовності.

\noindent\ref{N:1_5_17}. б) Нехай $Z$ --- обмежена множина в $L_2[0;1]$. Тоді $Z$
складається з інтегровних на $[0;1]$ функцій і обмежена в $L_1[0;1]$ (Коші-Буняковський).
$A(Z) \subset C[0;1]$; $A(Z)$ --- рівномірно обмежені за нормою $C[0;1]$ та одностайно
неперервні: $\left|(Ax)(t_1) - (Ax)(t_2)\right| \leq \int\limits^1_0 |x| \cdot j_{[t_1,t_2]} ds \leq
\sqrt{t_2 - t_1} \left(\int\limits^1_0 x^2(s) ds\right)^{\frac{1}{2}}$, (тут $t_1 < t_2$).
Тому $A(Z)$ --- передкомпакт в $C[0;1]$. Але $\varepsilon$-сітка в $A(Z)$ за нормою $C[0;1]$ 
також є $\varepsilon$-сіткою за нормою $L_2[0;1]$ $\left(\norm{x}_{L_2[0;1]} \leq 
\norm{x}_{C[0;1]} \right)$. $A$ --- компактний.

\noindent в) $A(B[0;1]) \supset \set{y(t) \cdot j_{[\frac{1}{2};1]} \mid y \in B[0; \frac{1}{2}]}$.

\noindent ґ) $(B_1 x)(t) = \int\limits^t_0 s x(s) ds$; $(B_2 x)(t) = t x(t)$.
$B_1$ --- компактний, $B_2$ --- обмежений, $A = B_2 \circ B_1$.

\noindent\ref{N:1_5_20}. Оператор вкладення $A: C^1[a; b] \rightarrow C[a; b]$ є компактним (т. Арцела-Асколі). Тому
$A\Big|_{Z}$ --- також компактний. Тож $Z$ як підпростір в $C[a; b]$ є зліченним об'єднанням куль, передкомпактних в нормі
$C[a; b]$. Оскільки $Z$ --- повний простір, а передкомпакт в нескінченновимірному просторі ніде не щільний, прийдемо до суперечності
на підставі теореми Бера.

\noindent\ref{N:1_5_23}. Нехай $\{e_n\}$ --- ортонормований базис в сепарабельному гільбертовому просторі $H$; $P_n$ --- ортопроектор
на лінійну оболонку $\{e_1, e_2, \dots, e_n\}$. Тоді $P_n \circ A \xrightarrow{s} A$. Область значень компактного оператора --- сепарабельний
підпростір. Якщо $A_n$ --- підпослідовність компактних операторів; $A_n \xrightarrow{s} A \in L(H)$, то 
$\mathrm{Im} A \subset \overline{\bigcup\limits_{n = 1}^{\infty} \mathrm{Im} A_n}$ --- сепарабельний підпростір в $H$.
Розгляньте оператор $I$ у несепарабельному просторі.

\noindent\ref{N:1_5_26}. б) $\Rightarrow$ в). Нехай $y_n \equiv y$. Тоді
$(x_n \xrightarrow[\text{сл}]{} x) \Rightarrow (Ax_n \xrightarrow[\text{сл}]{} Ax)$.
Тепер з \ref{N:1_4_13} $A\in L(H_1, H_2)$. Тепер нехай $y_n = Ax_n$. Тоді: 
$(x_n \xrightarrow[\text{сл}]{} x, Ax_n \xrightarrow[\text{сл}]{} Ax) \Rightarrow (\norm{A x_n} \rightarrow \norm{A x})$. Далі скористаємося \ref{N:1_4_16} б).
$(Ax_n \xrightarrow[\text{сл}]{} Ax; \norm{A x_n} \rightarrow \norm{A x}) \Rightarrow (A x_n \rightarrow A x)$.

\noindent в) $\Rightarrow$ б). Скористатися \ref{N:1_4_16} а).

\noindent\ref{N:1_5_31}. Без втрати загальності покладемо $A = 0$.
$(x_n \xrightarrow[\text{сл}]{} 0) \Rightarrow \left( (C x_n, x_n) \rightarrow 0 \right)$ (дивись \ref{N:1_5_26}).
$(0 \leq (B x_n , x_n) \leq (C x_n, x_n)) \Rightarrow ((B x_n, x_n) \rightarrow 0)$. Аналогічно:
$(y_n \xrightarrow[\text{сл}]{} 0) \Rightarrow ((B y_n, y_n) \rightarrow 0)$.

\noindent$(B x_n, y_n) = \frac{1}{4}\left( (B(x_n+y_n), x_n+y_n) - (B(x_n - y_n), x_n - y_n)\right)$. 

\noindentТому $(x_n \xrightarrow[\text{сл}]{} 0, y_n \xrightarrow[\text{сл}]{} 0) \Rightarrow ((B x_n, y_n) \rightarrow 0)$. Далі скористатися \ref{N:1_5_26}.

\noindent\ref{N:1_5_33}.
%\begin{minipage}{0.25\textwidth}
%    \begin{tikzpicture}[line cap=round,line join=round,>=triangle 45,x=1.0cm,y=1.0cm]
%        \clip(-2.,-2.) rectangle (2.,2.);
%        \draw [line width=.25pt] (0.,0.) circle [radius = 1.cm];
%        \draw [->,line width=.2pt] (0.,0.) -- (0.705637350341705,0.7085731647492289);
%        \draw [->,line width=.2pt] (0.705637350341705,0.7085731647492289) -- (0.24804622428767287,1.3468866592154136);
%        \begin{scriptsize}
%        \draw[color=black] (0.40565960486893554,0.27) node {$x$};
%       \draw[color=black] (0.5490845250768417,1.101728558923356) node {$z$};
%        \end{scriptsize}
%    \end{tikzpicture}
%\end{minipage}
%\begin{minipage}{0.7\textwidth}
    $y(t)$ --- крива на сфері $\{x \mid \norm{x} = 1\}; y(t_0) = x$. $\frac{d}{dt} \norm{A y(t)}^2 \Big|_{t = t_0} = 0$
    (в т. $t_0$ функція $\norm{A y(t)}$ приймає найбільше значення). Тож $\dotprod{A \dot{y} (t_0)}{A x} = 0$.
    $\dot{y} (t_0)$ --- дотичний вектор до сфери в т. $x$. Тож $\dot{y} (t_0) \perp x$. 
%\end{minipage}
Він переходить у вектор $A \dot{y} (t_0) \in \{Ax\}^{\perp}$. Залишилося зауважити, що кожний вектор $z$ з $\{x\}^{\perp}$ є дотичним вектором
гладкої кривої (кола), що є перетином сфери з л.о. $\{x, z\}$. При розв'язанні задачі використана лише обмеженість оператора $A$.

\noindent \ref{N:1_5_44}. a) $\{a_n\}$ задовольняє умову: $\exists \{b_n\}, 
\{c_n\}$ такі, що $\sum\limits_{n=1}^{\infty}|b_n|^2 < \infty$, 
$\sum\limits_{n=1}^{\infty}|c_n|^2 < \infty$ і при цьому $a_n = b_nc_n$.
Доведіть, що ця умова еквівалентна збіжності ряду $\sum\limits_{n=1}^\infty 
|a_n|$.

\noindent\ref{N:1_6_16}. в) $x(t) - t\intl{0}{1}sx(s)ds = y(t)$;\; $x(t) = y(t) +Ct$;\\
$y(t) +Ct = y(t) + t \intl{0}{1}s(y(s) + Cs)ds$;\\
$C = \intl{0}{1}sy(s)ds + C \intl{0}{1}s^2ds$;\;
$C = \frac{3}{2} \intl{0}{1}sy(s)ds$.\; Тож:
$x(t) = y(t) + \frac{3}{2}t\intl{0}{1}sy(s)ds$. 

\noindent\ref{N:1_6_17}. $x(t) + \lambda \intl{0}{1}(t+s)x(s)ds = y(t)$;\;
$a \coloneqq \intl{0}{1} x(s) ds$;\; $b \coloneqq \intl{0}{1} sx(s)ds$.\\
Тоді: $x(t) = y(t) - (at +b)\lambda$;\;
$y(t) - \lambda(at + b) + \lambda\intl{0}{1}(t+s)(y(s) - \lambda(as+b))ds = y(t)$;
$\lambda_1 = 0$; Нехай $\lambda \neq 0$
\[\begin{cases}
    - a + \intl{0}{1}(y(s) - \lambda(as + b))ds = 0\\
    - b + \intl{0}{1}s(y(s) - \lambda(as + b))ds = 0
\end{cases}\]
\[\begin{cases}
    a(1 + \frac{\lambda}{2}) + b\lambda = \intl{0}{1}y(s)ds\\
    a(\frac{\lambda}{3}) + b(1 + \frac{\lambda}{2}) = \intl{0}{1} sy(s)ds.
\end{cases}
\quad (*)\]
$\Delta = (1 + \frac{\lambda}{2})^2 - \frac{\lambda^2}{3} \neq 0$; \;
$\lambda^2 - 12\lambda -12 \neq 0$, при
$\lambda \not\in \{6 \pm 4\sqrt{3}\}$\\
Тож при $\lambda \not\in \{6 \pm 4\sqrt{3}\}$ існує неперервний обернений.
Якщо $\lambda \not\in \{0, 6 \pm 4\sqrt{3}\}$, то знаходимо $a$ та $b$ із системи $(*)$
і $x(t) = y(t) - \lambda(at+b)$.

\noindent\ref{N:1_6_22}. б) $X = L_2 [0; 1]$. Нехай $B: \int\limits_0^t x(s) ds \mapsto x(t)$. Доведіть необмеженість
$B$ за допомогою послідовності $x_n(t) = t^n$.

\noindent\ref{N:1_6_25}. Нехай $\vec{x}$ задовольняє рівняння: $(A+B) \vec{x} = \vec{e_1}$.
Тоді $B \vec{x} = (1, -x_1, -x_2, \dots)$; $\norm{B} > 1$.

\noindent\ref{N:1_6_28}. З умов задачі: $B$ --- неперервна лінійна бієкція $Y$ на $Z$. Тому
$B^{-1} \in L(Z, Y)$. $A = B^{-1} \circ (BA)$.

\noindent\ref{N:1_6_36}. Скористаємося результатом задачі \ref{N:1_6_6}. Позначимо: $C_n = A_n - A$.
Тоді $A_n^{-1} = (I+A^{-1}C_n)^{-1} A^{-1}$, якщо $\norm{C_n} \leq \frac{1}{2} \norm{A^{-1}}^{-1}$, і при цьому має місце
оцінка: $\norm{A_n^{-1}} < 2\norm{A^{-1}}$.
Достатність умови. Якщо $A_n$ --- неперервно оборотний, то $(\norm{C_n}<\frac{1}{M}) \Rightarrow (\norm{C_n} < \norm{A_n^{-1}}^{-1}) \Rightarrow (A$
 --- неперервно оборотний) (тут $M = \sup\limits_{n \geq N} \norm{A_n^{-1}}$).

\noindent\ref{N:1_7_9}. Оператор $A$ компактний. Якщо $\dim H \geq 2$, то $\mathrm{Ker} A = \{0\}$ і $\sigma (A) = \sigma_T (A)$.
 Власні вектори $x$ --- або ортогональні $x_0$ або колінеарні $x_1$. Тому 
 $(\dotprod{x_0}{x_1} = 0) \Rightarrow (\sigma (A) = \{0\})$; 
 $(\dotprod{x_0}{x_1} \neq 0) \Rightarrow (\sigma(A) = \{0; \dotprod{x_1}{x_0}\})$. $r(A) = |\dotprod{x_1}{x_0}|$. 
 Резольвенту $R_{\lambda} (A)$ при $\lambda \notin \{0; (x_1, x_02)\}$ шукаємо як розв'язок рівняння: $\lambda y - \dotprod{y}{x_0} x_1 = x$.
 Покладемо: $x = x' + x''$, де $x' \in$ л. о. $\{x_1\}$; $x'' \perp x_1$; $y = \alpha x_1 + y''$, де $y'' \perp x_{1}$;
 $x'' = x - \dotprod{x}{x_{1}}x_{1}$. Маємо: $y'' = \frac{1}{\lambda} x''$; 
 $\lambda \alpha x_1 - \alpha \dotprod{x_1}{x_0} x_1 - \frac{1}{\lambda}\dotprod{x''}{x_0}x_1 = x' = (x, x_1)x_1$.
 $\alpha(\lambda - (x_1, x_0))x_1 = \frac{1}{\lambda}(x'', x_0)x_1 + (x, x_1)x_1$.
 Якщо $x_1 \neq 0$, то $\alpha = \frac{1}{\lambda - (x_1, x_0)}\left(\frac{1}{\lambda}(x'', x_0)+(x, x_1)\right)$.
 
 Тому $R_{\lambda} (A): x \mapsto \left(\frac{1}{\lambda(\lambda - (x_1, x_0))}(x, x_0)+\frac{\lambda - \norm{x_0}^2}{\lambda}(x, x_1)\right)x_1 + \frac{1}{\lambda}(x - (x, x_1)x_1)$.
 
\noindent\ref{N:1_7_14}. а) Якщо $\lambda \notin [0; 1]$, то рівняння $(\lambda - t)y(t) = x(t)$ має в $L_2 [0; 1]$
 (єдиний) розв'язок: $y(t) = \frac{1}{\lambda - t}x(t)$, тому що $\frac{1}{\lambda - t}$ --- обмежена функція на $[0; 1]$.
 Якщо $\lambda \in [0; 1]$, то рівняння $(\lambda - t)y(t) = \frac{1}{(\lambda - t)^{\frac{1}{3}}} \in L_2[0; 1]$ не має розв'язку в $L_2[0,1]$,
 тому що $y(t) = \frac{1}{(\lambda - t)^{\frac{4}{3}}}$ майже скрізь.
 $\{(\lambda - t)x(t) \big| x(\cdot) \in L_2 [0;1]\}$ --- щільний підпростір в $L_2 [0; 1]$, оскільки він містить всі неперервні функції
 $y \in C[0; 1]$, для яких $y(\lambda) = 0$ та існує $y'(\lambda)$. Тож:
 $\sigma_{\text{н}} (A) = \sigma (A) = [0; 1]$; $r(A) = 1$; $(R_{\lambda} (A)x)(t) = \frac{1}{\lambda - t}x(t), (\lambda \notin [0;1])$.

\noindent в) Оскільки $A$ --- компактний оператор; $\dim L_2  [0; 1] = \infty$, то $\sigma(A) \ni 0$ і всі ненульові точки $\sigma(A)$ --- 
власні числа. Для $\lambda \neq 0$ розглянемо рівняння $(\lambda x)(t) = \int\limits_0^t x(s) ds$. Оскільки $x \in L_2 [0;1]$, то $\int\limits_0^t x(s) ds$ ---
неперервна функція від $t$. Тому $x \in C[0;1]$, а отже $\int\limits_0^t x(s) ds$ --- неперервно диференційовна.
Позначимо: $z(t) = \int\limits_0^t x(s) ds$. $\lambda z' - z = 0$; $z(t) = C \mathrm{e}^{\frac{1}{\lambda}t}$. Але $z(0) = 0$. Тож $C = 0$.
$\sigma(A) = \{0\}$. $\mathrm{Im} A$ --- щільний в $L_2 [0;1]$, тому що $C[0;1] \subset \mathrm{Im}A$. $\sigma_{\text{н}} (A) = \sigma(A) = \{0\}$.
$R_{\lambda} (A)$ знаходимо з рівняння: $(\lambda y)(t) - \int\limits_0^t x(s) ds = x(t) (\lambda \neq 0)$. Спочатку робимо <<халтуру>>: нехай $x(\cdot)$ та
$y(\cdot)$ диференційовні функції. Тоді заміною $z(t) = \int\limits_0^t x(s) ds$ приводимо рівняння до такого: $\lambda z' - z = x$; 
$z(t) = C\mathrm{e}^{-\frac{1}{\lambda} t} + \frac{1}{\lambda} \int\limits_0^t \mathrm{e}^{\frac{1}{\lambda}(t-s)} x(s) ds$;
$z(0) = 0$. Тому $z(t) = \frac{1}{\lambda} \int\limits_0^t \mathrm{e}^{\frac{1}{\lambda}(t-s)} x(s) ds$; 
$y(t) = z'(t) = \frac{1}{\lambda}x(t) + \frac{1}{\lambda^2} \int\limits_0^t \mathrm{e}^{\frac{1}{\lambda}(t-s)} x(s) ds$.
Інтеграл має сенс і для $x \in L_2 [0;1]$. Перевіримо, що $y$ --- шуканий розв'язок.
$(\lambda y)(t) - \int\limits_0^t y(s) ds = x(t) + \frac{1}{\lambda} \int\limits_0^t \mathrm{e}^{\frac{1}{\lambda}(t-s)} x(s) ds - \frac{1}{\lambda} \int\limits_0^t x(s) ds - $
$\frac{1}{\lambda^2} \int\limits_0^t ds \int\limits_0^s \mathrm{e}^{\frac{1}{\lambda}(s-\tau)} x(\tau) d\tau = x(t)$, тому що:
$\int\limits_0^t ds \int\limits_0^s \mathrm{e}^{\frac{1}{\lambda}(s-\tau)} x(\tau) d\tau = \int_0^t d\tau \int_{\tau}^t \mathrm{e}^{\frac{1}{\lambda}(s - \tau)}x(\tau)ds = \lambda \int\limits_0^t (\mathrm{e}^{\frac{t - \tau}{\lambda}} - 1)x(\tau)d\tau$.
\underline{Відповідь}: $(R_{\lambda} x)(t) = \frac{1}{\lambda}x(t)+\frac{1}{\lambda^2}\int_0^t \mathrm{e}^{\frac{t-s}{\lambda}}x(s)ds$.

\noindent\ref{N:1_7_25}. Скористайтесь задачею \ref{N:1_6_6}.

\noindent\ref{N:1_7_27}. а) безпосередній наслідок результату задачі \ref{N:1_6_6}.

\noindent б) Нехай $\varepsilon > 0$ і $F = \complex \setminus (\sigma(A))_{\varepsilon}$. $F$ --- замкнена множина в $\complex$.
$\lambda \mapsto R_{\lambda} (A)$ --- неперервна функція на $F$ та $\lim\limits_{\lambda \rightarrow \infty} R_{\lambda} (A) = 0$.
Тому $\inf\limits_F \{ \norm{R_{\lambda} (A)}^{-1} \} > 0$ і $\exists N, \forall n \geq N: \rho (A_n) \subset F$.

\noindent\ref{N:1_7_28}. $0 \in \rho(A); \norm{A^{-1}} = 1$. Тому $(|\lambda| < 1) \Rightarrow (|\lambda - 0| < \norm{R(0; A)}^{-1}) \Rightarrow (\lambda \in \rho(A))$.

\noindent\ref{N:1_7_30}. б) Припустимо спочатку, що та $M > |m|$. Тоді $M = \sup\limits_{\norm{x} = 1} \dotprod{Ax}{x} = \sup\limits_{\norm{x} = 1} |\dotprod{Ax}{x}| = \norm{A} = r(A)$.
Оскільки $\sigma(A)$ --- замкнена множина в $\mathbb{R}$, то $M \in \sigma(A)$.
Всі інші варіанти розташування $m$ і $M$ в $\mathbb{R}$ паралельним перенесенням зводяться до розглянутого і при цьому
$(\lambda \in \sigma(A)) \Leftrightarrow (\lambda + c \in \sigma (cI + A))$. Перетворення $A \mapsto B = -A$ переводять $m_A$ в 
$M_B$ і $(\lambda \in \sigma(B)) \Leftrightarrow (-\lambda \in \sigma(A))$.

\noindent\ref{N:1_7_33}. $\lambda \neq 0; A A^* x = \lambda x (x \neq 0)$. Тоді
$
    \begin{cases}
        A^* A (A^* x) = \lambda A^* x
        \\A^* x \neq 0,
    \end{cases}
$, $\lambda \in \sigma(A^* A)$. 

\noindent $\lambda = 0$. Контрприклад: $A$ --- оператор правого зсуву в $\ell_2$.

\noindent\ref{N:1_8_6}. $\sigma (A) = \{0\}$; в $X = C[0; 1]: 0 \in \sigma_{\text{з}}(A)$, тому що 
$\mathrm{Ker} A = \{0\}$ та $\mathrm{Im} A \subset \{x \in X \big| x(0) = 0\}$;
в $X = L_2 [0; 1]: 0 \in \sigma_{\text{н}} (A)$, тому що $(\int\limits_0^t x(s) ds = 0 \text{ в } L_2[0; 1]) \Rightarrow$
$(\int\limits_0^t x(s) ds = 0, \forall t \in [0, 1]$, оскільки функція $\int\limits_0^t x(s) ds$ неперервна по $t) \Rightarrow$
$(\forall $ числового проміжку $\Delta \subset [0, t]: \int\limits_{\Delta} x(s)ds = 0) \Rightarrow$
$(\forall $ лебегової множини $A \subset [0, 1): \int\limits_A x(s) ds = 0) \Rightarrow$
$(x(\cdot) = \frac{d(0)}{d \lambda_1} = 0$ (м. с.)).

\noindent Інший шлях: $y(t) = \int\limits_0^t x(s) ds$ --- абсолютно неперервна, а тому $x(t) = y'(t)$ (м. с.).
Крім того, $\mathrm{Im} A$ містить всі поліноми $p (\cdot)$, для яких $p(0) = 0$, а тому $\overline{\mathrm{Im}A} = L_2 [0; 1]$
(обміркуйте!).

\noindent\ref{N:1_8_9}. а) $A$ --- компактний самоспряжений оператор в $L_2 [0; 1]$. Тому крім $0$ всі точки спектра --- власні числа.
Нехай $\lambda \neq 0$. $\lambda x(t) = (1-t)\int\limits_0^t s x(s) ds + t \int\limits_t^1 (1-s) x(s) ds$. Права частина --- неперервна
по $t$ функція. Тоді $x \in C[0;1]$. Але тоді права частина --- неперервно диференційовна функція. Тому $x \in C^1 [0; 1]$. І далі: $x \in C^2 [0; 1]$.
Двічі диференціюємо і приходимо до крайової задачі: $\lambda x'' + x = 0; x(0) = x(1) = 0$.

\noindent\ref{N:1_8_12}. Скористаємось 1-ою теоремою Фредгольма і розглянемо супутнє рівняння: $x(t) - \int\limits_0^{\pi} \sin (s-t)x(s) ds = 0$.
$x(t) = \cos t \cdot \int\limits_0^{\pi} \sin s \cdot x(s) ds - \sin t \int\limits_0^{\pi} \cos s \cdot x(s) ds$. $x(t) = A \cos t + B \sin t$.
Одержимо: $A = B = 0$. Тому вихідне рівняння має розв'язок при кожному $f \in L_2 [0; \pi]$.

\noindent\ref{N:2_1_10}. Для перевірки необмеженості $A$ розгляньте послідовність $x_n (t) = \sin (nt)$.
Якщо $x_n \rightrightarrows x$; $y_n = A x_n \rightrightarrows y$ на $[0; 1]$, то $x_n (t) = t y_n(t) \rightrightarrows t y(t)$
на $[0; 1]$; $\lim\limits_{t \rightarrow 0+} \frac{1}{t} x(t) = y(0)$; $(t \in (0; 1]) \Rightarrow (y(t) = \frac{1}{t} x(t))$.

\noindent\ref{N:2_1_27}. $H = L_2 [0; 1]$. $(A u)(t) = j_{\Delta} \cdot u'(t)$; 
$D(A) = \{u$ --- поліном; $u(0) = 0\}$; $\Delta$ --- числовий відрізок $[a; b] \subset [0; 1]$, $[a; b] \neq [0; 1]$ ; $j_{\Delta}$ --- індикатор $\Delta$.

\noindent\ref{N:2_2_20}. \ul{а) $\Rightarrow$ б)} Скористатись \ref{N:2_2_8} і 
довести: $\exists \lim\limits_{t\rightarrow 0+0}\frac{1}{t}\norm{T(t) - I}$.

\noindent \ul{а) $\Rightarrow$ в)} Скористатись \ref{N:2_2_8}. 
$\norm{\lambda A \inv{(\lambda - A)}} = \norm{\sum\limits_{n=0}^\infty 
(\frac{1}{\lambda}A)^{n+1}} \leq \frac{\lambda \norm{A}}{\lambda - \norm{A}}$.

\noindent \ul{в) $\Rightarrow$ а)} Оператори $A_\lambda = \lambda A \inv{(\lambda - A)} = 
\lambda^2 \inv{(\lambda - A)} - \lambda I \in L(X)$ рівномірно обмежені за 
нормою на $(\omega, +\infty)$; для $x \in D(A)$: $A_\lambda x \rightarrow 
Ax$, $\lambda \rightarrow +\infty$. Тому $\forall x \in X$ $\exists Bx 
= \lim\limits_{\lambda \rightarrow +\infty} A_\lambda x$. 
Тож $(A \subset B;$ $B \in L(X))$ $\Rightarrow$ ($A = B$).

\noindent\ref{N:2_2_21}. \ul{a) $\Rightarrow$ б)} $\left(f, g \in D(A)\right)$ 
$\Rightarrow$ $\left(\exists \left.\frac{d}{ds}\right|_{s=0}T(s)f, 
\left.\frac{d}{ds}\right|_{s=0}T(s)g\right)$ $\Rightarrow$ 
$(\exists \left.\frac{d}{ds}\right|_{s=0}T(s)(f\cdot g) = Af \cdot g + f \cdot Ag)$

\ul{б) $\Rightarrow$ а)} $(f, g\in D(A))$ $\Rightarrow$ $\frac{d}{dt}(T(t)f \cdot T(t)g) = 
T(t)f AT(t)g + AT(t)f\cdot T(t)g =$ $=A(T(t)f\cdot T(t)g)$.
З корректності відповідної задачі Коші робимо висновок:
$T(t)f \cdot T(t)g = T(t)(f \cdot g)$.

Для довільних $f, g \in X$: $\exists f_n, g_n \in D(A)$ такі, що $f_n \rightrightarrows f$, 
$g_n \rightrightarrows g$. Тоді $f_n g_n \rightrightarrows fg$ і застосувати граничний перехід.

\noindent\ref{N:2_2_24}. $(x\in D(A)) \Rightarrow \left( x\in D(T'(0))\right)
\Rightarrow (\exists AT(t)x = T'(0)T(t)x = T(t)T'(0)x = T(t)Ax)$.
Далі: $\lambda > \omega$, де $\norm{T(t)} \leq M e^{\omega t}$.
$(x\in D(A)) \Rightarrow
\left( x\in D(T''(0))\right)\Rightarrow \left( \intl{0}{\infty} e^{-\lambda t}T(t)x dt =
\frac{1}{\lambda}x + \frac{1}{\lambda} \intl{0}{\infty} e^{-\lambda t}T'(0)T(t)x dt =
\frac{1}{\lambda}x + \frac{1}{\lambda} \intl{0}{\infty} e^{-\lambda t}AT(t)x dt
\right)$.
Оскільки $\exists \overline{A}$, то отримуємо для $x \in D(A)$ рівність:
$\lambda \intl{0}{\infty} e^{-\lambda t}T(t)x dt - \intl{0}{\infty} e^{-\lambda t}\overline{A}T(t)x dt = x$.
Збіжність невласного інтеграла та замкненість $\overline{A}$ дають:\\
$(*)\qquad (\lambda - \overline{A})\intl{0}{\infty} e^{-\lambda t}T(t)x dt = x$.\\
Оскільки оператор
$R_\lambda: x \mapsto \intl{0}{\infty} e^{-\lambda t}T(t)x dt$ обмежений,
то (з урахуванням замкненості оператора $\lambda - \overline{A}$) одержимо рівність $(*)$
для всіх $x \in X$. Тому $D(T'(0))\subset D(\overline{A})$; $\overline{A} = T'(0)$

\noindent\ref{N:3_1_4}. В просторі $H^1[a,b]$ щільною є підмножина функцій з $C^1[a,b]$
(але за нормою $H^1[a,b]$). Далі достатньо довести існування зліченної сім'ї функцій в
$C^1[a,b]$, яка щільна за нормою $C^1[a,b]$ (збіжність за нормою $C^1[a,b]$ індукує
збіжність за нормою $H^1[a,b]$).
$\forall \varepsilon >0$ існує многочлен $p(x)$ з раціональними коефіцієнтами, для
якого $\sup\limits_{a\leq x\leq b}|u'(x)-p(x)| \leq \varepsilon$. Тоді для многочлена
$q(x) = u(a) + \intl{a}{x}p(t)dt$:\; $\sup\limits_{a\leq x \leq b}|u(x) - q(x)| \leq \varepsilon(b-a)$.
Далі прийдіть до висновку, що многочлени з раціональними коефіцієнтами утворюють зліченну
щільну підмножину в $H^1[a,b]$.

\noindent\ref{N:3_1_15}. Оскільки вкладення $A: H^1[0; \pi] \to C[0; \pi]$ є компактним
оператором, то $\mathrm{Im} A$ є зліченним об'єднанням передкомпактних множин в $C[0; \pi]$.
Оскільки $\dim C[0; \pi] = \infty$, то передкомпакти в $C[0; \pi]$ ніде не щільні і в разі
рівності $\mathrm{Im} A = C[0;\pi]$ приходимо до суперечності з теоремою Бера.
(інформація із задачника: $f(t) = \suml{n=1}{\infty} \frac{\sin(nt)}{n\sqrt{n}} \not\in \mathrm{Im}A$)

\noindent\ref{N:3_1_17}. Нехай $y(t) = \intl{a}{t} x(s) ds$. Перевірте обмеженість
оператора $B: L_2[a,b] \ni x \mapsto y \in L_2[a,b]$ і зробіть звідси висновок, щільною
$\mathrm{Im}A = L_2[a,b]$; $\mathrm{Ker}A = \real \cdot \mathds{1} = \{\mathrm{const}\}$.
$A$ --- обмежений; рівність $\norm{A} = 1$ та відсутність компактності перевірте за допомогою
послідовності $x_n(t) = \frac{1}{n}\sin(\pi n t)$.

\noindent\ref{N:3_2_12}. д) $\pair{f}{\varphi} = \underset{\varepsilon \to 0+}{\lim}
\intl{\varepsilon}{+\infty} \frac{\varphi(x) + \varphi(-x) - 2\varphi(0)}{x^2} dx$.
Оскільки $\varphi(x) = \varphi(0) + x\varphi'(0) + \frac{1}{2}x^2 (\varphi''(0) + \alpha(x))$,
де $\alpha(x) \to 0$, $x\to 0$,
то $\frac{\varphi(x) + \varphi(-x) - 2\varphi(0)}{x^2} = 
\varphi''(0) + \frac{\alpha(x) + \alpha(-x)}{2} \to \varphi''(0)$, $x \to 0$.\\
$\varphi(x) = \varphi(0) + x\varphi'(0) + \frac{1}{2}x^2\varphi''(\nu(x)\cdot x)$, $0 < \nu(x) < 1$.
Тому  $\pair{f}{\varphi} = \frac{1}{2}\intl{0}{+\infty} (\varphi''(\nu(x)x) + \varphi''(\nu(-x)(-x)))dx$,
звідки одержуємо неперервність функціонала $f$.

\noindent\ref{N:3_2_21}. а) $\pair{|\sin x|'' + |\sin x|}{\varphi} = 
\intl{\real}{} |\sin x| (\varphi''(x) + \varphi(x)) dx = \suml{k\in \mathbb{Z}}{} \intl{k \pi - \frac{\pi}{2}}{k \pi + \frac{\pi}{2}} |\sin x| (\varphi''(x) + \varphi(x)) dx =
\suml{k\in \mathbb{Z}}{} (-1)^k \left( \intl{k \pi}{k \pi + \frac{\pi}{2}}  \sin x \left(\varphi''(x) + \varphi(x)\right) dx - 
\intl{k \pi - \frac{\pi}{2}}{k \pi}  \sin x \left(\varphi''(x) + \varphi(x)\right) dx\right) =
\left[ \intl{\alpha}{\beta} \sin x \left(\varphi''(x) + \varphi(x)\right) dx = \varphi'(x) \cdot \sin x \rvert_\alpha^\beta -
\varphi(x) \cdot \cos x \rvert_\alpha^\beta - \intl{\alpha}{\beta} \sin x \cdot \varphi(x) dx\right] = $

\noindent$\suml{k\in \mathbb{Z}}{} (-1)^k \left((-1)^k \varphi'(k \pi + \frac{\pi}{2}) + (-1)^k \varphi'(k \pi) + (-1)^{k-1} \varphi'(k \pi - \frac{\pi}{2}) + (-1)^k \varphi'(k \pi)\right) =$

\noindent$\suml{k\in \mathbb{Z}}{} \left( \varphi'(k \pi + \frac{\pi}{2}) - \varphi'(k \pi - \frac{\pi}{2})\right) + 2 \suml{k\in \mathbb{Z}}{} \varphi(k \pi) = \pair{2\suml{k\in \mathbb{Z}}{} \delta_{k \pi}}{\varphi}$.